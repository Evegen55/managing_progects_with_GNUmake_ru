%%%-------------------------------------------------------------------
%%% How to write a simple makefile
%%%-------------------------------------------------------------------
\chapter{Как написать простой \Makefile{}}
\label{chap:simple_makefile}
Механизм создания программ обычно довольно прост: редактирование
исходного кода, компиляция в исполняемый файл и отладка полученной
программы. Несмотря на то, что компиляция считается рутинной
операцией, будучи проведённой некорректно, она может породить ошибки,
на исправление которых у программиста уйдёт очень много времени.
Многие разработчики испытывают удивление, когда после исправления
какой-нибудь функции и запуска нового варианта программы видят,
что их изменения не исправляют ошибок. Позже они обнаружат, что их
модифицированный код никогда не выполнялся по причине ошибки процедуры
компиляции, компоновки или сборки JAR\hyp{}архива. Более того, с
ростом сложности программ эти рутинные задачи становятся источниками
всё более сложных ошибок, поскольку одновременно могут разрабатываться
различные версии приложения, например, для разных платформ или с
использованием разных версий библиотек.

Программа \GNUmake{} была разработана для того, чтобы автоматизировать
рутинные задачи трансформации исходного кода в исполняемые файлы.
Преимущество \GNUmake{} над простыми сценариями командного
интерпретатора состоит в возможности описать отношения между
элементами проекта. \GNUmake{}, основываясь на описанных отношениях и
данных о времени модификации файлов, сможет определить, какие именно
шаги необходимо осуществить для получения необходимой вам версии
программы. Используя эту информацию, \GNUmake{} также сможет
оптимизировать процесс сборки и избежать выполнения ненужных действий.

GNU \GNUmake{} предоставляет язык для описания отношений между
исходным кодом, промежуточными и исполняемыми файлами.
\GNUmake{} также включает функциональность для управления
конфигурациями, реализации библиотек спецификаций, пригодных для
повторного использования, и параметризации процесса сборки через
механизм макросов, определяемых пользователем. \GNUmake{} может
рассматриваться как каркас всего процесса разработки, выделяющий
компоненты приложения и описывающий способы связать их в единое целое.

Спецификация, используемая \GNUmake{}, обычно сохраняется в файле с
именем \Makefile{}. Ниже приведён пример \Makefile{}'а пригодного для
сборки традиционной программы <<Hello, World>>:

{\footnotesize
\begin{verbatim}
hello: hello.c
    gcc hello.c -o hello
\end{verbatim}
}

Для того, чтобы собрать программу, достаточно выполнить в командном
интерпретаторе следующую команду:

{\footnotesize
\begin{alltt}
\$ \textbf{make}
\end{alltt}
}

Это приведёт к запуску \GNUmake{}, который затем прочитает \Makefile{}
и соберёт первую цель, определённую в нём. В итоге вы увидите
следующее:

{\footnotesize
\begin{alltt}
\$ \textbf{make}
gcc hello.c -o hello
\end{alltt}
}

Цель может быть передана в качестве аргумента командной строки,
в этом случае \GNUmake{} будет пытаться собрать указанную вами цель.
В противном случае для сборки будет выбрана первая цель, определённая
\index{Цели!по умолчанию}
в \Makefile{}'е (называемая также \newword{целью по умолчанию}).

Обычно цель по умолчанию предназначена для сборки всего приложения, и
процесс сборки этой цели состоит из определённой последовательности
шагов.  Например, довольно часто исходный код приложения должен быть
составлен при помощи программ наподобие \utility{flex} или
\utility{bison}. Полученный исходный код нужно скомпилировать в
объектные файлы (файлы с расширением \filename{.o} или \filename{.obj}
для \Clang{}/\Cplusplus{} и \filename{.class} для Java).  Затем
объектные файлы (для случая \Clang{}/\Cplusplus{}) должны быть собраны
компоновщиком (обычно вызываемым компилятором \utility{gcc}) для
получения исполняемого файла.

Модификация любого из исходных файлов требует повторного вызова
\GNUmake{}, который инициирует повторение некоторых (обычно не всех)
из вышеописанных действий таким образом, чтобы получить исполняемый
файл с новой функциональностью. Файл спецификаций, \Makefile{},
описывает отношения между исходным кодом, промежуточными и
исполняемыми файлами, что позволяет \GNUmake{} выполнять минимум
необходимой работы для получения новой версии исполняемого файла.

Принципиальное значение \GNUmake{} заключается в его возможности
осуществлять сложные последовательности операций, необходимые для
сборки приложения, и производить оптимизацию выполнения этих операций
для максимального сокращения времени, занимаемого циклом
<<модификация\hyp{}компиляция\hyp{}отладка>>. Более того, этот
инструмент настолько гибок, что может быть использован везде, где
существуют зависимости между файлами, начиная с традиционных программ
на \Clang/\Cplusplus{} и заканчивая приложениями на \Java{},
документами \TeX{}, управлением базами данных и конвертацией
изображений.

%%--------------------------------------------------------------------
%% Targets and prerequisites
%%--------------------------------------------------------------------
\section{Цели и реквизиты}
По существу \Makefile{} содержит набор правил для сборки приложения.
\index{Правила!по умолчанию}
Первое правило, обнаруженное \GNUmake{}, становится \newword{правилом
по умолчанию}. Правила состоят из трёх составляющих: целей, реквизитов
и сценария сборки:

{\footnotesize
\begin{alltt}
\emph{цель: реквизит\subi{1} реквизит\subi{2}
    сценарий сборки
}
\end{alltt}
}

\index{Цели}
\newword{Цель}~--- это файл или некоторая сущность, требующая
обновления.
\index{Реквизиты}
\newword{Реквизиты}~--- это те файлы, которые должны существовать для
того, чтобы цель могла быть собрана.
\index{Сценарий сборки}
\newword{Сценарий сборки}~--- это команды интерпретатора, описывающие
способ создания цели из реквизитов.

Вот правила компиляции исходного файла на языке \Clang{}
\filename{foo.c} в объектный файл \filename{foo.o}:

{\footnotesize
\begin{verbatim}
foo.o: foo.c foo.h
    gcc -c foo.c
\end{verbatim}
}

Цель \filename{foo.o} указана слева от двоеточия, реквизиты
\filename{foo.c} и \filename{foo.h}~--- справа. Сценарий сборки обычно
начинается со следующей строки и предваряется символом табуляции.

Когда требуется выполнить правило, \GNUmake{} пытается найти все
файлы-реквизиты, ассоциированные с целью. Если хотя\hyp{}бы один из
реквизитов выступает в качестве цели какого\hyp{}либо правила, то
сначала выполняется правило для этого реквизита. Далее проверяются
даты последней модификации файла\hyp{}цели и реквизитов: если
какой-либо из реквизитов модифицировался позже цели, то выполняется
сценарий сборки. Каждая строка сценария выполняется в отдельном
процессе командного интерпретатора. Если хотя\hyp{}бы одна команда
приводит к ошибке, сборка цели завершается и \GNUmake{} прекращает
выполнение.

Ниже представлена программа подсчёта числа вхождений слов <<fee>>,
<<fie>>, <<foe>> и <<fum>> в содержимое стандартного потока ввода,
реализованная с помощью лексического анализатора \utility{flex}.

{\footnotesize
\begin{verbatim}
#include <stdio.h>

extern int fee_count, fie_count, foe_count, fum_count;

extern int yylex( void );

int main( int argc, char ** argv )
{
    yylex( );

    printf( "%d %d %d %d \n",
        fee_count, fie_count, foe_count, fum_count );

    exit( 0 );
}
\end{verbatim}
}

Код лексического анализатора очень прост:

{\footnotesize
\begin{verbatim}
        int fee_count = 0;
        int fie_count = 0;
        int foe_count = 0;
        int fum_count = 0;
%%
fee     fee_count++;
fie     fie_count++;
foe     foe_count++;
fum     fum_count++;
\end{verbatim}
}

Составление \Makefile{} для сборки этой программы также не вызывает
затруднений:

{\footnotesize
\begin{verbatim}
count_words: count_words.o lexer.o -lfl
        gcc count_words.o lexer.o -lfl -o count_words

count_words.o: count_words.c
        gcc -c count_words.c

lexer.o: lexer.c
        gcc -c lexer.c

lexer.c: lexer.l
        flex -t lexer.l > lexer.c
\end{verbatim}
}

Когда мы запустим \GNUmake{} впервые, мы увидим следующее:

{\footnotesize
\begin{alltt}
\$ \textbf{make}

gcc -c count\_words.c
flex -t lexer.l > lexer.c
gcc -c lexer.c
gcc counti\_words.o lexer.o -lfl -ocount\_words
\end{alltt}
}

Теперь мы имеем исполняемый файл программы. Естественно, реальные
приложения, как правило, состоят из большего числа модулей, чем наш
пример. Кроме того, позже мы увидим, что наш \Makefile{} не использует
большей части функциональности \GNUmake{} и потому гораздо более
объёмен чем мог бы быть. Тем не менее, это всё-же функциональный и
полезный \Makefile{}.

Как вы могли заметить, порядок, в котором команды реально выполняются,
и порядок их появления в \Makefile{}'е противоположны. Стиль
опеделения <<сверху вниз>> является общим стилем составления
спецификаций для \GNUmake{}. Обычно сначала описываются правила для
наиболее важных целей, а детали оставляются на потом. В \GNUmake{}
существует несколько механизмов поддержки такого стиля. Главными среди
\index{Переменные!рекурсивные}
них являются двухфазовая модель выполнения \GNUmake{} и рекурсивные
переменные. Мы обсудим эти важные детали в последующих главах.

%%--------------------------------------------------------------------
%% Dependency checking
%%--------------------------------------------------------------------
\section{Разрешение зависимостей}

Как же \GNUmake{} решает, что делать? Давайте рассмотрим выполнение
предыдущего примера более детально и выясним это.

Сначала \GNUmake{} замечает, что командная строка не содержит
аргументов и решает собрать цель по умолчанию, т.е.
\filename{count\_words}. Затем выполняется проверка реквизитов, их
обнаруживается три: \filename{count\_words.o}, \filename{lexer.o} и
\filename{-lfl}.  Затем \GNUmake{} ищет способ собрать цель
\filename{count\_words.o} и находит правило для неё. Снова происходит
проверка реквизитов. \GNUmake{} замечает, что цель
\filename{count\_words.c} не имеет правила, и файл с таким именем
существует, поэтому производится трансформация
\filename{count\_words.c} в \filename{count\_words.o}, для
осуществления которой выполняется следующая команда:

{\footnotesize
\begin{verbatim}
gcc -c count_words.c
\end{verbatim}
}

Подобная цепь переходов от целей к реквизитам и рассмотрение
реквизитов в качестве целей~--- типичный механизм, при помощи которого
\GNUmake{} проводит анализ \Makefile{}'а и решает, какие команды
подлежат выполнению.

Следующим реквизитом, рассматриваемым \GNUmake{}, является
\filename{lexer.o}. Цепочка правил, подобная рассмотренной, ведёт к
файлу \filename{lexer.c}, ещё не существующему. Далее \GNUmake{}
находит правило получения \filename{lexer.c} из \filename{lexer.l},
\index{flex}
согласно которому запускается программа \utility{flex}.
\index{gcc}
Теперь \filename{lexer.c} существует и можно запускать \utility{gcc}.

\index{Библиотека}
Наконец, \GNUmake{} проверяет реквизит \filename{-lfl}. Опция
\command{-l} сообщает \utility{gcc}, что приложение использует
системную библиотеку. Имя библиотеки <<fl>> преобразуется согласно
правилам именования библиотек в \filename{libfl.a}. GNU \GNUmake{}
имеет поддержку этого синтаксиса. Когда находится реквизит вида
\filename{-l<NAME>}, \GNUmake{} ищет файл с именем
\filename{libNAME.so} в стандартных каталогах библиотек. Если поиск
заканчивается неудачей, производится повторный поиск по имени
\filename{libNAME.a}.  В нашем случае поиск успешен, \GNUmake{}
находит файл \filename{/usr/lib/libfl.a} и производит финальное
действие~--- компоновку.

%%--------------------------------------------------------------------
%% Minimazing rebuilds
%%--------------------------------------------------------------------
\section{Минимизируем число действий}

Если мы запустим нашу программу, то обнаружим, что помимо вывода числа
вхождений слов fee, fies, foes и fums она также выводит содержимое
входного файла. Это совсем не то, что от неё ожидалось. Проблема в
том, что мы забыли добавить несколько правил в наш лексический
\index{flex}
анализатор, и \utility{flex} выдаёт нераспознанный текст на
стандартный поток вывода. Для решения этой проблемы мы просто добавим
правило для <<любого символа>>:

{\footnotesize
\begin{verbatim}
        int fee_count = 0;
        int fie_count = 0;
        int foe_count = 0;
        int fum_count = 0;
%%
fee     fee_count++;
fie     fie_count++;
foe     foe_count++;
fum     fum_count++;
.
\n
\end{verbatim}
}

После редактирования файла с исходным кодом анализатора нам нужно
собрать наше приложение заново и проверить его работу:

{\footnotesize
\begin{alltt}
\$ \textbf{make}

flex -t lexer.l > lexer.c
gcc -c lexer.c
gcc count\_words.o lexer.o -lfl -ocount\_words
\end{alltt}
}

На этот раз файл \filename{count\_words.c} не подвергся компиляции.
При анализе правил \GNUmake{} обнаружил, что файл
\filename{count\_words.o} существует и имеет более позднюю дату
модификации, чем цель, потому не было предпринято никаких действий по
сборке этой цели. Однако при анализе цели \filename{lexer.c} было
обнаружено, что реквизит \filename{lexer.l} имеет более позднюю дату
модификации, поэтому произошла повторная сборка цели
\filename{lexer.c}. Это, в свою очередь, вызвало повторную сборку
\filename{lexer.o}, а затем и \filename{count\_words}. Теперь наша
программа подсчёта слов работает правильно:

{\footnotesize
\begin{alltt}
\$ \textbf{count\_words < lexer.l}
3 3 3 3
\end{alltt}
}

%%--------------------------------------------------------------------
%% Invoking make
%%--------------------------------------------------------------------
\section{Вызов \GNUmake{}}
\label{sec:invoking_make}

В предыдущих примерах предполагается, что:
\begin{itemize}
%---------------------------------------------------------------------
\item Все исходные файлы проекта и файл спецификации \GNUmake{}
хранятся в одной директории.
%---------------------------------------------------------------------
\item Файл спецификации для \GNUmake{} называется
\filename{makefile}, \filename{Makefile} или \filename{GNUMakefile}.
%---------------------------------------------------------------------
\item \Makefile{} находится в текущей директории, когда пользователь
запускает команду \GNUmake{}.
%---------------------------------------------------------------------
\end{itemize}

Если \GNUmake{} запускается с соблюдением этих условий, автоматически
производится попытка собрать первую цель в файле спецификации. Чтобы
собрать другую цель (или несколько целей), необходимо указать имя
этой цели в качестве аргументов командной строки:

{\footnotesize
\begin{alltt}
\$ \textbf{make lexer.c}
\end{alltt}
}

В этом случае после запуска \GNUmake{} произведёт чтение файла
спецификации и определение цели для обновления. Если цель или один из
реквизитов устарели (или не существуют), тогда в точности один раз
\index{Сценарий сборки}
будет выполнен сценарий их сборки. После выполнения сценария цель
предполагается существующей и обновлённой, и процесс повторяется для
следующей цели; после сборки всех целей процесс завершается.

Если указанная вами цель не требует повторной сборки, \GNUmake{}
завершит работу с сообщением следующего вида:

{\footnotesize
\begin{alltt}
\$ \textbf{make lexer.c}
make:  `lexer.c' is up to date.
\end{alltt}
}

Если в качестве цели будет указана цель, не присутствующая в
\index{Правила!неявные}
\Makefile{}'е, и для которой не существует неявного правила (см.
главу~\ref{chap:rules}), то \GNUmake{} закончит выполнение с
сообщением следующего вида:

{\footnotesize
\begin{alltt}
\$ \textbf{make non-existent-target}
make: *** No rule to make target `non-existent-target'. Stop.
\end{alltt}
}

У команды \GNUmake{} есть множество опций командной строки. Одной из
\index{Опции!just-print@\command{-{}-just-print (-n)}}
самых полезных является опция \command{-{}-just\hyp{}print} (или просто
\command{-n}), которая сообщает \GNUmake{}, что нужно только отобразить
последовательность действий, необходимых для сборки цели. Эту
возможность очень удобно использовать для отладки \Makefile{}'ов.
Также полезной возможностью является передача или переопределение
значений переменных \GNUmake{} через аргументы командной строки.

%%--------------------------------------------------------------------
%% Basic makefile syntax
%%--------------------------------------------------------------------
\section{Основы синтаксиса \Makefile{}}
Теперь, когда вы уже получили базовое представление о \GNUmake{},
можно приступить к составлению собственных \Makefile{}'ов. В этом
разделе мы рассмотрим синтаксис и структуру \Makefile{}'а, чтобы вы
могли начать использование \GNUmake{}. 

Как правило, \Makefile{}'ы пишутся в манере <<сверху вниз>>, то есть
сначала описывается наиболее общая цель (как правило, она имеет имя
\index{Цели!по умолчанию} \index{Цели!стандартные!all}
\target{all}), которая будет целью по умолчанию. Далее следуют всё
более детализированные цели, и в самом конце описываются цели,
используемые для поддержки программного продукта (такие, например, как
\index{Цели!стандартные!clean}
\target{clean}, используемая для удаления ненужных вр'{е}менных
файлов). Именами целей вовсе не обязательно должны быть настоящие
файлы, можно использовать любое имя.

\index{Правило}
В предыдущем примере мы видели упрощённую форму правила. Более полной
(но всё ещё не законченной) формой правила является следующая:

{\footnotesize
\begin{alltt}
\emph{цель\subi{1} цель\subi{2} цель\subi{3} : реквизит\subi{1} реквизит\subi{2}
    команда\subi{1}
    команда\subi{2}
    команда\subi{3}
}
\end{alltt}
}

Одна или более целей указываются слева от двоеточия, справа от него
следуют реквизиты. Если реквизиты не указаны, то собираются только те
цели, которые ещё не существуют. Последовательность команд, которые
необходимо выполнить для сборки цели, иногда называют
\index{Сценарий сборки}
\newword{сценарием сборки}, но чаще просто \newword{командами}.

Каждая команда должна начинаться с символа табуляции. Такой синтаксис
сообщает \GNUmake{} о том, что следующие за табуляцией символы должны
быть переданы в командный интерпретатор для последующего выполнения.
Если вы случайно поставите символ табуляции в начале строки, не
являющейся командой, то в большинстве случаев \GNUmake{} будет
интерпретировать последующий текст в этой строке как команду. Однако
может случиться и так, что \GNUmake{} сможет распознать ваш символ
табуляции как синтаксическую ошибку, в этом случае вы увидите
сообщение, подобное следующему:

{\footnotesize
\begin{alltt}
\$ \textbf{make}
Makefile:6: *** commands commence before first target.  Stop.
\end{alltt}
}

Мы вернёмся к аспектам использования символа табуляции в
главе~\ref{chap:rules}.

