%%--------------------------------------------------------------------
%% Invoking make
%%--------------------------------------------------------------------
\section{Вызов \GNUmake{}}
\label{sec:invoking_make}

В предыдущих примерах предполагается, что:
\begin{itemize}
%---------------------------------------------------------------------
\item Все исходные файлы проекта и файл спецификации \GNUmake{}
хранятся в одной директории.
%---------------------------------------------------------------------
\item Файл спецификации для \GNUmake{} называется
\filename{makefile}, \filename{Makefile} или \filename{GNUMakefile}.
%---------------------------------------------------------------------
\item \Makefile{} находится в текущей директории, когда пользователь
запускает команду \GNUmake{}.
%---------------------------------------------------------------------
\end{itemize}

Если \GNUmake{} запускается с соблюдением этих условий, автоматически
производится попытка собрать первую цель в файле спецификации. Чтобы
собрать другую цель (или несколько целей), необходимо указать имя
этой цели в качестве аргументов командной строки:

{\footnotesize
\begin{alltt}
\$ \textbf{make lexer.c}
\end{alltt}
}

В этом случае после запуска \GNUmake{} произведёт чтение файла
спецификации и определение цели для обновления. Если цель или один из
реквизитов устарели (или не существуют), тогда в точности один раз
\index{Сценарий сборки}
будет выполнен сценарий их сборки. После выполнения сценария цель
предполагается существующей и обновлённой, и процесс повторяется для
следующей цели; после сборки всех целей процесс завершается.

Если указанная вами цель не требует повторной сборки, \GNUmake{}
завершит работу с сообщением следующего вида:

{\footnotesize
\begin{alltt}
\$ \textbf{make lexer.c}
make:  `lexer.c' is up to date.
\end{alltt}
}

Если в качестве цели будет указана цель, не присутствующая в
\index{Правила!неявные}
\Makefile{}'е, и для которой не существует неявного правила (см.
главу~\ref{chap:rules}), то \GNUmake{} закончит выполнение с
сообщением следующего вида:

{\footnotesize
\begin{alltt}
\$ \textbf{make non-existent-target}
make: *** No rule to make target `non-existent-target'. Stop.
\end{alltt}
}

У команды \GNUmake{} есть множество опций командной строки. Одной из
\index{Опции!just-print@\command{-{}-just-print (-n)}}
самых полезных является опция \command{-{}-just\hyp{}print} (или просто
\command{-n}), которая сообщает \GNUmake{}, что нужно только отобразить
последовательность действий, необходимых для сборки цели. Эту
возможность очень удобно использовать для отладки \Makefile{}'ов.
Также полезной возможностью является передача или переопределение
значений переменных \GNUmake{} через аргументы командной строки.
