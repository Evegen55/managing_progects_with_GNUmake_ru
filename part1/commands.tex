%%--------------------------------------------------------------------
%% Commands
%%--------------------------------------------------------------------
\chapter{Команды}
\label{chap:commands}

Мы уже рассмотрели б\'{о}льшую часть базовых концепций команд
\GNUmake{}, однако давайте всё же проведём их краткий обзор.

Команды являются однострочными сценариями командного интерпретатора.
\GNUmake{} считывает каждую команду и передаёт её в дочерний процесс
командного интерпретатора для выполнения. На самом деле \GNUmake{}
может осуществлять оптимизацию этого (относительно) ресурсоёмкого
fork/exec алгоритма, если это гарантированно не изменит поведения
программы. Для этого каждая команда сканируется на наличие специальных
символов командного интерпретатора, таких как шаблоны или перенаправления
ввода-вывода. Если эти символы не найдены, \GNUmake{} выполняет команду
самостоятельно без порождения дочернего процесса.

В качестве командного интерпретатора по умолчанию используется
\filename{/bin/sh}. За используемый интерпретатор отвечает переменная
\variable{SHELL}, не наследуемая из окружения. При старте \GNUmake{}
импортирует все переменные, кроме \variable{SHELL}, из окружения
пользователя, преобразуя их в переменные \GNUmake{}. Это сделано для
того, чтобы пользовательский выбор интерпретатора не вызвал ошибки
выполнения \Makefile{}'а (возможно, входящего в загруженный из сети
пакет программного обеспечения). Если пользователь действительно хочет
поменять интерпретатор по умолчанию, он должен явно присвоить
переменной \variable{SHELL} нужное значение прямо в \Makefile{}'е. Мы
вернёмся к этой теме в разделе <<\nameref{sec:which_shell_to_use}>> этой
главы.

%%--------------------------------------------------------------------
%% Parsing commands
%%--------------------------------------------------------------------
\section{Синтаксический анализ команд}
\label{sec:parsing_commands}

Строки, начинающиеся с символа табуляции и следующие за спецификацией
цели, считаются командами (если только предыдущая строка не
завершалась символом обратного слэша). Встретив символ табуляции в
другом контексте, GNU \GNUmake{} старается догадаться, что вы имели
ввиду. Например, присваивания переменных, комментарии и директивы
включения могут начинаться с символа табуляции, если это не приводит к
неоднозначности. Если \GNUmake{} обнаруживает команду, которая не
следует за спецификацией цели, выводится сообщение об ошибке:

{\footnotesize
\begin{verbatim}
makefile:20: *** commands commence before first target. Stop.
\end{verbatim}
}

Формулировка этого сообщения немного удивляет, ведь оно часто
появляется в середине \Makefile{}'а, много позже спецификации
<<первой>> цели, однако теперь мы в состоянии понять это сообщение без
особых проблем. Пожалуй, лучшей формулировкой была бы следующая:
<<обнаружена команда вне контекста определения правила>>.

Когда синтаксический анализатор обнаруживает команду в подходящем
контексте, он включает <<режим разбора команды>>, строя сценарий
сборки по строке за раз. Добавление текста к сценарию прекращается,
когда обнаруживается строка, не могущая быть частью сценария.  В
сценарии могут появляться следующие строки:

\begin{itemize}
%---------------------------------------------------------------------
\item Строки, начинающиеся с символа табуляции, являются командами,
предназначенными для выполнения в командном интерпретаторе. Даже
строки, которые могли бы быть интерпретированы как конструкции
\GNUmake{} (например, директивы \directive{ifdef}, \directive{include}
или комментарии), в режиме разбора команд трактуются как команды.
%---------------------------------------------------------------------
\item Пустые строки игнорируются, их выполнение не осуществляется.
%---------------------------------------------------------------------
\item Строки, начинающиеся с символа \command{\#}, возможно, с
начальными пробелами (не символами табуляции!), воспринимаются как
комментарии \GNUmake{} и игнорируются.
%---------------------------------------------------------------------
\item Директивы условной обработки, такие как \directive{ifdef} и
\directive{ifeq}, распознаются и правильно обрабатываются в контексте
сценария сборки.
%---------------------------------------------------------------------
\end{itemize}

Встроенные функции \GNUmake{}, не предваряемые символом табуляции, не
приводят к выходу из режима разбора команд. Это значит, что их
значением должны быть допустимые команды интерпретатора или пустая
строка.  Значением функций \function{warning} или \function{eval}
является пустая строка.

Тот факт, что пустые строки или комментарии \GNUmake{} допустимы в
сценариях сборки, поначалу может удивлять. Следующий пример
показывает, как это работает:

{\footnotesize
\begin{verbatim}
long-command:
    @echo Строка 2:  далее пустая строка

    @echo Строка 4:  далее комментарий shell
    # Комментарий командного интерпретатора \
      (начинается с символа табуляции)
    @echo Строка 6:  далее комментарий make
# комментарий make в начале строки
    @echo Строка 8:  далее комментарий make
  # выровненный пробелами комментарий make
    # ещё один выровненный пробелами комментарий make
    @echo Строка 11: далее директива make
  ifdef COMSPEC
    @echo мы работаем под Windows
  endif
    @echo Строка 15: далее <<команда>> warning
    $(warning предупреждение)
    @echo Строка 17: далее <<команда>> eval
    $(eval $(shell echo Вывод shell 1>&2))
\end{verbatim}
}

Заметим, что строки 5 и 10 выглядят очень похоже, но по сути они очень
разные. Строка 5 содержит комментарий командного интерпретатора,
начинающийся с символа табуляции, в то время как строка 10 содержит
комментарий \GNUmake{}, начинающийся с отступа в 4 пробела.  Очевидно,
что форматирование своих \Makefile{}'ов подобным образом~--- не самая
лучшая идея (если только вы не собираетесь участвовать в конкурсе
самых запутанных \Makefile{}'ов). Как вы можете видеть из следующего
листинга, комментарии \GNUmake{} не выполняются и не выводятся, даже
если они появляются в контексте сценария сборки:

{\footnotesize
\begin{verbatim}
$ make
makefile:2: предупреждение
Вывод shell
Строка 2:  далее пустая строка
Строка 4:  далее коментарий shell
# комментарий командного интерпретатора (начинается...)
Строка 6:  далее комментарий make
Строка 8:  далее комментарий make
Строка 11: далее директива make
мы работаем под Windows
Строка 15: далее <<команда>> warning
Строка 17: далее <<команда>> eval
\end{verbatim}
}

Сначала может показаться, что вывод функций \function{warning} и
\function{eval} появился слишком рано, однако это не так (мы обсудим
порядок вычисления в этой главе в разделе
<<\nameref{sec:evaluating_commands})>>. Тот факт, что сценарий сборки
может содержать произвольное число пустых строк и комментариев, может
быть источником трудно находимых ошибок.  Предположим, вы случайно
добавили строку с начальным символом табуляции.  Если выше неё
располагается определение цели, за которым следуют только комментарии
и пустые строки, \GNUmake{} будет трактовать вашу строку со случайным
символом табуляции как команду, ассоциированную с предыдущей целью.
Как вы уже видели, это вполне допустимо и не вызовет ошибки или
предупреждения, пока та же цель не встретится в составе другого
правила в другом месте \Makefile{}'а (или во включаемом файле).

Если вам повезёт, ваш \Makefile{} будет содержать непустую строку, не
содержащую комментария, между вашей ошибочной строкой и предыдущим
сценарием сборки. В этом случае вы получите сообщение <<commands
commence before first target>>.

Сейчас самое время упомянуть об инструментальных средствах. Я думаю,
теперь все согласны с тем, что использование символа табуляции для
обозначения команды было не самым удачным решением, однако теперь уже
поздно что-либо менять. Использование современного, понимающего
синтаксис текстового редактора может помочь вам предотвратить
потенциальные проблемы с помощью цветового выделения сомнительных
конструкций. GNU \utility{emacs} имеет довольно удобный режим для
редактирования \Makefile{}'ов. Этот режим осуществляет подсветку
синтаксиса и находит простые синтаксические ошибки, такие как пробелы
после символа переноса строки или смешанные символы табуляции и
пробелы. Мы вернёмся к теме совместного использования редактора
\utility{emacs} и \GNUmake{} немного позже.

%---------------------------------------------------------------------
% Continuing long commands
%---------------------------------------------------------------------
\subsection{Продолжение длинных команд}

Поскольку каждая команда выполняется в отдельном процессе командного
интерпретатора, последовательности выражений, которые должны
выполняться совместно, должны обрабатываться особым образом.
Предположим для примера, что нам нужно создать файл, содержащий список
файлов. Компилятор \Java{} принимает такие файлы в случае, если нужно
скомпилировать много исходного кода. Для этой цели мы можем написать
следующее правило:

{\footnotesize
\begin{verbatim}
.INTERMEDIATE: file_list

file_list:
    for d in logic ui
    do
      echo $d/*.java
    done > $@
\end{verbatim}
}

Очевидно, что этот пример не будет работать и вызовет ошибку при
запуске:

{\footnotesize
\begin{alltt}
  \${} \textbf{make}
  for d in logic ui
  /bin/sh: -c: line 2: syntax error: unexpected end of file
  make: *** [file\_{}list] Error 2
\end{alltt}
}

В качестве решения можно попробовать добавить символы продолжения в
конце каждой строки:

{\footnotesize
\begin{verbatim}
.INTERMEDIATE: file_list

file_list:
    for d in logic ui \
    do                \
      echo $d/*.java  \
    done > $@
\end{verbatim}
}

Однако теперь при запуске появляется другая ошибка:

{\footnotesize
\begin{alltt}
\${} \textbf{make}
for d in logic ui \textbackslash{}
do                \textbackslash{}
  echo /*.java    \textbackslash{}
done > file\_list
/bin/sh: -c: line 1: syntax error near unexpected token `>'
/bin/sh: -c: line 1: `for d in logic ui do       echo /*.java
make: *** [file\_list] Error 2
\end{alltt}
}

Что же произошло? В коде есть две ошибки. Во-первых, ссылка на
переменную цикла, \variable{d}, должна быть экранирована. Во-вторых,
поскольку цикл передаётся в интерпретатор одной строкой, мы должны
добавить точку с запятой после списка файлов и выражения в теле цикла:

{\footnotesize
\begin{verbatim}
.INTERMEDIATE: file_list

file_list:
    for d in logic ui; \
    do                 \
      echo $d/*.java;  \
    done > $@
\end{verbatim}
}

Теперь мы получим именно то, что ожидали. Поскольку цель
\target{file\_list} помечена как \variable{.INTERMEDIATE}, \GNUmake{}
удалит её после завершения компиляции.

В более реалистичном примере список файлов будет храниться в
переменной \GNUmake{}. Если есть уверенность в том, что этот список
достаточно мал, мы можем осуществить ту же операцию без использования
цикла, используя только встроенные функции \GNUmake{}:

{\footnotesize
\begin{verbatim}
.INTERMEDIATE: file_list

file_list:
    echo $(addsuffix /*.java,$(COMPILATION_DIRS)) > $@
\end{verbatim}
}

Однако у версии с циклом меньше шансов столкнуться с проблемой
конечности длины командной строки, если список каталогов будет расти
со временем.

Ещё одной общей проблемой является смена текущего каталога:

{\footnotesize
\begin{verbatim}
TAGS:
    cd src
    ctags --recurse
\end{verbatim}
}

Очевидно, что предыдущий пример не выполнит программу \utility{ctags}
в каталоге \filename{src}. Чтобы добиться желаемого эффекта, мы должны
либо разместить оба выражения в одной строке, либо экранировать символ
новой строки (разделив выражения точкой с запятой):

{\footnotesize
\begin{verbatim}
TAGS:
    cd src; \
    ctags --recurse
\end{verbatim}
}

Ещё более разумно проверять статус выполнения программы \utility{cd}
перед выполнением \utility{ctags}:

{\footnotesize
\begin{verbatim}
TAGS:
    cd src && \
    ctags --recurse
\end{verbatim}
}

Заметим, что при некоторых обстоятельствах можно опустить точку с
запятой, не вызвав ошибки командного интерпретатора или \GNUmake{}:

{\footnotesize
\begin{verbatim}
disk-free = echo "Проверяем размер дискового пространства..." \
    df . | awk '{ print $$4 }'
\end{verbatim}
}

Этот пример выводит простое сообщение, за которым следует число
свободных блоков на текущем устройстве. Или нет? Мы случайно забыли
поставить точку с запятой после команды \utility{echo}, в результате
чего программа \utility{df} никогда не будет запущена. Вместо этого мы
направим сообщение <<\command{Проверяем размер дискового
пространства...  df .}>> на вход программы \utility{awk}, которая
напечатает четвёртое поле строки, т.е. \command{пространства...}.

Возможно, вам приходилось использовать директиву \directive{define},
предназначенную для создания многострочных последовательностей. К
сожалению, она не решает проблемы переносов строк. Когда происходит
подстановка макроса, каждая его строка вставляется в сценарий сборки с
начальным символом табуляции, и \GNUmake{} работает с каждой строкой
независимо. Разные строки макроса выполняются в разных экземплярах
командного интерпретатора, поэтому вам следуюет обращать внимание на
перенос строк даже в определениях макросов.

%---------------------------------------------------------------------
% Command modifiers
%---------------------------------------------------------------------
\subsection{Модификаторы команд}
\label{sec:command_modifiers}

\index{Модификаторы команд}
Команды могут быть модифицированы при помощи нескольких префиксов.  Мы
уже встречались с <<молчаливым>> префиксом, \command{@}, ниже приведён
полный список возможных префиксов с некоторыми комментариями:

\begin{description}
%---------------------------------------------------------------------
\item[\command{@}] \hfill \\
Подавляет вывод команды. Для исторической совместимости вы можете
поместить вашу команду в реквизиты специальной цели \target{.SILENT},
если хотите, чтобы все команды, ассоциированные с вашей целью, были
скрыты. Однако использование \texttt{@} предпочтительней, поскольку
этот модификатор может быть применён к отдельным командам в сценарии.
Если вы хотите применить модификатор ко всем целям, используйте
\index{Опции!silent@\command{-{}-silent (-s)}}
опцию \command{-{}-si\-lent} (или просто \command{-s}).

Сокрытие команд может сделать вывод \GNUmake{} приятнее для глаза,
однако это также ведёт к затруднениям при отладке. Если вы обнаружите,
что часто удаляете модификатор \command{@}, а затем возвращаете его на
место, разумно завести переменную (к примеру, \variable{QUIET}),
содержащую модификатор, и использовать её в командах:

{\footnotesize
\begin{verbatim}
QUIET = @
hairy_script:
    $(QUIET) сложный сценарий ...
\end{verbatim}
}

В этом случае при необходимости увидеть сложный сценарий в том виде, в
котором его выполняет \GNUmake{}, просто сбросьте значение переменной
\variable{QUIET} в командной строке:

{\footnotesize
\begin{alltt}
\${} \textbf{make QUIET= hairy\_script}
сложный сценарий ...
\end{alltt}
}

%---------------------------------------------------------------------
\item[\command{-}] \hfill \\
Этот префикс сообщает \GNUmake{}, что ошибки, возникающие при
выполнении помеченной команды, следует игнорировать. По умолчанию
\GNUmake{} проверяет код возврата каждой программы или конвейера
программ. Если какая-то программа завершается с ненулевым кодом,
\GNUmake{} прерывает выполнение оставшейся части сценария и завершает
своё выполнение. Этот модификатор заставляет \GNUmake{} игнорировать
код возврата модифицированной строки и продолжать выполнение в обычном
режиме. Мы обсудим эту тему подробнее в следующем разделе.

Для исторической совместимости вы можете игнорировать ошибки в части
сценария сборки, поместив соответствующую цель в реквизиты специальной
цели \target{.IGNORE}. Вы можете игнорировать все ошибки в
\index{Опции!ignore-errors@\command{-{}-ignore\hyp{}errors (-i)}}
\Makefile{}'е используя опцию \command{-{}-ig\-no\-re\hyp{}errors}
(или \command{-i}).  Однако полезность этой возможности вызывает
сомнения.

%---------------------------------------------------------------------
\item[\texttt{+}] \hfill \\
Этот модификатор сообщает \GNUmake{}, что помеченная им команда должна
выполняться даже в том случае, если в опциях командной строки
встречается \command{-{}-just\hyp{}print} (или \command{-n}). Этот
модификатор используется при составлении рекурсивных \Makefile{}'ов.
Мы обсудим эту тему в разделе <<\nameref{sec:recursive_make}>>
главы~\ref{chap:managing_large_proj}.
%---------------------------------------------------------------------
\end{description}

Все модификаторы должны встречаться в строке только один раз.
Очевидно, перед выполнением команд модификаторы вырезаются.

%-------------------------------------------------------------------
% Errors and interrupts
%-------------------------------------------------------------------
\subsection{Ошибки и прерывания}

Каждая команда, выполняемая \GNUmake{}, возвращает код ошибки. Нулевой
код соответствует успешному выполнению команды, ненулевой~--- ошибочной
ситуации. Некоторые программы используют код возврата для передачи
более полезной информации, чем просто факта возникновения ошибки.
Например, программа \utility{grep} возвращает 0 в случае обнаружения
соответствия шаблону, 1 в случае отсутствия соответствий и 2 в случае
возникновения ошибки какого-либо рода.

Обычно при ошибке выполнения команды (т.е. при возвращении ненулевого
кода ошибки) \GNUmake{} прекращает выполнение команд и завершает
выполнение с ненулевым кодом возврата. Иногда нужно, чтобы \GNUmake{}
продолжал работу, собрав столько целей, сколько возможно. Например,
вам может понадобиться скомпилировать максимально возможное число
исходных файлов, чтобы увидеть все ошибки компиляции в один проход.
Вы можете добиться такого поведения при помощи опции
\index{Опции!keep-going@\command{-{}-keep\hyp{}going (-k)}}
\command{-{}-keep\hyp{}going} (или \command{-k}).

Поскольку модификатор \command{-} заставляет \GNUmake{} игнорировать
ошибки в отдельных командах, я стараюсь избегать его использования,
поскольку это усложняет автоматическую обработку ошибок и привносит в
код небрежность.

Когда \GNUmake{} игнорирует ошибку, выводится сообщение об ошибке с
именем цели в квадратных скобках. Ниже приведён вывод, полученный при
попытке удаления несуществующего файла:

{\footnotesize
\begin{verbatim}
rm non-existent-file
rm: cannot remove `non-existent-file': No such file or directory
make: [clean] Error 1 (ignored)
\end{verbatim}
}

Некоторые команды (например, \utility{rm}) имеют опции для подавления
ошибочных кодов возврата. Опция \command{-f} заставит программу
\utility{rm} вернуть нулевой код ошибки и подавить вывод
предупреждений. Использование этой опции лучше, чем зависимость от
модификатора.

Бывают случаи, когда возврат программой нулевого кода ошибки считается
неудачей и наоборот. В таких случаях можно просто инвертировать код
возврата программы:

{\footnotesize
\begin{verbatim}
# Убедимся, что в коде не осталось отладочных сообщений.
.PHONY: no_debug_printf

no_debug_printf: $(sources)
    ! grep --line-number '"debug:' $^
\end{verbatim}
}

К сожалению, версия GNU \GNUmake{} 3.80 содержит ошибку, мешающую
непосредственному использованию этой возможности: \GNUmake{} не
распознаёт символ \command{!} как часть команды, требующей вызова
командного интерпретатора, и выполняет оставшуюся часть строки
самостоятельно, вызывая ошибку. В качестве простого обхода проблемы
можно использовать в команде специальные символы как намёк для
\GNUmake{}:

{\footnotesize
\begin{verbatim}
# Убедимся, что в коде не осталось отладочных сообщений.
.PHONY: no_debug_printf

no_debug_printf: $(sources)
    ! grep --line-number '"debug:' $^ < /dev/null
\end{verbatim}
}

Другим источником неожиданных ошибок в командах является оператор
\command{if} без ветки \command{else}:

{\footnotesize
\begin{verbatim}
$(config): $(config_template)
    if [ ! -d $(dir $@) ];  \
    then                    \
        $(MKDIR) $(dir $@); \
    fi
    $(M4) $^ > $@
\end{verbatim}
}

Первая команда проверяет, существует ли целевой каталог и в случае
необходимости вызывает программу \utility{mkdir} для его создания. К
сожалению, если каталог существует, команда \command{if} вернёт
ненулевой код ошибки (код возврата программы \utility{test}), что
приведёт к завершению работы сценария. Одним из решений этой проблемы
является добавление ветви \command{else}:

{\footnotesize
\begin{verbatim}
$(config): $(config_template)
    if [ ! -d $(dir $@) ];  \
    then                    \
        $(MKDIR) $(dir $@); \
    else                    \
        true;               \
    fi
    $(M4) $^ > $@
\end{verbatim}
}

Двоеточие (\command{:})~--- это команда интерпретатора, которая всегда
возвращает истину, поэтому её можно использовать вместо \command{true}.
Альтернативной рабочей реализацией является следующая:

{\footnotesize
\begin{verbatim}
$(config): $(config_template)
    [[ -d $(dir $@) ]] || $(MKDIR) $(dir $@)
    $(M4) $^ > $@
\end{verbatim}
}

 Теперь первое выражение истинно, если целевой каталог существует или
 выполнение программы \utility{mkdir} завершилось успешно. Другой
 альтернативой является использование ключа \command{-t} программы
 \utility{mkdir}. Это приведёт к успешному завершению \utility{mkdir}
 даже в том случае, если требуемый каталог уже сужествует.

Все предыдущие реализации вызывали интерпретатор даже в том случае,
если каталог уже существовал. Использование функции
\function{wildcard} позволяет избежать выполнения команд в случае
наличия каталога:

{\footnotesize
\begin{verbatim}
# $(call make-dir, directory)
make-dir = $(if $(wildcard $1),,$(MKDIR) -p $1)

$(config): $(config_template)
    $(call make-dir, $(dir $@))
    $(M4) $^ > $@
\end{verbatim}
}

Поскольку каждая команда выполняется в отдельном экземпляре командного
интерпретатора, общей практикой является использование многострочных
команд, разделённых точками с запятой. Остерегайтесь случаев, когда
ошибка в таких сценариях может не привести к завершению выполнения
сборки:

{\footnotesize
\begin{verbatim}
target:
    rm rm-неудачен; echo но следующая команда выполняется 
\end{verbatim}
}

Лучшей практикой является минимизация длины сценария, что даёт
\GNUmake{} шанс обработать код возврата самостоятельно. Например:

{\footnotesize
\begin{verbatim}
path-fixup = \
    -e "s;[a-zA-Z:/]*/src/;$(SOURCE_DIR)/;g" \
    -e "s;[a-zA-Z:/]*/bin/;$(OUTPUT_DIR)/;g"

# хорошая версия
define fix-project-paths
  sed $(path-fixup) $1 > $2.fixed && \
  mv $2.fixed $2
endef

# отличная версия
define fix-project-paths
  sed $(path-fixup) $1 > $2.fixed
  mv $2.fixed $2
endef
\end{verbatim}
}

Этот макрос преобразует пути в стиле DOS (с прямыми слэшами) в пути
назначения для определённой структуры каталогов исходного кода и
бинарных файлов. Макрос принимает имена двух файлов: входного и
выходного. Кроме того, принимаются дополнительные действия, чтобы
выходной файл был перезаписан только в том случае, если \utility{sed}
завершится корректно. В <<хорошей>> версии это достигается за счёт
соединения программ \utility{sed} и \utility{mv} операцией
\command{\&\&}, в результате чего они выполняются в одном экземпляре
командного интерпретатора. <<Лучшая>> версия выполняет их как две
отдельных команды, позволяя \GNUmake{} завершить выполнение сценария,
если программа \utility{sed} завершится неудачей. Однако <<лучшая>>
версия не является менее производительной (программа \utility{mv} не
требует вызова командного интерпретатора и выполняется непосредственно
\GNUmake{}), к тому же её легче понять, а в случае возникновения
ошибки полученное сообщение будет более информативным (поскольку
\GNUmake{} укажет, какая именно команда закончилась неудачей).

Заметим, что предыдущая ситуация не имеет отношения к общей проблеме
команды \utility{cd}:

{\footnotesize
\begin{verbatim}
TAGS:
    cd src && \
    ctags --recurse
\end{verbatim}
}

В этом случае оба выражения должны выполняться в одном процессе
командного интерпретатора, поэтому должен использоваться разделитель
команд, например, \command{;} или \command{\&\&}.

%---------------------------------------------------------------------
% Deleting and preserving target files
%---------------------------------------------------------------------
\subsubsection*{Удаление и сохранение целевых файлов}
Если происходит ошибка, \GNUmake{} подразумевает, что повторная сборка
цели не может быть осуществлена. В этом случае любая другая цель,
имеющая текущую в качестве реквизита, также не может быть собрана,
поэтому \GNUmake{} не будет даже пытаться осуществить её сборку. Если
использована опция \command{-{}-keep\hyp{}going} (\command{-k}), будет
произведена попытка сборки следующей цели, иначе \GNUmake{} закончит
своё выполнение. Если текущей целью является файл, его содержимое
может быть повреждено, если команда из сценария завершится, не
закончив своей работы. К сожалению, \GNUmake{} оставит этот
потенциально повреждённый файл на диске из соображений исторической
совместимости. Поскольку время последнего изменения файла будет
изменено, все последующие вызовы \GNUmake{} не смогут поместить в этот
файл корректные данные. Вы можете избежать этой проблемы и заставить
\GNUmake{} удалять эти подозрительные файлы в случае возникновения
ошибки, указав целевой файл реквизитом специальной цели
\index{Цели!специальные!.DELETE\_ON\_ERROR@\target{.DELETE\_ON\_ERROR}}
\target{.DELETE\_ON\_ERROR}. Если цель \target{.DELETE\_ON\_ERROR}
используется без реквизитов, ошибка при сборке любого файла приведёт к
его удалению.

Дополнительные проблемы связаны с ситуацией, когда выполнение
\GNUmake{} прерывается сигналом, например, по нажатию клавиш Ctrl-C. В
этом случае \GNUmake{} удалит текущий целевой файл, если он был
модифицирован. Иногда удаление целевого файла не является желаемой
реакцией. Возможно, создание целевого файла~--- чрезвычайно затратная
операция, или получение части его содержимого желательней полного
его отсутствия. В таком случае вы можете защитить целевой файл, сделав
\index{Цели!специальные!.PRECIOUS@\target{.PRECIOUS}}
его реквизитом специальной цели \target{.PRECIOUS}.

%%-------------------------------------------------------------------
%% Which shell to use
%%-------------------------------------------------------------------
\section{Выбор командного интерпретатора}
\label{sec:which_shell_to_use}

Когда \GNUmake{} требуется передать команду интерпретатору, он
использует \utility{/bin/sh}. Вы можете изменить интерпретатор,
выставив соответствующим образом значение переменной \variable{SHELL}.
Однако хорошенько подумайте перед тем, как это сделать. Обычно
назначением \GNUmake{} является предоставление для команды
разработчиков инструмента сборки системы из исходного кода. Довольно
легко создать \Makefile{}, не соответствующий этому назначению,
используя инструменты, не доступные для других участников процесса
разработки или строя предположения, для них не справедливые.
Использование любых интерпретаторов, отличных от \utility{/bin/sh},
считается дурным тоном для любого широко распространённого приложения
(доступного через ftp или открытый репозиторий cvs). Мы обсудим
вопросы переносимости более детально в
главе~\ref{chap:portable_makefiles}.

Однако и есть другой контекст использования \GNUmake{}. В закрытых
средах разработки часто все участники проекта работают на ограниченном
множестве машин и операционных систем. На самом деле это именно та
среда, в которой мне чаще всего приходилось работать. В этой ситуации
вы имеете полное право настроить среду, в которой будет работать
\GNUmake{}, по своему усмотрению. Следует лишь инструктировать всех
разработчиков в вопросах настройки среды и работы со сборками.

В подобных средах я предпочитаю открыто жертвовать переносимостью в
некоторых аспектах. Я уверен, что это может сделать процесс разработки
гораздо более гладким. Одной их таких жертв является замена
стандартного значения переменной \variable{SHELL} на
\utility{/usr/bin/bash}. \utility{bash}~--- это переносимый,
\POSIX{}-совместимый командный интерпретатор (отсюда следует, что он
включает в себя все возможности \utility{sh}), являющийся
интерпретатором по умолчанию для GNU/Linux. Причиной многих проблем с
переносимостью \Makefile{}'ов является использование непереносимых
конструкций в сценариях сборки. Решением этих проблем является явное
использование одного стандартного интерпретатора вместо употребления
лишь переносимого подмножества команд \utility{sh}. У Пола Смита,
разработчика, занимающегося поддержкой GNU \GNUmake{}, ести
веб-страница <<Правила Пола для \Makefile{}'ов>> (<<Paul's Rules of
Makefiles>>,
\filename{\url{http://make.paulandlesley.org/rules.html}}), на которой
он делает следующее замечание: <<Не тратьте силы, пытаясь написать
переносимые \Makefile{}'ы, используйте переносимую версию
\GNUmake{}!>> (<<Don't hassle with writing portable makefiles, use
portable make instead!>>). Я могу добавить следующее: <<Когда есть
возможность, не тратьте силы, пытаясь написать переносимый сценарий,
используйте переносимый командный интерпретатор (bash)>>.
\utility{bash} работает на большинстве операционных систем, включая
практически все варианты \UNIX{}, Windows, BeOS, Amiga и OS/2.

В оставшейся части книги я буду явно указывать на случаи использования
специфичных возможностей \utility{bash}.

%%--------------------------------------------------------------------
%% Empty commands
%%--------------------------------------------------------------------
\section{Пустые команды}
\label{sec:empty_commands}

\index{Команды!пустые}
\newword{Пустая команда}~-- это команда, которая не производит никаких
действий:

{\footnotesize
\begin{verbatim}
header.h: ;
\end{verbatim}
}

Вспомним, что за списком реквизитов цели может следовать точка с
запятой и команда. Здесь используется только точка с запятой, что
означает, что команды не предполагаются. Вместо этого вы можете
поместить после определения цели строку, содержащую только один символ
табуляции, однако это будет невозможно прочитать. Пустые команды чаще
всего используются для предотвращения соответствия цели шаблонному
правилу и выполнения нежелательных команд.

Заметим, что в других версиях \GNUmake{} пустые цели иногда
используются в качестве абстрактных. В GNU \GNUmake{} следует
использовать специальную цель \target{.PHONY}, это безопасней и яснее.

%%-------------------------------------------------------------------
%% Command enviromnent
%%-------------------------------------------------------------------
\section{Команды и окружение}
\label{sec:command_environment}

Команды, выполняемые \GNUmake{}, наследуют окружение процесса
\GNUmake{}. Это окружение включает текущий каталог, файловые
дескрипторы и переменные окружения, передаваемые \GNUmake{}.

Когда создаётся дочерний процесс командного интерпретатора, \GNUmake{}
добавляется в окружение несколько переменных:

{\footnotesize
\begin{verbatim}
MAKEFLAGS
MFLAGS
MAKELEVEL
\end{verbatim}
}

Переменная \variable{MAKEFLAGS} включает опции командной строки,
переданные \GNUmake{}. Переменная \variable{MFLAGS} дублирует
содержимое \variable{MAKEFLAGS} и существует по историческим причинам.
Переменная \variable{MAKELEVEL} содержит число вложенных вызовов
\GNUmake{}. Таким образом, когда \GNUmake{} рекурсивно вызывает
\GNUmake{}, переменная \variable{MAKELEVEL} увеличивается на единицу.
Подпроцесс родительского процесса \GNUmake{} будет иметь переменную
\variable{MAKELEVEL}, значением которой будет единица. Все эти
переменные обычно используются для управления рекурсивным \GNUmake{}.
Мы обсудим эту тему в разделе <<\nameref{sec:recursive_make}>>
главы~\ref{chap:managing_large_proj}.

Конечно, пользователь может передать в окружение дочернего процесса
любую переменную по своему усмотрению, используя директиву
\index{Директивы!export@\directive{export}}
\directive{export}.

Текущий рабочий каталог исполняемой команды совпадает с рабочим
каталогом родителького процесса \GNUmake{}. Обычно это тот же каталог,
из которого была вызвана программа \GNUmake{}, однако его можно
заменить при помощи опции \command{--directory=\emph{каталог}} (или
\command{-C}). Заметим, что спецификация \Makefile{}'а при помощи
опции \command{-{}-fi\-le} не изменяет рабочий каталог, только
устанавливает \Makefile{}, который нужно прочитать.

Каждый подпроцесс, порождаемый \GNUmake{}, наследует три стандартных
файловых дескриптора: \filename{stdin}, \filename{stdout} и
\filename{stderr}. Здесь нет ничего особенного, за исключением одного
следствия: сценарий сборки может считывать данные из стандартного
потока ввода. Как только сценарий считает все данные из потока,
оставшиеся команды выполняются в обычном порядке. Однако ожидается,
что \Makefile{}'ы должны работать корректно без этого типа
взаимодействия. Пользователь часто расчитывает на возможность просто
запустить \GNUmake{} и далее не принимать никакого участия в процессе
сборки, проверив лишь результаты по завершению. И, конечно, сложно
придумать полезное применение чтению стандартного потока ввода в
контексте автоматизированных сборок, основанных на использовании
\utility{cron}.

Общей ошибкой является случайное чтение стандартного потока ввода:

{\footnotesize
\begin{verbatim}
$(DATA_FILE): $(RAW_DATA)
    grep pattern $(RAW_DATA_FILES) > $@
\end{verbatim}
}

Здесь входные файлы для \utility{grep} хранятся в переменной (при
использовании которой произошла опечатка). Если вместо значения
переменной подставится пустая строка, \utility{grep} останется только
читать данные со стандартного потока ввода, без каких либо объяснений
причины <<зависания>> \GNUmake{}. Простым способом избежать такой
проблемы является включение в команду дополнительного файла устройства
\filename{/dev/null}:

{\footnotesize
\begin{verbatim}
$(DATA_FILE): $(RAW_DATA)
    grep pattern $(RAW_DATA_FILES) /dev/null > $@
\end{verbatim}
}

Такая команда никогда не примет попытки чтения стандартного потока
ввода. Естественно, отладка \Makefile{}'ов также помогает избежать
неприятностей.

%%--------------------------------------------------------------------
%% Evaluating commands
%%--------------------------------------------------------------------
\section{Выполнение команд}
\label{sec:evaluating_commands}

Обработка командного сценария происходит в четыре этапа: чтение кода,
подстановка переменных, вычисление выражений \GNUmake{} и выполнение
команд. Давайте посмотрим, как все эти этапы применяются к сложному
сценарию. Рассмотрим следующий (немного надуманный) \Makefile{}.
Приложение компонуется, затем от полученного исполняемого файла
отделяется таблица символов, после чего он сжимается при помощи
компрессора исполняемых файлов \utility{upx}:

{\footnotesize
\begin{verbatim}
# $(call strip-program, file)
define strip-program
  strip $1
endef

complex_script:
    $(CC) $^ -o $@
  ifdef STRIP
    $(call strip-program, $@)
  endif
    $(if $(PACK), upx --best $@)
    $(warning Final size: $(shell ls -s $@))
\end{verbatim}
}

Вычисление командных сценариев откладывается до того момента, когда
оно действительно потребуется, однако директивы \directive{ifdef}
обрабатывается сразу после их обнаружения. Поэтому \GNUmake{}
считывает команды сценария, игнорируя их содержимое и сохраняя каждую
строку, пока не обнаружит строку \command{ifdef STRIP}. \GNUmake{}
выполняет тест, и если переменная \variable{STRIP} не определена,
\GNUmake{} считывает и отбрасывает весь текст сценария, пока не
натолкнётся на закрывающую директиву \directive{endif}. После этого
\GNUmake{} считывает и сохраняет оставшуюся часть сценария.

Когда приходит время выполнения сценария, \GNUmake{} сначала сканирует
команды на наличие конструкций \GNUmake{}, требующих подстановки. После
подстановки макросов каждая строка сценария начинается с символа
табуляции. Вычисление макросов \emph{перед} выполнением команд может
привести к неожиданным результатам. Последняя строка в нашем сценарии
некорректна. Функции \function{shell} и \function{warning} выполняются
\emph{до} компоновки приложения. Поэтому команда \utility{ls} будет
выполнена до того, как целевой файл будет собран. Это объясняет
<<неправильный>> порядок выполнения, который мы наблюдали в разделе
<<\nameref{sec:parsing_commands}>>.

Также заметим, что строка \command{ifdef STRIP} выполняется во время
чтения \Makefile{}'а, однако строка \command{\$(if...)} вычисляется
непосредственно перед выполнения сценария сборки цели
\target{complex\_script}. Использование функции \function{if}
допускает написание более гибких сценариев, поскольку предоставляет
больше возможностей для контроля определения переменных, однако такой
подход не очень хорошо приспособлен для управления большими блоками
текста.

Как показывает наш пример, всегда очень важно обращать внимание на то,
какая программа вычисляет выражение (т.е. \GNUmake{} или командный
интерпретатор), и когда именно это вычисление происходит:

{\footnotesize
\begin{verbatim}
$(LINK.c) $(shell find                              \
            $(if $(ALL),$(wildcard core ext*),core) \
              -name '*.o')
\end{verbatim}
}

Это запутанный командный сценарий компоновки множества объектных
файлов. Порядок вычисления операций таков (в скобках указана
программа, выполняющая соответствующую операцию):

\begin{enumerate}
\item Вычисление \variable{\$ALL} (\GNUmake{}).
\item Вычисление \function{if}  (\GNUmake{}).
\item Вычисление \function{wildcard} в предположении, что
  \variable{ALL} содержит непустое значение (\GNUmake{}).
\item Вычисление \function{shell} (\GNUmake{}).
\item Вычисление \utility{find} (\utility{sh}).
\item После завершения подстановок и вычисления конструкций
  \GNUmake{}, происходит выполнение команды компоновки
  (\utility{sh}).
\end{enumerate}

%%--------------------------------------------------------------------
%% Command-line limits
%%--------------------------------------------------------------------
\section{Ограничения командной строки}
\label{sec:command_line_limits}

Во время работы с крупными проектами вы можете столкнуться с
ограничениями на длину команд, которые \GNUmake{} пытается выполнить.
Ограничения на длину командной строки варьируются в зависимости от
операционной системы. Red Hat 9 GNU/Linux позволяет выполнять команды
длиной не более 128 Кб, а Windows XP ограничивает длину 32 Кб.
Сообщения об ошибке также варьируются. Если вы вызовите команду
\utility{ls} со слишком большим списком параметров, в Cygwin под
Windows, то получите следующее сообщение:

{\footnotesize
\begin{verbatim}
C:\usr\cygwin\bin\bash: /usr/bin/ls: Invalid argument
\end{verbatim}
}

На Red Hat 9 сообщение выглядит иначе:

{\footnotesize
\begin{verbatim}
/bin/ls: argument list too long
\end{verbatim}
}

Даже 32 Кб выглядит как довольно большой объём данных для командной
строки, однако когда ваш проект содержит 3000 файлов и 100
подкаталогов, и вы хотите манипулировать ими всеми, это ограничение
может быть довольно существенным.

Существует два основных пути, которые ведут к неприятностям с
ограничениями на длину строки: вычисление базовых значений при помощи
инструментов командного интерпретатора или использование \GNUmake{}
для присваивания переменной значения очень большой длины. Предположим
для примера, что мы хотим скомпилировать все исходные файлы одной
командой:

{\footnotesize
\begin{verbatim}
ompile_all:
$(JAVAC) $(wildcard $(addsuffix /*.java,$(source_dirs)))
\end{verbatim}
}

Переменная \GNUmake{} \variable{source\_dirs} может содержать всего
несколько сотен слов, однако после добавления шаблона исходных файлов
\Java{} и применения функции \function{wildcard} этот список может
превысить предельную длину командной строки вашей системы. Кстати,
\GNUmake{} не имеет собственных ограничений на длину строки, позволяя
вам хранить столько данных, сколько может вместить виртуальная память.

Когда сталкиваешься с подобной ситуацией, возникает ощущение, что
играешь в старую игру <<Приключение>> (Adventure): <<Вы находитесь в
лабиринте из одинаковых извилистых коридоров>>. Например, вы можете
попробовать использовать \utility{xargs} для решения проблемы, так как
\utility{xargs} разделяет строки на части в соответствии с
ограничениями текущей системы:

{\footnotesize
\begin{verbatim}
compile_all:
    echo $(wildcard
           $(addsuffix /*.java,$(source_dirs))) | \
    xargs $(JAVAC)
\end{verbatim}
}

К сожалению, так мы просто переместили проблему ограничений из команды
\utility{javac} в команду \utility{echo}. Мы также не можем
использовать \utility{echo} или \utility{printf} для записи данных в
файл (предполагается, что компилятор может читать список файлов из
файла).

Нет, для решения этой проблемы нужно в первую очередь избежать
создания одного большого списка файлов. Вместо этого мы можем
просматривать по одному каталогу за раз, используя шаблоны командного
интерпретатора:

{\footnotesize
\begin{verbatim}
compile_all:
    for d in $(source_dirs); \
    do                       \
        $(JAVAC) $$d/*.java; \
    done
\end{verbatim}
}

Также можно использовать канал в \utility{xargs}, чтобы достигнуть
желаемого результата за меньшее количество вызовов компилятора:

{\footnotesize
\begin{verbatim}
compile_all:
    for d in $(source_dirs); \
    do                       \
        echo $$d/*.java;     \
    done |                   \
    xargs $(JAVAC)
\end{verbatim}
}

К сожалению, ни один из этих сценариев не обрабатывает должным образом
ошибки компиляции. Лучшим подходом является сохранение полного списка
файлов и последующая передача его компилятору, если, конечно,
компилятор поддерживает чтение аргументов из файла. Компилятор \Java{}
поддерживает эту возможность:

{\footnotesize
\begin{verbatim}
compile_all: $(FILE_LIST)
    $(JAVA) @$<

.INTERMEDIATE: $(FILE_LIST)
$(FILE_LIST):
    for d in $(source_dirs); \
    do                       \
        echo $$d/*.java;     \
    done > $@

\end{verbatim}
}

Обратите внимание на тонкую ошибку в цикле \command{for}. Если
какой-либо из каталогов не содержит исходных файлов \Java{}, строка
\filename{*.java} будет включена в список файлов и компилятор \Java{}
выдаст сообщение об ошибке: <<File not found>> (файл не найден). Мы
можем приказать \utility{bash} подставлять пустые строки на место
шаблонов, которым не соответствует ни один файл, использовав опцию
\command{nullglob}:

{\footnotesize
\begin{verbatim}
compile_all: $(FILE_LIST)
    $(JAVA) @$<

.INTERMEDIATE: $(FILE_LIST)
$(FILE_LIST):
    shopt -s nullglob;       \
    for d in $(source_dirs); \
    do                       \
        echo $$d/*.java;     \
    done > $@
\end{verbatim}
}

Многим проектам приходится создавать список файлов. Ниже представлен
макрос, содержащий сценарий \utility{bash}, создающий список файлов.
Первым аргументом является корневой каталог, пути всех найденных
файлов будут указываться относительно этого каталога. Вторым
параметром является список каталогов, в которых нужно искать файлы,
соответствующие шаблону. Третий и четвёртый аргументы опциональны и
содержат расширения интересующих файлов.

{\footnotesize
\begin{verbatim}
# $(call collect-names,root-dir,dir-list,
#        suffix1-opt,suffix2-opt)
define collect-names
  echo Making $@ from directory list...             
  cd $1;                                              \
  shopt -s nullglob;                                  \
  for f in $(foreach file,$2,'$(file)'); do           \
    files=( $$f$(if $3,/*.{$3$(if $4,$(comma)$4)}) ); \
    if (( $${#files[@]} > 0 ));                       \
    then                                              \
      printf '"%s"\n' $${files[@]};                   \
    else :; fi;                                       \
  done
endef
\end{verbatim}
}

Так выглядит шаблонное правило создания списка файлов изображений:

{\footnotesize
\begin{verbatim}
%.images:
    @$(call collect-names,$(SOURCE_DIR),$^,gif,jpeg) > $@
\end{verbatim}
}

Вычисление макроса скрыто с помощью модификатора \command{@},
поскольку сценарий достаточно велик, а причину для копирования и
вставки полученного кода найти трудно. Список каталогов указан в
реквизитах. После смены текущего каталога сценарий включает опцию
\texttt{nullglob}. Остаток макроса~--- цикл \emph{for}, осуществляющий
проход по всем каталогам, которые нужно обработать. Первое выражение
поиска файлов~--- это список слов, переданный в качестве второго
параметра (\variable{\${}2}). Сценарий экранирует слова в списке
файлов с помощью апострофа, так как они могут содержать символы,
имеющие для командного интерпретатора специальный смысл. В частности,
имена файлов в некоторых языках программирования (например, \Java{})
могут содержать символы доллара:

{\footnotesize
\begin{verbatim}
for f in $(foreach file,$2,'$(file)'); do
\end{verbatim}
}

Мы производим поиск файлов в каталоге, заполняя массив
\variable{files} результатами вычислений подстановок. Если полученный
массив содержит элементы, мы используем \function{printf} для того,
чтобы напечатать каждое слово на новой строке. Использование массива
позволяет макросу правильно обрабатывать пути, содержащие
пробелы. Возможность наличия пробелов в путях~--- это ещё одна причина,
по которой аргумент \function{printf} окружён кавычками.

Список файлов создаётся при помощи следующей строки:

{\footnotesize
\begin{verbatim}
files=( $$f$(if $3,/*.{$3$(if $4,$(comma)$4)}) );
\end{verbatim}
}

Переменная \variable{\$\$f}~--- это каталог или файл, переданный
макросу в составе аргумента. Следующее выражение~--- это функция
\function{if}, проверяющая третий аргумент на непустоту. Это один из
путей, который можно использовать для реализации необязательных
аргументов. Если третий аргумент пуст, четвёртый также подразумевается
пустым. В этом случае файл, переданный пользователем, должен быть
включён в список как есть. Это позволяет макросу строить списки
обычных файлов, для которых использование шаблонов не подходит. Если
третий аргумент не пуст, функция \function{if} добавляет к корневому
каталогу строку \texttt{/*\{\$3\}}. Если передан четвёртый аргумент,
после \variable{\$3} происходит вставка \variable{\$4}. Обратите
внимание на то, как происходит вставка запятой в шаблон. Поместив
символ запятой в переменную \GNUmake{}, мы можем незаметно передать её
в выражение, явное использование запятой было бы воспринято как
отделение \emph{then} части от \emph{else} части функции
\function{if}. Определение переменной \variable{comma} очевидно:

{\footnotesize
\begin{verbatim}
comma = ,
\end{verbatim}
}
 
Все рассмотренные циклы \function{for} зависели от пределов длины
командной строки, поскольку использовали шаблонные выражения. Разница
в том, что результат применения шаблона для поиска файлов в одном
каталоге имеет гораздо меньше шансов превысить предел.

Что будет, если какая-то переменная \GNUmake{} содержит длинный список
файлов? Чтож, тогда мы столкнулись с настоящей неприятностью. Я нашёл
лишь два пути передать длинную переменную \GNUmake{} в интерпретатор.
Первый подход~--- передавать содержимое переменной по частям,
используя фильтры, основанные на применении функции
\function{wordlist}:

{\footnotesize
\begin{verbatim}
compile_all:
    $(JAVAC) $(wordlist 1, 499, $(all-source-files))
    $(JAVAC) $(wordlist 500, 999, $(all-source-files))
    $(JAVAC) $(wordlist 1000, 1499, $(all-source-files))
\end{verbatim}
}

Второй путь~--- использовать функцию \function{filter}, однако в этом
случае результаты менее предсказуемы, поскольку число отбираемых
фильтром файлов может зависеть от числа слов, соответствующему
каждому из выбранных шаблонов. В следующем примере используются
шаблоны, основанные на алфавитном порядке:

{\footnotesize
\begin{verbatim}
compile_all:
    $(JAVAC) $(filter a%, $(all-source-files))
    $(JAVAC) $(filter b%, $(all-source-files))
\end{verbatim}
}

Ваши шаблоны могут использовать специальные свойства имён файлов.

Обратите внимание на то, как сложно автоматизировать этот процесс. Мы
могли бы попробовать использовать алфавитный подход совместно с циклом
\function{foreach}:

{\footnotesize
\begin{verbatim}
compile_all:
    $(foreach l,a b c d e ...,                 \
      $(if $(filter $l%, $(all-source-files)), \
        $(JAVAC) $(filter $l%, $(all-source-files));))
\end{verbatim}
}

Однако такой подход не работает. \GNUmake{} превратит этот сценарий в
одну строку текста, что только усугубит проблемы с длиной команд.
Вместо этого можно использовать \function{eval}:

{\footnotesize
\begin{verbatim}
compile_all:
    $(foreach l,a b c d e ...,                 \
      $(if $(filter $l%, $(all-source-files)), \
        $(eval                                 \
          $(shell                              \
            $(JAVAC) $(filter $l%, $(all-source-files));))))
\end{verbatim}
}

Этот вариант будет работать правильно, потому что функция
\function{eval} выполняет команду \function{shell} незамедлительно, и
результатом её вычисления является пустая строка. Таким образом,
результатом вычисления функции \function{foreach} является также
пустая строка. Проблема заключается в том, что проверка ошибок в этом
контексте не происходит, поэтому ошибки компиляции не будут переданы
\GNUmake{} напрямую.

Подход, основанный на использовании \function{wordlist} значительно
хуже. Из-за ограниченных возможностей \GNUmake{} в области численных
операций, применить эту технику в цикле не получится. В общем,
сколь-нибудь удовлетворительных техник обращения с огромными списками
файлов практически не существует.

