%%%-------------------------------------------------------------------
%%% Rules
%%%-------------------------------------------------------------------
\chapter{Правила}
\label{chap:rules}

В предыдущей главе мы рассмотрели несколько правил для компиляции и
компоновки программы подсчёта слов. Каждое из этих правил определяло
цель~--- файл, который требуется собрать. Каждая цель зависела от
множества реквизитов, которые также являлись файлами. Когда
требовалось обновить цель, \GNUmake{} выполнял сценарий сборки только
в том случае, если файлы реквизитов имели дату модификации более
позднюю, чем цель. Поскольку цель одного правила может быть реквизитом
другого, множество целей и реквизитов может быть представлено в форме
\index{Граф зависимостей}
\newword{графа зависимостей} (\newword{dependency graph}). Составление
и обработка графа зависимостей является основной работой \GNUmake{}.

\index{Правила!явные}
Поскольку правила так важны для \GNUmake{}, существует несколько их
разновидностей. \newword{Явные правила}, наподобие тех, что были
использованы в предыдущей главе, указывают на необходимость обновления
цели при модификации или отсутствии файлов-реквизитов. Правила этого
типа вы будете писать наиболее часто.
\index{Правила!шаблонные}
\newword{Шаблонные правила} используют подстановки (wildcards) вместо
явного указания имён файлов.  Это позволяет \GNUmake{} применять такие
правила каждый раз при соответствии имени цели некоторому шаблону.
\index{Правила!неявные}
\newword{Неявные правила}~--- это явные или суффиксные правила,
встроенные в базу данных правил \GNUmake{}. Наличие встроенной базы
данных правил упрощает написание \Makefile'ов, поскольку для многих
общих задач уже известны типы файлов, суффиксы и сценарии сборки
целей.
\index{Правила!шаблонные!статические}
\newword{Статические шаблонные правила} отличаются от обычных
шаблонных правил тем, что могут быть применены только к определённому
списку целей.

GNU \GNUmake{} может быть использован как замена для многих других
версий \GNUmake{}. Он включает в себя множество возможностей,
сохранённых для поддержания обратной совместимости. Например,
\index{Правила!суффиксные}
\newword{Суффиксные правила} были реализованы в одной из первых версий
\GNUmake{} для написания общих правил. \GNUmake{} имеет поддержку
сиффиксных правил, однако они признаны устаревшими, поскольку могут
быть заменены более простыми и более \'{о}бщными шаблонными правилами.

%%--------------------------------------------------------------------
%% Explicit rules
%%--------------------------------------------------------------------
\section{Явные правила}
\label{sec:explicit_rules}

Чаще всего вам придётся писать именно явные правила, указывающие
некоторые файлы как цели и реквизиты. Правило может иметь более одной
цели. Это значит, что каждая из указанных целей имеет в точности то же
множество реквизитов, что и остальные цели. Если цели требуется
обновить, для каждой из них будет выполнен один и тот же сценарий. Вот
пример такого правила:

{\footnotesize
\begin{verbatim}
vpath.o variable.o: make.h config.h getopt.h gettext.h dep.h
\end{verbatim}
}

Это правило означает, что цели \filename{vpath.o} и
\filename{variable.o} зависят от одного и того же множества
заголовочных файлов. Оно имеет в точности тот же эффект, что и
следующая спецификация:

{\footnotesize
\begin{verbatim}
vpath.o: make.h config.h getopt.h gettext.h dep.h

variable.o: make.h config.h getopt.h gettext.h dep.h
\end{verbatim}
}

Обе цели собираются независимо. Если один из объектных файлов имеет
более раннюю дату модификации, чем один из указанных заголовочных
файлов, \GNUmake{} инициирует сборку и выполнит команды,
ассоциированные с правилом.

Правило не обязательно указывать полностью сразу. Каждый раз, когда
\GNUmake{} обнаруживает файл в качестве цели, он добавляет цель и
реквизиты в граф зависимостей. Если такая цель уже существовала в
графе, к записи о цели добавляются новые реквизиты. Одним из
элементарных применений этого свойства является разбиение длинной
строки на несколько более коротких для улучшения читабельности файла:

{\footnotesize
\begin{verbatim}
vpath.o: make.h config.h getopt.h gettext.h dep.h
vpath.o: filedef.h hash.h job.h commands.h variable.h vpath.h
\end{verbatim}
}

В наиболее сложных случаях список реквизитов может состоять из файлов, способы
обработки которых различны:

{\footnotesize
\begin{verbatim}
# Убедимся, что файл lexer.c существует до компиляции vpath.c
vpath.o: lexer.c

...

# Компилируем vpath.c с определёнными флагами
vpath.o: vpath.c
	$(COMPILE.c) $(RULE_FLAGS) $(OUTPUT_OPTION) $<

...

# Включаем файл зависимостей, составленный программой
include auto-generated-dependencies.d
\end{verbatim}
}

Первое правило декларирует, что цель \filename{vpath.o} должна быть
собрана заново при изменении файла \filename{lexer.c} (возможно,
генерация этого файла имеет некий побочный эффект). Правило также
может быть использовано для того, чтобы убедиться, что все реквизиты
существуют (или, в случае необходимости, обновлены) перед сборкой
цели. Нужно отметить двустороннюю сущность правил. При прямом чтении
правило означает, что, если файл \filename{lexer.c} изменился,
требуется выполнить действия по обновлению \filename{vpath.o}. При
чтении в обратном направлении правило означает, что если требуется
обновить \filename{vpath.o}, то нужно убедиться, что файл
\filename{lexer.c} существует. Это правило может быть помещено рядом с
остальными правилами, касающимися файла \filename{lexer.c}, чтобы
разработчики помнили об этой тонкой взаимосвязи. Далее, рассмотрим
правило компиляции \filename{vpath.o}. Сценарий сборки для этого
правила использует три переменных \GNUmake{}. Переменные будут
детально описаны позже, пока важно знать лишь то, что обращение к
переменной происходит с помощью знака доллара (\command{\$}), за
которым следует либо один символ, либо слово в круглых скобках.
Наконец, зависимости типа \filename{.o/.h} включаются из отдельного
файла, полученного при помощи внешней программы.

В качестве особого случая GNU \GNUmake{} поддерживает упрощённый
синтаксис для правил с одной командой.

{\footnotesize
\begin{alltt}
\emph{цель: ; команда}
\end{alltt}
}

На практике такие правила встречаются редко, однако всё же иногда они
могут быть полезны, особенно когда нужно сберечь место на экране
монитора или листе бумаги. 

%---------------------------------------------------------------------
% Wildcards
%---------------------------------------------------------------------
\subsection{Шаблоны}
Часто \Makefile'ы содержат огромное количество файлов. Для упрощения
работы с ними \GNUmake{} поддерживает шаблоны, идентичные шаблонам
командного интерпретатора \utility{Bourne shell}: \verb|~|, \verb|*|,
\verb|?|, \verb|[...]| и \verb|[^...]|.  Например, шаблону \verb|*.*|
соответствуют все файлы, содержащие в имени точку. Знак вопроса
означает один символ, а \verb|[...]|~--- класс символов. Для выбора
дополнения класса символов нужно использовать \verb|[^...]|. Знак
тильды (\verb|~|) может быть использован для обозначения домашнего
каталога текущего пользователя системы.  Если за тильдой следует имя
пользователя, будет подставлен домашний каталог указанного
пользователя. \GNUmake{} автоматически раскрывает шаблоны, когда они
встречаются в названиях целей, реквизитов или командных сценариях. В
другом контексте шаблоны могут быть раскрыты явным вызовом функции.
Шаблоны чрезвычайно полезны для написания более адаптивных
\Makefile{}'ов. Например, вместо того, чтобы явно перечислять все
файлы, входящие в состав исходного кода программы, вы можете
использовать шаблоны\footnote{В более серьёзных приложениях применение
шаблонов для выбора компилируемых файлов является плохой практикой,
поскольку может вызвать компоновку с посторонним опасным кодом. В
правилах удаления промежуточных файлов шаблоны могут быть фатальными
для проекта (прим. автора).}:

{\footnotesize
\begin{verbatim}
prog: *.c
	$(CC) -o $@ $^
\end{verbatim}
}

Однако очень важно быть осторожным с шаблонами, ими легко
злоупотребить. Рассмотрим пример:

{\footnotesize
\begin{verbatim}
*.o: constants.h
\end{verbatim}
}

Намерения очевидны: все объектные файлы зависят от заголовочного файла
\filename{constants.h}. Однако посмотрим, как раскроется шаблон в
каталоге, не содержащем объектных файлов:

{\footnotesize
\begin{verbatim}
: constants.h
\end{verbatim}
}

Это допустимое выражение \GNUmake{}, оно не вызовет ошибки, однако оно
также не выразит той зависимости, которую имел в виду пользователь.
Одним из корректных способов реализации этого правила является
использование шаблона для получения файлов с исходным кодом (которые,
как правило, присутствуют) и трансформация полученного списка в список
объектных файлов. Мы рассмотрим эту технику при обсуждении функций
в главе~{\ref{chap:functions}}.

Наконец, стоит отметить, что раскрытие шаблонов в тот момент, когда
они появляются в качестве целей или реквизитов, осуществляет
непосредственно \GNUmake{}. Однако раскрытие шаблонов в сценариях
происходит в дочернем процессе командного интерпретатора. Это может
быть важной деталью, поскольку \GNUmake{} раскрывает шаблоны во время
чтения \Makefile{}'а, а командный интерпретатор раскрывает их много
позже, во время непосредственного выполнения команд. Когда
производятся сложные манипуляции с файлами, результаты раскрытия
одинаковых шаблонов в разные моменты времени могут сильно отличаться.
Проблематичной может быть ситуация, когда некоторые файлы являются
результатом сборки, и \GNUmake{} не видит их во время обработки
\Makefile{}'а. К таким случаям нужно относится особенно осторожно.

%---------------------------------------------------------------------
% Phony targets
%---------------------------------------------------------------------
\subsection{Абстрактные цели}
\label{sec:phony_targets}
\index{Цели!абстрактные}
До этого момента все цели и реквизиты, рассматриваемые нами, были
файлами, которые нужно было создать или обновить. Хоть это и типичный
способ использования целей, часто бывает полезным представлять цель в
качестве метки для командного сценария. Например, ранее упоминалось,
что стандартной целью для многих \Makefile{}'ов является \target{all}.
Цели, не представляющие файлов, называют \newword{абстрактными целями}
(\newword{phony targets}). Ещё одной стандартной абстрактной целью
является \target{clean}:

{\footnotesize
\begin{verbatim}
clean:
    rm -f *.o lexer.c
\end{verbatim}
}

Абстрактные цели должны собираться всегда, потому что команды,
ассоциированные с правилом, не создают файл с именем цели.

Важно заметить, что \GNUmake{} не отличает абстрактных целей от
целей, являющимися файлами. Если по случайности файл с именем
абстрактной цели существует, \GNUmake{} будет ассоциировать этот файл
с абстрактной целью в графе зависимостей. Например, если в текущей
директории существует файл \filename{clean}, то запуск команды
\BoldMono{make clean} приведёт к появлению довольно неожиданного
сообщения:

{\footnotesize
\begin{alltt}
\$ \textbf{make clean}
make: `clean' is up to date.
\end{alltt}
}

Довольно часто абстрактные цели не имеют реквизитов; цель
\target{clean} всегда будет рассматриваться как не требующая
обновления, и ассоциированные с ней команды никогда не будут
выполнены.

Чтобы избежать этой проблемы, GNU \GNUmake{} имеет специальную цель,
\target{.PHONY}, позволяющую сообщить \GNUmake{}, что цель не является
настоящим файлом. Любая цель может быть объявлена как абстрактная с
путём включения её в список реквизитов цели \target{.PHONY}:

{\footnotesize
\begin{verbatim}
.PHONY: clean
clean: 
    rm -f *.o lexer.c
\end{verbatim}
}

Теперь \GNUmake{} всегда будет выполнять команды, ассоциированные с
целью \target{clean}, даже если файл с таким именем существует. В
добавок к пометке цели как требующей обновления, спецификация цели как
абстрактной сообщает \GNUmake{}, для этой цели не нужно использовать
стандартное правило получения файла цели из исходного кода. Это
позволяет \GNUmake{} провести оптимизацию обычного процесса поиска
правил для достижения более высокой производительности.

Довольно редко имеет смысл включать абстрактную цель в качестве
реквизита реального файла, поскольку это будет приводить к
безусловному обновлению цели. Указание же реквизитов абстрактных целей
довольно часто приносит пользу. Например, цель \target{all} имеет в
качестве реквизитов список программ, которые нужно собрать:

\begin{alltt}
.PHONY: all
all: bash bashbug
\end{alltt}

В предыдущем примере цель \target{all} собирает командный
интерпретатор \utility{bash} и инструмент отправки сообщений об
ошибках \utility{bashbug}.

Абстрактные цели могут рассматриваться как сценарии интерпретатора,
встроенные в \Makefile{}. Объявление абстрактной цели в качестве
реквизита другой цели вызовет запуск сценария, ассоциированного с
абстрактной целью, перед сборкой основной цели. Предположим, мы
ограничены в использовании дискового пространства, и хотим отобразить
количество доступного места на диске перед выполнением действий,
требующих значительных затрат дискового пространства. Одно из решений
демонстрирует следующий пример:

{\footnotesize
\begin{verbatim}
.PHONY: make-documentation
make-documentation:
    df -k . | awk 'NR == 2 { printf( "%d available\n", $$4 ) }'
    javadoc ...
\end{verbatim}
}

Проблема заключается в том, что нам может понадобиться указать
команды \utility{df} и \utility{awk} несколько раз для разных целей.
Это является проблемой с точки зрения поддержки, поскольку нам
придётся изменять каждое вхождение этих команд, если, например, в
другой системе формат выдачи данных утилиты \utility{df} отличается.
Поэтому более изящным решением является следующее:

{\footnotesize
\begin{verbatim}
.PHONY: make-documentation
make-documentation: df
    javadoc ...

.PHONY: df
df:
    df -k . | awk 'NR == 2 { printf( "%d available\n", \$\$4 ) }'
\end{verbatim}
}

Мы можем сообщить \GNUmake{} о необходимости вызова сценария,
ассоциированного с целью \utility{df}, перед созданием документации,
указав \utility{df} как реквизит цели \target{make-documentation}. Это
допустимо, поскольку \target{make-documentation} также является
абстрактной целью. Такой подход даёт нам ещё одно преимущество: теперь
мы можем легко использовать \utility{df} в других целях.

Существует много примеров удачного применения абстрактных целей.

Сообщения \GNUmake{} довольно трудны для чтения и отладки. На это есть
несколько причин: составление \Makefile{}'ов сверху\hyp{}вниз, в то
время как команды выполняются снизу\hyp{}вверх; кроме того, не
указывается, какая цель выполняется в данный момент. Чтобы исправить
ситуацию, полезно выводить сообщения о начале выполнения основных
целей. Абстрактные цели являются простым средством реализации этой
идеи. Ниже приведён отрывок из \Makefile{}'а командного интерпретатора
\utility{bash}:

{\footnotesize
\begin{verbatim}
$(Program): build_msg $(OBJECTS) $(BUILTINS_DEP) $(LIBDEP)
    $(RM) $@
    $(CC) $(LDFLAGS) -o $(Program) $(OBJECTS) $(LIBS)
    ls -l $(Program)
    size $(Program)

.PHONY: build_msg
build\_msg:
    @printf "#\n# Building $(Program)\n#\n"
\end{verbatim}
}

Поскольку \target{build\_msg} является абстрактной целью, сообщение
выводится непосредственно перед проверкой остальных реквизитов.  Если
бы сообщение о начале сборки было первой командой сценария сборки
\variable{\$(Program)}, то оно выводилось бы только после сборки всех
зависимых файлов. Также важно заметить, что, поскольку абстрактные цели
всегда помечены как требующие обновления, указание абстрактной цели
\target{build\_msg} в качестве реквизита \variable{\$(Program)}
вызовет безусловную сборку этой цели, даже если на самом деле
этого не требуется. В нашем случае это выглядит разумным, поскольку
основная работа заключается в компиляции исходных файлов в объектные,
а на финальном этапе будет производиться только компоновка.

Абстрактные цели также могут быть использованы для улучшения
<<пользовательского интерфейса>> \Makefile{}'а. Имена целей часто
содержат длинные стоки с путями к каталогам, дополнительные имена
компонентов (например, номера версий) и стандартные суффиксы. Это
может сделать указание имени нужной цели довольно неудобным занятием.
Этой проблемы можно избежать, добавив абстрактную цель указав в
качестве её реквизита имя нужной реальной цели.

Есть ряд абстрактных целей, являющихся более или менее стандартными.
Не смотря на то, что их имена являются лишь соглашением, эти цели
встречаются в большинстве \Makefile{}'ов. Список этих целей содержится
в Таблице~\ref{tab:std_phony_targets}.

\begin{table}
\begin{tabular}{|l|l|}
\hline
\textbf{Цель} & \textbf{Назначение}\\
\hline
\texttt{all} & Произвести сборку приложения.\\
\hline
\target{install} & Произвести установку собранного приложения.\\
\hline
\target{clean} & Удалить все бинарные файлы, полученные после сборки.\\
\hline
\target{distclean} & Удалить все файлы, не входящие в базовый дистрибутив.\\
\hline
\target{TAGS} & Создать таблицу тэгов для текстового редактора.\\
\hline
\target{info} & Создать файлы GNU info из файлов Texinfo.\\
\hline
\target{check} & Запустить все тесты, ассоциированные с приложением.\\
\hline
\end{tabular}
\caption{Стандартные абстрактные цели.}\label{tab:std_phony_targets}
\end{table}

Цель \target{TAGS} на самом деле не является абстрактной, поскольку
программы \utility{ctags} и \utility{etags} создают файл с именем
\filename{TAGS}. Эта цель включена в таблицу потому, что это
единственная стандартная реальная цель.

%---------------------------------------------------------------------
% Empty targets
%---------------------------------------------------------------------
\subsection{Пустые цели}
\index{Цели!пустые}
Пустые цели подобны абстрактным в том плане, что позволяют расширить
возможности \GNUmake{}. Абстрактные цели всегда требуют обновления и
вызывают сборку всех целей, \emph{зависимых} от абстрактной (т.е.
содержащих её в списке реквизитов). Предположим, однако, что у нас
есть команда, не ассоциированная с файлом, которую нужно выполнять
время от времени, причём зависимые цели не должны при этом
обновляться. Для этого мы можем воспользоваться целью, ассоциированной
с пустым файлом:

{\footnotesize
\begin{verbatim}
prog: size prog.o
    $(CC) $(LDFLAGS) -o $@ $^

size: prog.o
    size $^
    touch size
\end{verbatim}
}

Заметим, что правило \target{size} использует программу
\utility{touch} после своего завершения. Пустой файл используется
только для хранения времени последней модификации, и \GNUmake{} будет
выполнять правило \target{size} только в том случае, если файл
\filename{prog.o} подвергся изменению. Более того, спецификация
\target{size} как реквизита \filename{prog} будет вызывать обновление
\target{prog} только в том случае, если соответствующий объектный файл
изменялся.

\index{Переменные!автоматические!\${}?@\variable{\${}?}}
Пустые файлы бывают полезны в сочетании с автоматической переменной
\variable{\$?}. Мы обсудим автоматические переменные в разделе
<<\nameref{sec:automatic_vars}>>, но краткое описание этой переменной
здесь не повредит. Внутри сценария сборки каждого правила \GNUmake{}
определяет переменную \variable{\$?} как множество реквизитов, имеющих
более позднюю дату модификации, чем цель.  Вот пример правила,
печатающего имена всех файлов, изменившихся с момента последнего
выполнения команды \command{make print}:

{\footnotesize
\begin{verbatim}
print: *.[hc]
    lpr $?
    touch $@
\end{verbatim}
}

%%--------------------------------------------------------------------
%% Variables
%%--------------------------------------------------------------------
\section{Переменные}

Рассмотрим некоторые из тех переменных, которые мы использовали в
наших примерах. Самые простые из них имели следующий синтаксис:

{\footnotesize
\begin{alltt}
\emph{\$(имя-переменной)}
\end{alltt}
}

Эта запись означает, что мы хотим получить значение переменной с
именем \variable{имя\hyp{}переменной}. Переменные могут содержать
практически произвольный текст, а имена переменных допускают
использование большинства символов, включая знаки пунктуации.
Например, переменная, содержащая имя команды для компиляции исходного
кода на языке \Clang{}, имеет имя \variable{COMPILE.c}. Как правило,
для подстановки значения переменной её имя окружают символами
\command{\$(} и \command{)}. В случае, когда имя переменной состоит из
одного символа, скобки можно опускать.

Как правило, \Makefile{}'ы содержат много объявлений переменных. Кроме
того, существует множество переменных, определяемых непосредственно
\GNUmake{}. Некоторые из них предназначены для контроля пользователем
поведения \GNUmake{}, другие выставляются \GNUmake{} для
взаимодействия с пользовательским \Makefile{}'ом.

%---------------------------------------------------------------------
% Automatic variables
%---------------------------------------------------------------------
\subsection*{Автоматические переменные}
\label{sec:automatic_vars}

\index{Переменные!автоматические}
\newword{Автоматические переменные} вычисляются \GNUmake{} заново для
каждого исполняемого правила. Они предоставляют доступ к цели и списку
реквизитов, избавляя от необходимости явно указывать имена файлов.
Автоматические переменные полезны для избежания дублирования кода и
необходимы для написания шаблонных правил (их мы рассмотрим позже).

Существует семь автоматических переменных:

\begin{description}
%---------------------------------------------------------------------
% $@
%---------------------------------------------------------------------
\item[\variable{\$@}] \hfill \\
%---------------------------------------------------------------------
Имя файла цели правила. Если цель является элементом архива (archive
member), то <<\variable{\$@}>> обозначает имя файла архива.

%---------------------------------------------------------------------
% $%
%---------------------------------------------------------------------
\item[\variable{\$\%}] \hfill \\
%---------------------------------------------------------------------
Для целей, являющихся элементами архива, обозначает имя
элемента. Если цель не является элементом архива, то \variable{\$\%}
содержит пустое значение.

%---------------------------------------------------------------------
% $<
%---------------------------------------------------------------------
\item[\variable{\${}<}] \hfill \\
%---------------------------------------------------------------------
Имя первого реквизита в списке реквизитов.

%---------------------------------------------------------------------
% $<
%---------------------------------------------------------------------
\item[\variable{\${}?}] \hfill \\
%---------------------------------------------------------------------
Имена всех реквизитов, имеющих более позднюю дату модификации, чем
цель.

%---------------------------------------------------------------------
% $^
%---------------------------------------------------------------------
\item[\variable{\$\^}] \hfill \\
%---------------------------------------------------------------------
Имена всех реквизитов, разделённые пробелами. В списке отсутствуют
повторения элементов, поскольку для большинства задач (копирование,
компиляция и т.д.) повторения нежелательны.

%---------------------------------------------------------------------
% $+
%---------------------------------------------------------------------
\item[\variable{\$+}] \hfill \\
%---------------------------------------------------------------------
Подобно \variable{\$?}, содержит список имён реквизитов, разделённых
пробелами, с тем отличием, что может содержать повторения. Эта
переменная была введена для специфических ситуаций, таких как
компоновка, где повторение аргументов несёт особый смысл.

%---------------------------------------------------------------------
% $*
%---------------------------------------------------------------------
\item[\texttt{\$*}] \hfill \\
%---------------------------------------------------------------------
Основа имени файла цели. Как правило, основой является имя файла с
отброшенным суффиксом (мы рассмотрим вычисление основы в разделе
<<\nameref{sec:pattern_rules}>>). Использование этой переменной вне
шаблонных правил настоятельно не рекомендуется.
%---------------------------------------------------------------------
\end{description}

Кроме того, каждая из вышеперечисленных переменных имеет два варианта
для совместимости с другими версиями \GNUmake{}. Один из этих вариантов
возвращает название каталога, в которой находится соответствующий
файл. Этот вариант обозначается добавлением символа <<D>> к имени
переменной: \variable{\$(@D)}, \variable{\${}(<D)} и т.д. Второй
вариант возвращает только имена файлов без имени каталога, в котором
они находятся. Этот вариант обозначается добавлением символа <<F>> к
имени переменной: \variable{\$(@F)}, \variable{\$(<F)} и т.д.
Поскольку эти варианты имён содержат более одного символа, они должны
заключаться в круглые скобки. GNU \GNUmake{} предоставляет более
читабельные альтернативы в лице функций \function{dir} и
\function{notdir}. Мы обсудим эти функции в
главе~\ref{chap:functions}.

Вот пример нашего \Makefile{}'а, в котором явно указанные имена
заменены подходящими автоматическими переменными.

{\footnotesize
\begin{verbatim}
count_words: count_words.o counter.o lexer.o -lfl
    gcc $^ -o $@

count_words.o: count_words.c
    gcc -c $<

counter.o: counter.c
    gcc -c $<

lexer.o: lexer.c
    gcc -c $<

lexer.c: lexer.l
    flex -t $< > $@
\end{verbatim}
}

%%--------------------------------------------------------------------
%% Finding files with VPATH and vpath
%%--------------------------------------------------------------------
\section{Поиск файлов с помощью VPATH и vpath}
\label{sec:vpath}

Наши примеры были довольно просты, поэтому \Makefile{} и исходные
файлы находились в одном каталоге. Реальные программы гораздо сложнее
(когда в последний раз вы работали над проектом, все файлы которого
размещались в одном каталоге?). Давайте изменим наш пример, разместив
файлы более реалистичным образом. Мы может модифицировать нашу
программу подсчёта слов, перенеся часть работы, выполняемой функцией
\function{main}, в функцию \function{counter}:

{\footnotesize
\begin{verbatim}
#include <lexer.h>
#include <counter.h>

void counter( int counts[4] )
{
    while ( yylex( ) )
        ;

    counts[0] = fee_count;
    counts[1] = fie_count;
    counts[2] = foe_count;
    counts[3] = fum_count;
}
\end{verbatim}
}

Поместим объявление этой функции в заголовочный файл
\filename{counter.h}:

{\footnotesize
\begin{verbatim}
#ifdef COUNTER_H_
#define COUNTER_H_

extern void
counter( int counts[4] );

#endif
\end{verbatim}
}

Мы также можем поместить объявления функций из \filename{lexer.l} в
файл \filename{lexer.h}:

{\footnotesize
\begin{verbatim}
#ifndef LEXER_H_
#define LEXER_H_

extern int fee_count, fie_count, foe_count, fum_count;

extern int yylex( void );

#endif
\end{verbatim}
}

В соответствии с негласным правилами размещения исходного кода в
файловой системе, все заголовочные файлы помещаются в каталог
\filename{include}, а исходные файлы~--- в каталог \filename{src}.
Разместим наши файлы подобным образом, оставив \Makefile{} в корневом
каталоге проекта. Результирующее дерево изображено на
рисунке~\ref{fig:ex_src_tree}.

\begin{figure}
{\footnotesize
\begin{verbatim}
.
|--include
|  |--counter.h
|  `--lexer.h
|--src
|  |--count\_words.c
|  |--counter.c
|  `--lexer.l
`--Makefile
\end{verbatim}
}
\caption{Структура каталогов проекта программы подсчёта слов.}
\label{fig:ex_src_tree}
\end{figure}

Поскольку теперь наши файлы с исходным кодом включают заголовочные
файлы, эта новая зависимость должна быть отражена в нашем \Makefile{},
то есть если модифицируется заголовочный файл, то соответствующий
объектный файл должен быть заново скомпилирован из исходного кода.

{\footnotesize
\begin{verbatim}
count_words: count_words.o counter.o lexer.o -lfl
    gcc $^ -o $@

count_words.o: count_words.c include/counter.h
    gcc -c $<

counter.o: counter.c include/counter.h include/lexer.h
    gcc -c $<

lexer.o: lexer.c include/lexer.h
    gcc -c $<

lexer.c: lexer.l
    flex -t $< > $@
\end{verbatim}
}

После запуска \GNUmake{} получаем следующий результат:

{\footnotesize
\begin{alltt}
\$ \textbf{make}
make:  *** No rule to make target `count\_words.c',
       needed by `count\_words.o'.  Stop.
\end{alltt}
}

Что же произошло? \GNUmake{} попытался собрать исходный файл
\filename{count\_words.c}! Давайте проследим за работой \GNUmake{}.
Первым реквизитом является файл \filename{count\_words.o}. Такой файл
не существует, следовательно, его нужно создать. Явное правило для
создания \filename{count\_words.o} указывает на
\filename{count\_words.c}. Но \GNUmake{} не сможет найти этот
исходный файл, ведь он находится не в текущем каталоге, а в каталоге
\filename{src}. Пока вы не укажите другого, \GNUmake{} будет искать
цели и реквизиты в текущем каталоге. Как сообщить \GNUmake{}, что
исходные файлы нужно искать в каталоге \filename{src}, или, если
поставить вопрос более \'{о}бщно, где нужно искать файлы с исходным
кодом?

Вы можете указать \GNUmake{} каталоги, в которых нужно искать
файлы, используя переменную \variable{VPATH} и директиву
\directive{vpath}. Для того, чтобы решить нашу проблему, мы может
добавить определение \variable{VPATH} в наш \Makefile{}:

{\footnotesize
\begin{verbatim}
VPATH = src
\end{verbatim}
}

Это означает, что \GNUmake{} должен искать в каталоге \filename{src}
те файлы, которые не обнаружены в текущем каталоге. После запустка
\GNUmake{}, мы получим следующий вывод:

{\footnotesize
\begin{alltt}
\$ \textbf{make}
gcc -c src/count\_words.c -o count\_words.o
src/count\_words.c:2:21: counter.h: No such file or directory
make: *** [count\_words.o] Error 1
\end{alltt}
}

Заметим, что в этот раз \GNUmake{} успешно запустил команду
компиляции, правильно подставив относительный путь к исходному файлу.
В этом заключается ещё один плюс использования автоматических
переменных: \GNUmake{} не сможет использовать подходящий путь к
исходному файлу, если вы явно укажите его имя. К несчастью, процесс
компиляции окончился неудачей, так как \utility{gcc} не смог найти
заголовочного файла. Эту проблему можно решить, настроив неявное
правило компиляции подходящей опцией \command{-I}:

{\footnotesize
\begin{verbatim}
CPPFLAGS = -I include
\end{verbatim}
}

Теперь сборка проходит успешно:

{\footnotesize
\begin{alltt}
\$ \textbf{make}

gcc -I include -c src/count\_words.c -o count\_words.o
gcc -I include -c src/counter.c -o counter.o
flex -t src/lexer.l > lexer.c
gcc -I include -c lexer.c -o lexer.o
gcc count\_words.o counter.o lexer.o /lib/libfl.a -o count\_words
\end{alltt}
}

Переменная \variable{VPATH} содержит список всех каталогов, в которых
\GNUmake{} будет искать недостающие файлы. В этих каталогах будет
происходить поиск как целей, так и реквизитов, но не файлов, указанных
в сценариях сборки. Элементы списка могут разделяться пробелами или
двоеточиями в \UNIX{} и пробелами или точками с запятой в Windows.
Предпочтительней пользоваться пробелами, поскольку такой формат
подходит для любой операционной системы и позволяет избежать путаницы
с точками с запятой и двоеточиями. Кроме того, имена каталогов,
разделённые пробелами, легче читать.

Переменная \variable{VAPTH} решает нашу проблему поиска, но всё же она
является слишком мощным инструментом. \GNUmake{} ищет в каждом
указанном в \variable{VPATH} каталоге \emph{каждый} недостающий
файл. Если в разных каталогах существуют файлы с одинаковыми
именами, \GNUmake{} возьмёт первый найденный. Иногда это может быть
причиной неприятностей.

Директива \directive{vpath} больше подходит для наших нужд. Синтаксис
вызова этой директивы представлен ниже:

{\footnotesize
\begin{alltt}
vpath \emph{шаблон список-каталогов}
\end{alltt}
}

Теперь мы можем исключить использование переменной \variable{VPATH},
добавив следующие строки в наш \Makefile{}:

{\footnotesize
\begin{verbatim}
vpath %.c src
vpath %.h include
\end{verbatim}
}

Теперь мы сообщили \GNUmake{}, что искать все файлы с расширением
\filename{.c} нужно в каталоге \filename{src}, а файлы с расширением
\filename{.h}~--- в каталоге \filename{include} (теперь мы можем
удалить префикс \filename{include} у заголовочных файлов в списке
реквизитов). В более сложных приложениях такой подход позволяет
сохранить нервы и время, потраченные на отладку.

В нашем примере мы использовали \directive{vpath} для решения проблемы
поиска файлов, разнесённых по разным директориям. Существует также
родственная проблема сборки приложения таким образом, чтобы объектные
файлы помещались в отдельное <<бинарное>> дерево каталогов, в то время
как исходные файлы хранились дереве каталогов <<исходного кода>>.
Правильное использование \directive{vpath} помогает решить и эту
проблему, однако эта задача быстро становиться очень сложной, так что
использование одной лишь директивы \directive{vpath} становится
неэффективным.  Мы обсудим эту проблему более детально в следующих
разделах.

%%--------------------------------------------------------------------
%% Pattern rules
%%--------------------------------------------------------------------
\section{Шаблонные правила}
\label{sec:pattern_rules}

\index{Правила!шаблонные}
Все примеры \Makefile{}'ов, рассмотренные нами, были слишком подробны.
Для маленькой программы, содержащей десяток файлов это не имеет
большого значения, но явная спецификация всех целей, реквизитов и
сценариев сборки для программы, содержащей сотни и тысячи файлов, не
представляется возможным. Более того, команды сами по себе вносят
дублирующийся код в наш \Makefile{}. Это может быть источником ошибок
и большой проблемой при поддержке.

Многие программы, читающие один тип файлов и выводящие другой следуют
стандартным соглашениям. Например, все компиляторы языка \Clang{}
предполагают, что файлы с расширением \filename{.c} содержат исходный
код на языке \Clang{}, а имена объектных файлов могут быть получены
заменой расширения \filename{.c} на расширение \filename{.o} (или
\filename{.obj} для некоторых компиляторов в Windows). В предыдущей
главе мы увидели, что исходные файлы генератора \utility{flex} имеют
расширение \filename{.l}, а расширением генерируемых им файлов
является \filename{.c}.

Эти и другие соглашения позволяют \GNUmake{} упростить создание правил
с помощью распознавания шаблонов и предоставления встроенных правил
сборки. Например, используя встроенные правила, мы можем сократить наш
\Makefile{} с семнадцати строк до шести:

{\footnotesize
\begin{verbatim}
VPATH = src include
CPPFLAGS = -I include

count_words: counter.o lexer.o -lfl
count_words.o: counter.h
counter.o: counter.h lexer.h
lexer.o: lexer.h
\end{verbatim}
}

\index{Правила!встроенные}
Все встроенные правила являются шаблонными. Шаблонные правила похожи
\index{Основа файла}
на обычные за тем исключением, что \newword{основа} имени файла
(часть, не содержащая расширение) представлена символом \command{\%}.
Наш \Makefile{} выполняет свою работу благодаря трём встроенным
правилам. Первое указывает, как произвести компиляцию файла с
расширением \filename{.o} из файла с расширением \filename{.c}:

{\footnotesize
\begin{verbatim}
%.o: %.c
    $(COMPILE.c) $(OUTPUT_OPTION) $<
\end{verbatim}
}

Второе правило указывает, как получить файлы с расширением
\filename{.c} из файлов с расширением \filename{.l}:

{\footnotesize
\begin{verbatim}
%.c: %.l
    @$(RM) $@
    $(LEX.l) $< > $@
\end{verbatim}
}

Наконец, существует специальное правило получения файла без расширения
(он всегда считается исполняемым) из файлов с расширением
\filename{.o}:

{\footnotesize
\begin{verbatim}
%: %.o
    $(LINK.o) $^ $(LOADLIBES) $(LDLIBS) -o $@
\end{verbatim}
}

Прежде, чем углубиться в детали синтаксиса, давайте внимательно
рассмотрим вывод \GNUmake{} и разберёмся, как он применяет встроенные
правила.

Если запустить \GNUmake{} с нашим исправленным \Makefile{}'ом, вывод
будет следующим:

{\footnotesize
\begin{alltt}
\$ \textbf{make}
gcc  -I include  -c -o count\_words.o src/count\_words.c
gcc  -I include  -c -o counter.o src/counter.c
flex  -t src/lexer.l > lexer.c
gcc  -I include  -c -o lexer.o lexer.c
gcc   count\_words.o counter.o lexer.o /lib/libfl.a -o count\_words
rm lexer.c
\end{alltt}
}

Сначала \GNUmake{} читает \Makefile{} и выставляет целью по умолчанию
файл \filename{count\_words}, поскольку аргументы командной строки не
заданы. Рассматривая цель по умолчанию, \GNUmake{} определяет
её реквизиты: \filename{count\_words.o} (этот реквизит не указан в
\Makefile{}, он следует из неявного правила), \filename{counter.o},
\filename{lexer.o} и \filename{-lfl}. Далее \GNUmake{} пытается
собрать каждый из этих реквизитов.

Когда \GNUmake{} рассматривает первый реквизит,
\filename{count\_words.o}, он не находит явного правила, однако
обнаруживает неявное. Не найдя исходного файла в текущем каталоге,
\index{Переменные!встроенные!VPATH@\variable{VPATH}}
он просматривает директории, указанные в \variable{VPATH}, и
обнаруживает исходный файл в каталоге \filename{src}. Поскольку файл
\filename{src/count\_words.c} не имеет реквизитов, \GNUmake{}
запускает сценарий компиляции \filename{count\_words.o},
ассоциированный с неявным правилом.  Таким же образом получается файл
\filename{counter.o}. Когда \GNUmake{} рассматривает
\filename{lexer.o}, соответствующий исходный файл не обнаруживается,
поэтому \GNUmake{} предполагает, что \filename{lexer.c} является
промежуточным файлом, и ищет подходящее правило для его получения.
Обнаружив правило создания файла с расширением \filename{.c} из файла
с расширением \filename{.l}, \GNUmake{} замечает, что файл
\filename{lexer.l} существует. Поскольку \filename{lexer.l} не имеет
реквизитов, вызывается сценарий генерации файла \filename{lexer.c},
содержащий вызов команды \utility{flex}. Затем происходит компиляция с
получением объектного файла. Использование последовательностей правил
\index{Цепочка правил}
для сборки цели, подобных предыдущим, называется \newword{цепочкой
правил}.

Далее, \GNUmake{} исследует спецификацию библиотеки \filename{-lfl}.
Осуществляется поиск в каталогах из списка стандартных каталогов
библиотек, результатом которого является файл \filename{/lib/libfl.a}.

Теперь \GNUmake{} имеет все реквизиты для получения
\filename{count\_words}, запускается команда компоновки \utility{gcc}.
Наконец, \GNUmake{} вспоминает, что был создан промежуточный файл,
который больше не нужно хранить, и осуществляет удаление этого файла.

Как мы убедились, использование правил в \Makefile{}'ах позволяет
избежать спецификации многих деталей. Правила могут иметь сложные
отношения, которые порождают чрезвычайно широкие возможности. В
частности, наличие встроенной базы данных общих правил значительно
упрощает спецификацию многих \Makefile{}'ов.

Встроенные правила могут быть настроены путём изменения значений
переменных, фигурирующих в сценариях сборки. Типичные сценарии имеют
множество переменных, начиная с имени команды, подлежащей выполнению,
и заканчивая большими группами опций командной строки, такими, как имя
выходного файла, флаги оптимизации, включение символьной информации и
т.д. Вы можете увидеть стандартную базу данных \GNUmake{} с помощью
команды \command{make --print\hyp{}data\hyp{}base}.

%---------------------------------------------------------------------
% The patterns
%---------------------------------------------------------------------
\subsection{Шаблоны}
\index{Шаблоны}
Символ процента в шаблонном правиле практически эквивалентен
специальному символу \command{*} командного интерпретатора \UNIX{}. Он
соответствует произвольному числу символов. Знак процента может
появиться в любом месте шаблона ровно один раз. Вот примеры
допустимого использования этого символа.

{\footnotesize
\begin{verbatim}
%,v
s%.o
wrapper_%
\end{verbatim}
}

Символы, входящие в шаблон и отличные от процента, соответствуют самим
себе.  Шаблон может содержать префикс, суффикс или и то, и другое.
Когда \GNUmake{} ищет соответствие шаблонному правилу, сначала
проверяется соответствие шаблону цели. Шаблон цели должен начинаться с
префикса и заканчиваться суффиксом (если они есть). Если соответствие
найдено, символы между префиксом и суффиксом становятся основой имени
файла. Затем \GNUmake{} производит поиск реквизитов шаблонного
правила, подставляя основу имени файла в шаблон реквизита. Если
результирующий файл существует или может быть получен с помощью
другого правила, соответствие считается установленным и правило
применяется. Основа файла должна содержать хотя бы один символ.

Допускается также иметь шаблоны с одним только символом процента.
Наиболее частое использование этого шаблона~--- сборка исполняемых
файлов \UNIX{}. Например, вот несколько встроенных шаблонных правил,
использующихся \GNUmake{} для сборки программ:

{\footnotesize
\begin{verbatim}
%: %.mod
    $(COMPILE.mod) -o $@ -e $@ $^

%: %.cpp

    $(LINK.cpp) $^ $(LOADLIBES) $(LDLIBS) -o $@

%: %.sh
    cat $< >$@
    chmod a+x $@
\end{verbatim}
}

Эти правила будут использованы для сборки программы из исходных файлов
Modula, исходных файлов \Clang{}, обработанных препроцессором, или
командных сценариев интерпретатора Bourne shell соответственно. Мы
рассмотрим другие неявные правила в разделе
<<\nameref{sec:implicit_rule_db}>>.

%---------------------------------------------------------------------
% Static pattern rules
%---------------------------------------------------------------------
\subsection{Статические шаблонные правила}
Статические шаблонные правила применяются только к определённому
списку целей.

{\footnotesize
\begin{verbatim}
$(OBJECTS): %.o: %c
        $(CC) -c $(CFLAGS) $< -o $@
\end{verbatim}
}

Единственная разница между этим правилом и обычным шаблонным
правилом~--- наличие спецификации \command{\$(OBJECTS):}. Эта
спецификация ограничивает область действия правила файлами,
перечисленными в переменной \variable{\$(OBJECTS)}. Каждый объектный
файл из этой переменной проверяется на соответствие шаблону
\command{\%.c}, извлекается основа его имени, которая затем
подставляется в шаблон \command{\%.c}, порождая имя реквизита цели.
Если файл из списка не соответствует шаблону цели, \GNUmake{} выдаст
предупреждение.

Используйте статические шаблонные правила везде, где легче перечислить
файлы целей явно, нежели определять их по расширению или шаблону.

%---------------------------------------------------------------------
% Suffix rules
%---------------------------------------------------------------------
\subsection{Суффиксные правила} \label{sec:suffix_rules}
\index{Правила!суффиксные}
Суффиксные правила~--- это первоначальный (и ныне устаревший) способ
определения неявных правил. Поскольку другие версии \GNUmake{} могут
не поддерживать синтаксис шаблонов GNU \GNUmake{}, вы ещё можете
встретить такие правила в \Makefile{}'ах дистрибутивов,
предназначенных для широкого распространения, поэтому важно
уметь читать и понимать их синтаксис. Итак, несмотря на то, что
компиляция GNU \GNUmake{} для целевой платформы является
предпочтительным методом переноса \Makefile{}'ов, при некоторых
обстоятельствах вам, возможно, придётся писать суффиксные правила.

Суффиксные правила выглядят как два написанных слитно суффикса,
выступающие в роли правила:

{\footnotesize
\begin{verbatim}
.c.o:
    $(COMPILE.c) $(OUTPUT_OPTION) $<
\end{verbatim}
}

Тот факт, что первым указывается суффикс реквизита, а вторым~---
суффикс цели, может внести некоторую путаницу. Предыдущему правилу
соответствуют в точности те же цели и реквизиты, что и следующему:

{\footnotesize
\begin{verbatim}
%.o: %.c
    $(COMPILE.c) $(OUTPUT_OPTION) $<
\end{verbatim}
}

Суффиксные правила формируют основу имени файла путём удаления
суффикса; имя файла реквизита получается в результате замены суффикса
цели суффиксом реквизита. Суффиксное правило распознаётся \GNUmake{}
только в том случае, если оба суффикса находятся в списке известных
суффиксов.

Предыдущее суффиксное правило называется также двусуффиксным,
поскольку оно содержит два суффикса. Существуют также односуффиксные
правила. Такие правила содержат только один суффикс~--- суффикс
реквизита. Эти правила используются для сборки исполняемых файлов
\UNIX{}, не имеющих расширения:

{\footnotesize
\begin{verbatim}
.p:
    $(LINK.p) $^ $(LOADLIBES) $(LDLIBS) -o $@
\end{verbatim}
}

Предыдущее правило описывает создание исполняемого файла из исходного
файла Pascal. Оно полностью эквивалентно следующему правилу:

{\footnotesize
\begin{verbatim}
%: %.p
    $(LINK.p) $^ $(LOADLIBES) $(LDLIBS) -o $@
\end{verbatim}
}

Список известных суффиксов является наиболее необычной частью
синтаксиса. Для определения этого списка используется специальная
цель \target{.SUFFIXES}. Вот первая часть стандартного определения
\target{.SUFFIXES}:

{\footnotesize
\begin{verbatim}
.SUFFIXES: .out .a .ln .o .c .cc .C .cpp .p .f .F .r .y .l
\end{verbatim}
}

Вы можете добавить собственные суффиксы, добавив правило
\target{.SUFFIXES} в ваш \Makefile{}:

{\footnotesize
\begin{verbatim}
.SUFFIXES: .pdf .fo .html .xml
\end{verbatim}
}

Если вы хотите удалить все известные суффиксы (например, они
перекрываются с вашими специальными суффиксами), определите цель
\target{.SUFFIXES} как не имеющую реквизитов:

{\footnotesize
\begin{verbatim}
.SUFFIXES:
\end{verbatim}
}

Вы также можете использовать опцию командной строки
\index{Опции!no-builtin-rules@\command{-{}-no\hyp{}builtin\hyp{}rules (-r)}}
\command{-{}-no\hyp{}builtin\hyp{}rules} (или \command{-r}).

Мы не будем использовать суффиксные правила в этой книге, поскольку
шаблонные правила GNU \GNUmake{} более выразительны и \'{о}бщны.

%%--------------------------------------------------------------------
%% Implicit rules database
%%--------------------------------------------------------------------
\section{База данных неявных правил}
\label{sec:implicit_rule_db}

\index{База данных!неявных правил}
GNU \GNUmake{} 3.80 содержит примерно 90 встроенных неявных правил
(шаблонных и суффиксных), предназначенных для работы с исходными
файлами \Clang{}, \Cplusplus{}, Pascal, \FORTRAN{}, ratfor, Modula,
Texinfo, \TeX{} (включая инструменты \utility{tangle} и
\utility{weave}), Emacs Lisp, RCS и SCCS, а также правила для
поддержки программ, предназначенных для работы с этими языками:
\utility{cpp}, \utility{as}, \utility{yacc},  \utility{lex},
\utility{tangle}, \utility{weave} и инструменты для работы с
\utility{dvi}.

Если вы используете одну из этих программ, то, возможно, вас вполне
устроят возможности, предоставляемые встроенными правилами. Если же вы
используете некоторые не поддерживаемые языки наподобие \Java{} или
XML, вам придётся писать правила самим. Впрочем, обычно для добавления
поддержки языка достаточно написать всего несколько простых правил.

Чтобы увидеть встроенные правила \GNUmake{}, нужно запустить его с
\index{Опции!print-data-base@\command{-{}-print-data-base (-p)}}
опцией \command{-{}-print\hyp{}data\hyp{}base} (или просто
\command{-p}). Вывод составит около тысячи строк текста: после номера
версии и текста лицензии \GNUmake{} выведет на экран определения всех
переменных с описанием их <<происхождения>>. Например, переменные
могут быть взяты из окружения, являться стандартными или
автоматическими. После описания переменных последуют правила. Текущий
формат правил GNU \GNUmake{} выглядит следующим образом:

{\footnotesize
\begin{verbatim}
%: %.C
#  Команды для выполнения (встроенные):
    $(LINK.C) $^ $(LOADLIBES) $(LDLIBS) -o $@
\end{verbatim}
}

Комментарии к правилам, определённым в \Makefile{}, будут содержать
имя файла и номер строки, в которых эти правила определены:

{\footnotesize
\begin{verbatim}
%.html: %.xml
#  Команды для выполнения (из `Makefile', строка 168):
    $(XMLTO) $(XMLTO\_FLAGS) html-nochunks $<
\end{verbatim}
}

%---------------------------------------------------------------------
% Working with implicit rules
%---------------------------------------------------------------------
\subsection{Работа с неявными правилами}
Встроенные правила применяются каждый раз, когда рассматривается цель,
для которой не указан сценарий сборки. Таким образом, использовать
встроенные правила очень просто~--- достаточно не указывать команд при
добавлении цели в \Makefile{}. Это будет сигналом для \GNUmake{},
призывающим к поиску подходящего правила в базе данных встроенных
правил. Обычно это приводит к нужному результату, но в редких случаях
ваша среда разработки может породить некоторые проблемы.  Предположим,
у вас имеется смешанная среда разработки, состоящая из исходных файлов
C и Lisp. Если файлы \filename{editor.l} и \filename{editor.c}
находятся в одном каталоге (например, один из них является
низкоуровневой реализацией, к которой имеет доступ другой), \GNUmake{}
будет считать, что файл с исходным кодом на Lisp на самом деле
является файлом \utility{flex} (повторим, \utility{flex} использует
файлы с расширением \filename{.l}), а исходный файл \Clang{}~---
результат запуска \utility{flex}. Если файл \filename{editor.o}
является целью, а \filename{editor.l} модифицировался позднее
\filename{editor.c}, \GNUmake{} попытается обновить исходный файл
\Clang{} выводом команды \utility{flex}, заместив ваш исходный код.

Чтобы избежать этой проблемы, вы можете удалить из базы данных
правила, вызывающие \utility{flex}, следующим образом:

{\footnotesize
\begin{verbatim}
%.o: %.l
%.c: %.l
\end{verbatim}
}

Шаблонное правило без сценария сборки удаляет правило из базы данных
\GNUmake{}. На практике ситуации, подобные описанной, встречаются
крайне редко. Тем не менее, важно помнить, что правила из встроенной
базы данных могут взаимодействовать с вашими \Makefile{}'ами
неожиданным для вас образом.

Мы уже рассмотрели несколько примеров того, как \GNUmake{} выстраивает
цепочки правил для сборки цели. Такое поведение может порождать
довольно сложные конструкции. Когда \GNUmake{} рассматривает варианты
сборки цели, происходит поиск неявных правил, шаблону цели которых
соответствует имя текущей цели. Для каждого подходящего шаблона цели
\GNUmake{} осуществляет проверку соответствия реквизитов. Таким
образом, проверка реквизитов осуществляется \emph{сразу} после
нахождения подходящего шаблона цели. Если реквизиты найдены, правило
применяется. Для некоторых шаблонов цели может быть несколько
возможных видов реквизитов. Например, файлы с расширением
\filename{.o} можно получить из файлов с расширением \filename{.c},
\filename{.cc}, \filename{.cpp}, \filename{.p}, \filename{.f},
\filename{.r}, \filename{.s} и \filename{.mod}. Если исходные файлы не
обнаружены при переборе всех возможных вариантов, \GNUmake{}
произведёт повторный поиск правил, рассматривая исходные файлы в
качестве новых целей. Повторяя этот поиск рекурсивно, \GNUmake{}
выстроит цепь правил, позволяющих произвести сборку цели. Мы уже имели
возможность убедиться в этом на примере файла \filename{lexer.o}:
\GNUmake{} смог получить \filename{lexer.o} из \filename{lexer.l} даже
при отсутствующем файле \filename{lexer.c}, вызвав сначала правило
\filename{.l}$\,\,\rightarrow{}$\filename{.c}, а затем правило
\filename{.c}$\,\,\rightarrow{}$\filename{.o}.

Давайте рассмотрим одну из самых впечатляющих цепочек, которую
\GNUmake{} может породить с помощью встроенной базы данных правил.
Сначала создадим пустой исходный файл \utility{yacc} и зарегистрируем
его в системе контроля ревизий RCS командой \utility{ci}:

\begin{alltt}
\footnotesize
\$ \textbf{touch foo.y}
\$ \textbf{ci foo.y}
foo.y,v  <--  foo.y
.
initial revision: 1.1
done
\end{alltt}

Теперь попросим \GNUmake{} создать исполняемый файл \filename{foo}.
\index{Опции!just-print@\command{-{}-just-print (-n)}}
Опция \command{-{}-just\hyp{}print} (или просто \command{-n})
означает, что от \GNUmake{} требуется лишь описать, какие действия
будут выполнены.  Заметим, что у нас нет \Makefile{}'а и исходного
кода, только RCS-файл.

\begin{alltt}
\footnotesize
\$ \textbf{make -n foo}
co  foo.y,v foo.y
foo.y,v  -->  foo.y
revision 1.1
done
bison -y  foo.y
mv -f y.tab.c foo.c
gcc -c -o foo.o foo.c
gcc foo.o -o foo
rm foo.c foo.o foo.y
\end{alltt}

Следуя по цепочке неявных правил и реквизитов, \GNUmake{} определяет,
что сборка исполняемого файла \filename{foo} возможна при наличии
объектного файла \filename{foo.o}, который можно получить из исходного
файла \filename{foo.c}. В свою очередь, \filename{foo.c} может быть
получен из файла \filename{yacc} \filename{foo.y}, доступного при
извлечении его из файла RCS \filename{foo.y,v}, имеющегося в наличии.
Составив план действий, \GNUmake{} выполняет извлечение
\filename{foo.y} с помощью команды \utility{co}, преобразует его в
исходный файл \filename{foo.c} командой \utility{bison}, компилирует
полученный исходный файл C в объектный файл \filename{foo.o} вызовом
\utility{gcc} и производит компоновку для получения исполняемого файла
\filename{foo} повторным вызовом \utility{gcc}. Всё это происходит с
использованием лишь встроенной базы данных правил. Впечатляет.

Файлы, порождаемые применением цепочки правил, называются
\emph{промежуточными} и обрабатываются \GNUmake{} особым образом.
Во-первых, поскольку промежуточные файлы не специфицируются в качестве
целей (иначе они бы не были промежуточными), \GNUmake{} никогда не
удовлетвориться просто сборкой этого файла. Во-вторых, поскольку
\GNUmake{} создаёт эти файлы самостоятельно как побочные эффекты
сборки цели, такие файлы должны быть удалены перед завершением работы.
Вы можете убедиться в этом, посмотрев на последнюю строку предыдущего
примера.

%---------------------------------------------------------------------
% Rule structure
%---------------------------------------------------------------------
\subsection{Структура правил}
Встроенные правила имеют стандартную структуру, направленную на
предоставление возможности простой настройки этих правил.
Рассмотрим сначала общую структуру, затем обсудим тонкости настройки.
Ниже представлено уже знакомое правило компиляции исходного файла
\Clang{} в объектный файл:

{\footnotesize
\begin{verbatim}
%.o: %.c
    $(COMPILE.c) $(OUTPUT_OPTION) $<
\end{verbatim}
}

Настройка этого правила осуществляется за счёт переменных, используемых
им. Мы видим две переменные, однако переменная
\variable{\$(COMPILE.c)}, например, определена через другие переменные:

{\footnotesize
\begin{verbatim}
COMPILE.c = $(CC) $(CFLAGS) $(CPPFLAGS) $(TARGET\_ARCH) -c
CC = gcc
OUTPUT_OPTION = -o $@
\end{verbatim}
}

Таким образом, замена компилятора языка \Clang{} может быть
произведена изменением значения переменной \variable{CC}. Другие
переменные используются для определения опций компиляции
(\variable{CFLAGS}), опций препроцессора (\variable{CPPFLAGS}) и
архитектурно\hyp{}зависимых опций (\variable{TARGET\_ARCH}).

Все переменные, использующиеся во встроенных правилах, нацелены на
максимально простую настройку правил. По этой причине очень важно
быть осторожным при определении значений этих переменных в вашем
\Makefile{}'е. Если вы будете определять переменные <<по наитию>>,
вы можете нарушить возможность настройки правил конечным
пользователем. Рассмотрим пример присваивания в \Makefile{}'е:

{\footnotesize
\begin{verbatim}
CPPFLAGS = -I project/include
\end{verbatim}
}

Если пользователь хочет самостоятельно определить значение переменной
в командной строке, он обычно поступает следующим образом:

\begin{alltt}
\footnotesize
\$ \textbf{make CPPFLAGS=-DDEBUG}
\end{alltt}

Однако такой вызов случайно удалит опцию \command{-I} (которая,
предположительно, необходима для компиляции) поскольку значения
переменных, определённые в командной строке, перекрывают все другие
присваивания этим переменным\footnote{Для более подробных сведений о
присваиваниях в командной строке следует обратиться к
разделу <<\nameref{sec:where_vars_come_from}>>.}. Таким образом,
ненадлежащее определение переменной \variable{CPPFLAGS} нарушило
возможность настройки, на наличие которой рассчитывает б\'{о}льшая
часть пользователей. Вместо использования простых присваиваний,
рассмотрим переопределение переменной компиляции с добавлением новых
переменных:

{\footnotesize
\begin{verbatim}
COMPILE.c = $(CC) $(CFLAGS) $(INCLUDES) $(CPPFLAGS) $(TARGET_ARCH) -c
INCLUDES = -I project/include
\end{verbatim}
}

Ещё одним выходом из ситуации является использование доопределения
переменных вместо присваивания, этот подход обсуждается в разделе
<<\nameref{sec:other_types_of_assign}>> главы~\ref{chap:vars}.

%---------------------------------------------------------------------
% Implicit rules for source control
%---------------------------------------------------------------------
\subsection{Неявные правила для управления ревизиями}
В \GNUmake{} реализована на уровне встроенных правил поддержка двух
\index{RCS} \index{SCCS}
систем контроля ревизий: RCS и SCCS. Складывается впечатление, что
искусство управления исходным кодом на текущем этапе своего развития
и современная компьютерная инженерия оставили \GNUmake{} далеко позади.
На практике поддержка управления ревизиями в \GNUmake{} практически
не используется. И на это есть ряд причин.

Во\hyp{}первых, инструменты контроля ревизий, поддерживаемые
\GNUmake{}, RCS и SCCS, в прошлом весьма значимые и ценные, были
\index{CVS}
повсеместно вытеснены CVS (Concurrent Version System) или
коммерческими инструментами. Несмотря на то, что CVS использует RCS
для внутреннего управления одиночными файлами, прямое использование
RCS порождает значительные проблемы, когда проект состоит из более чем
одного каталога, или в нём задействовано более одного разработчика. В
частности, CVS была разработана для заполнения пробелов в
функциональности RCS в упомянутых аспектах. Поддержки CVS в \GNUmake{}
никогда не было, и это, возможно, является правильным
решением\footnote{CVS, в свою очередь, постепенно вытесняется более
современными инструментами. На данный момент наиболее часто
встречается система управления исходным кодом
\index{Subversion}
Subversion (\filename{\url{http://subversion.tiqris.org}})
(прим. автора).}.

Общепризнано, что жизненный цикл разработки программного обеспечения
становится всё более сложным. Приложения редко плавно продвигаются от
одного релиза к другому. Как правило, одновременно может
использоваться (и, следовательно, требовать исправления ошибок)
несколько версий приложения, в то время как ещё несколько версий могут
находиться в активной разработке. CVS предоставляет богатые
возможности управления параллельной разработкой программного
обеспечения. Однако это требует от разработчика чёткого осознания
того, над какой версией исходного кода он работает. Если бы \GNUmake{}
автоматически извлекал из хранилища исходный код для компиляции, сразу
бы вставали вопросы актуальности извлечённой версии и совместимости
этой версии с кодом, расположенным в рабочих каталогах разработчика.
Во многих проектах разработчики работают с тремя и более различными
версиями программного продукта в течение одного дня. Проверка
целостности сложной системы достаточно сложна и без наличия
инструмента, безмолвно изменяющего ваш исходный код.

К тому же, одной из самых важных возможностей CVS является возможность
доступа к удалённому хранилищу. В большинстве проектов CVS
хранилище (база данных версионных файлов) располагается на сервере,
а не на машине разработчика. Несмотря на то, что удалённый доступ
достаточно быстр (по крайней мере, в локальных сетях), запуск
\GNUmake{}, проверяющего каждый файл на удалённом сервере, не является
хорошей идеей, поскольку производительность упадёт катастрофически.

Таким образом, можно использовать встроенные правила для довольно
прозрачного взаимодействия с RCS и SCCS, но правил для доступа к
хранилищам CVS для поиска файлов не существует. По большому счёту,
даже если бы такие правила существовали, им трудно было бы найти
разумное применение. С другой стороны, использование CVS для
управления версиями \Makefile{}'ов чрезвычайно полезно и порождает
много интересных применений, таких как проверка правильности помещения
файла в хранилище, управление нумерацией релизов и корректное
выполнение автоматического тестирования. Все это возможно при
использовании CVS авторами \Makefile{}'ов, а не за счёт интеграции
\GNUmake{} с CVS.

%---------------------------------------------------------------------
% A simple help command
%---------------------------------------------------------------------
\subsection{Пример простой справки}
Большие \Makefile{}'ы могут содержать много целей, трудных для
запоминания. Одним из способов устранения этой проблемы является
выбор в качестве цели по умолчанию абстрактной цели вывода краткой
справки. Однако поддержка текста этой справки в актуальном состоянии
является довольно утомительным занятием. К счастью, для составления
справки могут быть использованы команды из встроенной базы данных
правил \GNUmake{}. Ниже приведён пример цели, выводящей отсортированный
список доступных целей:

{\footnotesize
\begin{verbatim}
# help - цель по умолчанию
.PHONY: help
help:
    $(MAKE) --print-data-base --question |            \
    $(AWK) '/\^[\^.\%][-A-Za-z0-9\_]*:/               \
           { print substr($$1, 1, length($$1)-1) }' | \
    $(SORT) |                                         \
    $(PR) --omit-pagination --width=80 --columns=4
\end{verbatim}
}

Сценарий состоит из одного конвейера программ. Список правил \GNUmake{}
\index{Опции!print-data-base@\command{-{}-print-data-base (-p)}}
выводится при помощи ключа \command{-{}-print\hyp{}data\hyp{}base}, ключ
\index{Опции!question@\command{-{}-question (-q)}}
\command{-{}-question} исключает выполнение сценариев сборки. Затем
база данных пропускается через простой сценарий \utility{awk},
извлекающий из потока правил все цели, имена которых не начинаются
с символов процента или точки (то все шаблонные и суффиксные правила,
соответственно) и вырезающий всю прочую информацию в строке. Наконец,
цели сортируются по имени и выводятся в четыре колонки.

Другим возможным решением проблемы является применение специального
сценария \utility{awk} непосредственно к \Makefile{}'у. Это потребует
специальной обработки включения \Makefile{}'ов (см. раздел
<<\nameref{sec:include_directive}>> главы~\ref{chap:vars}) и сделает
невозможным обработку правил, созданных самим \GNUmake{} на основе
шаблонных и суффиксных правил. Версия, представленная выше, избавлена
от этих недостатков благодаря вызову \GNUmake{}, осуществляющему все
необходимые действия самостоятельно.

%%--------------------------------------------------------------------
%% Special targets
%%--------------------------------------------------------------------
\section{Специальные цели}

\index{Цели!специальные}
\newword{Специальной целью} называют встроенную абстрактную цель,
предназначенную для спецификации поведения \GNUmake{}. Например, уже
известная нам \target{.PHONY} является специальной целю, реквизиты
которой не связаны с реальными файлами и всегда требуют обновления.

Для специальных целей используется стандартный синтаксис
\ItalicMono{цель: реквизиты}, но \ItalicMono{цель} не является файлом
или обычной абстрактной целью. Более всего специальные цели похожи на
директивы, изменяющие внутренние алгоритмы \GNUmake{}.

Существует двенадцать специальных целей. Их можно разделить на три
категории: одни изменяют поведение \GNUmake{}, вторые являются просто
флагами, наконец, специальная цель \variable{.SUFFIXES} используется
для спецификации суффиксных правил (обсуждавшихся в разделе
<<\nameref{sec:suffix_rules}>>).

Наиболее полезны следующие специальные цели:

\begin{description}
\item[\target{.INTERMEDIATE}] \hfill\\
Реквизиты этой цели интерпретируются как промежуточные файлы. Если
\GNUmake{} создаст файл во время сборки другой цели, файл будет
автоматически удалён перед завершением работы \GNUmake{}. Если файл
уже существует в момент сборки цели, файл не будет удалён.

Такое поведение может быть очень полезным при построении цепочки
правил. Например, большинство Java утилит принимают списки файлов в
стиле Windows. Создание правил для сохранения списков
файлов и спецификация их как промежуточных позволяет \GNUmake{}
автоматически удалять множество временных файлов.

\item[\target{.SECONDARY}] \hfill\\
Реквизиты этой специальной цели интерпретируются как промежуточные
файлы, которые не удаляются автоматически. Наиболее часто
\target{.SECONDARY} используется для пометки объектных файлов,
хранимых в виде библиотек. Обычно такие объекты будут удалены сразу
после добавления их в архив. Иногда удобнее не удалять объектные
файлы во время разработки.

\item[\target{.PRECIOUS}] \hfill\\
Когда \GNUmake{} аварийно завершает выполнение, файл цели может быть
удалён, если он изменился с момента старта \GNUmake{}. Таким образом
избегается возможность сохранения частично собранных (и, возможно,
повреждённых) файлов. Иногда вы можете захотеть от \GNUmake{} другого
поведения, например, если файл велик и требует много вычислений для
своего создания. Если вы укажете имя такого файла в качестве
реквизитов цели \target{.PRECIOUS}, \GNUmake{} не будет удалять этот
файл в случае аварийного завершения. Эта цель используется довольно
редко, но если её применение действительно необходимо, её наличие
сохранит разработчику много времени и сил. Заметим, что \GNUmake{} не
осуществляет автоматического удаления в случае ошибки в выполняемой
команде, только в случае получения сигнала останова.


\item[\target{.DELETE\_ON\_ERROR}] \hfill\\
Эта цель~--- противоположность \target{.PRECIOUS}. Спецификация файла
как реквизита этой цели означает, что файл должен быть удалён в случае
любой ошибки в сценарии сборки, ассоциированном с соответствующим
правилом. Обычно \GNUmake{} удаляет цель только в случае получении
сигнала останова.
\end{description}

Остальные специальные цели будут рассмотрены в момент их
непосредственного использования. Цели, относящиеся к параллельному
выполнению, будут рассмотрены в
главе~\ref{chap:improving_the_performance}, цель
\target{.EXPORT\_ALL\_VARIABLES}~--- в главе~\ref{chap:vars}.

%%--------------------------------------------------------------------
%% Automatic dependency generation
%%--------------------------------------------------------------------
\section{Автоматическое определение зависимостей}
\label{sec:auto_dep_gen}
Когда мы изменили нашу программу подсчёта слов так, чтобы часть
объявлений была описана в заголовочных файлах, мы, сами того не
замечая, добавили новую проблему. Мы описали зависимости между
объектными и заголовочными файлами в наш \Makefile{} самостоятельно. В
нашем случае сделать это было нетрудно, но в реальных программах (а не
в игрушечных примерах) это может быть весьма утомительным и
порождающим ошибки процессом. На самом деле, в большинстве программ
указание зависимостей практически невозможно, поскольку заголовочные
файлы могут включать другие заголовочные файлы, образуя сложное дерево
включений.
\index{Заголовочный файл}
Например, в моей системе один заголовочный файл \filename{stdio.h}
(наиболее часто используемый заголовочный файл стандартной библиотеки
языка \Clang{}) в общем счёте включает 15 других заголовочных файлов.
Разрешение подобных зависимостей вручную является практически
безнадёжным занятием. Однако неудавшаяся компиляция ведёт к часам
потраченного на отладку времени или, что ещё хуже, к проблемам в уже
выпущенном программном обеспечении. Что же нам делать?

К счастью, компьютеры весьма хорошо справляются с задачами поиска и
нахождения соответствий шаблону. Давайте используем программу для
определения зависимостей между исходными файлами, и даже записи этих
зависимостей в соответствии со стандартным синтаксисом \GNUmake{}. Как
вы, возможно, уже догадались, такая программа уже существует, по крайней
мере, для исходных файлов на \Clang{}/\Cplusplus{}.
\index{gcc}
\index{Опции!компилятора}
Компилятор \utility{gcc}, как и многие другие компиляторы, имеет опцию
для чтения исходных файлов и составления зависимостей для \GNUmake{}.
Например, так мы можем определить зависимости для \filename{stdio.h}

\begin{alltt}
\footnotesize
\$ \textbf{echo "#include <stdio.h>" > stdio.c}
\$ \textbf{gcc -M stdio.c}
stdio.o: stdio.c /usr/include/stdio.h /usr/include/\_ansi.h \textbackslash{}
/usr/include/newlib.h /usr/include/sys/config.h \textbackslash{}
/usr/include/machine/ieeefp.h /usr/include/cygwin/config.h \textbackslash{}
/usr/lib/gcc-lib/i686-pc-cygwin/3.2/include/stddef.h \textbackslash{}
/usr/lib/gcc-lib/i686-pc-cygwin/3.2/include/stdarg.h \textbackslash{}
/usr/include/sys/reent.h /usr/include/sys/\_types.h \textbackslash{}
/usr/include/sys/types.h /usr/include/machine/types.h \textbackslash{}
/usr/include/sys/features.h /usr/include/cygwin/types.h \textbackslash{}
/usr/include/sys/sysmacros.h /usr/include/stdint.h \textbackslash{}
/usr/include/sys/stdio.h
\end{alltt}

<<Отлично,>>~--- скажете вы,~---<<Теперь мне придётся запускать gcc,
открывать текстовый редактор и вставлять результаты работы компилятора
с ключом \command{-M} в свой \Makefile{}. Какой ужас.>>. И вы были бы
правы, если бы это была вся правда. Существует два стандартных способа
включения автоматически составленных зависимостей в \Makefile{}.
Первый, он же самый старый, заключается в добавлении комментария
наподобие следующего:

{\footnotesize
\begin{verbatim}
# Далее следуют автоматически составленные зависимости:
# НЕ РЕДАКТИРОВАТЬ
\end{verbatim}
}

{\flushleft
в конец \Makefile{}'а и написании сценария командного интерпретатора
для автоматического обновления этого раздела. Это, безусловно, гораздо
лучше ручного обновления, но всё ещё довольно неудобно. Второй метод
заключается в добавлении директивы include. Б\'{о}льшая часть версий
\GNUmake{} поддерживает эту директиву, и, безусловно, GNU \GNUmake{} в
их числе. Идея заключается в спецификации цели, с которой
ассоциированы действия по запуску \utility{gcc} с ключом \command{-M},
сохранении результатов в файле зависимостей и повторный запуск
\GNUmake{} с включением составленного файла зависимостей в основной
\Makefile{}. До появления GNU \GNUmake{} это делалось правилом
следующего вида:
}

{\footnotesize
\begin{verbatim}
depend: count\_words.c lexer.c counter.c
    $(CC) -M $(CPPFLAGS) $^ > $@
include depend
\end{verbatim}
}

Сначала вы запускаете \GNUmake{} с целью составить файл зависимостей,
и только после этого производите повторный пуск для сборки программы.
На момент появления этой возможности она выглядела неплохо, однако
часто люди добавляли или удаляли зависимости из исходного кода, забыв
заново составить файл зависимостей. Это становилось причиной
неправильной компиляции со всеми вытекающими неприятностями. GNU
\GNUmake{} решил эту неприятную проблему с помощью мощной
функциональности и довольно простого алгоритма. Рассмотрим сначала
алгоритм. Если мы составим для каждого исходного файла собственный
файл зависимостей, скажем, файл с расширением \filename{.d}, и добавим
этот файл в качестве цели к соответствующему правилу, то сможем
сообщить \GNUmake{}, что \filename{.d} файл нуждается в обновлении
(наряду с объектным файлом) при изменении исходного файла:

{\footnotesize
\begin{verbatim}
counter.o counter.d: src/counter.c include/counter.h include/lexer.h
\end{verbatim}
}

Составление этого правила может быть завершено шаблонным правилом и
довольно неуклюжим сценарием (взятым прямо из руководства по GNU
\GNUmake{}) \footnote{Этот довольно выразительный сценарий, по моему
мнению, всё же требует некоторого объяснения. Сначала мы используем
компилятор \Clang{} с опцией \command{-M} для создания временного
файла, содержащего список зависимостей цели. Имя временного файла
получается из названия цели \command{\${}@} и добавочного уникального
числового суффикса \command{.\${}\${}\${}\${}}. В командном
интерпретаторе \utility{sh} переменная \command{\${}\${}} содержит
идентификатор текущего запущенного процесса командного интерпретатора.
Поскольку этот идентификатор является уникальным, имя нашего
временного файла также получается уникальным.  Затем мы используем
\utility{sed} для добавления файла с расширением \filename{.d} в
качестве цели правила. Выражение \utility{sed} состоит из шаблона
поиска
\command{\textbackslash{}(\${}\textbackslash{})\textbackslash{}1.o[
:]*} и подстановки \command{\textbackslash{}1.o \${}@ :}, разделённых
запятыми. Шаблон поиска состоит из основы имени цели \command{\${}*},
заключенной в группу регулярного выражения
\command{\textbackslash{}(\textbackslash{})}, за которой следует
суффикс \command{.o}. После имени цели могут следовать пробелы или
двоеточия (\command{[ :]*}). Подстановка восстанавливает
первоначальную цель с помощью ссылки на первую группу регулярного
выражения с добавлением суффикса (\command{\textbackslash{}1.o}) и
добавляет файл зависимостей в качестве второй цели правила
(\command{\${}@}).}:

{\footnotesize
\begin{verbatim}
%.d: %.c
    $(CC) -M $(CPPFLAGS) $< > $@.$$$$;                  \
    sed 's,\($*\)\.o[ :]*,\1.o $@ : ,g' < $@.$$$$ > $@; \
    rm -f $@.$$$$
\end{verbatim}
}

Теперь рассмотрим вышеупомянутую функциональность. GNU \GNUmake{}
будет рассматривать каждый включаемый файл в качестве цели,
нуждающейся в обновлении.  Таким образом, когда мы будем упоминать
\filename{.d} файлы, \GNUmake{} автоматически попытается создать эти
файлы во время чтения \Makefile{}'а. Ниже представлен наш пример с
добавлением автоматического управления зависимостями:

{\footnotesize
\begin{verbatim}
VPATH    = src include
CPPFLAGS = -I include
SOURCES  = count_words.c \
           counter.c     \
           lexer.c 
count_words: counter.o lexer.o -lfl
count_words.o: counter.h
counter.o: counter.h lexer.h
lexer.o: lexer.h

include $(subst .c,.d,$(SOURCES))

%.d: %.c
    $(CC) -M $(CPPFLAGS) $< > $@.$$$$;                  \
    sed 's,\($*\)\.o[ :]*,\1.o $@ : ,g' < $@.$$$$ > $@; \
    rm -f $@.$$$$
\end{verbatim}
}

Директива включения должна появляться только после записанных вручную
правил, чтобы не подменить цель по умолчанию целью из включаемого
\index{Директивы!include@\directive{include}}
файла. Директива \directive{include} принимает в качестве аргумента
список файлов (чьи имена могут включать шаблоны). В предыдущем примере
\index{Функции!встроенные!substr@\function{substr}}
мы использовали встроенную функцию \GNUmake{} \function{substr} для
трансформации списка исходных файлов в список файлов зависимостей (мы
рассмотрим \function{substr} более подробно в разделе
<<\nameref{sec:str_func}>> главы~\ref{chap:functions}). Пока просто
примите к сведению, что мы используем эту функцию для замены строки
\filename{.c} на строку \filename{.d} в каждом слове списка
\variable{\${}(SOURCES)}.

\index{Опции!just-print@\command{-{}-just-print (-n)}}
Если теперь мы запустим \GNUmake{} с опцией
\command{-{}-just\hyp{}print}, то получим следующее:

\begin{alltt}
\footnotesize
\$ \textbf{make --just-print}
Makefile:13: count\_words.d: No such file or directory
Makefile:13: lexer.d: No such file or directory
Makefile:13: counter.d: No such file or directory
\verb#gcc -M -I include src/counter.c > counter.d.$$;       \#
\verb#sed 's,\(counter\)\.o[ :]*,\1.o counter.d : ,g'       \#
\verb#< counter.d.$$ > counter.d;                           \#
rm -f counter.d.\$\$
flex -t src/lexer.l > lexer.c
\verb#gcc -M -I include lexer.c > lexer.d.$$;           \#
\verb#sed 's,\(lexer\)\.o[ :]*,\1.o lexer.d : ,g'       \#
\verb#< lexer.d.$$ > lexer.d;                           \#
rm -f lexer.d.\$\$
\verb#gcc -M -I include src/count_words.c > count_words.d.$$; \#
\verb#sed 's,\(count_words\)\.o[ :]*,\1.o count_words.d : ,g' \#
\verb#< count_words.d.$$ count_words.d;                       \#
rm -f count\_words.d.\$\$
rm lexer.c
gcc -I include -c -o count\_words.o src/count\_words.c
gcc -I include -c -o counter.o src/counter.c
gcc -I include -c -o lexer.o lexer.c
gcc count\_words.o counter.o lexer.o /lib/libfl.a -o count\_words
\end{alltt}

Сначала \GNUmake{} выводит несколько предупреждений, с виду
напоминающих ошибки. Не стоит волноваться, это всего лишь
предупреждения. \GNUmake{} производит поиск файлов, указанных в
директиве \index{Директивы!sinclude@\directive{-include}}
\directive{include}, не находит их, и перед началом поиска правила для
создания этих файлов выводит предупреждение \command{No such file or
directory}. Эти предупреждения могут быть подавлены при помощи символа
\command{-}, добавленного перед директивой \directive{include}.
Следующие строки демонстрируют вызов \utility{gcc} с опцией
\command{-M} и запуск команды \utility{sed}.  Обратите внимание на то,
что \GNUmake{} должен вызвать \utility{flex} для создания
\filename{lexer.c}, удаляемый перед началом сборки цели по умолчанию.

\index{Автоматическое определение зависимостей}
Теперь у вас есть представление об автоматическом определении
зависимостей. Эта тема содержит ещё много интересных вопросов,
например, построение зависимостей для других языков программирования,
или вывод зависимостей в виде дерева. Мы вернёмся к этим темам во
второй части книги.

%%--------------------------------------------------------------------
%% Managing libraries
%%--------------------------------------------------------------------
\section{Управление библиотеками}
\label{sec:managing_libs}

\index{Библиотечный архив} \index{archive!library}
\index{Элемент архива} \index{archive!member}
\newword{Библиотечный архив} (\newword{archive library}), обычно
называемый просто библиотекой или архивом,~--- это специальный файл,
содержащий в себе другие файлы, именуемые \newword{элементами архива}
(\newword{archive members}). Например, стандартная библиотека языка
\Clang{} \filename{libc.a} содержит низкоуровневые функции. Библиотеки
используются настолько часто, что \GNUmake{} имеет специализированную
функциональность для создания, поддержки и компоновки архивов. Архивы
\index{Программы!ar@\utility{ar}}
создаются и модифицируются при помощи программы \utility{ar}.

Давайте вернёмся к нашему примеру. Мы можем модифицировать нашу
программу подсчёта слов, упаковав все её компоненты, пригодные для
повторного использования, в библиотеку.  Наша библиотека будет
состоять из двух файлов: \filename{counter.o} и \filename{lexer.o}.
Для создания библиотеки вызовем команду \filename{ar}:

{\footnotesize
\begin{alltt}
\$ \textbf{ar rv libcounter.a counter.o lexer.o}
a - counter.o
a - lexer.
\end{alltt}
}

Опции \command{rv} означают, что мы хотим заменить элементы библиотеки
указанными объектными файлами, и что \utility{ar} должен выводить отчёт
о своих действиях. Мы можем использовать действие замены даже в том
случае, если указанная библиотека не существует. Первым аргументом
после опций является имя библиотеки, за ним следуют имена объектных
файлов (некоторые версии \filename{ar} требуют опции \utility{c} в
случае, если библиотека ещё не существует, но GNU \utility{ar} не
требует этого). Две строки, следующие за вызовом команды \utility{ar},
являются отчётом о том, что объектные файлы были добавлены в
библиотеку.

Использование опции замены позволяет создавать и изменять архив
последовательно:

{\footnotesize
\begin{alltt}
\$ \textbf{ar rv libcounter.a counter.o}
r - counter.o
\$ \textbf{ar rv libcounter.a lexer.o}
r - lexer.o
\end{alltt}
}

Теперь \utility{ar} предваряет имена файлов символом "r". Это значит,
что файлы в архиве были заменены.

Библиотека может быть скомпонована в исполняемый файл несколькими
способами. Самый простой способ~--- просто указать имя библиотеки в
списке аргументов компилятора.  В свою очередь, компилятор или
компоновщик будут использовать расширение для определения типа каждого
из указанных в командной строке файлов:

{\footnotesize
\begin{verbatim}
cc count_words.o libcounter.a /lib/libfl.a -o count_words
\end{verbatim}
}

Компилятор \utility{cc} распознает два файла \filename{libcounter.a} и
\filename{/lib/libfl.a} как библиотеки и будет искать в них
недостающие символы.  Ещё одним способом ссылки на библиотеку является
опция \command{-l}:

{\footnotesize
\begin{verbatim}
cc count_words.o -lcounter -lfl -o count_words
\end{verbatim}
}

Как вы можете видеть, при использовании этой опции опускается префикс
и суффикс имени библиотеки. Опция \command{-l} делает командную строку
более компактной и удобочитаемой, однако, при использовании этой опции
вы получаете гораздо более весомое преимущество. Когда компилятор
\utility{cc} видит опцию \command{-l}, он \emph{ищет} библиотеку в
стандартных каталогах системных библиотек. Это избавляет программиста
от необходимости знать точный путь к файлу библиотеки и делает команду
компоновки более переносимой. К тому же, в системах, поддерживающих
разделяемые библиотеки (библиотеки с расширением \filename{.so} на
системах семейства \UNIX{}), компоновщик будет искать сначала
разделяемые библиотеки, и только если подходящей не обнаружено, будет
осуществлён поиск библиотечного архива. Такой подход позволяет
программам пользоваться преимуществами разделяемых библиотек без их
явной спецификации. Таково стандартное поведение компилятора и
компоновщика GNU. Старые компоновщики и компиляторы могут не
осуществлять такой оптимизации.

Список каталогов, в которых компилятор должен осуществлять поиск
библиотек, может быть изменён с помощью опции \command{-L},
указывающей список и порядок каталогов, в которых нужно искать
библиотеки. Эти каталоги будут добавлены в список прямо перед
системными каталогами библиотеки и будут использоваться для всех опций
\command{-l} в командной строке. На самом деле, компиляции в
предыдущем примере не завершится успехом, поскольку текущий каталог не
входит в список каталогов библиотек \filename{cc}. Мы можем решить эту
проблему добавлением текущего каталога в список как показано ниже:

{\footnotesize
\begin{verbatim}
cc count_words.o -L. -lcounter -lfl -o count_words
\end{verbatim}
}

Библиотеки вносят некоторые трудности в процесс сборки программ. Какие
возможности предоставляет \GNUmake{} для упрощения этого процесса? GNU
\GNUmake{} включает функциональность как по созданию библиотек, так и
использованию библиотек при компоновке. Давайте посмотрим, как это
работает.

%---------------------------------------------------------------------
% Creating and updating libraries
%---------------------------------------------------------------------
\subsection{Создаём и изменяем библиотеки}

Библиотеки фигурируют в \Makefile{}'е в качестве обычных файлов. Ниже
представлено простое правило для создания нашей библиотеки:

{\footnotesize
\begin{verbatim}
libcounter.a: counter.o lexer.o
    $(AR) $(ARFLAGS) $@ $^
\end{verbatim}
}

Это правило использует встроенные переменные \variable{AR} и
\variable{ARFLAGS}, содержащие имя программы \utility{ar} и
стандартные опции \command{rv} соответственно. Для спецификации файла
архива используется автоматическая переменная \variable{\$@}, а для
спецификации реквизитов~--- автоматическая переменная \variable{\$\^}.

Теперь, если вы укажите файл \filename{libcounter.a} в качестве
реквизита цели \target{count\_words}, \GNUmake{} обновит нашу
библиотеку перед компоновкой исполняемого файла. Обратите внимание на
одну деталь. \emph{Все} элементы архива будут замещены, даже если
среди них есть не изменявшиеся с момента последнего обновления архива
элементы. Чтобы не терять время впустую, мы можем написать более
подходящее правило:

{\footnotesize
\begin{verbatim}
libcounter.a: counter.o lexer.o
    $(AR) $(ARFLAGS) $@ $?
\end{verbatim}
}

Если вы используете \variable{\$?} вместо \variable{\$\^},
\GNUmake{} будет подставлять в список аргументов только те объектные
файлы, которые имеют более позднюю дату модификации, чем цель.

Можем ли мы ещё улучшить это правило? Может быть да, а может и нет.
\GNUmake{} имеет встроенную поддержку обновления отдельных файлов в
архиве, но прежде, чем мы вдадимся в эти детали, стоит сделать
несколько важных замечаний относительно такого подхода к работе с
библиотеками. Одна из основных задач \GNUmake{} состоит в том, чтобы
эффективно использовать время процессора и собирать только те файлы,
которые действительно в этом нуждаются. К сожалению, вызов
\utility{ar} для каждого элемента архива по отдельности при наличии
несколько десятков файлов занимает настолько много времени, что
перевешивает преимущество элегантного синтаксиса, рассмотренного
далее. Используя простой метод, представленный выше, мы можем вызвать
\utility{ar} один раз для всех изменившихся файлов и избежать
множества ненужных системных вызовов \filename{fork/exec}. Кроме того,
на многих системах использование ключа \command{r} при вызове
\utility{ar} очень неэффективно. На моём компьютере 1.9 GHz Pentium 4
создание большого архива, содержащего 14216 элементов общим размером
55 MB, занимает 4 минуты 24 секунды, в то время как замена одного
элемента в этом архиве требует 28 секунд. Таким образом, создание
архива заново будет более быстрой альтернативой замене элементов при
наличии более 10 (из 14216!) изменившихся файлов. В такой ситуации
более разумным подходом будет единовременное обновление архива с
использованием автоматической переменной \variable{\$?}. Для небольших
библиотек и более быстрых компьютеров нет причин отказываться от
элегантного подхода, описанного ниже, в пользу более простого, но и
более быстрого.

В GNU \GNUmake{} элемент архива может быть специфицирован при помощи
следующей нотации:

{\footnotesize
\begin{verbatim}
libgraphics.a(bitblt.o): bitblt.o
    $(AR) $(ARFLAGS) $@ $<
\end{verbatim}
}

Здесь \filename{libgraphics.a}~--- это имя библиотеки, а
\filename{bitblt.o} (сокращение от \newword{bit block transfer,
передача битовых блоков})~--- имя её элемента. Синтаксис
\filename{libgraphics.a(bitblt.o)} означает модуль, содержащийся в
библиотеке. Реквизитом для цели является сам объектный файл, а
командой~--- добавление этого файла в архив. Автоматическая переменная
\variable{\$<} используется для получения первого реквизита. На самом
деле существует встроенное шаблонное правило, предоставляющее в
точности ту же функциональность.

Когда мы соединим всё это воедино, наш \Makefile{} будет выглядеть
следующим образом:

{\footnotesize
\begin{verbatim}
VPATH    = src include
CPPFLAGS = -I include

count_words: libcounter.a /lib/libfl.a

libcounter.a: libcounter.a(lexer.o) libcounter.a(counter.o)

libcounter.a(lexer.o): lexer.o
    $(AR) $(ARFLAGS) $@ $<

libcounter.a(counter.o): counter.o
    $(AR) $(ARFLAGS) $@ $<

count_words.o: counter.h

counter.o: counter.h lexer.h

lexer.o: lexer.h
\end{verbatim}
}

При запуске \GNUmake{} выводит следующее:

{\footnotesize
\begin{alltt}
\$ \textbf{make}
gcc -I include -c -o count\_words.o src/count\_words.c
flex -t src/lexer.l > lexer.c
gcc -I include -c -o lexer.o lexer.c
ar rv libcounter.a lexer.o
ar: creating libcounter.a
a - lexer.o
gcc -I include -c -o counter.o src/counter.c
ar rv libcounter.a counter.o
a - counter.o
gcc count\_words.o libcounter.a /lib/libfl.a -o count\_words
rm lexer.c
\end{alltt}
}

Обратите внимание на правило обновления архива. Автоматическая
переменная \variable{\$@} приняла значение имени библиотеки, несмотря
на то, что имя цели в \Makefile{}'е было
\filename{libcounter.a(lexer.o)}.

Наконец, нужно отметить, что библиотечный архив включает индекс всех
символов, содержащихся в нём. Новые программы архиваторов, такие как
GNU \utility{ar}, обновляют этот индекс автоматически при добавлении в
архив нового символа. Более старые версии архиваторов могут этого не
делать. Для создания и обновления индекса архива используется
\index{Программы!runlib}
программа \utility{ranlib}. В системах со старой версией архиватора
должно использоваться правило следующего вида:

{\footnotesize
\begin{verbatim}
libcounter.a: libcounter.a(lexer.o) libcounter.a(counter.o)
    $(RANLIB) $@
\end{verbatim}
}

Вы также можете использовать альтернативный подход для больших
архивов:

{\footnotesize
\begin{verbatim}
libcounter.a: counter.o lexer.o
    $(RM) $@
    $(AR) $(ARFLGS) $@ $^
    $(RANLIB) $@
\end{verbatim}
}

Конечно, синтаксис управления элементами архива может использоваться с
применением встроенных правил. GNU \GNUmake{} содержит встроенные
правила обновления архивов. Если мы используем эти правила, наш
\Makefile{} будет выглядеть следующим образом:

{\footnotesize
\begin{verbatim}
VPATH    = src include
CPPFLAGS = -I include

count_words: libcounter.a -lfl

libcounter.a: libcounter.a(lexer.o) libcounter.a(counter.o)

count_words.o: counter.h

counter.o: counter.h lexer.h

lexer.o: lexer.h
\end{verbatim}
}

%---------------------------------------------------------------------
% Using libraries as prerequisites
%---------------------------------------------------------------------
\subsection{Использование библиотек в качестве реквизитов}

Когда библиотеки появляются в качестве реквизитов, они могут быть
обозначены с помощью расширения файла или опции \command{-l}. Если
указать имя файла библиотеки:

{\footnotesize
\begin{verbatim}
xpong: $(OBJECTS) /lib/X11/libX11.a /lib/X11/libXaw.a
    $(LINK) $^ -o $@
\end{verbatim}
}

{\noindent то компоновщик просто прочитает библиотечные файлы из
командой строки.  При использовании опции \command{-l} реквизиты вовсе
не выглядят обычными файлами:}

{\footnotesize
\begin{verbatim}
xpong: $(OBJECTS) -lX11 -lXaw
    $(LINK) $^ -o $@
\end{verbatim}
}

Когда в реквизитах используется форма \command{-l}, \GNUmake{}
производит поиск библиотеки (предпочитая разделяемую версию) и
подставляет абсолютный путь в переменные \variable{\$\^} и
\variable{\$?}. Одно из преимуществ такого подхода состоит в
возможности производить автоматический поиск библиотек даже в том
случае, если компоновщик в вашей системе не поддерживает такой
возможности. Другим преимуществом является возможность настройки
путей поиска \GNUmake{}, что позволяет вам производить поиск
собственных библиотек наравне с системными. В приведённом примере
первая форма (с использованием абсолютных путей) будет игнорировать
разделяемые библиотеки. При использовании же второй формы \GNUmake{}
будет знать, что разделяемые библиотеки более предпочтительны, поэтому
сначала произведёт поиск разделяемой версии \filename{X11}, и только в
случае неудачи будет выбрана статическая библиотека. Шаблоны
для распознавания имён библиотек хранятся в виде реквизитов специальной
\index{Цели!специальные!LIBPATTERNS@\target{.LIBPATTERNS}}
цели \target{.LIBPATTERNS} и могут быть настроены для различных
форматов имён библиотек.

К сожалению, есть одна неприятная мелочь. Если в какая-либо цель в
\Makefile{}'е специфицирует библиотеку, на неё нельзя ссылаться в
реквизитах с помощью опции \command{-l}.  Например, запуск \GNUmake{}
с таким \Makefile{}'ом:

{\footnotesize
\begin{verbatim}
count_words: count_words.o -lcounter -lfl
    $(CC) $^ -o $@ libcounter.a: libcounter.a(lexer.o)

libcounter.a(counter.o)
\end{verbatim}
}

{\noindent завершится неудачей со следующей ошибкой:}

{\footnotesize
\begin{verbatim}
No rule to make target `-lcounter', needed by `count_words'
\end{verbatim}
}

Причиной ошибки является то, что \GNUmake{} не совершил подстановку
\filename{libcounter.a} вместо \filename{-lcounter} и поиск подходящей
цели. Вместо этого был осуществлён обычный поиск библиотеки. Таким
образом, для библиотек, собранных в \GNUmake{}, должно указывается
непосредственно имя файла.

Компоновка больших программ без возникновения ошибок подобна искусству
чёрной магии. Компоновщик производит поиск библиотек в том порядке, в
каком они указаны в командной строке. Таким образом, если библиотека
\filename{A} содержит неопределённый символ, например, \textit{open},
определённый в библиотеке \filename{B}, то \filename{A} должна быть
указана в командной строке \emph{перед} \filename{B} (именно так,
\filename{A} требует \filename{B}). Иначе, когда компоновщик прочитает
\filename{A} и не найдёт определения символа \filename{open}, будет
слишком поздно возвращаться назад к \filename{B}. Компоновщик никогда не
осуществляет поиск в уже просмотренных библиотеках. Таким образом,
порядок появления библиотек в командной строке играет фундаментальное
значение.

Когда реквизиты цели сохраняются в переменных \variable{\$\^} и
\variable{\$?}, порядок их следования также сохраняется. Это
справедливо даже для реквизитов, размещённых в нескольких правилах. В
этом случае реквизиты каждого правила добавляются к списку реквизитов
в том порядке, в котором они появляются.

Родственной проблемой является проблема перекрёстных ссылок между
\index{Циклические ссылки}
библиотеками,также известных как \newword{циклические ссылки}
\index{Зацикливания}
(\newword{circular references}) или \newword{зацикливания}
(\newword{circularities}). Предположим, что после некоторой
модификации библиотека \filename{B} использует символ из \filename{A}.
Мы уже знаем, что \filename{A} должна быть указана до \filename{B}, но
теперь ещё и \filename{B} должно быть указана до \filename{A}.
Решением является ссылка на \filename{A} и до, и после ссылки на
\filename{B}: \command{-lA -lB -lA}. В больших и сложных программах
библиотеки часто должны повторяться подобным образом, иногда более
одного раза.

Такая ситуация ставит небольшую проблему при использовании \GNUmake{},
поскольку автоматические переменные, как правило, не содержат
дубликатов. Например, предположим, что нам нужно повторить библиотеку
в реквизитах для устранения циклических ссылок:

{\footnotesize
\begin{verbatim}
xpong: xpong.o libui.a libdynamics.a libui.a -lX11
    $(CC) $^ -o $@
\end{verbatim}
}

Этот список реквизитов после подстановки переменных будет выглядеть
следующим образом:

{\footnotesize
\begin{verbatim}
gcc xpong.o libui.a libdynamics.a /usr/lib/X11R6/libX11.a -o xpong
\end{verbatim}
}

Для подавления последствий такого поведения переменной \variable{\$\^}
в \GNUmake{} была добавлена переменная \variable{\$+}. Эта переменная
идентична \variable{\$\^} с той лишь разницей, что в списке реквизитов
сохраняются дубликаты. Используем \variable{\$+}:

{\footnotesize
\begin{verbatim}
xpong: xpong.o libui.a libdynamics.a libui.a -lX11
    $(CC) $+ -o $@
\end{verbatim}
}

Теперь список реквизитов породит следующую команду компоновки:

\begin{verbatim}
gcc xpong.o libui.a libdynamics.a libui.a \
/usr/lib/X11R6/libX11.a -o xpong
\end{verbatim}

%---------------------------------------------------------------------
% Double-colon rules
%---------------------------------------------------------------------
\subsection{Правила с двойным двоеточием}

\index{Правила!с двойным двоеточием}
Правила с двойным двоеточием~--- это реализация функциональности,
позволяющей собирать одну и ту же цель с помощью разных сценариев, в
зависимости от того, какое из подмножеств реквизитов было
модифицировано. Обычно если цель появляется более одного раза, все её
реквизиты соединяются в один список, сценарий сборки же для одной цели
может быть указан только один раз. При использовании же правил с
двойным двоеточием каждое появление цели рассматривается как отдельное
правило и обрабатывается индивидуально. Это значит, что для какой-то
определённой цели все правила должны быть одного типа: либо с одним
двоеточием, либо с двумя.

По-настоящему полезные применения этой возможности придумать довольно
сложно, поэтому давайте рассмотрим следующий искусственный пример:

{\footnotesize
\begin{verbatim}
file-list:: generate-list-script
    chmod +x $<
    generate-list-script $(files) > file-list

file-list:: $(files)
    generate-list-script $(files) > file-list
\end{verbatim}
}

Мы можем создать цель \target{file-list} двумя способами. Если
сценарий составления списка файлов изменился, то добавим файлу
сценария права на запуск и выполним его. Если изменились исходные
файлы, мы просто запускаем сценарий.  Несмотря на свою надуманность,
пример наглядно демонстрирует, как можно использовать эту
функциональность.

Мы рассмотрели большую часть функциональности \GNUmake{}, связанной с
правилами, которые, наряду с переменными и сценариями, составляют
самую сущность \GNUmake{}. Мы фокусировали внимание главным образом на
специфику синтаксиса и поведение различных возможностей, практически
не останавливаясь на способах их применения в более сложных ситуациях.
Это будет главным объектом нашего внимания во второй части книги. А
сейчас продолжим обсуждение переменных и команд.

