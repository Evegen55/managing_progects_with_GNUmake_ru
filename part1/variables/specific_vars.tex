%%--------------------------------------------------------------------
%% Target- and pattern-specific variables
%%--------------------------------------------------------------------
\section{Переменные, зависящие от цели или шаблона}

Обычно переменные принимают только одно значение во время выполнения
\GNUmake{}. Это гарантируется двухэтапной обработкой \Makefile{}'а.
На первом этапе \GNUmake{} читает \Makefile{}, производит присваивание
и вычисление переменных и строит граф зависимостей. На втором этапе
производится анализ графа зависимостей. Таким образом, когда происходит
выполнение команд, обработка переменных уже закончена. Предположим,
однако, что нам нужно переопределить значение переменной только для
какой-то цели или шаблона.

В приведённом ниже примере файл, который мы компилируем, требует
использования дополнительной опции \command{-DUSE\_NEW\_MALLOC=1},
которая не должна быть использована при компиляции других файлов:

{\footnotesize
\begin{verbatim}
gui.o: gui.h
    $(COMPILE.c) -DUSE_NEW_MALLOC=1 $(OUTPUT_OPTION) $<
\end{verbatim}
}

Проблема решена добавлением дубликата правила компиляции, включающего
необходимую опцию. Такой подход неудовлетворителен по нескольким
причинам. Во-первых, мы повторяем код. Если правило когда-нибудь
изменится, или если мы решим заменить встроенное правило собственным,
этот код потребует изменения, о котором легко можно забыть. Во-вторых,
если множество файлов требует специализированных опций, копирование и
вставка кусков кода быстро становится утомительным и рискованным
занятием (представьте, что у вас сотни таких файлов).

Для решения таких проблем \GNUmake{} предоставляет функциональность
\index{Переменные!зависящие от цели}
\newword{переменных, зависящих от цели}. Определения этих переменных
привязаны к цели и действительны только во время обработки цели или её
реквизитов. Используя эту функциональность, мы можем переписать наш
пример следующим образом:

{\footnotesize
\begin{verbatim}
gui.o: CPPFLAGS += -DUSE_NEW_MALLOC=1
gui.o: gui.h
$(COMPILE.c) $(OUTPUT_OPTION) $<
\end{verbatim}
}

Переменная \variable{CPPFLAGS} встроена в стандартное правило
компиляции исходных файлов \Clang{} и предназначена для опций
препроцессора \Clang{}. При помощи оператора \command{+=} мы добавляем
новую опцию к уже существующим. Теперь сценарий компиляции может
быть удалён полностью:

{\footnotesize
\begin{verbatim}
gui.o: CPPFLAGS += -DUSE_NEW_MALLOC=1
gui.o: gui.h
\end{verbatim}
}

Пока происходит сборка цели \target{gui.o}, значение переменной
\variable{CPPFLAGS} вдобавок к оригинальному значению будет содержать
строку \command{-DUSE\_NEW\_MALLOC=1}. Когда цель \filename{gui.o}
будет собрана, значение \variable{CPPFLAGS} будет восстановлено.

Общий синтаксис определения переменных, зависящих от цели, таков:

{\footnotesize
\begin{alltt}
\emph{цель ...: переменная}  = \emph{значение}
\emph{цель ...: переменная} := \emph{значение}
\emph{цель ...: переменная} += \emph{значение}
\emph{цель ...: переменная} ?= \emph{значение}
\end{alltt}
}

Как вы могли заметить, для определения таких переменных применимы все
формы оператора присваивания. Переменная не обязательно должна
существовать до присваивания.

Более того, присваивание переменным значения не осуществляется до
начала сборки цели. Таким образом, правая часть присваивания может
содержать ссылку на значение другой переменной, зависящей от этой
цели. Переменная также действительна во время сборки всех реквизитов.
