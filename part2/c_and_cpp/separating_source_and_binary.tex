%%--------------------------------------------------------------------
%% Separating source and binary
%%--------------------------------------------------------------------
\section{Разделение исходных и бинарных файлов}
\label{sec:separating_source_and_binary}

Итак, что же нам делать, если мы хотим поддерживать единственное
дерево каталогов с исходным кодом, но множество платформ и множество
сборок для каждой платформы, разделяя при необходимости деревья
каталогов исходных и бинарных файлов? Изначально программа \GNUmake{}
была написана для эффективной работы с файлами, находящимися в одном
каталоге. И хотя со времени своего создания она очень изменилась,
истоки забыты не были. Лучше всего \GNUmake{} работает с несколькими
каталогами, если модифицируемые им файлы находятся в текущем каталоге
или в его подкаталогах.

%---------------------------------------------------------------------
% The easy way
%---------------------------------------------------------------------
\subsection*{Простой способ}

Наиболее простой способ заставить \GNUmake{} помещать бинарные файлы в
отдельный каталог~--- это запустить \GNUmake{} из этого каталога.
Доступ к выходным файлам осуществляется через относительные пути, как
показано в предыдущей главе, в то время как исходные файлы должны быть
указаны явно или при помощи \directive{vpath}. В любом случае, нам
придётся ссылаться на каталог с исходными файлами из нескольких мест,
поэтому нам нужно завести переменную для хранения пути к нему:

{\footnotesize
\begin{verbatim}
SOURCE_DIR := ../mp3_player
\end{verbatim}
}

Возьмём за основу наш предыдущий \Makefile{}. Функция
\function{source-to-object} остаётся неизменной, а вот функцию
\function{subdirectory} нужно изменить так, чтобы она учитывала
относительные пути к исходным файлам.

{\footnotesize
\begin{verbatim}
# $(call source-to-object, source-file-list)
source-to-object = $(subst .c,.o,$(filter %.c,$1)) \
                   $(subst .y,.o,$(filter %.y,$1)) \
                   $(subst .l,.o,$(filter %.l,$1))
# $(subdirectory)
subdirectory = $(patsubst $(SOURCE_DIR)/%/module.mk,%, \
                 $(word                                \
                   $(words $(MAKEFILE_LIST)),$(MAKEFILE_LIST)))

\end{verbatim}
}

Файлы, перечисленные в переменной \variable{MAKEFILE\_LIST}, будут
включать относительные пути к исходным файлам. Таким образом, чтобы
получить относительный путь к каталогу модуля, нам нужно помимо
суффикса \filename{module.mk} отсечь и префикс.

Далее, чтобы помочь \GNUmake{} найти исходные файлы, мы используем
директиву \directive{vpath}:

{\footnotesize
\begin{verbatim}
vpath %.y $(SOURCE_DIR)
vpath %.l $(SOURCE_DIR)
vpath %.c $(SOURCE_DIR)
\end{verbatim}
}

Это позволит нам использовать для исходных файлов такие же простые
относительные пути, как и для выходных файлов. Когда \GNUmake{} будет
искать исходный файл и не сможет найти его в текущем каталоге, он
будет сканировать каталог \variable{SOURCE\_DIR}. Далее, нам нужно
обновить значение переменной \variable{include\_dirs}:

{\footnotesize
\begin{verbatim}
include_dirs := lib $(SOURCE_DIR)/lib $(SOURCE_DIR)/include
\end{verbatim}
}

В дополнение к каталогам исходных файлов это переменная включает
подкаталог \filename{lib} дерева бинарных файлов, ведь именно туда
будут попадать заголовочные файлы, составленные \utility{lex} и
\utility{yacc}.

Также нужно обновить директиву \directive{include} для получения
доступа к файлам \filename{module.mk}, поскольку \GNUmake{} не
использует директиву \directive{vpath} для поиска включаемых файлов:

{\footnotesize
\begin{verbatim}
include $(patsubst %,$(SOURCE_DIR)/%/module.mk,$(modules))
\end{verbatim}
}

Наконец, мы создаём каталоги для размещения выходных файлов:

{\footnotesize
\begin{verbatim}
create-output-directories :=                    \
    $(shell for f in $(modules);                \
            do                                  \
              $(TEST) -d $$f || $(MKDIR) $$f;   \
            done)
\end{verbatim}
}

Это присваивание создаёт пустую переменную, значение которой никогда
не используется. Это даёт гарантию, что необходимые каталоги будут
созданы до начала основной работы \GNUmake{}. Нам приходится самим
создавать каталоги, так как \utility{lex}, \utility{yacc} и
инструменты автоматической генерации зависимостей не сделают этого за
нас.

Ещё один способ убедиться в том, что необходимые каталоги созданы~---
добавить эти каталоги в качестве реквизитов файлов зависимостей (эти
файлы имеют расширение \filename{.d}). Однако это плохая идея, так как
каталог на самом деле не является реквизитом. Файлы \utility{yacc},
\utility{lex} или файлы зависимостей не зависят от \emph{содержимого}
каталога, и не должны создаваться заново только от того, что
временн\'{а}я метка каталога изменилась. На самом деле это будет
источником неэффективности в случае добавления или удаления файлов из
каталога, содержащего выходные файлы.

Изменения, которые нужно сделать в файле \filename{module.mk}, ещё
проще:

{\footnotesize
\begin{verbatim}
local_src := $(addprefix $(subdirectory)/,playlist.y scanner.l)

$(eval $(call make-library, $(subdirectory)/libdb.a, $(local_src)))

.SECONDARY: $(call generated-source, $(local_src))

$(subdirectory)/scanner.d: $(subdirectory)/playlist.d
\end{verbatim}
}

Эта версия не использует функцию \function{wildcard} для определения
исходных файлов. Пусть восстановление этого функционала станет
упражнением для читателя. В первоначальном варианте был небольшой
сбой. Когда я запустил этот \Makefile{} я обнаружил, что файл
зависимостей \filename{scanner.d} был создан до файла
\filename{playlist.h}, от которого он зависит. Эта зависимость не была
отражена в исходном \Makefile{}'е, однако по счастливой случайности
всё работало правильно. Правильное определение \emph{всех}
зависимостей является трудной задачей даже для небольших проектов.

Если предположить, что исходные файлы находятся в подкаталоге
\filename{mp3\_player}, то сборка проекта будет осуществляться
следующим образом:

{\footnotesize
\begin{alltt}
\$ \textbf{mkdir mp3\_player\_out}
\$ \textbf{cd mp3\_player\_out}
\$ \textbf{make --file=../mp3\_player/makefile}
\end{alltt}
}

Наш \Makefile{} правилен и хорошо работает, однако немного раздражает
то, что приходится изменять каталог и добавлять опцию
\command{-{}-file} (\command{-f}). Эту проблему можно решить с помощью
простого сценария:

{\footnotesize
\begin{verbatim}
#! /bin/bash
if [[ ! -d $OUTPUT_DIR ]]
then
  if ! mkdir -p $OUTPUT_DIR
  then
    echo "Cannot create output directory" > /dev/stderr
  exit 1
  fi
fi

cd $OUTPUT_DIR
make --file=$SOURCE_DIR/makefile "$@"
\end{verbatim}
}

Этот сценарий подразумевает, что каталог с исходными файлами и каталог
выходных файлов хранятся в переменных окружения \variable{SOURCE\_DIR}
и \variable{OUTPUT\_DIR} соответственно. Это является обычной
практикой, позволяющей разработчикам легко переключать деревья
каталогов, не печатая при этом пути слишком часто.

Наш \Makefile{} не содержит никаких средств, позволяющих запретить
разработчикам выполнять \Makefile{} из каталога с исходными файлами.
Это частая ошибка, и некоторые командные сценарии могут вести себя
неправильно. Например, цель \target{clean}:

{\footnotesize
\begin{verbatim}
.PHONY: clean
clean:
    $(RM) -r *
\end{verbatim}
}

{\flushleft удалит всё дерево каталогов с исходным кодом пользователя!
Благоразумным решением является добавление соответствующей проверки
в самом начале выполнения \Makefile{}'а. Ниже приведён возможный
вариант такой проверки:}

{\footnotesize
\begin{verbatim}
$(if $(filter $(notdir $(SOURCE_DIR)),$(notdir $(CURDIR))),\
  $(error Пожалуйста, запустите makefile из бинарного дерева каталогов.))
\end{verbatim}
}

Этот код проверяет совпадение имени текущего рабочего каталога
\command{(\$(notdir \$(CURDIR)))} с именем каталога, содержащего
исходный код: \command{(\$(notdir \$(SOURCE\_DIR)))}. Если эти два
выражения совпадают, выводится сообщение об ошибке, и выполнение
\GNUmake{} прекращается. Поскольку результатом вычисления функций
\function{if} и \function{error} является пустая строка, мы можем
поместить эти две строчки кода сразу после определения переменной
\variable{SOURCE\_DIR}.

%---------------------------------------------------------------------
% The hard way
%---------------------------------------------------------------------
\subsection*{Сложный способ}

Некоторые разработчики находят необходимость перемещения в дерево
каталогов, содержащих бинарные файлы, настолько раздражающей, что
готовы проделать огромную работу для того, чтобы этого избежать. Или,
возможно, разработчик \Makefile{}'а использует среду, в которой
использование сценариев\hyp{}обёрток или псевдонимов программ
недопустимо.  В любом случае, можно изменить \Makefile{} таким
образом, чтобы была возможность производить запуск \GNUmake{} из
дерева каталогов, содержащих исходный код, и помещать бинарные файлы в
другое дерево каталогов с помощью добавления к путям выходных файлов
соответствующих префиксов. Обычно в этом случае я использую абсолютные
пути, поскольку такой подход предоставляет больше гибкости, хотя и
обостряет проблему конечности длины командной строки. Для исходных
файлов используются простые относительные пути (они вычисляются
относительно каталога, в котором располагается \Makefile{}).

Следующий пример содержит модифицированный \Makefile{}, позволяющий
запускать \GNUmake{} из каталога с исходным кодом и записывающий
бинарные файлы в отдельное дерево каталогов:

{\footnotesize
\begin{verbatim}
SOURCE_DIR := /test/book/examples/ch07-separate-binaries-1
BINARY_DIR := /test/book/out/mp3_player_out

# $(call source-dir-to-binary-dir, directory-list)
source-dir-to-binary-dir = $(addprefix $(BINARY_DIR)/, $1)

# $(call source-to-object, source-file-list)
source-to-object = $(call source-dir-to-binary-dir,  \
                     $(subst .c,.o,$(filter %.c,$1)) \
                     $(subst .y,.o,$(filter %.y,$1)) \
                     $(subst .l,.o,$(filter %.l,$1)))
# $(subdirectory)
subdirectory = $(patsubst %/module.mk,%,  \
                 $(word                   \
                   $(words $(MAKEFILE_LIST)),$(MAKEFILE_LIST)))

# $(call make-library, library-name, source-file-list)
define make-library
  libraries += $(BINARY_DIR)/$1
  sources   += $2
  $(BINARY_DIR)/$1: $(call source-dir-to-binary-dir,  \
                      $(subst .c,.o,$(filter %.c,$2)) \
                      $(subst .y,.o,$(filter %.y,$2)) \
                      $(subst .l,.o,$(filter %.l,$2)))
      $(AR) $(ARFLAGS) $$@ $$^
endef

# $(call generated-source, source-file-list)
generated-source = $(call source-dir-to-binary-dir,   \
                     $(subst .y,.c,$(filter %.y,$1))  \
                     $(subst .y,.h,$(filter %.y,$1))  \
                     $(subst .l,.c,$(filter %.l,$1))) \
                   $(filter %.c,$1)

# $(compile-rules)
define compile-rules
  $(foreach f, $(local_src),\
  $(call one-compile-rule,$(call source-to-object,$f),$f))
endef

# $(call one-compile-rule, binary-file, source-files)
define one-compile-rule
  $1: $(call generated-source,$2)
      $(COMPILE.c) -o $$@ $$<

  $(subst .o,.d,$1): $(call generated-source,$2)
      $(CC) $(CFLAGS) $(CPPFLAGS) $(TARGET_ARCH) -M $$< | \
      $(SED) 's,\($$(notdir $$*)\.o\) *:,$$(dir $$@)\1 $$@: ,' > $$@.tmp
      $(MV) $$@.tmp $$@
endef

modules      := lib/codec lib/db lib/ui app/player
programs     :=
libraries    :=
sources      :=

objects      = $(call source-to-object,$(sources))
dependencies = $(subst .o,.d,$(objects))

include_dirs := $(BINARY_DIR)/lib lib include
CPPFLAGS     += $(addprefix -I ,$(include_dirs))
vpath %.h $(include_dirs)

MKDIR := mkdir -p
MV    := mv -f
RM    := rm -f
SED   := sed
TEST  := test

create-output-directories :=                                      \
    $(shell for f in $(call source-dir-to-binary-dir,$(modules)); \
            do                                                    \
              $(TEST) -d $$f || $(MKDIR) $$f;                     \
            done)

all:

include $(addsuffix /module.mk,$(modules))

.PHONY: all
all: $(programs)

.PHONY: libraries
libraries: $(libraries)

.PHONY: clean
clean:
    $(RM) -r $(BINARY_DIR)

ifneq "$(MAKECMDGOALS)" "clean"
include $(dependencies)
endif
\end{verbatim}
}

В этой версии функция \function{source-to-object} модифицирована так,
чтобы осуществлять исправление путей к бинарным файлам. Поскольку эта
операция осуществляется несколько раз, реализуем её в виде функции:

{\footnotesize
\begin{verbatim}
SOURCE_DIR := /test/book/examples/ch07-separate-binaries-1
BINARY_DIR := /test/book/out/mp3_player_out

# $(call source-dir-to-binary-dir, directory-list)
source-dir-to-binary-dir = $(addprefix $(BINARY_DIR)/, $1)

# $(call source-to-object, source-file-list)
source-to-object = $(call source-dir-to-binary-dir,  \
                     $(subst .c,.o,$(filter %.c,$1)) \
                     $(subst .y,.o,$(filter %.y,$1)) \
                     $(subst .l,.o,$(filter %.l,$1)))
\end{verbatim}
}

Функция \function{make-library} изменена подобным образом, теперь она
добавляет к выходным файлам префикс \variable{BINARY\_DIR}. Теперь мы
используем предыдущую функции \function{subdirectory}, поскольку путь
к каталогам с включаемыми файлами снова стал простым относительным
путём. Есть одна небольшая загвоздка: ошибка в \GNUmake{} 3.80
препятствует вызову функции \function{source-to-object} из новой
версии функции \function{make-library}. Эта ошибка была исправлена в
версии 3.81. Мы можем обойти эту ошибку, подставив вместо вызова
функции \function{source-to-object} её тело.

Рассмотрим теперь по-настоящему неуклюжую часть. Неявные правила не
работают, если доступ к выходному файлу нельзя осуществить с помощью
относительного пути. Например, основное правило компиляции
\command{\%.o: \%.c} отлично работает, когда оба файла находятся в
одном каталоге, или даже если исходный файл располагается в
каком-нибудь подкаталоге, например, \filename{lib/codec/codec.c}.
Если исходный файл располагается в удалённом каталоге, мы можем
указать \GNUmake{} на необходимость его поиска при помощи директивы
\directive{vpath}. Однако если объектный файл располагается в
удалённом каталоге, \GNUmake{} не сможет определить местоположения
этого файла, и цепочка правил будет нарушена. Единственный способ
информировать \GNUmake{} о расположении выходного файла~--- это
предоставить явное правило, соединяющее исходный и объектный файлы:

{\footnotesize
\begin{verbatim}
$(BINARY_DIR)/lib/codec/codec.o: lib/codec/codec.c
\end{verbatim}
}

Такая операция должна быть осуществлена для каждого объектного файла.

Хуже того, эта пара не соответствует неявному правилу \command{\%.o:
\%.c}. Это значит, что нам самим нужно предоставить командный
сценарий, повторяющий правило встроенной базы правил, и, возможно,
повторить этот сценарий много раз. Проблема также касается правила
автоматического составления файлов зависимостей, которое мы
используем. Добавление двух явных правил для каждого объектного файла,
упоминающегося в \Makefile{}'е,~--- это кошмар для человека, который
будет поддерживать программный продукт. Однако мы можем минимизировать
дублирование кода, написав функцию для автоматического составления
этих правил:

{\footnotesize
\begin{verbatim}
# $(call one-compile-rule, binary-file, source-files)
define one-compile-rule
  $1: $(call generated-source,$2)
      $(COMPILE.c) $$@ $$<

  $(subst .o,.d,$1): $(call generated-source,$2)
      $(CC) $(CFLAGS) $(CPPFLAGS) $(TARGET_ARCH) -M $$< | \
      $(SED) 's,\($$(notdir $$*)\.o\) *:,$$(dir $$@)\1 $$@: ,' > $$@.tmp
      $(MV) $$@.tmp $$@
endef
\end{verbatim}
}

Первые две строки функции~--- это явные правила для описания
зависимости объектного файла от исходного. Реквизит для этого правила
должен быть вычислен при помощи функции \function{generated-source},
описанной в главе~\ref{chap:managing_large_proj}. Если этого не
сделать, исходные файлы \utility{lex} и \utility{yacc} вызовут ошибку
компиляции при появлении в командном сценарии (после подстановки
переменной \variable{\$\^}). Автоматические переменные экранированы,
поэтому они будут вычислены позднее, при выполнении командного
сценария (а не при выполнении функции \function{eval}). Функция
\function{generated-source} была модифицирована так, чтобы возвращать
пути к исходным файлам, написанным на языке \Clang{}, неизменными:

{\footnotesize
\begin{verbatim}
# $(call generated-source, source-file-list)
generated-source = $(call source-dir-to-binary-dir,   \
                     $(subst .y,.c,$(filter %.y,$1))  \
                     $(subst .y,.h,$(filter %.y,$1))  \
                     $(subst .l,.c,$(filter %.l,$1))) \
                   $(filter %.c,$1)
\end{verbatim}
}

С учётом этого изменения функция будет возвращать результаты,
представленные в следующей таблице:

\begin{center}
\begin{tabular}{ll}
Аргумент & Результат \\
\filename{lib/db/playlist.y} &
\filename{/c/mp3\_player\_out/lib/db/playlist.c} \\
 & \filename{/c/mp3\_player\_out/lib/db/playlist.h} \\
\filename{lib/db/scanner.l} &
\filename{/c/mp3\_player\_out/lib/db/scanner.c} \\
\filename{app/player/play\_mp3.c} &
\filename{app/player/play\_mp3.c} \\
\end{tabular}
\end{center}

Явное правило для создания файлов зависимостей устроено точно так же.
Ещё раз обратите внимание на экранирование (двойные символы доллара),
которое требуется для правильной работы сценария.

Теперь для каждого исходного файла в модуле нужно вычислить нашу новую
функцию:

{\footnotesize
\begin{verbatim}
# $(compile-rules)
define compile-rules
  $(foreach f, $(local_src),\
    $(call one-compile-rule,$(call source-to-object,$f),$f))
endef
\end{verbatim}
}

Эта функция использует глобальную переменную \variable{local\_src},
переопределяемую в файлах \filename{module.mk}. Более общий подход
заключается в передаче списка файлов в качестве аргумента, однако в
нашем проекте это делать не обязательно. Просто добавим эти функции в
файлы \filename{module.mk}:

{\footnotesize
\begin{verbatim}
local_src := $(subdirectory)/codec.c

$(eval $(call make-library,$(subdirectory)/libcodec.a,$(local_src)))

$(eval $(compile-rules))
\end{verbatim}
}

Мы должны использовать функцию \function{eval}, так как результат
вычисления функции \function{compile-rules} содержит более одной
строки кода \GNUmake{}.

Наконец, последняя сложность. Так как стандартные шаблонные правила
компиляции исходных файлов \Clang{} не могут определить путь к
выходным файлам, неявное правило для \utility{lex} и наше шаблонное
правило для \utility{yacc} также не смогут этого сделать. Мы можем
легко изменить эти правила самостоятельно. Поскольку они больше не
применимы к остальным \utility{lex} и \utility{yacc} файлам, мы можем
вынести их в файл \filename{lib/db/module.mk}:

{\footnotesize
\begin{verbatim}
local_dir := $(BINARY_DIR)/$(subdirectory)
local_src := $(addprefix $(subdirectory)/,playlist.y scanner.l)

$(eval $(call make-library,$(subdirectory)/libdb.a,$(local_src)))

$(eval $(compile-rules))

.SECONDARY: $(call generated-source, $(local_src))

$(local_dir)/scanner.d: $(local_dir)/playlist.d

$(local_dir)/%.c $(local_dir)/%.h: $(subdirectory)/%.y
    $(YACC.y) --defines $<
    $(MV) y.tab.c $(dir $@)$*.c
    $(MV) y.tab.h $(dir $@)$*.h

$(local_dir)/scanner.c: $(subdirectory)/scanner.l
    @$(RM) $@
    $(LEX.l) $< > $@
\end{verbatim}
}

Правила для \utility{lex} файлов реализованы как обычные явные
правила, а правила для \utility{yacc} файлов~--- как шаблонные.
Почему? Потому что правила для \utility{yacc} файлов используются для
сборки двух целей: исходного и заголовочного файла на языке \Clang{}.
Если мы используем обычное явное правило, \GNUmake{} выполнит
командный сценарий дважды: один раз для исходного файла, а второй~---
для заголовочного. Однако \GNUmake{} считает, что шаблонное правило,
определяющее несколько целей, при выполнении обновляет обе цели.  Если
возможно, вместо \Makefile{}'ов, приведённых в этом разделе, я буду
использовать более простой подход, основанный на компиляции из дерева
каталогов бинарных файлов. Как вы могли заметить, при попытке
компиляции из дерева исходных файлов немедленно возникают трудности, и
с ростом размеров проекта эти трудности только растут.
