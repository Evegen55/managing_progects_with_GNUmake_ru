%%--------------------------------------------------------------------
%% The Book Makefile
%%--------------------------------------------------------------------
\section{\Makefile{} этой книги}

Написание книги о программировании само по себе является интересным
упражнением в построении систем сборки. Текст книги состоит из
множества файлов, каждый из которых требует различной
обработки. Примеры являются реальными программами, каждая из которых
должна быть запущена, а их вывод нуждается в сборе, обработке и
включении в основной текст (благодаря этому вывод не нужно копировать
и вставлять в текст, что избавляет от риска внесения ошибок). В
процессе написания полезно иметь возможность просмотреть текст в
различных форматах. Наконец, доставка материала требует
архивирования. Разумеется, все шаги должны быть воспроизводимыми и
относительно простыми в поддержке.

Выглядит как работа для \GNUmake{}! Возможность применения для
удивительно разнородных потребностей ~--- одна из самых замечательных
особенностей \GNUmake{}. Эта книга была написана в формате DocBook
(т.е. XML). \GNUmake{} ~--- стандартный инструмент при работе с
\TeX{}, \LaTeX{} и \command{troff}.

Следующий пример содержит полный \Makefile{} книги. В нём примерно 440
строк. \Makefile{} разделяется на следующие базовые задачи:

\begin{itemize}
  \item{} Управление примерами.
  \item{} Обработка XML.
  \item{} Генерация документов различных форматов.
  \item{} Проверка исходного кода.
  \item{} Базовые задачи поддержки.
\end{itemize}

\footnotesize{
\begin{verbatim}
# Сборка книги.
#
# Основные цели этого файла:
#
# show_pdf  Генерация pdf и запуск программы просмотра
# pdf       Генерация pdf
# print     Печать pdf
# show_html Генерация html и запуск программы просмотра
# html      Генерация html
# xml       Генерация xml
# release   Создание архива релиза
# clean     Удаление файлов, созданных в процессе сборки
#

BOOK_DIR     := /test/book
SOURCE_DIR   := text
OUTPUT_DIR   := out
EXAMPLES_DIR := examples

QUIET = @

SHELL       =  shell
AWK         := awk
CP          := cp
EGREP       := egrep
HTML_VIEWER := cygstart
KILL        := /bin/kill
M4          := m4
MV          := mv
PDF_VIEWER  := cygstart
RM          := rm -f
MKDIR       := mkdir -p
LNDIR       := lndir
SED         := sed
SORT        := sort
TOUCH       := touch
XMLTO       := xmlto
XMLTO_FLAGS =  -o $(OUTPUT_DIR) $(XML_VERBOSE)
process-pgm := bin/process-include
make-depend := bin/make-depend

m4-macros := text/macros.m4

# $(call process-includes, input-file, output-file)
# Осуществляет удаление символов табуляции, подстановку
# макросов и обработку директив включения.
define process-includes
  expand $1 |                                             \
  $(M4) --prefix-builtins --include=text $(m4-macros) - | \
  $(process-pgm) > $2
endef

# $(call file-exists, file-name)
# Возвращает ненулевое значение в случае существования
# файла с заданным именем
file-exists = $(wildcard $1)

# $(call maybe-mkdir, directory-name-opt)
# Создаёт каталог, если он ещё не существует.
# Если параметр directory-name-opt опущен, в качестве имени
# каталога используется значение $@.
maybe-mkdir = $(if $(call file-exists,        \
                     $(if $1,$1,$(dir $@))),, \
                $(MKDIR) $(if $1,$1,$(dir $@)))

# $(kill-acroread)
# Завершает процесс Acrobat Reader
define kill-acroread
  $(QUIET) ps -W |                                 \
  $(AWK) 'BEGIN { FIELDWIDTHS = "9 47 100" }       \
          /AcroRd32/ {                             \
                       print "Killing " $$3;       \
                       system( "$(KILL) -f " $$1 ) \
                     }'
endef

# $(call source-to-output, file-name)
# Преобразует имя исходного файла в имя выходного файла.
define source-to-output
$(subst $(SOURCE_DIR),$(OUTPUT_DIR),$1)
endef

# $(call run-script-example, script-name, output-file)
# Запускает makefile примера.
define run-script-example
  ( cd $(dir $1);                                   \
    $(notdir $1) 2>&1 |                             \
    if $(EGREP) --silent '\$$\(MAKE\)' [mM]akefile; \
    then                                            \
      $(SED) -e 's/^++*/$$/';                       \
    else                                            \
      $(SED) -e 's/^++*/$$/'                        \
             -e '/ing directory /d'                 \
             -e 's/\[[0-9]\]//';                    \
    fi )                                            \
  > $(TMP)/out.$$$$ &                               \
  $(MV) $(TMP)/out.$$$$ $2
endef

# $(call generic-program-example,example-directory)
# Создаёт общие правила сборки примера
define generic-program-example
  $(eval $1_dir      := $(OUTPUT_DIR)/$1)
  $(eval $1_make_out := $($1_dir)/make.out)
  $(eval $1_run_out  := $($1_dir)/run.out)
  $(eval $1_clean    := $($1_dir)/clean)
  $(eval $1_run_make := $($1_dir)/run-make)
  $(eval $1_run_run  := $($1_dir)/run-run)
  $(eval $1_sources  := $(filter-out %/CVS, \
                          $(wildcard $(EXAMPLES_DIR)/$1/*)))
  $($1_run_out): $($1_make_out) $($1_run_run)
      $$(call run-script-example, $($1_run_run), $$@)

  $($1_make_out): $($1_clean) $($1_run_make)
      $$(call run-script-example, $($1_run_make), $$@)

  $($1_clean): $($1_sources) Makefile
      $(RM) -r $($1_dir)
      $(MKDIR) $($1_dir)
      $(LNDIR) -silent ../../$(EXAMPLES_DIR)/$1 $($1_dir)
      $(TOUCH) $$@

  $($1_run_make):
      printf "#! /bin/bash -x\nmake\n" > $$@
endef

# Конечные форматы книги
BOOK_XML_OUT     := $(OUTPUT_DIR)/book.xml
BOOK_HTML_OUT    := $(subst xml,html,$(BOOK_XML_OUT))
BOOK_FO_OUT      := $(subst xml,fo,$(BOOK_XML_OUT))
BOOK_PDF_OUT     := $(subst xml,pdf,$(BOOK_XML_OUT))
ALL_XML_SRC      := $(wildcard $(SOURCE_DIR)/*.xml)
ALL_XML_OUT      := $(call source-to-output,$(ALL_XML_SRC))
DEPENDENCY_FILES := $(call source-to-output,\
                           $(subst .xml,.d,$(ALL_XML_SRC)))
# xml/html/pdf - Производит желаемые конечный формат книги.
.PHONY: xml html pdf
xml:  $(OUTPUT_DIR)/validate
html: $(BOOK_HTML_OUT)
pdf:  $(BOOK_PDF_OUT)

# show_pdf - Создаёт pdf документ и отображает его.
.PHONY: show_pdf show_html print
show_pdf: $(BOOK_PDF_OUT)
    $(kill-acroread)
    $(PDF_VIEWER) $(BOOK_PDF_OUT)

# show_html - Создаёт html файл и отображает его.
show_html: $(BOOK_HTML_OUT)
    $(HTML_VIEWER) $(BOOK_HTML_OUT)

# print - Печатает заданные страницы книги.
print: $(BOOK_FO_OUT)
    $(kill-acroread)
    java -Dstart=15 -Dend=15 $(FOP) $< -print > /dev/null

# $(BOOK_PDF_OUT) - Создаёт pdf файл.
$(BOOK_PDF_OUT): $(BOOK_FO_OUT) Makefile

# $(BOOK_HTML_OUT) - Создаёт html файл.
$(BOOK_HTML_OUT): $(ALL_XML_OUT) $(OUTPUT_DIR)/validate Makefile

# $(BOOK_FO_OUT) - Создаёт временный файл в формате fo.
.INTERMEDIATE: $(BOOK_FO_OUT)
    $(BOOK_FO_OUT): $(ALL_XML_OUT) $(OUTPUT_DIR)/validate Makefile

# $(BOOK_XML_OUT) - Обрабатывает все входные xml файлы.
$(BOOK_XML_OUT): Makefile

#################################################################
# Поддержка FOP
#
FOP := org.apache.fop.apps.Fop

# DEBUG_FOP - Определите этот макрос для просмотра вывода fop.
ifndef DEBUG_FOP
  FOP_FLAGS := -q
  FOP_OUTPUT := | $(SED) -e '/not implemented/d'       \
                         -e '/relative-align/d'        \
                         -e '/xsl-footnote-separator/d'
endif

# CLASSPATH - Compute the appropriate CLASSPATH for fop.
export CLASSPATH
CLASSPATH = $(patsubst %;,%,                                \
              $(subst ; ,;,                                 \
                 $(addprefix c:/usr/xslt-process-2.2/java/, \
                  $(addsuffix .jar;,                        \
                    xalan                                   \
                    xercesImpl                              \
                    batik                                   \
                    fop                                     \
                    jimi-1.0                                \
                    avalon-framework-cvs-20020315))))

# %.pdf - Шаблонное правило создания pdf из fo.
%.pdf: %.fo
    $(kill-acroread)
    java -Xmx128M $(FOP) $(FOP_FLAGS) $< $@ $(FOP_OUTPUT)

# %.fo - Шаблонное правило для создания fo из xml.
PAPER_SIZE := letter
%.fo: %.xml
    XSLT_FLAGS="--stringparam paper.type $(PAPER_SIZE)" \
    $(XMLTO) $(XMLTO_FLAGS) fo $<

# %.html - Шаблонное правило для создания html из xml.
%.html: %.xml
    $(XMLTO) $(XMLTO_FLAGS) html-nochunks $<

# fop_help - Отображение спавки по использованию fop.
.PHONY: fop_help
fop_help:
    -java org.apache.fop.apps.Fop -help
    -java org.apache.fop.apps.Fop -print help

#################################################################
# release - Создаёт релиз книги
#
RELEASE_TAR   := mpwm-$(shell date +%F).tar.gz
RELEASE_FILES := README Makefile *.pdf bin examples out text
.PHONY: release
release: $(BOOK_PDF_OUT)
    ln -sf $(BOOK_PDF_OUT) .
    tar --create                 \
        --gzip                   \
        --file=$(RELEASE_TAR)    \
        --exclude=CVS            \
        --exclude=semantic.cache \
        --exclude=*~             \
        $(RELEASE_FILES)
    ls -l $(RELEASE_TAR)

#################################################################
# Правила для примеров из первой главы.
#
# Все каталоги с примерами.
EXAMPLES :=
    ch01-bogus-tab
    ch01-cw1
    ch01-hello
    ch01-cw2
    ch01-cw2a
    ch02-cw3
    ch02-cw4
    ch02-cw4a
    ch02-cw5
    ch02-cw5a
    ch02-cw5b
    ch02-cw6
    ch02-make-clean
    ch03-assert-not-null
    ch03-debug-trace
    ch03-debug-trace-1
    ch03-debug-trace-2
    ch03-filter-failure
    ch03-find-program-1
    ch03-find-program-2
    ch03-findstring-1
    ch03-grep
    ch03-include
    ch03-invalid-variable
    ch03-kill-acroread
    ch03-kill-program
    ch03-letters
    ch03-program-variables-1
    ch03-program-variables-2
    ch03-program-variables-3
    ch03-program-variables-5
    ch03-scoping-issue
    ch03-shell
    ch03-trailing-space
    ch04-extent
    ch04-for-loop-1
    ch04-for-loop-2
    ch04-for-loop-3
    ch06-simple
    appb-defstruct
    appb-arithmetic

# Я бы с радостью использовал этот цикл foreach, но ошибка
# в версии 3.80 приводит к аварийному останову.
#$(foreach e,$(EXAMPLES),$(eval $(call generic-program-example,$e)))

# Вместо этого приходитя раскрывать цикл вручную:
$(eval $(call generic-program-example,ch01-bogus-tab))
$(eval $(call generic-program-example,ch01-cw1))
$(eval $(call generic-program-example,ch01-hello))
$(eval $(call generic-program-example,ch01-cw2))
$(eval $(call generic-program-example,ch01-cw2a))
$(eval $(call generic-program-example,ch02-cw3))
$(eval $(call generic-program-example,ch02-cw4))
$(eval $(call generic-program-example,ch02-cw4a))
$(eval $(call generic-program-example,ch02-cw5))
$(eval $(call generic-program-example,ch02-cw5a))
$(eval $(call generic-program-example,ch02-cw5b))
$(eval $(call generic-program-example,ch02-cw6))
$(eval $(call generic-program-example,ch02-make-clean))
$(eval $(call generic-program-example,ch03-assert-not-null))
$(eval $(call generic-program-example,ch03-debug-trace))
$(eval $(call generic-program-example,ch03-debug-trace-1))
$(eval $(call generic-program-example,ch03-debug-trace-2))
$(eval $(call generic-program-example,ch03-filter-failure))
$(eval $(call generic-program-example,ch03-find-program-1))
$(eval $(call generic-program-example,ch03-find-program-2))
$(eval $(call generic-program-example,ch03-findstring-1))
$(eval $(call generic-program-example,ch03-grep))
$(eval $(call generic-program-example,ch03-include))
$(eval $(call generic-program-example,ch03-invalid-variable))
$(eval $(call generic-program-example,ch03-kill-acroread))
$(eval $(call generic-program-example,ch03-kill-program))
$(eval $(call generic-program-example,ch03-letters))
$(eval $(call generic-program-example,ch03-program-variables-1))
$(eval $(call generic-program-example,ch03-program-variables-2))
$(eval $(call generic-program-example,ch03-program-variables-3))
$(eval $(call generic-program-example,ch03-program-variables-5))
$(eval $(call generic-program-example,ch03-scoping-issue))
$(eval $(call generic-program-example,ch03-shell))
$(eval $(call generic-program-example,ch03-trailing-space))
$(eval $(call generic-program-example,ch04-extent))
$(eval $(call generic-program-example,ch04-for-loop-1))
$(eval $(call generic-program-example,ch04-for-loop-2))
$(eval $(call generic-program-example,ch04-for-loop-3))
$(eval $(call generic-program-example,ch06-simple))
$(eval $(call generic-program-example,ch10-echo-bash))
$(eval $(call generic-program-example,appb-defstruct))
$(eval $(call generic-program-example,appb-arithmetic))

#################################################################
# validate
#
# Производит проверку
# a) неподставленных макросов m4;
# b) символов табуляций;
# c) комментариев FIXME;
# d) RM: мои ответы Andy;
# e) дубликатов макросов m4.
#
validation_checks := $(OUTPUT_DIR)/chk_macros_tabs      \
                     $(OUTPUT_DIR)/chk_fixme            \
                     $(OUTPUT_DIR)/chk_duplicate_macros \
                     $(OUTPUT_DIR)/chk_orphaned_examples

.PHONY: validate-only
validate-only: $(OUTPUT_DIR)/validate
$(OUTPUT_DIR)/validate: $(validation_checks)
    $(TOUCH) $@

$(OUTPUT_DIR)/chk_macros_tabs: $(ALL_XML_OUT)
    # Ищем макросы и символы табуляции...
    $(QUIET)! $(EGREP) --ignore-case          \
                       --line-number          \
                       --regexp='\b(m4_|mp_)' \
                       --regexp='\011'
                       $^
    $(TOUCH) $@

$(OUTPUT_DIR)/chk_fixme: $(ALL_XML_OUT)
    # Ищем комментарии RM: и FIXME...
    $(QUIET)$(AWK)                                                \
            '/FIXME/ { printf "%s:%s: %s\n", FILENAME, NR, $$0 }  \
             /^ *RM:/ {                                           \
                        if ( $$0 !~ /RM: Done/ )                  \
                        printf "%s:%s: %s\n", FILENAME, NR, $$0   \
                      }' $(subst $(OUTPUT_DIR)/,$(SOURCE_DIR)/,$^) 
    $(TOUCH) $@ 

$(OUTPUT_DIR)/chk_duplicate_macros: $(SOURCE_DIR)/macros.m4
    # Ищем дубликаты макросов...
    $(QUIET)! $(EGREP) --only-matching              \
        "\`[^']+'," $< |                            \
    $(SORT) |                                       \
    uniq -c |                                       \
    $(AWK) '$$1 > 1 { printf "$<:0: %s\n", $$0 }' | \
    $(EGREP) "^"
    $(TOUCH) $@

ALL_EXAMPLES := $(TMP)/all_examples

$(OUTPUT_DIR)/chk_orphaned_examples: $(ALL_EXAMPLES) $(DEPENDENCY_FILES)
    $(QUIET)$(AWK) -F/ '/(EXAMPLES|OUTPUT)_DIR/ { print $$3 }' \
            $(filter %.d,$^) |                                 \
    $(SORT) -u |                                               \
    comm -13 - $(filter-out %.d,$^)
    $(TOUCH) $@

.INTERMEDIATE: $(ALL_EXAMPLES)
$(ALL_EXAMPLES):
    # Ищем неиспользованные примеры...
    $(QUIET) ls -p $(EXAMPLES_DIR) | \
    $(AWK) '/CVS/ { next }           \
            /\// { print substr($$0, 1, length - 1) }' > $@
#################################################################
# clean
#
clean:
    $(kill-acroread)
    $(RM) -r $(OUTPUT_DIR)
    $(RM) $(SOURCE_DIR)/*~ $(SOURCE_DIR)/*.log semantic.cache
    $(RM) book.pdf

#################################################################
# Управление зависимостями
#
# Если выполняется цель clean, не стоит производить чтение или
# запуск включаемых файлов.
#
ifneq "$(MAKECMDGOALS)" "clean"
  -include $(DEPENDENCY_FILES)
endif

vpath %.xml $(SOURCE_DIR)
vpath %.tif $(SOURCE_DIR)
vpath %.eps $(SOURCE_DIR)

$(OUTPUT_DIR)/%.xml: %.xml $(process-pgm) $(m4-macros)
    $(call process-includes, $<, $@)

$(OUTPUT_DIR)/%.tif: %.tif
    $(CP) $< $@

$(OUTPUT_DIR)/%.eps: %.eps
    $(CP) $< $@

$(OUTPUT_DIR)/%.d: %.xml $(make-depend)
    $(make-depend) $< > $@

#################################################################
# Создание каталогов для вывода
#
# Создаём каталоги для вывода по мере необходимости.
#
DOCBOOK_IMAGES := $(OUTPUT_DIR)/release/images
DRAFT_PNG      := /usr/share/docbook-xsl/images/draft.png

ifneq "$(MAKECMDGOALS)" "clean"
  _CREATE_OUTPUT_DIR :=                                                 \
    $(shell                                                             \
      $(MKDIR) $(DOCBOOK_IMAGES) &                                      \
      $(CP) $(DRAFT_PNG) $(DOCBOOK_IMAGES);                             \
      if ! [[ $(foreach d,                                              \
                $(notdir                                                \
                  $(wildcard $(EXAMPLES_DIR)/ch*)),                     \
                -e $(OUTPUT_DIR)/$d &) -e . ]];                         \
      then                                                              \
        echo Компоновка примеров... > /dev/stderr;                      \
        $(LNDIR) $(BOOK_DIR)/$(EXAMPLES_DIR) $(BOOK_DIR)/$(OUTPUT_DIR); \
      fi)
endif
\end{verbatim}
}

Этот \Makefile{} написан для запуска в Cygwin без серьёзных претензий
на переносимость в \UNIX{}. Тем не менее, я уверен, что в нём очень
мало (если вообще есть) несовместимостей с \UNIX{}, которые нельзя
было бы решить переопределением значений переменных или, возможно,
введением новых переменных.

Раздел глобальных переменных определяет расположение корневого
каталога и относительные пути к каталогам с текстом и примерами, а
также к каталогу для вывода. Имя каждой нетривиальной программы,
используемой в \makefile{е}, оформляется в виде переменной.

%---------------------------------------------------------------------
% Managing Examples
%---------------------------------------------------------------------
\subsection{Управление примерами}

%---------------------------------------------------------------------
% XML Processing
%---------------------------------------------------------------------
\subsection{Обработка XML}

%---------------------------------------------------------------------
% Generating Output
%---------------------------------------------------------------------
\subsection{Генерация документов}

%---------------------------------------------------------------------
% Validating the Source
%---------------------------------------------------------------------
\subsection{Проверка исходного кода}
