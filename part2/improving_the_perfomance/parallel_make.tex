%%--------------------------------------------------------------------
%% Parallel make
%%--------------------------------------------------------------------
\section{Параллельное выполнение \GNUmake{}}

Ещё одним способом увеличения производительности сборок является
использование параллелизма, присущего проблеме обработки
\makefile{а}. Большинство \makefile{ов} предназначены для
осуществления задач, многие из которых можно обрабатывать параллельно,
например, компиляцию исходных файлов \Clang{} в объектные или создание
библиотек из объектных файлов. Более того, сама структура хорошо
написанных \makefile{ов} предоставляет всю необходимую информацию для
автоматического управления конкурирующими процессами.

Следующий пример демонстрирует сборку нашей программы mp3 плеера с
опцией управления задачами, \command{-{}-jobs=2} (или \command{-j
  2}). На рисунке~\ref{fig:parallel_make} изображён тот же самый
запуск \GNUmake{}, представленный с помощью диаграммы
UML. Использование опции \command{-{}-jobs} сообщает \GNUmake{}, что
при возможности следует обновлять параллельно две цели. Когда
\GNUmake{} обновляет цели параллельно, он выводит команды в том
порядке, в каком они выполняются, чередуя в выводе команды сборки
разных целей. Это может затруднить чтение вывода \GNUmake{},
осуществляющего параллельную сборку. Давайте внимательно рассмотрим
вывод.

{\footnotesize
\begin{alltt}
\$ \textbf{make -f ../ch07-separate-binaries/makefile --jobs=2}
\end{alltt}
\begin{verbatim}
1  bison -y --defines ../ch07-separate-binaries/lib/db/playlist.y
2  flex -t ../ch07-separate-binaries/lib/db/scanner.l >
   lib/db/scanner.c
3  gcc -I lib -I ../ch07-separate-binaries/lib
   -I ../ch07-separate-binaries/include
   -M ../ch07-separate-binaries/app/player/play_mp3.c | \
   sed 's,\(play_mp3\.o\) *:,app/player/\1 app/player/play_mp3.d: ,'
   > app/player/play_mp3.d.tmp
4  mv -f y.tab.c lib/db/playlist.c
5  mv -f y.tab.h lib/db/playlist.h
6  gcc -I lib -I ../ch07-separate-binaries/lib
   -I ../ch07-separate-binaries/include
   -M ../ch07-separate-binaries/lib/codec/codec.c | \
   sed 's,\(codec\.o\) *:,lib/codec/\1 lib/codec/codec.d: ,' >
   lib/codec/codec.d.tmp
7  mv -f app/player/play_mp3.d.tmp app/player/play_mp3.d
8  gcc -I lib -I ../ch07-separate-binaries/lib
   -I ../ch07-separate-binaries/include -M lib/db/playlist.c | \
   sed 's,\(playlist\.o\) *:,lib/db/\1 lib/db/playlist.d: ,' >
   lib/db/playlist.d.tmp
9  mv -f lib/codec/codec.d.tmp lib/codec/codec.d
10 gcc -I lib -I ../ch07-separate-binaries/lib
   -I ../ch07-separate-binaries/include
   -M ../ch07-separate-binaries/lib/ui/ui.c | \
   sed 's,\(ui\.o\) *:,lib/ui/\1 lib/ui/ui.d: ,' > lib/ui/ui.d.tmp
11 mv -f lib/db/playlist.d.tmp lib/db/playlist.d
12 gcc -I lib -I ../ch07-separate-binaries/lib
   -I ../ch07-separate-binaries/include
   -M lib/db/scanner.c | \
   sed 's,\(scanner\.o\) *:,lib/db/\1 lib/db/scanner.d: ,' >
   lib/db/scanner.d.tmp
13 mv -f lib/ui/ui.d.tmp lib/ui/ui.d
14 mv -f lib/db/scanner.d.tmp lib/db/scanner.d
15 gcc -I lib -I ../ch07-separate-binaries/lib
   -I ../ch07-separate-binaries/include -c
   -o app/player/play_mp3.o
   ../ch07-separate-binaries/app/player/play_mp3.c
16 gcc -I lib -I ../ch07-separate-binaries/lib
   -I ../ch07-separate-binaries/include -c
   -o lib/codec/codec.o
   ../ch07-separate-binaries/lib/codec/codec.c
17 gcc -I lib -I ../ch07-separate-binaries/lib
   -I ../ch07-separate-binaries/include -c
   -o lib/db/playlist.o lib/db/playlist.c
18 gcc -I lib -I ../ch07-separate-binaries/lib
   -I ../ch07-separate-binaries/include -c
   -o lib/db/scanner.o lib/db/scanner.c
   ../ch07-separate-binaries/lib/db/scanner.l: In function yylex:
   ../ch07-separate-binaries/lib/db/scanner.l:9: warning:
   return makes integer from pointer without a cast
19 gcc -I lib -I ../ch07-separate-binaries/lib
   -I ../ch07-separate-binaries/include -c
   -o lib/ui/ui.o ../ch07-separate-binaries/lib/ui/ui.c
20 ar rv lib/codec/libcodec.a lib/codec/codec.o
   ar: creating lib/codec/libcodec.a
   a - lib/codec/codec.o
21 ar rv lib/db/libdb.a lib/db/playlist.o lib/db/scanner.o
   ar: creating lib/db/libdb.a
   a - lib/db/playlist.o
   a - lib/db/scanner.o
22 ar rv lib/ui/libui.a lib/ui/ui.o
   ar: creating lib/ui/libui.a
   a - lib/ui/ui.o
23 gcc app/player/play_mp3.o lib/codec/libcodec.a lib/db/libdb.a
   lib/ui/libui.a app/player/play_mp3
\end{verbatim}
}

\begin{figure}
\begin{center}
\includegraphics{figures/parallel_make.eps}
\end{center}
\caption{Диаграмма выполнения \GNUmake{} при \command{-{}-jobs=2}}
\label{fig:parallel_make}
\end{figure}

Сначала \GNUmake{} должен сгенерировать исходные файлы и файлы
зависимостей. Два сгенерированных исходных файла являются выводом
команд \utility{yacc} и \utility{lex} (команды 1 и 2
соответственно). Третья команда генерирует файл зависимостей для
\filename{play\_mp3.c}, её выполнение начинается до завершения
генерации файлов зависимостей для \filename{playlist.c} или
\filename{scanner.c} (командами 4, 5, 8, 9, 12 и 14). Таким образом,
\GNUmake{} выполняет одновременно три задачи, не смотря на то, что
опция командной строки требует одновременного выполнения двух задач.

Команды \command{mv} (4 и 5) завершают генерацию исходного файла
\filename{playlist.c}, начавшуюся первой командой. Команда 6 начинает
генерацию очередного файла зависимостей. Каждый командный сценарий
выполняется одним процессом \GNUmake{}, однако каждая цель и каждый
реквизит формируют отдельный поток. Таким образом, команда 7,
являющаяся второй командой сценария генерации файла зависимостей,
исполняется тем же процессом \GNUmake{}, что и команда 3. Команда 6
выполняется, скорее всего, процессом \GNUmake{}, порождённым сразу
после выполнения команд 1-4-5 (обработки грамматики \utility{yacc}),
но до генерации файла зависимостей, осуществляющейся командой 8.

Определение зависимостей продолжается в том же духе вплоть до команды
14. Все файлы зависимостей должны быть созданы до того, как \GNUmake{}
сможет приступить к следующей фазе обработки~--- повторному считыванию
\makefile{а}. Эта фаза образует естественную точку синхронизации.

Как только завершается повторное чтение \makefile{а}, \GNUmake{} может
снова продолжить параллельное выполнение сборки. В этот раз \GNUmake{}
решает произвести компиляцию всех объектных файлов до того, как начать
упаковывать их в библиотечные архивы. Этот порядок является
недетерминированным. В следующий раз \GNUmake{} может собрать
библиотеку \filename{libcodec.a} до того, как будет скомпилирован файл
\filename{playlist.c}, поскольку библиотека не требует наличия никаких
объектных файлов, кроме \filename{codec.o}. Таким образом, этот пример
демонстрирует один порядок выполнения из множества возможных.

Наконец, происходит компоновка программы. В нашем случае фаза
компоновки также является естественной точкой синхронизации и всегда
будет происходить в последнюю очередь. Если, однако, целью была не
единственная программа, а множество программ или библиотек, последняя
исполняемая инструкция может также быть другой.

Запуск множества задач на многопроцессорном компьютере может иметь
смысл, однако запуск более одной задачи на процессор может быть также
очень полезным. Причина кроется в латентности дискового ввода/вывода и
наличии кэширования на большинстве систем. Например, если процесс, к
примеру, \utility{gcc}, ожидает поступления данных с диска, данные для
других задач (таких как \utility{mv} или \utility{yacc}) могут
располагаться в памяти компьютера. В этом случае лучшим выходом будет
разрешить задаче обработку данных. В общем случае запуск \GNUmake{} с
несколькими потоками выполнения на однопроцессорной системе
практически всегда быстрее, чем запуск однопоточной сборки, и не так уж
редко запуск трёх или даже четырёх потоков приводит к лучшему
результату, чем при запуске двух потоков.

Опция \command{-{}-jobs} может использоваться без аргумента. В этом
случае \GNUmake{} порождает по одному потоку на каждую обновляемую
цель. Как правило, это плохая идея, поскольку на переключение
контекста может уходить настолько много времени, что итоговая
производительность будет гораздо ниже, чем в случае однопоточного
выполнения.

Ещё одним способом управления множеством задач является использование
средней загрузки системы как отправной точки. Средняя загрузка
системы~--- это число запущенных задач, усреднённое за некоторый
промежуток времени (как правило,1, 5 или 15 минут). Средняя загрузка
выражается как число с плавающей точкой. Опция
\command{-{}-load-average} (или \command{-l}) задаёт верхнюю границу
числа порождаемых задач. Например, следующая команда:

{\footnotesize
\begin{alltt}
\$ \textbf{make --load-average=3.5}
\end{alltt}
}

сообщает \GNUmake{}, что потоки задач должны порождаться таким
образом, чтобы средняя средняя загрузка системы была не более
3.5. Если средняя загрузка выше, \GNUmake{} будет ждать, пока она не
уменьшится до допустимых пределов, или пока все потоки не закончат
свою работу.

Когда вы пишите \makefile{} для параллельного выполнения, задача
правильного указания реквизитов становятся ещё более важной. Как уже
было замечено, когда опция \command{-{}-jobs} имеет значение 1, список
реквизитов обычно вычисляется слева направо. Когда \command{-{}-jobs}
больше 1, реквизиты могут вычисляться параллельно. Поэтому любые
отношения зависимости, неявно присутствующие в порядке вычисления
реквизитов, при параллельном запуске должны быть указаны явно.

Ещё одной неприятностью при использовании параллельных сборок является
проблема разделяемых временных файлов. Например, если каталог содержит
файлы \filename{foo.y} и \filename{bar.y}, параллельный запуск
\utility{yacc} может привести к тому, что один из экземпляров файла
\filename{y.tab.c} или \filename{y.tab.h} перепишет другой. С подобной
проблемой вы также сталкиваетесь при использовании в своих сценариях
временных файлов с фиксированными именами.

Ещё одной идиомой, препятствующей параллельному выполнению, является
рекурсивный вызов \GNUmake{} из цикла \command{for}:

{\footnotesize
\begin{verbatim}
dir:
    for d in $(SUBDIRS);         \
    do                           \
        $(MAKE) --directory=$$d; \
    done
\end{verbatim}
}

Как уже упоминалось в разделе \nameref{sec:recursive_make} главы
\ref{chap:managing_large_proj}, \GNUmake{} не может выполнять
рекурсивные вызовы параллельно. Чтобы достичь параллельного
выполнения, объявите каталоги абстрактными целями:

{\footnotesize
\begin{verbatim}
.PHONY: $(SUBDIRS)
$(SUBDIRS):
    $(MAKE) --directory=$@
\end{verbatim}
}
