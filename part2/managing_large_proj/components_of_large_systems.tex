%%--------------------------------------------------------------------
%% Components of large systems
%%--------------------------------------------------------------------
\section{Компоненты больших систем}

Мы рассмотрим две популярных на сегодняшний день модели разработки:
модель свободного программного обеспечения и коммерческую модель.

В модели свободного программного обеспечения каждый разработчик в
основном может полагаться только на себя. Проект имеет \Makefile{} и
\filename{README} файл, и ожидается, что разработчикам потребуется
лишь немного помощи для начала работы. Приоритетами таких проектов, как
правило, являются качество и привлечение к участию всего сообщества,
однако наибольшую ценность имеет участие умелых и хорошо
мотивированных членов сообщества. Это не критика. С этой точки
зрения программное обеспечение должно быть высокого качества,
соблюдение временных ограничений имеет меньший приоритет.

В коммерческой модели разработки разработчики могут иметь различный
уровень подготовки, и все из них должны быть способны выполнить работу
к назначенному сроку. Разработчик, который не может разобраться, как
выполнить свою работу, расходует деньги понапрасну. Если система не
компилируется или не запускается должным образом, вся команда
разработчиков может простаивать: этот сценарий является наиболее
затратным из всех возможных. Для решения подобных проблем процесс
разработки управляется командой поддержки, координирующей процесс
сборки, конфигурацию инструментов разработки, поддержку и разработку,
а также менеджмент новых релизов. В подобной среде процессом управляют 
критерии эффективности.

Как правило, для коммерческой модели характерны более продуманные
системы сборки. Основной причиной такого перевеса является стремление
сократить стоимость разработки программного обеспечения за счёт
повышения эффективности труда программистов. Это, в свою очередь,
должно вести к увеличению прибыли. Это именно та модель, которая
требует от \GNUmake{} наибольшего функционала. Тем не менее, техники,
которые мы обсудим, применимы в случае необходиости и к модели
свободного программного обеспечения.

Этот раздел содержит много общей информации, чуть-чуть специфики
и совсем не содержит примеров. Причина этого заключается в том, что
очень многое зависит от языка разработки и используемой операционной
среды. В главах~\ref{chap:c_and_cpp} и \ref{chap:java} мы рассмотрим
примеры реализации многих возможностей, рассмотренных в этом разделе.

%---------------------------------------------------------------------
% Requirements
%---------------------------------------------------------------------
\subsection*{Требования}

Разумеется, требования различны для каждого проекта и каждой рабочей
среды. Здесь мы рассмотрим основные требования, считающиеся важными во
многих коммерческих средах разработки.

Наиболее общей потребностью команд разработчиков является отделение
исходного кода от бинарных файлов. То есть объектные файлы,
полученные в результате компиляции, должны располагаться в отдельном
дереве каталогов. Это, в свою очередь, позволяет включать ещё
множество возможностей. Отделение дерева каталогов бинарных файлов
сулит множество преимуществ:

\begin{itemize}
%---------------------------------------------------------------------
\item Когда расположение дерева каталогов бинарных файлов определено,
гораздо легче управлять дисковым пространством.
%---------------------------------------------------------------------
\item Различные версии деревьев бинарных файлов могут управляться
параллельно. Например, единственному дереву исходных файлов могут
соответствовать оптимизированное, отладочное, и профилировочное
деревья бинарных файлов.
%---------------------------------------------------------------------
\item Существует возможность одновременной поддержки различных
платформ. Правильным образом реализованное дерево исходных файлов
может быть использовано для параллельной компиляции исполняемых файлов
для различных платформ.
%---------------------------------------------------------------------
\item Разработчики могут взять небольшую часть исходного кода и
позволить системе сборки самостоятельно <<заполнять>> недостающие
файлы с помощью зависимостей исходных файлов и деревьев каталогов
объектных файлов. Это не обязательно требует отделения исходных файлов
от объектных, однако без отделения больше вероятность того, что
система сборки не сможет правильно определить, где следует искать
бинарные файлы.
%---------------------------------------------------------------------
\item Дерево исходного кода может быть сделано доступно только для
чтения. Это даёт дополнительную уверенность в том, что сборка отражает
реальное состояние репозитория.
%---------------------------------------------------------------------
\item Некоторые цели, подобные \target{clean}, можно реализовать
тривиальным образом (и выполнять с колоссальным выигрышем в
производительности), если всё дерево каталогов может быть рассмотрено
как отдельная единица, не требующая поиска и манипуляций файлами.
%---------------------------------------------------------------------
\end{itemize}

Б\'{о}льшая часть этих пунктов является важными преимуществами системы
сборки и может быть проектным требованием.

Возможность управления историей сборок проекта часто является важным
качеством системы сборки. Основная идея заключается в том, что сборка
исходного кода осуществляется по ночам, обычно при помощи задачи
\utility{cron}.
\index{Справочные деревья каталогов}
Поскольку результирующие деревья каталогов, содержащие
исходный код и бинарные файлы, являются не модифицируемыми с точки
зрения CVS, я буду называть их справочными. Эта идея имеет множество
применений.

Во-первых, справочное дерево каталогов исходного кода может
использоваться программистами и менеджерами, которым нужно просмотреть
исходный код. Это может показаться банальным, однако когда число
файлов и релизов растёт, извлечение всего исходного кода из
репозитория ради просмотра одного файла может быть не очень разумно. К
тому же, хоть инструменты для просмотра CVS репозиториев достаточно
распространены, они обычно не предоставляют средств для простого
поиска по всему исходному коду проекта. Для этих целей больше подходят
таблицы символов или даже команды \utility{find}/\utility{grep} (или
\utility{grep -R}).

Во-вторых, справочные деревья бинарных файлов являются индикатором
того, что соответствующая сборка прошла успешно. Когда разработчики
начинают утром свою работу, они уже знают, является ли система
работоспособной. Если проект использует систему автоматического
тестирования, свежие сборки могут использоваться для запуска
автоматических тестов. Каждый день разработчики могут проверять отчёты
с результатами тестов, чтобы определить жизнеспособность системы, не
тратя время на запуск тестов. Если же разработчик имеет на руках
только модифицированную версию исходного кода, имеет место дополнительное
сокращение затрат проекта, поскольку в этом случает разработчику не
нужно терять время на получение исходной версии и сборку. Наконец,
справочные сборки могут запускаться разработчиками для тестирования и
сравнения функциональности определённых компонентов.

Есть и другие способы использования справочных сборок. Для проекта,
состоящего из множества библиотек, прекомпилированные библиотеки,
полученные в результате ночных сборок, могут использоваться
программистами для компоновки собственных приложений с теми
библиотеками, которые они не модифицируют. Это позволяет сократить
цикл разработки за счёт исключения необходимости компиляции большей
части исходного кода при запуске собственных сборок. Разумеется,
лёгкий доступ к исходному коду проекта, располагающегося на локальном
файловом сервере, чрезвычайно удобен, если разработчикам, не имеющим
полной рабочей копии проекта, нужно просмотреть исходный код.

При таком разнообразии применений справочных деревьев каталогов
поддержание целостности их структуры становится чрезвычайно важным.
Одним из простых и эффективных способов повышения надёжности является
объявление дерева каталогов с исходным кодом доступным только для
чтения. Это гарантирует, что дерево отражает состояние репозитория в
момент сборки. Этот аспект может потребовать особого внимания,
поскольку во многих случаях система сборки может принимать попытки
записи в дерево каталогов, в частности, при генерации исходного кода
или создании временных файлов. Объявления дерева каталогов с исходным
кодом доступным только для чтения также предотвращает случайное его
повреждение обычными пользователями, что случается наиболее
часто.

Ещё одним общим требованием к проектной системе сборки является
возможность лёгкого управления различными конфигурациями компиляции,
компоновки и развёртывания системы. Система сборки обычно должна
оперировать различными версиями проекта (которые могут быть различными
ветками в репозитории).

Множество крупных проектов зависят от программного обеспечения третих
разработчиков, представленном в форме библиотек или инструментов
разработки. Если нет других инструментов для управления конфигурацией
программного обеспечения (а обычно их нет), использование
\Makefile{}'а и системы сборки для этих целей часто является разумным
выбором.

Наконец, когда программное обеспечение доставляется заказчику, оно часто
упаковывается на базе текущей рабочей версии. Это может быть также
сложно, как конструирование \filename{setup.exe} файла для Windows или
также просто, как редактирование HTML файла и связывание его с
\filename{jar} архивом. Иногда операция инсталляции сочетается с
обычным процессом сборки. Я предпочитаю разделять сборку и генерацию
инсталлятора на два независимых шага, поскольку, как правило, они
используют совершенно разные процессы. В любом случае, скорее всего,
обе эти операции будут влиять на систему сборки.
