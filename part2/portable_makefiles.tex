%%%------------------------------------------------------------------
%%% Portable makefiles
%%%------------------------------------------------------------------
\chapter{Переносимые \Makefile{}'ы}
\label{chap:portable_makefiles}

Какой же \Makefile{} мы будем считать переносимым? В качестве
экстремального примера мы хотим иметь \Makefile{}, запускающийся без
изменений на любой платформе, позволяющей запустить GNU \GNUmake{}.
Однако это практически невозможно по причине огромного количества
различных операционных систем. Более разумным будет назвать
переносимым \Makefile{}, который легко изменить для запуска на другой
платформе. Однако перенос на другую платформу не должен препятствовать
поддержке всех предыдущих платформ, это будет дополнительным важным
ограничением.

Мы можем достичь этого уровня переносимости \Makefile{}'ов, используя
те же техники, что и в традиционном программировании: инкапсуляция и
абстракция. Используя переменные и функции, определяемые
пользователем, мы можем инкапсулировать приложения и алгоритмы.
Определяя переменные для аргументов командной строки и параметров, мы
можем абстрагироваться от элементов, варьирующихся от платформы к
платформе.

Затем вам потребуется определить, какие инструменты может предложить
каждая платформа для выполнения вашей работы, и какие из них нужно
использовать в случае каждой конкретной платформы. Наибольшую
переносимость приносит использование только тех инструментов, которые
присутствуют на всех интересующих платформах. Обычно это называют
подходом <<наименьшего общего знаменателя>>, который, очевидно, может
сделать базовый набор инструментов довольно скудным.

Другой версией подхода наименьшего общего знаменателя является
следующая парадигма: используйте мощный набор инструментов и
убедитесь, что можете взять его с собой на любую платформу. Это
гарантирует, что команды, которые вы вызываете в \Makefile{}'е,
работают совершенно одинаково в любой системе. Осуществить это, как
правило, нелегко, и с административной точки зрения, и в плане
убеждения вашей организации в необходимости кооперации её систем с
вашими  наработками. Однако такой подход может приносить результаты, и
позже я приведу приведу пример с пакетом Cygwin для Windows. Как вы
увидите, стандартизация инструментов не решает всех проблем, всегда
найдутся отличия операционных систем, которые нужно будет обрабатывать
особым образом.

Наконец, вы можете принять различия между системами как данность и
обходить их с помощью аккуратного выбора функций и макросов. Я приведу
пример такого подхода в этой главе.

Таким образом, рассудительно используя переменные и функции,
определяемые пользователем, минимизируя использование экзотических
возможностей и полагаясь на стандартные инструменты, мы можем
увеличить переносимость наших \Makefile{}'ов. Как уже было замечено,
идеальная переносимость недостижима, поэтому нашей задачей является
нахождение баланса между затратами и переносимостью. Однако прежде,
чем мы исследуем специфические техники, давайте произведём обзор
основных проблем переносимости \Makefile{}'ов.

%%--------------------------------------------------------------------
%% Portability issues
%%--------------------------------------------------------------------
\section{Проблемы переносимости}

Проблемы переносимости может быть нелегко охарактеризовать, поскольку
они могут варьироваться от тотальной смены парадигмы (например,
отличие классической Mac OS от System V \UNIX{}) до исправлений
тривиальных ошибок (таких, как исправление кода возврата программы).
Тем не менее, ниже перечислены основные проблемы переносимости, с
которыми рано или поздно сталкивается любой \Makefile{}:

\begin{description}
%---------------------------------------------------------------------
% Program names
%---------------------------------------------------------------------
\item[Имена программ] \hfill \\
Довольно часто в различных платформах для программ, реализующих схожую
(или даже одинаковую) функциональность, используются различные имена.
Наиболее ярким примером являются имена компиляторов языков \Clang{} и 
\Cplusplus{} (например, \utility{cc} и \utility{xlc}). Также общим
является добавления префикса \filename{g} для GNU-версий программ,
установленных на не GNU системах (например, \utility{gmake},
\utility{gawk}). 

%---------------------------------------------------------------------
% Paths
%---------------------------------------------------------------------
\item[Пути] \hfill \\
Расположение файлов и программ варьируется от платформы к платформе. 
Например, в операционной системе Solaris каталогом X-сервера
является \filename{/usr/X}, в то время как на многих других системах
этим каталогом является \filename{/usr/X11R6}. К тому же, различие
между \filename{/bin}, \filename{/usr/bin}, \filename{/sbin} и
\filename{/usr/sbin} часто неясно при переходе от одной системе к
другой.

%---------------------------------------------------------------------
% Options
%---------------------------------------------------------------------
\item[Аргументы командной строки] \hfill \\
Аргументы командной строки программы могут отличаться, в частности при
использовании альтернативной реализации. Более того, если на какой-то
платформе отсутствует нужная вам программа (или присутствующая версия
этой программы вам не подходит), вам, возможно, придётся заменить эту
программу другой, использующей другие аргументы командной строки.

%---------------------------------------------------------------------
% Shell features
%---------------------------------------------------------------------
\item[Возможности интерпретатора] \hfill \\
По умолчанию \GNUmake{} выполняет командные сценарии с помощью
\filename{/bin/sh}, однако возможности различных реализаций
интерпретатора \utility{sh} варьируются от платформы к платформе. В
частности, интерпретаторы, выпущенные до принятия стандарта \POSIX{},
не имеют множества возможностей и не принимают синтаксис современных
интерпретаторов.

У Open Group есть очень полезная статья, описывающая различия между
интерпретаторами System V и \POSIX{}. Её можно найти по адресу
\filename{\url{http://www.unix-systems.org/whitepapers/shdiffs.html}}.
Те, кому интересны детали, смогут найти спецификацию командного
интерпретатора \POSIX{} по адресу
\filename{\url{http://www.opengroup.org/onlinepubs/
007904975/utilities/xcu_chap02.html}}

%---------------------------------------------------------------------
% Program behavior
%---------------------------------------------------------------------
\item[Поведение программ] \hfill \\
Переносимым \Makefile{}'ам приходится бороться с программам, которые
ведут себя по-разному на различных платформах. Это встречается
повсеместно, поскольку различные поставщики исправляют (и совершают)
ошибки и добавляют новые возможности. Существуют также обновления
программ, которые поставщик может включить или не включить в релиз.
Например, в 1987 году программа \utility{awk} сменила старший номер
версии. Тем не менее, даже спустя двадцать лет некоторые системы всё
ещё не используют новую версию в качестве стандартной программы
\utility{awk}.

%---------------------------------------------------------------------
% Operating system
%---------------------------------------------------------------------
\item[Операционная система] \hfill \\
Наконец, существуют проблемы переносимости, связанные с совершенно
различными операционными системами, например, Windows и \UNIX{}, Linux
и VMS.
\end{description}

%%--------------------------------------------------------------------
%% Cygwin
%%--------------------------------------------------------------------
\section{Cygwin}

Несмотря на то, что есть порт \GNUmake{} под Win32, это лишь малая
часть проблемы переноса \Makefile{}'ов на Windows, поскольку командным
интерпретатором, используемым этим портом по умолчанию, является
\filename{cmd.exe} (или \filename{command.exe}). Это, наряду с
отсутствием большинства инструментов \UNIX{}, делает реализацию
кросс-платформенной переносимости очень сложной задачей. К счастью,
проект Cygwin (\filename{\url{http://www.cygwin.com}}) реализовал
для Windows библиотеку совместимости с Linux, с использованием которой
были перенесены многие программы. Я уверен, что Windows разработчики,
желающие достичь совместимости с Linux или получить доступ к
инструментам GNU, не найти смогут лучшего инструмента.

Я использовал Cygwin на протяжении десяти лет для различных проектов,
начиная с CAD-приложения, построенного на смеси \Cplusplus{} и Lisp, и
заканчивая приложением для управления рабочим процессом, написанным на
чистом \Java{}. Набор инструментов Cygwin включает компиляторы и
интерпретаторы многих языков программирования. Однако Cygwin можно
выгодно использовать даже в том случае, если приложение реализовано
без использования набора компиляторов и интерпретаторов Cygwin. Набор
инструментов Cygwin можно использовать как вспомогательное средство
для координации процессов разработки и сборки. Другими словами, совсем
не обязательно писать <<Cygwin>> приложение или использовать средства
разработки Cygwin, чтобы извлечь выгоду из Cygwin-окружения.

Тем не менее, Linux~--- это не Windows (слава богам!), поэтому при
использовании Cygwin инструментов применительно к <<родным>>
приложениям Windows возникает ряд проблем. Практически все эти
проблемы решаются на уровне символов окончаний строки в файлах и форм
путей к файлам, передающихся между Cygwin и Windows.

%---------------------------------------------------------------------
% Line termination
%---------------------------------------------------------------------
\subsection*{Окончания строк}

Файловая система Windows использует для индикации окончания строки
последовательность из двух символов: символа возврата каретки и
символа окончания строки (CRLF). \POSIX{} системы используют для этой
цели один символ~--- символ окончания строки (LF). Иногда это различие
может стать причиной удивления, если программа вдруг сообщит о
синтаксической ошибке или перейдёт к неверной позиции в файле.
Библиотека Cygwin делает всё возможное для избежания этих
неприятностей. Во время установки Cygwin (или при использовании
команды \utility{mount}) вы можете выбрать, следует ли Cygwin
выполнять преобразование файлов, содержащих последовательность CRLF в
качестве индикатора окончания строки. Если выбран формат файлов DOS,
Cygwin будет заменять последовательной CRLF на символ LF при открытии
файла и производить обратное преобразование при записи текста, таким
образом, \UNIX{}-программы могут правильно работать с текстовыми
файлами DOS.  Если вы планируете использовать родные инструменты
наподобие Visual C++ или Sun Java SDK, выбирайте формат файлов DOS.
Если же вы планируете использовать компиляторы Cygwin, используйте
формат \UNIX{} (вы сможете изменить своё решение в любое время).

В добавок ко всему, Cygwin поставляется с набором инструментов для
перевода форматов файлов. Программы \utility{dos2unix} и
\utility{unix2dos} помогут преобразовать файлы в нужный формат в
случае необходимости.

%---------------------------------------------------------------------
% Filesystem
%---------------------------------------------------------------------
\subsection*{Файловая система}

Cygwin предоставляет \POSIX{}-взгляд на файловую систему Windows.
Корневой каталог файловой системы \POSIX{}, \filename{/}, отображается
в каталог, в который установлен Cygwin. Диски Windows доступны из
псевдокаталога \filename{/cygdrive/\texttt{буква}}. Таким образом,
если Cygwin установлен в каталог
\filename{C:\textbackslash{}usr\textbackslash{}cygwin} (я предпочитаю
именно этот каталог), будет производиться отображение каталогов,
представленное в таблице~\ref{tab:cygwin_dir_mapping}.

\begin{table}
\footnotesize
\center
\begin{tabular}{|l|l|l|}
\hline
Путь Windows & Путь Cygwin & Альтернативный путь Cygwin \\
\hline
\hline
\filename{c:\textbackslash{}usr\textbackslash{}cygwin} &%
\filename{/} &%
\filename{/cygdrive/c/usr/cygwin} \\
\hline
\filename{c:\textbackslash{}Program Files} &%
\filename{/cygdrive/c/Program Files} & \\
\hline
\filename{c:\textbackslash{}usr\textbackslash{}cygwin\textbackslash{}bin} &%
\filename{/bin} &%
\filename{/cygdrive/c/usr/cygwin/bin} \\
\hline
\end{tabular}
\caption{Стандартное отображение каталогов Cygwin}
\label{tab:cygwin_dir_mapping}
\end{table}

Поначалу такое преобразование может быть немного непривычным, однако
на работу программ оно никак не влияет. Cygwin также предоставляет
команду \utility{mount}, позволяющую пользователям получать доступ к
файлам и каталогам более удобным способом. Oпция \utility{mount}
\index{Опции!Cygwin!change-cygdrive-prefix@\command{-{}-change-cygdrive-prefix}}
\command{-{}-change\hyp{}cygdrive\hyp{}prefix} позволяет вам изменить
префикс. Мне кажется, что изменение префикса на \filename{/} может
быть полезно, поскольку в этом случае доступ к дискам становится более
естественным:

\begin{alltt}
\footnotesize
\$ \textbf{mount --change-cygdrive-prefix} /
\$ \textbf{ls /c}
AUTOEXEC.BAT            IO.SYS                     WINDOWS
BOOT.INI                MSDOS.SYS                  WUTemp
CD                      NTDETECT.COM               hp
CONFIG.SYS              PERSIST                    ntldr
C\_DILLA                 Program Files              pagefile.sys
Documents and Settings  RECYCLER                   tmp
Home                    System Volume Information  usr
I386                    Temp                       work
\end{alltt}

Как только вы произведёте это действие, предыдущее отображение
каталогов поменяется на отображение, представленное в
таблице~\ref{tab:modified_dir_mapping}.

\begin{table}
\footnotesize
\center
\begin{tabular}{|l|l|l|}
\hline
Путь Windows & Путь Cygwin & Альтернативный путь Cygwin \\
\hline
\hline
\filename{c:\textbackslash{}usr\textbackslash{}cygwin} &%
\filename{/} &%
\filename{/c/usr/cygwin} \\
\hline
\filename{c:\textbackslash{}Program Files} &%
\filename{/c/Program Files} & \\
\hline
\filename{c:\textbackslash{}usr\textbackslash{}cygwin\textbackslash{}bin} &%
\filename{/bin} &%
\filename{/c/usr/cygwin/bin} \\
\hline
\end{tabular}
\caption{Модифицированное отображение каталогов Cygwin}
\label{tab:modified_dir_mapping}
\end{table}

Если вам нужно передать имя файла Windows-программе (например,
компилятору Visual \Cplusplus{}), вы можете просто передать
относительный путь к файлу, использовав \POSIX{} стиль, предполагающий
использование прямых слэшей. Win32 API не различает прямых и обратных
слэшей. К сожалению, некоторые программы, осуществляющие разбор
аргументов командной строки, интерпретируют все прямые слэши как
опции. Одной из таких программ является команда DOS \utility{print},
ещё одним примером является команда \utility{net}.

Если же используется абсолютный путь, синтаксис, основанный на именах
дисков, всегда вызывает проблемы. Несмотря на то, что программы
Windows обычно легко воспринимают прямые слэши в именах файлов, они
совершенно не способны воспринять синтаксис \filename{/c}. Имя диска
всегда должно преобразовываться в формат \filename{c:}. Для
осуществления прямых и обратных преобразований путей \POSIX{} в пути
Windows Cygwin предоставляет команду \utility{cygpath}:

\begin{alltt}
\footnotesize
\$ \textbf{cygpath --windows /c/work/src/lib/foo.c}
c:\textbackslash{}work\textbackslash{}src\textbackslash{}lib\textbackslash{}foo.c
\$ \textbf{cygpath --mixed /c/work/src/lib/foo.c}
c:/work/src/lib/foo.c
\$ \textbf{cygpath --mixed --path "/c/work/src:/c/work/include"}
c:/work/src;c:/work/include
\end{alltt}

\index{Опции!Cygwin!windows@\command{-{}-windows}}
Опция \command{-{}-windows} преобразует заданный путь \POSIX{} в путь
Windows (или, при указании соответствующего аргумента, наоборот). Я
\index{Опции!Cygwin!mixed@\command{-{}-mixed}}
предпочитаю использовать опцию \command{--mixed}, возвращающую путь
Windows, в котором все обратные слэши заменены на прямые (таком
образом, этот путь может использоваться для работы с программами
Windows). Такие пути гораздо удобнее использовать в командном
интерпретаторе Cygwin, воспринимающем обратный слэш как символ
экранирования. Программа \utility{cygpath} имеет множество опций,
предоставляющих переносимый доступ к важным каталогам Windows:

\begin{alltt}
\footnotesize
\$ \textbf{cygpath --desktop}
/c/Documents and Settings/Owner/Desktop
\$ \textbf{cygpath --homeroot}
/c/Documents and Settings
\$ \textbf{cygpath --smprograms}
/c/Documents and Settings/Owner/Start Menu/Programs
\$ \textbf{cygpath --sysdir}
/c/WINDOWS/SYSTEM32
\$ \textbf{cygpath --windir}
/c/WINDOWS
\end{alltt}

Если вы используете \utility{cygpath} в смешанной Windows/\UNIX{}
среде, вы можете захотеть обернуть его вызовы в переносимые функции:

{\footnotesize
\begin{verbatim}
ifdef COMSPEC
  cygpath-mixed         = $(shell cygpath -m "$1")
  cygpath-unix          = $(shell cygpath -u "$1")
  drive-letter-to-slash = /$(subst :,,$1)
else
  cygpath-mixed         = $1
  cygpath-unix          = $1
  drive-letter-to-slash = $1
endif
\end{verbatim}
}

Если вам нужно только преобразовать букву диска в \POSIX{} форму,
функция \function{drive-letter-to-slash} будет работать быстрее, чем
запуск программы \utility{cygpath}.

Наконец, Cygwin не может спрятать все причуды Windows. Имена файлов,
недопустимые в Windows, также недопустимы в Cygwin. Например, такие
имена, как \filename{aux.h}, \filename{com1} и \filename{prn} не могут
использоваться в \POSIX{} путях, даже при наличии расширения.

%---------------------------------------------------------------------
% Program conflicts
%---------------------------------------------------------------------
\subsection*{Конфликты программ}

Несколько программ Windows имеют точно такие же имена, что и
\UNIX{}-программы. Разумеется, программы Windows не принимают тех же
самых аргументов командной строки и не ведут себя совместимым с
\UNIX{}-программами образом. Если вы случайно вызвали Windows версию
программы, обычным результатом является серьёзное недоумение. Наиболее
проблемными программами в этом плане являются \utility{find},
\utility{sort}, \utility{ftp} и \utility{telnet}. Для достижения
максимальной переносимости убедитесь в том, что вы используете
абсолютные пути к этим программам при работе с \UNIX{}, Windows
и Cygwin.

Если вы тесно используете Cygwin и для сборки вам не нужны базовые
инструменты Windows, вы можете спокойно поместить каталог
\filename{/bin} в начало переменной окружения \variable{PATH}. Это
будет гарантией того, что в первую очередь будут использоваться
инструменты Cygwin, а не их Windows аналоги.

Если ваш \Makefile{} использует инструменты \Java{}, учтите, что
Cygwin включает программу GNU \utility{jar}, не совместимую по формату
со стандартными Sun \filename{jar} файлами. Поэтому каталог \Java{}
jdk \filename{bin} следует поместить в вашей переменной
\variable{PATH} раньше каталога Cygwin \filename{/bin}. Это поможет
вам избежать использования программы Cygwin \filename{jar}.

%%--------------------------------------------------------------------
%% Managing programs and files
%%--------------------------------------------------------------------
\section{Управление программами и файлами}

Наиболее общий способ управления программами заключается в
использовании переменных для имён программ или путей, которые могут
измениться. Переменные могут быть определены в простом блоке, как мы
уже видели прежде:
{\footnotesize
\begin{verbatim}
MV ?= mv -f
RM ?= rm -f
\end{verbatim}
}
{\flushleft или же в условном блоке:}
{\footnotesize
\begin{verbatim}
ifdef COMSPEC
  MV ?= move
  RM ?= del
else
  MV ?= mv -f
  RM ?= rm -f
endif
\end{verbatim}
}

Если используется простой блок, значения переменных могут измениться
при использовании аргументов командной строки, при редактировании
\Makefile{}'а или (именно для этого случая мы использовали оператор
условного присваивания \texttt{?=}) при наличии соответствующей
переменной окружения. Как уже было ранее замечено, одним из способов
определения текущей платформы является проверка существования
переменной \variable{COMSPEC}, используемой всеми версиями
операционной системы Windows. Иногда в коррекции нуждаются только
пути:

{\footnotesize
\begin{verbatim}
ifdef COMSPEC
  OUTPUT_ROOT := d:
  GCC_HOME    := c:/gnu/usr/bin
else
  OUTPUT_ROOT := $(HOME)
  GCC_HOME    := /usr/bin
endif

OUTPUT_DIR := $(OUTPUT_ROOT)/work/binaries
CC := $(GCC_HOME)/gcc
\end{verbatim}
}

Этот стиль приводит к \Makefile{}'ам, в которых б\'{о}льшая часть
программ вызывается при помощи переменных \GNUmake{}. Пока вы не
привыкните к этому, читать такие \Makefile{}'ы будет немного сложнее.
Однако использовать переменные в любом случае разумнее, поскольку их
имена могут быть значительно короче, чем имена программ, в частности,
если используются абсолютные пути.

Та же техника может быть использована для управления опциями командной
строки. Например, встроенные правила содержат переменную
\variable{TARGET\_ARCH}, которая может быть использована для указания
опций, зависящих от платформы:

{\footnotesize
\begin{verbatim}
ifeq "$(MACHINE)" "hpux-hppa"
  TARGET_ARCH := -mdisable-fpregs
endif
\end{verbatim}
}

При определении собственных программных переменных можно использовать
подобный подход:

{\footnotesize
\begin{verbatim}
MV := mv $(MV_FLAGS)

ifeq "$(MACHINE)" "solaris-sparc"
  MV_FLAGS := -f
endif
\end{verbatim}
}

Если вы переносите продукт на несколько платформ, цепочки секций
условной обработки могут стать неуклюжими и трудными в поддержке.
Вместо использования директивы \directive{ifdef} поместите каждый
набор переменных, зависящих от платформы, в собственный файл, имя
которого содержит название платформы. Например, если вы определяете
платформу по параметрам команды \utility{uname}, можете выбрать
соответствующий файл для включения следующим образом:

{\footnotesize
\begin{verbatim}
MACHINE := $(shell uname -smo | sed 's/ /-/g')
include $(MACHINE)-defines.mk
\end{verbatim}
}

Имена файлов, содержащие пробелы, являются особенно раздражающей
проблемой при использовании \GNUmake{}. Предположение о том, что
пробелы используются для разделения символов при синтаксическом
разборе, является для \GNUmake{} фундаментальным. Множество встроенных
функций, таких как \function{word}, \function{filter} и
\function{wildcard}, предполагают, что их аргументами является список
слов, разделённых пробелами. Тем не менее, есть несколько приёмов,
которые могут немного помочь в этом вопросе.  Первый приём, описанный
в разделе~\nameref{sec:supporting_multiple_binary_trees}
главы~\ref{chap:c_and_cpp}, заключается в замене пробелов другими
символами при помощи функции \function{subst}:

{\footnotesize
\begin{verbatim}
space = $(empty) $(empty)
# $(call space-to-question,file-name)
space-to-question = $(subst $(space),?,$1)
\end{verbatim}
}

Функция \function{space-to-question} заменяет все пробелы символом
вопросительного знака, используемым командным интерпретатором при
определении шаблонов. Теперь мы можем реализовать функции
\function{wildcard} и \function{file-exists}, умеющие правильно
работать с пробелами:

{\footnotesize
\begin{verbatim}
# $(call wildcard-spaces,file-name)
wildcard-spaces = $(wildcard $(call space-to-question,$1))

# $(call file-exists,file-name)
file-exists = $(strip                                \
                $(if $1,,$(warning $1 has no value)) \
                $(call wildcard-spaces,$1))
\end{verbatim}
}

Функция \function{wildcard-spaces} использует
\function{space-to-question} для осуществления операции поиска по
шаблону, содержащему пробелы. Можно использовать функцию
\function{wildcard-spaces} для реализации функции
\function{file-exists}. Разумеется, использование символа знака
вопроса может привести к тому, что функция \function{wildcard-spaces}
будет возвращать файлы, не соответствующие первоначальному шаблону
поиска (например, <<my documents.doc>> и <<my-documents.doc>>), однако
вряд ли можно найти что-то получше.

Функцию \function{space-to-question} можно использовать для
преобразования имён файлов, содержащих пробелы, в спецификациях целей
и реквизитов, поскольку они допускают использование шаблонов:

{\footnotesize
\begin{verbatim}
space := $(empty) $(empty)

# $(call space-to-question,file-name)
space-to-question = $(subst $(space),?,$1)

# $(call question-to-space,file-name)
question-to-space = $(subst ?,$(space),$1)
$(call space-to-question,foo bar): $(call space-to-question,bar baz)
       touch "$(call question-to-space,$@)"
\end{verbatim}
}

Если файл <<\filename{bar baz}>> существует, при первом выполнении
\Makefile{}'а реквизит будет найден, поскольку существующий файл
соответствует шаблону. Однако поиск файла по шаблону, соответствующего
цели, закончится неудачей, поскольку файл цели не существует. В
результате переменная \variable{\$@} примет значение \texttt{foo?bar}.
После этого командный сценарий вызовет функцию
\function{question-to-space}, чтобы преобразовать значение переменной
\variable{\$@} обратно в имя файла, содержащее пробел. При следующем
запуске файл цели, содержащий в имени пробел, будет найден по шаблону.
Этот приём выглядит немного неуклюже, однако я нашёл ему применение в
реальных \Makefile{}'ах.

%---------------------------------------------------------------------
% Source tree layout
%---------------------------------------------------------------------
\subsection*{Структура каталогов исходного кода}

Другим аспектом переносимости является возможность предоставления
разработчикам свободы в управлении средой разработки по собственному
усмотрению. Если система сборки будет требовать от них, к примеру,
помещать исходный код, бинарные файлы, библиотеки и инструменты
разработки в один и тот же каталог или диск Windows, рано или поздно
возникнут проблемы. В конце концов, разработчики, ограниченные в
дисковом пространстве, будут вынуждены разделить эти файлы.

Вместо этого имеет смысл реализовать \Makefile{} с использованием
переменных для хранения коллекций файлов и инициализировать эти
переменные разумными значениями по умолчанию. Для доступа к каждой
используемой библиотеке или инструменту может быть использована
соответствующая переменная, это позволит разработчикам настраивать
местоположение файлов по собственному усмотрению. Используйте оператор
условного присваивания при определении таких переменных, это даст
разработчикам простой способ переопределения их значений через
переменные окружения.

К тому же, возможность простой поддержки нескольких копий дерева
каталогов с исходными и бинарными файлами является благом для
разработчиков. Даже если им не приходится поддерживать несколько
платформ или использовать различные опции компиляции, разработчикам
часто приходится работать с несколькими рабочими копиями исходного
кода в целях отладки или при параллельной работе в нескольких
проектах. Мы уже рассмотрели два возможных пути реализации этой
возможности: использование высокоуровневых переменных окружения для
идентификации корневого каталога дерева исходных и бинарных файлов,
либо использование каталога, в котором находится \Makefile{}, в
совокупности с фиксированным относительным путём для определения
корневого каталога дерева бинарных файлов. Любой из этих подходов
предоставляет разработчикам механизм для поддержки нескольких деревьев
каталогов.

%%--------------------------------------------------------------------
%% Working with nonportable tools
%%--------------------------------------------------------------------
\section{Работа с непереносимыми инструментами}

Как уже было замечено, одной из альтернатив написания \Makefile{}'ов
по принципу наименьшего общего знаменателя является адаптация
стандартного набора инструментов. Разумеется, цель этого подхода~---
убедиться в том, что стандартный набор инструментов по меньшей мере
так же переносим, как и ваше приложение. Очевидным выбором переносимых
инструментов является набор программ проекта GNU, однако существует
довольно много проектов переносимых инструментов. Два других
инструмента, приходящие на ум~--- Perl и Python.

При отсутствии переносимых инструментов хорошей альтернативой является
инкапсуляция непереносимых инструментов в функции \GNUmake{}.
Например, для поддержки различных компиляторов Enterprise JavaBeans
(каждый из которых имеет собственный синтаксис вызова), мы можем
написать функцию для компиляции архива EJB и параметризовать её для
возможности подключения другого компилятора.

{\footnotesize
\begin{verbatim}
EJB_TMP_JAR = $(TMPDIR)/temp.jar
# $(call compile-generic-bean, bean-type, jar-name,
#        bean-files-wildcard, manifest-name-opt )
define compile-generic-bean
  $(RM) $(dir $(META_INF))
  $(MKDIR) $(META_INF)
  $(if $(filter %.xml %.xmi, $3),             \
    cp $(filter %.xml %.xmi, $3) $(META_INF))
  $(call compile-$1-bean-hook,$2)
  cd $(OUTPUT_DIR) &&                         \
  $(JAR) -cf0 $(EJB_TMP_JAR)                  \
         $(call jar-file-arg,$(META_INF))     \
         $(call bean-classes,$3)
  $(call $1-compile-command,$2)
  $(call create-manifest,$(if $4,$4,$2),,)
endef
\end{verbatim}
}

Первым аргументом этой общей функции компиляции EJB~--- это тип
компилятора компонентов, такого как Weblogic, Websphere и т.д.
Остальными аргументами являются имя архива, список файлов архива
(включая конфигурационные файлы) и необязательный файл манифеста.
Сначала шаблонная функция создаёт пустой временный каталог, удаляя и
создавая заново предыдущий временный каталог. Затем функция производит
копирование \filename{xml} и \filename{xmi} файлов, указанных в
качестве реквизитов каталога \variable{\$(META\_INF)}. На данном этапе
нам может понадобиться осуществление вспомогательных действий, будь то
очистка каталога \filename{META-INF} или подготовка \filename{.class}
файлов. Для поддержки этих операций мы включили функцию-триггер,
\function{compile-\$1-bean-hook}, которую пользователь может
реализовать по собственному усмотрению. Например, если компилятор
Websphere требует дополнительный контрольный файл, например,
\filename{xsl} файл, мы можем реализовать триггер следующим образом:

{\footnotesize
\begin{verbatim}
# $(call compile-websphere-bean-hook, file-list)
define compile-websphere-bean-hook
  cp $(filter %.xsl, $1) $(META_INF)
endef
\end{verbatim}
}

Просто определив эту функцию, мы убеждаемся в том, что вызов
\function{call} в функции \function{compile-generic-bean} будет
осуществлён успешно. Если не будем писать триггер, соответствующий
вызов в \function{compile-generic-bean} вычислится в пустую строку.

Затем наша функция создаёт jar архив. Вспомогательная функция
\function{jar\hyp{}file\hyp{}arg} производит преобразование обычного
пути к файлу в конкатенацию опции \texttt{-C} и относительного пути:

{\footnotesize
\begin{verbatim}
# $(call jar-file-arg, file-name)
define jar-file-arg
  -C "$(patsubst %/,%,$(dir $1))" $(notdir $1)
endef
\end{verbatim}
}

Вспомогательная функция \function{bean\hyp{}classes} извлекает
подходящий class файл из списка исходных файлов (в jar архив нужно
включать только интерфейсы и home классы):

{\footnotesize
\begin{verbatim}
# $(call bean-classes, bean-files-list)
define bean-classes
  $(subst $(SOURCE_DIR)/,,                 \
    $(filter %Interface.class %Home.class, \
      $(subst .java,.class,$1)))
endef
\end{verbatim}
}

Затем общая функция вызывает соответствующую команду компиляции
\texttt{\$(call \$1\hyp{}compile\hyp{}command,\$2)}:

{\footnotesize
\begin{verbatim}
define weblogic-compile-command
  cd $(TMPDIR) && \
  $(JVM) weblogic.ejbc -compiler $(EJB_JAVAC) $(EJB_TMP_JAR) $1
endef
\end{verbatim}
}

Наконец, общая функция добавляет файл манифеста.

После определения функции \function{compile-generic-bean} мы можем
обернуть её вызов в специальную функцию для каждого компилятора,
который мы хотим поддерживать.

{\footnotesize
\begin{verbatim}
# $(call compile-weblogic-bean, jar-name,
#        bean-files-wildcard, manifest-name-opt )
define compile-weblogic-bean
  $(call compile-generic-bean,weblogic,$1,$2,$3)
endef
\end{verbatim}
}

%---------------------------------------------------------------------
% A standard shell
%---------------------------------------------------------------------
\subsection*{Стандартный интерпретатор}

Следует ещё раз подчеркнуть, что одним из самых досадных источников
непереносимости при переходе на другую систему являются возможности
интерпретатора \filename{/bin/sh}, используемого \GNUmake{} по
умолчанию. Если вам приходится настраивать командные сценарии вашего
\Makefile{}'а, рассмотрите возможность стандартизации вашего
интерпретатора. Разумеется, это не очень подходит для типичных
проектов с открытым исходным кодом, \Makefile{}'ы которых выполняются
в неконтролируемой среде. Однако в случае, если вы управляете средой и
контролируете число машин, которые нужно настроить, такой подход
вполне разумен.

Многие интерпретаторы предоставляют возможности, которые могут
исключить использование большого числа небольших программ. Например,
\utility{bash} включает расширенные возможности работы с переменными,
такие как \texttt{\%\%} и \texttt{\#\#}, которые могут помочь избежать
использования инструментов, таких как \utility{sed} и \utility{expr}.

%%--------------------------------------------------------------------
%% Automake
%%--------------------------------------------------------------------
\section{Automake}

В этой главе мы сосредоточились на использовании GNU \GNUmake{} и
эффективной поддержке инструментов для достижения переносимости систем
сборки. Однако иногда даже эти скромные цели недостижимы. Если вы не
можете использовать мощные возможности GNU \GNUmake{} и вынуждены
полагаться на ограниченный набор возможностей, продиктованный подходом
наименьшего общего знаменателя, вам следует рассмотреть
\index{automake}
возможность использования программы \utility{automake},
\filename{\url{http://www.gnu.org/software/automake/automake.html}}.


Программа \utility{automake} принимает на вход стилизованный
\Makefile{} и производит переносимый \Makefile{}. Работа
\utility{automake} основывается на применении макроязыка \utility{m4},
допускающего довольно сжатый синтаксис входных файлов (обычно их имя
\filename{makefile.am}). Как правило, \utility{automake} используется
в совокупности с программой \utility{autoconf}, пакетом поддержки
переносимости программ, написанных на \Clang{}/\Cplusplus{}, однако
использование \utility{autoconf} не обязательно.

В то время как \utility{automake} является хорошим решением для систем
сборки, требующих максимальной переносимости, \Makefile{}'ы, которые
производит эта программа, не имеют доступа к богатым возможностям GNU
\GNUmake{} (за исключением оператора \texttt{+=}, для поддержки
которого используются особые средства). Более того, синтаксис входных
файлов \utility{automake} имеет мало общего с синтаксисом обычных
\Makefile{}'ов. Поэтому использование \utility{automake}
(без \utility{autoconf}) не очень сильно отличается от подхода
наименьшего общего знаменателя.

