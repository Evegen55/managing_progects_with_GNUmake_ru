%%--------------------------------------------------------------------
%% Automake
%%--------------------------------------------------------------------
\section{Automake}

В этой главе мы сосредоточились на использовании GNU \GNUmake{} и
эффективной поддержке инструментов для достижения переносимости систем
сборки. Однако иногда даже эти скромные цели недостижимы. Если вы не
можете использовать мощные возможности GNU \GNUmake{} и вынуждены
полагаться на ограниченный набор возможностей, продиктованный подходом
наименьшего общего знаменателя, вам следует рассмотреть
\index{automake}
возможность использования программы \utility{automake},
\filename{\url{http://www.gnu.org/software/automake/automake.html}}.


Программа \utility{automake} принимает на вход стилизованный
\Makefile{} и производит переносимый \Makefile{}. Работа
\utility{automake} основывается на применении макроязыка \utility{m4},
допускающего довольно сжатый синтаксис входных файлов (обычно их имя
\filename{makefile.am}). Как правило, \utility{automake} используется
в совокупности с программой \utility{autoconf}, пакетом поддержки
переносимости программ, написанных на \Clang{}/\Cplusplus{}, однако
использование \utility{autoconf} не обязательно.

В то время как \utility{automake} является хорошим решением для систем
сборки, требующих максимальной переносимости, \Makefile{}'ы, которые
производит эта программа, не имеют доступа к богатым возможностям GNU
\GNUmake{} (за исключением оператора \texttt{+=}, для поддержки
которого используются особые средства). Более того, синтаксис входных
файлов \utility{automake} имеет мало общего с синтаксисом обычных
\Makefile{}'ов. Поэтому использование \utility{automake}
(без \utility{autoconf}) не очень сильно отличается от подхода
наименьшего общего знаменателя.
