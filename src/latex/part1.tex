%%%%------------------------------------------------------------------
%%%% Basic concepts
%%%%------------------------------------------------------------------
\part{Основные концепции}
\label{part:basics}

В этой части мы сфокусируемся на возможностях \GNUmake{} и их
правильном использовании. Мы начнём с краткого введения и обзора,
которых должно быть достаточно для того, чтобы вы могли написать свой
первый \Makefile{}.  Главы этой части покрывают правила, переменные,
функции и сценарии сборки.

После прочтения этой части у вас будет законченное представление о
том, как работает GNU \GNUmake{}.

%%%-------------------------------------------------------------------
%%% How to write a simple makefile
%%%-------------------------------------------------------------------
\chapter{Как написать простой \Makefile{}}
\label{chap:simple_makefile}
Механизм создания программ обычно довольно прост: редактирование
исходного кода, компиляция в исполняемый файл и отладка полученной
программы. Несмотря на то, что компиляция считается рутинной
операцией, будучи проведённой некорректно, она может породить ошибки,
на исправление которых у программиста уйдёт очень много времени.
Многие разработчики испытывают удивление, когда после исправления
какой-нибудь функции и запуска нового варианта программы видят,
что их изменения не исправляют ошибок. Позже они обнаружат, что их
модифицированный код никогда не выполнялся по причине ошибки процедуры
компиляции, компоновки или сборки JAR\hyp{}архива. Более того, с
ростом сложности программ эти рутинные задачи становятся источниками
всё более сложных ошибок, поскольку одновременно могут разрабатываться
различные версии приложения, например, для разных платформ или с
использованием разных версий библиотек.

Программа \GNUmake{} была разработана для того, чтобы автоматизировать
рутинные задачи трансформации исходного кода в исполняемые файлы.
Преимущество \GNUmake{} над простыми сценариями командного
интерпретатора состоит в возможности описать отношения между
элементами проекта. \GNUmake{}, основываясь на описанных отношениях и
данных о времени модификации файлов, сможет определить, какие именно
шаги необходимо осуществить для получения необходимой вам версии
программы. Используя эту информацию, \GNUmake{} также сможет
оптимизировать процесс сборки и избежать выполнения ненужных действий.

GNU \GNUmake{} предоставляет язык для описания отношений между
исходным кодом, промежуточными и исполняемыми файлами.
\GNUmake{} также включает функциональность для управления
конфигурациями, реализации библиотек спецификаций, пригодных для
повторного использования, и параметризации процесса сборки через
механизм макросов, определяемых пользователем. \GNUmake{} может
рассматриваться как каркас всего процесса разработки, выделяющий
компоненты приложения и описывающий способы связать их в единое целое.

Спецификация, используемая \GNUmake{}, обычно сохраняется в файле с
именем \Makefile{}. Ниже приведён пример \Makefile{}'а пригодного для
сборки традиционной программы <<Hello, World>>:

{\footnotesize
\begin{verbatim}
hello: hello.c
    gcc hello.c -o hello
\end{verbatim}
}

Для того, чтобы собрать программу, достаточно выполнить в командном
интерпретаторе следующую команду:

{\footnotesize
\begin{alltt}
\$ \textbf{make}
\end{alltt}
}

Это приведёт к запуску \GNUmake{}, который затем прочитает \Makefile{}
и соберёт первую цель, определённую в нём. В итоге вы увидите
следующее:

{\footnotesize
\begin{alltt}
\$ \textbf{make}
gcc hello.c -o hello
\end{alltt}
}

Цель может быть передана в качестве аргумента командной строки,
в этом случае \GNUmake{} будет пытаться собрать указанную вами цель.
В противном случае для сборки будет выбрана первая цель, определённая
\index{Цели!по умолчанию}
в \Makefile{}'е (называемая также \newword{целью по умолчанию}).

Обычно цель по умолчанию предназначена для сборки всего приложения, и
процесс сборки этой цели состоит из определённой последовательности
шагов.  Например, довольно часто исходный код приложения должен быть
составлен при помощи программ наподобие \utility{flex} или
\utility{bison}. Полученный исходный код нужно скомпилировать в
объектные файлы (файлы с расширением \filename{.o} или \filename{.obj}
для \Clang{}/\Cplusplus{} и \filename{.class} для Java).  Затем
объектные файлы (для случая \Clang{}/\Cplusplus{}) должны быть собраны
компоновщиком (обычно вызываемым компилятором \utility{gcc}) для
получения исполняемого файла.

Модификация любого из исходных файлов требует повторного вызова
\GNUmake{}, который инициирует повторение некоторых (обычно не всех)
из вышеописанных действий таким образом, чтобы получить исполняемый
файл с новой функциональностью. Файл спецификаций, \Makefile{},
описывает отношения между исходным кодом, промежуточными и
исполняемыми файлами, что позволяет \GNUmake{} выполнять минимум
необходимой работы для получения новой версии исполняемого файла.

Принципиальное значение \GNUmake{} заключается в его возможности
осуществлять сложные последовательности операций, необходимые для
сборки приложения, и производить оптимизацию выполнения этих операций
для максимального сокращения времени, занимаемого циклом
<<модификация\hyp{}компиляция\hyp{}отладка>>. Более того, этот
инструмент настолько гибок, что может быть использован везде, где
существуют зависимости между файлами, начиная с традиционных программ
на \Clang/\Cplusplus{} и заканчивая приложениями на \Java{},
документами \TeX{}, управлением базами данных и конвертацией
изображений.

%%--------------------------------------------------------------------
%% Targets and prerequisites
%%--------------------------------------------------------------------
\section{Цели и реквизиты}
По существу \Makefile{} содержит набор правил для сборки приложения.
\index{Правила!по умолчанию}
Первое правило, обнаруженное \GNUmake{}, становится \newword{правилом
по умолчанию}. Правила состоят из трёх составляющих: целей, реквизитов
и сценария сборки:

{\footnotesize
\begin{alltt}
\emph{цель: реквизит\subi{1} реквизит\subi{2}
    сценарий сборки
}
\end{alltt}
}

\index{Цели}
\newword{Цель}~--- это файл или некоторая сущность, требующая
обновления.
\index{Реквизиты}
\newword{Реквизиты}~--- это те файлы, которые должны существовать для
того, чтобы цель могла быть собрана.
\index{Сценарий сборки}
\newword{Сценарий сборки}~--- это команды интерпретатора, описывающие
способ создания цели из реквизитов.

Вот правила компиляции исходного файла на языке \Clang{}
\filename{foo.c} в объектный файл \filename{foo.o}:

{\footnotesize
\begin{verbatim}
foo.o: foo.c foo.h
    gcc -c foo.c
\end{verbatim}
}

Цель \filename{foo.o} указана слева от двоеточия, реквизиты
\filename{foo.c} и \filename{foo.h}~--- справа. Сценарий сборки обычно
начинается со следующей строки и предваряется символом табуляции.

Когда требуется выполнить правило, \GNUmake{} пытается найти все
файлы-реквизиты, ассоциированные с целью. Если хотя\hyp{}бы один из
реквизитов выступает в качестве цели какого\hyp{}либо правила, то
сначала выполняется правило для этого реквизита. Далее проверяются
даты последней модификации файла\hyp{}цели и реквизитов: если
какой-либо из реквизитов модифицировался позже цели, то выполняется
сценарий сборки. Каждая строка сценария выполняется в отдельном
процессе командного интерпретатора. Если хотя\hyp{}бы одна команда
приводит к ошибке, сборка цели завершается и \GNUmake{} прекращает
выполнение.

Ниже представлена программа подсчёта числа вхождений слов <<fee>>,
<<fie>>, <<foe>> и <<fum>> в содержимое стандартного потока ввода,
реализованная с помощью лексического анализатора \utility{flex}.

{\footnotesize
\begin{verbatim}
#include <stdio.h>

extern int fee_count, fie_count, foe_count, fum_count;

extern int yylex( void );

int main( int argc, char ** argv )
{
    yylex( );

    printf( "%d %d %d %d \n",
        fee_count, fie_count, foe_count, fum_count );

    exit( 0 );
}
\end{verbatim}
}

Код лексического анализатора очень прост:

{\footnotesize
\begin{verbatim}
        int fee_count = 0;
        int fie_count = 0;
        int foe_count = 0;
        int fum_count = 0;
%%
fee     fee_count++;
fie     fie_count++;
foe     foe_count++;
fum     fum_count++;
\end{verbatim}
}

Составление \Makefile{} для сборки этой программы также не вызывает
затруднений:

{\footnotesize
\begin{verbatim}
count_words: count_words.o lexer.o -lfl
        gcc count_words.o lexer.o -lfl -o count_words

count_words.o: count_words.c
        gcc -c count_words.c

lexer.o: lexer.c
        gcc -c lexer.c

lexer.c: lexer.l
        flex -t lexer.l > lexer.c
\end{verbatim}
}

Когда мы запустим \GNUmake{} впервые, мы увидим следующее:

{\footnotesize
\begin{alltt}
\$ \textbf{make}

gcc -c count\_words.c
flex -t lexer.l > lexer.c
gcc -c lexer.c
gcc counti\_words.o lexer.o -lfl -ocount\_words
\end{alltt}
}

Теперь мы имеем исполняемый файл программы. Естественно, реальные
приложения, как правило, состоят из большего числа модулей, чем наш
пример. Кроме того, позже мы увидим, что наш \Makefile{} не использует
большей части функциональности \GNUmake{} и потому гораздо более
объёмен чем мог бы быть. Тем не менее, это всё-же функциональный и
полезный \Makefile{}.

Как вы могли заметить, порядок, в котором команды реально выполняются,
и порядок их появления в \Makefile{}'е противоположны. Стиль
опеделения <<сверху вниз>> является общим стилем составления
спецификаций для \GNUmake{}. Обычно сначала описываются правила для
наиболее важных целей, а детали оставляются на потом. В \GNUmake{}
существует несколько механизмов поддержки такого стиля. Главными среди
\index{Переменные!рекурсивные}
них являются двухфазовая модель выполнения \GNUmake{} и рекурсивные
переменные. Мы обсудим эти важные детали в последующих главах.

%%--------------------------------------------------------------------
%% Dependency checking
%%--------------------------------------------------------------------
\section{Разрешение зависимостей}

Как же \GNUmake{} решает, что делать? Давайте рассмотрим выполнение
предыдущего примера более детально и выясним это.

Сначала \GNUmake{} замечает, что командная строка не содержит
аргументов и решает собрать цель по умолчанию, т.е.
\filename{count\_words}. Затем выполняется проверка реквизитов, их
обнаруживается три: \filename{count\_words.o}, \filename{lexer.o} и
\filename{-lfl}.  Затем \GNUmake{} ищет способ собрать цель
\filename{count\_words.o} и находит правило для неё. Снова происходит
проверка реквизитов. \GNUmake{} замечает, что цель
\filename{count\_words.c} не имеет правила, и файл с таким именем
существует, поэтому производится трансформация
\filename{count\_words.c} в \filename{count\_words.o}, для
осуществления которой выполняется следующая команда:

{\footnotesize
\begin{verbatim}
gcc -c count_words.c
\end{verbatim}
}

Подобная цепь переходов от целей к реквизитам и рассмотрение
реквизитов в качестве целей~--- типичный механизм, при помощи которого
\GNUmake{} проводит анализ \Makefile{}'а и решает, какие команды
подлежат выполнению.

Следующим реквизитом, рассматриваемым \GNUmake{}, является
\filename{lexer.o}. Цепочка правил, подобная рассмотренной, ведёт к
файлу \filename{lexer.c}, ещё не существующему. Далее \GNUmake{}
находит правило получения \filename{lexer.c} из \filename{lexer.l},
\index{flex}
согласно которому запускается программа \utility{flex}.
\index{gcc}
Теперь \filename{lexer.c} существует и можно запускать \utility{gcc}.

\index{Библиотека}
Наконец, \GNUmake{} проверяет реквизит \filename{-lfl}. Опция
\command{-l} сообщает \utility{gcc}, что приложение использует
системную библиотеку. Имя библиотеки <<fl>> преобразуется согласно
правилам именования библиотек в \filename{libfl.a}. GNU \GNUmake{}
имеет поддержку этого синтаксиса. Когда находится реквизит вида
\filename{-l<NAME>}, \GNUmake{} ищет файл с именем
\filename{libNAME.so} в стандартных каталогах библиотек. Если поиск
заканчивается неудачей, производится повторный поиск по имени
\filename{libNAME.a}.  В нашем случае поиск успешен, \GNUmake{}
находит файл \filename{/usr/lib/libfl.a} и производит финальное
действие~--- компоновку.

%%--------------------------------------------------------------------
%% Minimazing rebuilds
%%--------------------------------------------------------------------
\section{Минимизируем число действий}

Если мы запустим нашу программу, то обнаружим, что помимо вывода числа
вхождений слов fee, fies, foes и fums она также выводит содержимое
входного файла. Это совсем не то, что от неё ожидалось. Проблема в
том, что мы забыли добавить несколько правил в наш лексический
\index{flex}
анализатор, и \utility{flex} выдаёт нераспознанный текст на
стандартный поток вывода. Для решения этой проблемы мы просто добавим
правило для <<любого символа>>:

{\footnotesize
\begin{verbatim}
        int fee_count = 0;
        int fie_count = 0;
        int foe_count = 0;
        int fum_count = 0;
%%
fee     fee_count++;
fie     fie_count++;
foe     foe_count++;
fum     fum_count++;
.
\n
\end{verbatim}
}

После редактирования файла с исходным кодом анализатора нам нужно
собрать наше приложение заново и проверить его работу:

{\footnotesize
\begin{alltt}
\$ \textbf{make}

flex -t lexer.l > lexer.c
gcc -c lexer.c
gcc count\_words.o lexer.o -lfl -ocount\_words
\end{alltt}
}

На этот раз файл \filename{count\_words.c} не подвергся компиляции.
При анализе правил \GNUmake{} обнаружил, что файл
\filename{count\_words.o} существует и имеет более позднюю дату
модификации, чем цель, потому не было предпринято никаких действий по
сборке этой цели. Однако при анализе цели \filename{lexer.c} было
обнаружено, что реквизит \filename{lexer.l} имеет более позднюю дату
модификации, поэтому произошла повторная сборка цели
\filename{lexer.c}. Это, в свою очередь, вызвало повторную сборку
\filename{lexer.o}, а затем и \filename{count\_words}. Теперь наша
программа подсчёта слов работает правильно:

{\footnotesize
\begin{alltt}
\$ \textbf{count\_words < lexer.l}
3 3 3 3
\end{alltt}
}

%%--------------------------------------------------------------------
%% Invoking make
%%--------------------------------------------------------------------
\section{Вызов \GNUmake{}}
\label{sec:invoking_make}

В предыдущих примерах предполагается, что:
\begin{itemize}
%---------------------------------------------------------------------
\item Все исходные файлы проекта и файл спецификации \GNUmake{}
хранятся в одной директории.
%---------------------------------------------------------------------
\item Файл спецификации для \GNUmake{} называется
\filename{makefile}, \filename{Makefile} или \filename{GNUMakefile}.
%---------------------------------------------------------------------
\item \Makefile{} находится в текущей директории, когда пользователь
запускает команду \GNUmake{}.
%---------------------------------------------------------------------
\end{itemize}

Если \GNUmake{} запускается с соблюдением этих условий, автоматически
производится попытка собрать первую цель в файле спецификации. Чтобы
собрать другую цель (или несколько целей), необходимо указать имя
этой цели в качестве аргументов командной строки:

{\footnotesize
\begin{alltt}
\$ \textbf{make lexer.c}
\end{alltt}
}

В этом случае после запуска \GNUmake{} произведёт чтение файла
спецификации и определение цели для обновления. Если цель или один из
реквизитов устарели (или не существуют), тогда в точности один раз
\index{Сценарий сборки}
будет выполнен сценарий их сборки. После выполнения сценария цель
предполагается существующей и обновлённой, и процесс повторяется для
следующей цели; после сборки всех целей процесс завершается.

Если указанная вами цель не требует повторной сборки, \GNUmake{}
завершит работу с сообщением следующего вида:

{\footnotesize
\begin{alltt}
\$ \textbf{make lexer.c}
make:  `lexer.c' is up to date.
\end{alltt}
}

Если в качестве цели будет указана цель, не присутствующая в
\index{Правила!неявные}
\Makefile{}'е, и для которой не существует неявного правила (см.
главу~\ref{chap:rules}), то \GNUmake{} закончит выполнение с
сообщением следующего вида:

{\footnotesize
\begin{alltt}
\$ \textbf{make non-existent-target}
make: *** No rule to make target `non-existent-target'. Stop.
\end{alltt}
}

У команды \GNUmake{} есть множество опций командной строки. Одной из
\index{Опции!just-print@\command{-{}-just-print (-n)}}
самых полезных является опция \command{-{}-just\hyp{}print} (или просто
\command{-n}), которая сообщает \GNUmake{}, что нужно только отобразить
последовательность действий, необходимых для сборки цели. Эту
возможность очень удобно использовать для отладки \Makefile{}'ов.
Также полезной возможностью является передача или переопределение
значений переменных \GNUmake{} через аргументы командной строки.

%%--------------------------------------------------------------------
%% Basic makefile syntax
%%--------------------------------------------------------------------
\section{Основы синтаксиса \Makefile{}}
Теперь, когда вы уже получили базовое представление о \GNUmake{},
можно приступить к составлению собственных \Makefile{}'ов. В этом
разделе мы рассмотрим синтаксис и структуру \Makefile{}'а, чтобы вы
могли начать использование \GNUmake{}. 

Как правило, \Makefile{}'ы пишутся в манере <<сверху вниз>>, то есть
сначала описывается наиболее общая цель (как правило, она имеет имя
\index{Цели!по умолчанию} \index{Цели!стандартные!all}
\target{all}), которая будет целью по умолчанию. Далее следуют всё
более детализированные цели, и в самом конце описываются цели,
используемые для поддержки программного продукта (такие, например, как
\index{Цели!стандартные!clean}
\target{clean}, используемая для удаления ненужных вр'{е}менных
файлов). Именами целей вовсе не обязательно должны быть настоящие
файлы, можно использовать любое имя.

\index{Правило}
В предыдущем примере мы видели упрощённую форму правила. Более полной
(но всё ещё не законченной) формой правила является следующая:

{\footnotesize
\begin{alltt}
\emph{цель\subi{1} цель\subi{2} цель\subi{3} : реквизит\subi{1} реквизит\subi{2}
    команда\subi{1}
    команда\subi{2}
    команда\subi{3}
}
\end{alltt}
}

Одна или более целей указываются слева от двоеточия, справа от него
следуют реквизиты. Если реквизиты не указаны, то собираются только те
цели, которые ещё не существуют. Последовательность команд, которые
необходимо выполнить для сборки цели, иногда называют
\index{Сценарий сборки}
\newword{сценарием сборки}, но чаще просто \newword{командами}.

Каждая команда должна начинаться с символа табуляции. Такой синтаксис
сообщает \GNUmake{} о том, что следующие за табуляцией символы должны
быть переданы в командный интерпретатор для последующего выполнения.
Если вы случайно поставите символ табуляции в начале строки, не
являющейся командой, то в большинстве случаев \GNUmake{} будет
интерпретировать последующий текст в этой строке как команду. Однако
может случиться и так, что \GNUmake{} сможет распознать ваш символ
табуляции как синтаксическую ошибку, в этом случае вы увидите
сообщение, подобное следующему:

{\footnotesize
\begin{alltt}
\$ \textbf{make}
Makefile:6: *** commands commence before first target.  Stop.
\end{alltt}
}

Мы вернёмся к аспектам использования символа табуляции в
главе~\ref{chap:rules}.


%%%-------------------------------------------------------------------
%%% Rules
%%%-------------------------------------------------------------------
\chapter{Правила}
\label{chap:rules}

В предыдущей главе мы рассмотрели несколько правил для компиляции и
компоновки программы подсчёта слов. Каждое из этих правил определяло
цель~--- файл, который требуется собрать. Каждая цель зависела от
множества реквизитов, которые также являлись файлами. Когда
требовалось обновить цель, \GNUmake{} выполнял сценарий сборки только
в том случае, если файлы реквизитов имели дату модификации более
позднюю, чем цель. Поскольку цель одного правила может быть реквизитом
другого, множество целей и реквизитов может быть представлено в форме
\index{Граф зависимостей}
\newword{графа зависимостей} (\newword{dependency graph}). Составление
и обработка графа зависимостей является основной работой \GNUmake{}.

\index{Правила!явные}
Поскольку правила так важны для \GNUmake{}, существует несколько их
разновидностей. \newword{Явные правила}, наподобие тех, что были
использованы в предыдущей главе, указывают на необходимость обновления
цели при модификации или отсутствии файлов-реквизитов. Правила этого
типа вы будете писать наиболее часто.
\index{Правила!шаблонные}
\newword{Шаблонные правила} используют подстановки (wildcards) вместо
явного указания имён файлов.  Это позволяет \GNUmake{} применять такие
правила каждый раз при соответствии имени цели некоторому шаблону.
\index{Правила!неявные}
\newword{Неявные правила}~--- это явные или суффиксные правила,
встроенные в базу данных правил \GNUmake{}. Наличие встроенной базы
данных правил упрощает написание \Makefile'ов, поскольку для многих
общих задач уже известны типы файлов, суффиксы и сценарии сборки
целей.
\index{Правила!шаблонные!статические}
\newword{Статические шаблонные правила} отличаются от обычных
шаблонных правил тем, что могут быть применены только к определённому
списку целей.

GNU \GNUmake{} может быть использован как замена для многих других
версий \GNUmake{}. Он включает в себя множество возможностей,
сохранённых для поддержания обратной совместимости. Например,
\index{Правила!суффиксные}
\newword{Суффиксные правила} были реализованы в одной из первых версий
\GNUmake{} для написания общих правил. \GNUmake{} имеет поддержку
сиффиксных правил, однако они признаны устаревшими, поскольку могут
быть заменены более простыми и более \'{о}бщными шаблонными правилами.

%%--------------------------------------------------------------------
%% Explicit rules
%%--------------------------------------------------------------------
\section{Явные правила}
\label{sec:explicit_rules}

Чаще всего вам придётся писать именно явные правила, указывающие
некоторые файлы как цели и реквизиты. Правило может иметь более одной
цели. Это значит, что каждая из указанных целей имеет в точности то же
множество реквизитов, что и остальные цели. Если цели требуется
обновить, для каждой из них будет выполнен один и тот же сценарий. Вот
пример такого правила:

{\footnotesize
\begin{verbatim}
vpath.o variable.o: make.h config.h getopt.h gettext.h dep.h
\end{verbatim}
}

Это правило означает, что цели \filename{vpath.o} и
\filename{variable.o} зависят от одного и того же множества
заголовочных файлов. Оно имеет в точности тот же эффект, что и
следующая спецификация:

{\footnotesize
\begin{verbatim}
vpath.o: make.h config.h getopt.h gettext.h dep.h

variable.o: make.h config.h getopt.h gettext.h dep.h
\end{verbatim}
}

Обе цели собираются независимо. Если один из объектных файлов имеет
более раннюю дату модификации, чем один из указанных заголовочных
файлов, \GNUmake{} инициирует сборку и выполнит команды,
ассоциированные с правилом.

Правило не обязательно указывать полностью сразу. Каждый раз, когда
\GNUmake{} обнаруживает файл в качестве цели, он добавляет цель и
реквизиты в граф зависимостей. Если такая цель уже существовала в
графе, к записи о цели добавляются новые реквизиты. Одним из
элементарных применений этого свойства является разбиение длинной
строки на несколько более коротких для улучшения читабельности файла:

{\footnotesize
\begin{verbatim}
vpath.o: make.h config.h getopt.h gettext.h dep.h
vpath.o: filedef.h hash.h job.h commands.h variable.h vpath.h
\end{verbatim}
}

В наиболее сложных случаях список реквизитов может состоять из файлов, способы
обработки которых различны:

{\footnotesize
\begin{verbatim}
# Убедимся, что файл lexer.c существует до компиляции vpath.c
vpath.o: lexer.c

...

# Компилируем vpath.c с определёнными флагами
vpath.o: vpath.c
	$(COMPILE.c) $(RULE_FLAGS) $(OUTPUT_OPTION) $<

...

# Включаем файл зависимостей, составленный программой
include auto-generated-dependencies.d
\end{verbatim}
}

Первое правило декларирует, что цель \filename{vpath.o} должна быть
собрана заново при изменении файла \filename{lexer.c} (возможно,
генерация этого файла имеет некий побочный эффект). Правило также
может быть использовано для того, чтобы убедиться, что все реквизиты
существуют (или, в случае необходимости, обновлены) перед сборкой
цели. Нужно отметить двустороннюю сущность правил. При прямом чтении
правило означает, что, если файл \filename{lexer.c} изменился,
требуется выполнить действия по обновлению \filename{vpath.o}. При
чтении в обратном направлении правило означает, что если требуется
обновить \filename{vpath.o}, то нужно убедиться, что файл
\filename{lexer.c} существует. Это правило может быть помещено рядом с
остальными правилами, касающимися файла \filename{lexer.c}, чтобы
разработчики помнили об этой тонкой взаимосвязи. Далее, рассмотрим
правило компиляции \filename{vpath.o}. Сценарий сборки для этого
правила использует три переменных \GNUmake{}. Переменные будут
детально описаны позже, пока важно знать лишь то, что обращение к
переменной происходит с помощью знака доллара (\command{\$}), за
которым следует либо один символ, либо слово в круглых скобках.
Наконец, зависимости типа \filename{.o/.h} включаются из отдельного
файла, полученного при помощи внешней программы.

В качестве особого случая GNU \GNUmake{} поддерживает упрощённый
синтаксис для правил с одной командой.

{\footnotesize
\begin{alltt}
\emph{цель: ; команда}
\end{alltt}
}

На практике такие правила встречаются редко, однако всё же иногда они
могут быть полезны, особенно когда нужно сберечь место на экране
монитора или листе бумаги. 

%---------------------------------------------------------------------
% Wildcards
%---------------------------------------------------------------------
\subsection{Шаблоны}
Часто \Makefile'ы содержат огромное количество файлов. Для упрощения
работы с ними \GNUmake{} поддерживает шаблоны, идентичные шаблонам
командного интерпретатора \utility{Bourne shell}: \verb|~|, \verb|*|,
\verb|?|, \verb|[...]| и \verb|[^...]|.  Например, шаблону \verb|*.*|
соответствуют все файлы, содержащие в имени точку. Знак вопроса
означает один символ, а \verb|[...]|~--- класс символов. Для выбора
дополнения класса символов нужно использовать \verb|[^...]|. Знак
тильды (\verb|~|) может быть использован для обозначения домашнего
каталога текущего пользователя системы.  Если за тильдой следует имя
пользователя, будет подставлен домашний каталог указанного
пользователя. \GNUmake{} автоматически раскрывает шаблоны, когда они
встречаются в названиях целей, реквизитов или командных сценариях. В
другом контексте шаблоны могут быть раскрыты явным вызовом функции.
Шаблоны чрезвычайно полезны для написания более адаптивных
\Makefile{}'ов. Например, вместо того, чтобы явно перечислять все
файлы, входящие в состав исходного кода программы, вы можете
использовать шаблоны\footnote{В более серьёзных приложениях применение
шаблонов для выбора компилируемых файлов является плохой практикой,
поскольку может вызвать компоновку с посторонним опасным кодом. В
правилах удаления промежуточных файлов шаблоны могут быть фатальными
для проекта (прим. автора).}:

{\footnotesize
\begin{verbatim}
prog: *.c
	$(CC) -o $@ $^
\end{verbatim}
}

Однако очень важно быть осторожным с шаблонами, ими легко
злоупотребить. Рассмотрим пример:

{\footnotesize
\begin{verbatim}
*.o: constants.h
\end{verbatim}
}

Намерения очевидны: все объектные файлы зависят от заголовочного файла
\filename{constants.h}. Однако посмотрим, как раскроется шаблон в
каталоге, не содержащем объектных файлов:

{\footnotesize
\begin{verbatim}
: constants.h
\end{verbatim}
}

Это допустимое выражение \GNUmake{}, оно не вызовет ошибки, однако оно
также не выразит той зависимости, которую имел в виду пользователь.
Одним из корректных способов реализации этого правила является
использование шаблона для получения файлов с исходным кодом (которые,
как правило, присутствуют) и трансформация полученного списка в список
объектных файлов. Мы рассмотрим эту технику при обсуждении функций
в главе~{\ref{chap:functions}}.

Наконец, стоит отметить, что раскрытие шаблонов в тот момент, когда
они появляются в качестве целей или реквизитов, осуществляет
непосредственно \GNUmake{}. Однако раскрытие шаблонов в сценариях
происходит в дочернем процессе командного интерпретатора. Это может
быть важной деталью, поскольку \GNUmake{} раскрывает шаблоны во время
чтения \Makefile{}'а, а командный интерпретатор раскрывает их много
позже, во время непосредственного выполнения команд. Когда
производятся сложные манипуляции с файлами, результаты раскрытия
одинаковых шаблонов в разные моменты времени могут сильно отличаться.
Проблематичной может быть ситуация, когда некоторые файлы являются
результатом сборки, и \GNUmake{} не видит их во время обработки
\Makefile{}'а. К таким случаям нужно относится особенно осторожно.

%---------------------------------------------------------------------
% Phony targets
%---------------------------------------------------------------------
\subsection{Абстрактные цели}
\label{sec:phony_targets}
\index{Цели!абстрактные}
До этого момента все цели и реквизиты, рассматриваемые нами, были
файлами, которые нужно было создать или обновить. Хоть это и типичный
способ использования целей, часто бывает полезным представлять цель в
качестве метки для командного сценария. Например, ранее упоминалось,
что стандартной целью для многих \Makefile{}'ов является \target{all}.
Цели, не представляющие файлов, называют \newword{абстрактными целями}
(\newword{phony targets}). Ещё одной стандартной абстрактной целью
является \target{clean}:

{\footnotesize
\begin{verbatim}
clean:
    rm -f *.o lexer.c
\end{verbatim}
}

Абстрактные цели должны собираться всегда, потому что команды,
ассоциированные с правилом, не создают файл с именем цели.

Важно заметить, что \GNUmake{} не отличает абстрактных целей от
целей, являющимися файлами. Если по случайности файл с именем
абстрактной цели существует, \GNUmake{} будет ассоциировать этот файл
с абстрактной целью в графе зависимостей. Например, если в текущей
директории существует файл \filename{clean}, то запуск команды
\BoldMono{make clean} приведёт к появлению довольно неожиданного
сообщения:

{\footnotesize
\begin{alltt}
\$ \textbf{make clean}
make: `clean' is up to date.
\end{alltt}
}

Довольно часто абстрактные цели не имеют реквизитов; цель
\target{clean} всегда будет рассматриваться как не требующая
обновления, и ассоциированные с ней команды никогда не будут
выполнены.

Чтобы избежать этой проблемы, GNU \GNUmake{} имеет специальную цель,
\target{.PHONY}, позволяющую сообщить \GNUmake{}, что цель не является
настоящим файлом. Любая цель может быть объявлена как абстрактная с
путём включения её в список реквизитов цели \target{.PHONY}:

{\footnotesize
\begin{verbatim}
.PHONY: clean
clean: 
    rm -f *.o lexer.c
\end{verbatim}
}

Теперь \GNUmake{} всегда будет выполнять команды, ассоциированные с
целью \target{clean}, даже если файл с таким именем существует. В
добавок к пометке цели как требующей обновления, спецификация цели как
абстрактной сообщает \GNUmake{}, для этой цели не нужно использовать
стандартное правило получения файла цели из исходного кода. Это
позволяет \GNUmake{} провести оптимизацию обычного процесса поиска
правил для достижения более высокой производительности.

Довольно редко имеет смысл включать абстрактную цель в качестве
реквизита реального файла, поскольку это будет приводить к
безусловному обновлению цели. Указание же реквизитов абстрактных целей
довольно часто приносит пользу. Например, цель \target{all} имеет в
качестве реквизитов список программ, которые нужно собрать:

\begin{alltt}
.PHONY: all
all: bash bashbug
\end{alltt}

В предыдущем примере цель \target{all} собирает командный
интерпретатор \utility{bash} и инструмент отправки сообщений об
ошибках \utility{bashbug}.

Абстрактные цели могут рассматриваться как сценарии интерпретатора,
встроенные в \Makefile{}. Объявление абстрактной цели в качестве
реквизита другой цели вызовет запуск сценария, ассоциированного с
абстрактной целью, перед сборкой основной цели. Предположим, мы
ограничены в использовании дискового пространства, и хотим отобразить
количество доступного места на диске перед выполнением действий,
требующих значительных затрат дискового пространства. Одно из решений
демонстрирует следующий пример:

{\footnotesize
\begin{verbatim}
.PHONY: make-documentation
make-documentation:
    df -k . | awk 'NR == 2 { printf( "%d available\n", $$4 ) }'
    javadoc ...
\end{verbatim}
}

Проблема заключается в том, что нам может понадобиться указать
команды \utility{df} и \utility{awk} несколько раз для разных целей.
Это является проблемой с точки зрения поддержки, поскольку нам
придётся изменять каждое вхождение этих команд, если, например, в
другой системе формат выдачи данных утилиты \utility{df} отличается.
Поэтому более изящным решением является следующее:

{\footnotesize
\begin{verbatim}
.PHONY: make-documentation
make-documentation: df
    javadoc ...

.PHONY: df
df:
    df -k . | awk 'NR == 2 { printf( "%d available\n", \$\$4 ) }'
\end{verbatim}
}

Мы можем сообщить \GNUmake{} о необходимости вызова сценария,
ассоциированного с целью \utility{df}, перед созданием документации,
указав \utility{df} как реквизит цели \target{make-documentation}. Это
допустимо, поскольку \target{make-documentation} также является
абстрактной целью. Такой подход даёт нам ещё одно преимущество: теперь
мы можем легко использовать \utility{df} в других целях.

Существует много примеров удачного применения абстрактных целей.

Сообщения \GNUmake{} довольно трудны для чтения и отладки. На это есть
несколько причин: составление \Makefile{}'ов сверху\hyp{}вниз, в то
время как команды выполняются снизу\hyp{}вверх; кроме того, не
указывается, какая цель выполняется в данный момент. Чтобы исправить
ситуацию, полезно выводить сообщения о начале выполнения основных
целей. Абстрактные цели являются простым средством реализации этой
идеи. Ниже приведён отрывок из \Makefile{}'а командного интерпретатора
\utility{bash}:

{\footnotesize
\begin{verbatim}
$(Program): build_msg $(OBJECTS) $(BUILTINS_DEP) $(LIBDEP)
    $(RM) $@
    $(CC) $(LDFLAGS) -o $(Program) $(OBJECTS) $(LIBS)
    ls -l $(Program)
    size $(Program)

.PHONY: build_msg
build\_msg:
    @printf "#\n# Building $(Program)\n#\n"
\end{verbatim}
}

Поскольку \target{build\_msg} является абстрактной целью, сообщение
выводится непосредственно перед проверкой остальных реквизитов.  Если
бы сообщение о начале сборки было первой командой сценария сборки
\variable{\$(Program)}, то оно выводилось бы только после сборки всех
зависимых файлов. Также важно заметить, что, поскольку абстрактные цели
всегда помечены как требующие обновления, указание абстрактной цели
\target{build\_msg} в качестве реквизита \variable{\$(Program)}
вызовет безусловную сборку этой цели, даже если на самом деле
этого не требуется. В нашем случае это выглядит разумным, поскольку
основная работа заключается в компиляции исходных файлов в объектные,
а на финальном этапе будет производиться только компоновка.

Абстрактные цели также могут быть использованы для улучшения
<<пользовательского интерфейса>> \Makefile{}'а. Имена целей часто
содержат длинные стоки с путями к каталогам, дополнительные имена
компонентов (например, номера версий) и стандартные суффиксы. Это
может сделать указание имени нужной цели довольно неудобным занятием.
Этой проблемы можно избежать, добавив абстрактную цель указав в
качестве её реквизита имя нужной реальной цели.

Есть ряд абстрактных целей, являющихся более или менее стандартными.
Не смотря на то, что их имена являются лишь соглашением, эти цели
встречаются в большинстве \Makefile{}'ов. Список этих целей содержится
в Таблице~\ref{tab:std_phony_targets}.

\begin{table}
\begin{tabular}{|l|l|}
\hline
\textbf{Цель} & \textbf{Назначение}\\
\hline
\texttt{all} & Произвести сборку приложения.\\
\hline
\target{install} & Произвести установку собранного приложения.\\
\hline
\target{clean} & Удалить все бинарные файлы, полученные после сборки.\\
\hline
\target{distclean} & Удалить все файлы, не входящие в базовый дистрибутив.\\
\hline
\target{TAGS} & Создать таблицу тэгов для текстового редактора.\\
\hline
\target{info} & Создать файлы GNU info из файлов Texinfo.\\
\hline
\target{check} & Запустить все тесты, ассоциированные с приложением.\\
\hline
\end{tabular}
\caption{Стандартные абстрактные цели.}\label{tab:std_phony_targets}
\end{table}

Цель \target{TAGS} на самом деле не является абстрактной, поскольку
программы \utility{ctags} и \utility{etags} создают файл с именем
\filename{TAGS}. Эта цель включена в таблицу потому, что это
единственная стандартная реальная цель.

%---------------------------------------------------------------------
% Empty targets
%---------------------------------------------------------------------
\subsection{Пустые цели}
\index{Цели!пустые}
Пустые цели подобны абстрактным в том плане, что позволяют расширить
возможности \GNUmake{}. Абстрактные цели всегда требуют обновления и
вызывают сборку всех целей, \emph{зависимых} от абстрактной (т.е.
содержащих её в списке реквизитов). Предположим, однако, что у нас
есть команда, не ассоциированная с файлом, которую нужно выполнять
время от времени, причём зависимые цели не должны при этом
обновляться. Для этого мы можем воспользоваться целью, ассоциированной
с пустым файлом:

{\footnotesize
\begin{verbatim}
prog: size prog.o
    $(CC) $(LDFLAGS) -o $@ $^

size: prog.o
    size $^
    touch size
\end{verbatim}
}

Заметим, что правило \target{size} использует программу
\utility{touch} после своего завершения. Пустой файл используется
только для хранения времени последней модификации, и \GNUmake{} будет
выполнять правило \target{size} только в том случае, если файл
\filename{prog.o} подвергся изменению. Более того, спецификация
\target{size} как реквизита \filename{prog} будет вызывать обновление
\target{prog} только в том случае, если соответствующий объектный файл
изменялся.

\index{Переменные!автоматические!\${}?@\variable{\${}?}}
Пустые файлы бывают полезны в сочетании с автоматической переменной
\variable{\$?}. Мы обсудим автоматические переменные в разделе
<<\nameref{sec:automatic_vars}>>, но краткое описание этой переменной
здесь не повредит. Внутри сценария сборки каждого правила \GNUmake{}
определяет переменную \variable{\$?} как множество реквизитов, имеющих
более позднюю дату модификации, чем цель.  Вот пример правила,
печатающего имена всех файлов, изменившихся с момента последнего
выполнения команды \command{make print}:

{\footnotesize
\begin{verbatim}
print: *.[hc]
    lpr $?
    touch $@
\end{verbatim}
}

%%--------------------------------------------------------------------
%% Variables
%%--------------------------------------------------------------------
\section{Переменные}

Рассмотрим некоторые из тех переменных, которые мы использовали в
наших примерах. Самые простые из них имели следующий синтаксис:

{\footnotesize
\begin{alltt}
\emph{\$(имя-переменной)}
\end{alltt}
}

Эта запись означает, что мы хотим получить значение переменной с
именем \variable{имя\hyp{}переменной}. Переменные могут содержать
практически произвольный текст, а имена переменных допускают
использование большинства символов, включая знаки пунктуации.
Например, переменная, содержащая имя команды для компиляции исходного
кода на языке \Clang{}, имеет имя \variable{COMPILE.c}. Как правило,
для подстановки значения переменной её имя окружают символами
\command{\$(} и \command{)}. В случае, когда имя переменной состоит из
одного символа, скобки можно опускать.

Как правило, \Makefile{}'ы содержат много объявлений переменных. Кроме
того, существует множество переменных, определяемых непосредственно
\GNUmake{}. Некоторые из них предназначены для контроля пользователем
поведения \GNUmake{}, другие выставляются \GNUmake{} для
взаимодействия с пользовательским \Makefile{}'ом.

%---------------------------------------------------------------------
% Automatic variables
%---------------------------------------------------------------------
\subsection*{Автоматические переменные}
\label{sec:automatic_vars}

\index{Переменные!автоматические}
\newword{Автоматические переменные} вычисляются \GNUmake{} заново для
каждого исполняемого правила. Они предоставляют доступ к цели и списку
реквизитов, избавляя от необходимости явно указывать имена файлов.
Автоматические переменные полезны для избежания дублирования кода и
необходимы для написания шаблонных правил (их мы рассмотрим позже).

Существует семь автоматических переменных:

\begin{description}
%---------------------------------------------------------------------
% $@
%---------------------------------------------------------------------
\item[\variable{\$@}] \hfill \\
%---------------------------------------------------------------------
Имя файла цели правила. Если цель является элементом архива (archive
member), то <<\variable{\$@}>> обозначает имя файла архива.

%---------------------------------------------------------------------
% $%
%---------------------------------------------------------------------
\item[\variable{\$\%}] \hfill \\
%---------------------------------------------------------------------
Для целей, являющихся элементами архива, обозначает имя
элемента. Если цель не является элементом архива, то \variable{\$\%}
содержит пустое значение.

%---------------------------------------------------------------------
% $<
%---------------------------------------------------------------------
\item[\variable{\${}<}] \hfill \\
%---------------------------------------------------------------------
Имя первого реквизита в списке реквизитов.

%---------------------------------------------------------------------
% $<
%---------------------------------------------------------------------
\item[\variable{\${}?}] \hfill \\
%---------------------------------------------------------------------
Имена всех реквизитов, имеющих более позднюю дату модификации, чем
цель.

%---------------------------------------------------------------------
% $^
%---------------------------------------------------------------------
\item[\variable{\$\^}] \hfill \\
%---------------------------------------------------------------------
Имена всех реквизитов, разделённые пробелами. В списке отсутствуют
повторения элементов, поскольку для большинства задач (копирование,
компиляция и т.д.) повторения нежелательны.

%---------------------------------------------------------------------
% $+
%---------------------------------------------------------------------
\item[\variable{\$+}] \hfill \\
%---------------------------------------------------------------------
Подобно \variable{\$?}, содержит список имён реквизитов, разделённых
пробелами, с тем отличием, что может содержать повторения. Эта
переменная была введена для специфических ситуаций, таких как
компоновка, где повторение аргументов несёт особый смысл.

%---------------------------------------------------------------------
% $*
%---------------------------------------------------------------------
\item[\texttt{\$*}] \hfill \\
%---------------------------------------------------------------------
Основа имени файла цели. Как правило, основой является имя файла с
отброшенным суффиксом (мы рассмотрим вычисление основы в разделе
<<\nameref{sec:pattern_rules}>>). Использование этой переменной вне
шаблонных правил настоятельно не рекомендуется.
%---------------------------------------------------------------------
\end{description}

Кроме того, каждая из вышеперечисленных переменных имеет два варианта
для совместимости с другими версиями \GNUmake{}. Один из этих вариантов
возвращает название каталога, в которой находится соответствующий
файл. Этот вариант обозначается добавлением символа <<D>> к имени
переменной: \variable{\$(@D)}, \variable{\${}(<D)} и т.д. Второй
вариант возвращает только имена файлов без имени каталога, в котором
они находятся. Этот вариант обозначается добавлением символа <<F>> к
имени переменной: \variable{\$(@F)}, \variable{\$(<F)} и т.д.
Поскольку эти варианты имён содержат более одного символа, они должны
заключаться в круглые скобки. GNU \GNUmake{} предоставляет более
читабельные альтернативы в лице функций \function{dir} и
\function{notdir}. Мы обсудим эти функции в
главе~\ref{chap:functions}.

Вот пример нашего \Makefile{}'а, в котором явно указанные имена
заменены подходящими автоматическими переменными.

{\footnotesize
\begin{verbatim}
count_words: count_words.o counter.o lexer.o -lfl
    gcc $^ -o $@

count_words.o: count_words.c
    gcc -c $<

counter.o: counter.c
    gcc -c $<

lexer.o: lexer.c
    gcc -c $<

lexer.c: lexer.l
    flex -t $< > $@
\end{verbatim}
}

%%--------------------------------------------------------------------
%% Finding files with VPATH and vpath
%%--------------------------------------------------------------------
\section{Поиск файлов с помощью VPATH и vpath}
\label{sec:vpath}

Наши примеры были довольно просты, поэтому \Makefile{} и исходные
файлы находились в одном каталоге. Реальные программы гораздо сложнее
(когда в последний раз вы работали над проектом, все файлы которого
размещались в одном каталоге?). Давайте изменим наш пример, разместив
файлы более реалистичным образом. Мы может модифицировать нашу
программу подсчёта слов, перенеся часть работы, выполняемой функцией
\function{main}, в функцию \function{counter}:

{\footnotesize
\begin{verbatim}
#include <lexer.h>
#include <counter.h>

void counter( int counts[4] )
{
    while ( yylex( ) )
        ;

    counts[0] = fee_count;
    counts[1] = fie_count;
    counts[2] = foe_count;
    counts[3] = fum_count;
}
\end{verbatim}
}

Поместим объявление этой функции в заголовочный файл
\filename{counter.h}:

{\footnotesize
\begin{verbatim}
#ifdef COUNTER_H_
#define COUNTER_H_

extern void
counter( int counts[4] );

#endif
\end{verbatim}
}

Мы также можем поместить объявления функций из \filename{lexer.l} в
файл \filename{lexer.h}:

{\footnotesize
\begin{verbatim}
#ifndef LEXER_H_
#define LEXER_H_

extern int fee_count, fie_count, foe_count, fum_count;

extern int yylex( void );

#endif
\end{verbatim}
}

В соответствии с негласным правилами размещения исходного кода в
файловой системе, все заголовочные файлы помещаются в каталог
\filename{include}, а исходные файлы~--- в каталог \filename{src}.
Разместим наши файлы подобным образом, оставив \Makefile{} в корневом
каталоге проекта. Результирующее дерево изображено на
рисунке~\ref{fig:ex_src_tree}.

\begin{figure}
{\footnotesize
\begin{verbatim}
.
|--include
|  |--counter.h
|  `--lexer.h
|--src
|  |--count\_words.c
|  |--counter.c
|  `--lexer.l
`--Makefile
\end{verbatim}
}
\caption{Структура каталогов проекта программы подсчёта слов.}
\label{fig:ex_src_tree}
\end{figure}

Поскольку теперь наши файлы с исходным кодом включают заголовочные
файлы, эта новая зависимость должна быть отражена в нашем \Makefile{},
то есть если модифицируется заголовочный файл, то соответствующий
объектный файл должен быть заново скомпилирован из исходного кода.

{\footnotesize
\begin{verbatim}
count_words: count_words.o counter.o lexer.o -lfl
    gcc $^ -o $@

count_words.o: count_words.c include/counter.h
    gcc -c $<

counter.o: counter.c include/counter.h include/lexer.h
    gcc -c $<

lexer.o: lexer.c include/lexer.h
    gcc -c $<

lexer.c: lexer.l
    flex -t $< > $@
\end{verbatim}
}

После запуска \GNUmake{} получаем следующий результат:

{\footnotesize
\begin{alltt}
\$ \textbf{make}
make:  *** No rule to make target `count\_words.c',
       needed by `count\_words.o'.  Stop.
\end{alltt}
}

Что же произошло? \GNUmake{} попытался собрать исходный файл
\filename{count\_words.c}! Давайте проследим за работой \GNUmake{}.
Первым реквизитом является файл \filename{count\_words.o}. Такой файл
не существует, следовательно, его нужно создать. Явное правило для
создания \filename{count\_words.o} указывает на
\filename{count\_words.c}. Но \GNUmake{} не сможет найти этот
исходный файл, ведь он находится не в текущем каталоге, а в каталоге
\filename{src}. Пока вы не укажите другого, \GNUmake{} будет искать
цели и реквизиты в текущем каталоге. Как сообщить \GNUmake{}, что
исходные файлы нужно искать в каталоге \filename{src}, или, если
поставить вопрос более \'{о}бщно, где нужно искать файлы с исходным
кодом?

Вы можете указать \GNUmake{} каталоги, в которых нужно искать
файлы, используя переменную \variable{VPATH} и директиву
\directive{vpath}. Для того, чтобы решить нашу проблему, мы может
добавить определение \variable{VPATH} в наш \Makefile{}:

{\footnotesize
\begin{verbatim}
VPATH = src
\end{verbatim}
}

Это означает, что \GNUmake{} должен искать в каталоге \filename{src}
те файлы, которые не обнаружены в текущем каталоге. После запустка
\GNUmake{}, мы получим следующий вывод:

{\footnotesize
\begin{alltt}
\$ \textbf{make}
gcc -c src/count\_words.c -o count\_words.o
src/count\_words.c:2:21: counter.h: No such file or directory
make: *** [count\_words.o] Error 1
\end{alltt}
}

Заметим, что в этот раз \GNUmake{} успешно запустил команду
компиляции, правильно подставив относительный путь к исходному файлу.
В этом заключается ещё один плюс использования автоматических
переменных: \GNUmake{} не сможет использовать подходящий путь к
исходному файлу, если вы явно укажите его имя. К несчастью, процесс
компиляции окончился неудачей, так как \utility{gcc} не смог найти
заголовочного файла. Эту проблему можно решить, настроив неявное
правило компиляции подходящей опцией \command{-I}:

{\footnotesize
\begin{verbatim}
CPPFLAGS = -I include
\end{verbatim}
}

Теперь сборка проходит успешно:

{\footnotesize
\begin{alltt}
\$ \textbf{make}

gcc -I include -c src/count\_words.c -o count\_words.o
gcc -I include -c src/counter.c -o counter.o
flex -t src/lexer.l > lexer.c
gcc -I include -c lexer.c -o lexer.o
gcc count\_words.o counter.o lexer.o /lib/libfl.a -o count\_words
\end{alltt}
}

Переменная \variable{VPATH} содержит список всех каталогов, в которых
\GNUmake{} будет искать недостающие файлы. В этих каталогах будет
происходить поиск как целей, так и реквизитов, но не файлов, указанных
в сценариях сборки. Элементы списка могут разделяться пробелами или
двоеточиями в \UNIX{} и пробелами или точками с запятой в Windows.
Предпочтительней пользоваться пробелами, поскольку такой формат
подходит для любой операционной системы и позволяет избежать путаницы
с точками с запятой и двоеточиями. Кроме того, имена каталогов,
разделённые пробелами, легче читать.

Переменная \variable{VAPTH} решает нашу проблему поиска, но всё же она
является слишком мощным инструментом. \GNUmake{} ищет в каждом
указанном в \variable{VPATH} каталоге \emph{каждый} недостающий
файл. Если в разных каталогах существуют файлы с одинаковыми
именами, \GNUmake{} возьмёт первый найденный. Иногда это может быть
причиной неприятностей.

Директива \directive{vpath} больше подходит для наших нужд. Синтаксис
вызова этой директивы представлен ниже:

{\footnotesize
\begin{alltt}
vpath \emph{шаблон список-каталогов}
\end{alltt}
}

Теперь мы можем исключить использование переменной \variable{VPATH},
добавив следующие строки в наш \Makefile{}:

{\footnotesize
\begin{verbatim}
vpath %.c src
vpath %.h include
\end{verbatim}
}

Теперь мы сообщили \GNUmake{}, что искать все файлы с расширением
\filename{.c} нужно в каталоге \filename{src}, а файлы с расширением
\filename{.h}~--- в каталоге \filename{include} (теперь мы можем
удалить префикс \filename{include} у заголовочных файлов в списке
реквизитов). В более сложных приложениях такой подход позволяет
сохранить нервы и время, потраченные на отладку.

В нашем примере мы использовали \directive{vpath} для решения проблемы
поиска файлов, разнесённых по разным директориям. Существует также
родственная проблема сборки приложения таким образом, чтобы объектные
файлы помещались в отдельное <<бинарное>> дерево каталогов, в то время
как исходные файлы хранились дереве каталогов <<исходного кода>>.
Правильное использование \directive{vpath} помогает решить и эту
проблему, однако эта задача быстро становиться очень сложной, так что
использование одной лишь директивы \directive{vpath} становится
неэффективным.  Мы обсудим эту проблему более детально в следующих
разделах.

%%--------------------------------------------------------------------
%% Pattern rules
%%--------------------------------------------------------------------
\section{Шаблонные правила}
\label{sec:pattern_rules}

\index{Правила!шаблонные}
Все примеры \Makefile{}'ов, рассмотренные нами, были слишком подробны.
Для маленькой программы, содержащей десяток файлов это не имеет
большого значения, но явная спецификация всех целей, реквизитов и
сценариев сборки для программы, содержащей сотни и тысячи файлов, не
представляется возможным. Более того, команды сами по себе вносят
дублирующийся код в наш \Makefile{}. Это может быть источником ошибок
и большой проблемой при поддержке.

Многие программы, читающие один тип файлов и выводящие другой следуют
стандартным соглашениям. Например, все компиляторы языка \Clang{}
предполагают, что файлы с расширением \filename{.c} содержат исходный
код на языке \Clang{}, а имена объектных файлов могут быть получены
заменой расширения \filename{.c} на расширение \filename{.o} (или
\filename{.obj} для некоторых компиляторов в Windows). В предыдущей
главе мы увидели, что исходные файлы генератора \utility{flex} имеют
расширение \filename{.l}, а расширением генерируемых им файлов
является \filename{.c}.

Эти и другие соглашения позволяют \GNUmake{} упростить создание правил
с помощью распознавания шаблонов и предоставления встроенных правил
сборки. Например, используя встроенные правила, мы можем сократить наш
\Makefile{} с семнадцати строк до шести:

{\footnotesize
\begin{verbatim}
VPATH = src include
CPPFLAGS = -I include

count_words: counter.o lexer.o -lfl
count_words.o: counter.h
counter.o: counter.h lexer.h
lexer.o: lexer.h
\end{verbatim}
}

\index{Правила!встроенные}
Все встроенные правила являются шаблонными. Шаблонные правила похожи
\index{Основа файла}
на обычные за тем исключением, что \newword{основа} имени файла
(часть, не содержащая расширение) представлена символом \command{\%}.
Наш \Makefile{} выполняет свою работу благодаря трём встроенным
правилам. Первое указывает, как произвести компиляцию файла с
расширением \filename{.o} из файла с расширением \filename{.c}:

{\footnotesize
\begin{verbatim}
%.o: %.c
    $(COMPILE.c) $(OUTPUT_OPTION) $<
\end{verbatim}
}

Второе правило указывает, как получить файлы с расширением
\filename{.c} из файлов с расширением \filename{.l}:

{\footnotesize
\begin{verbatim}
%.c: %.l
    @$(RM) $@
    $(LEX.l) $< > $@
\end{verbatim}
}

Наконец, существует специальное правило получения файла без расширения
(он всегда считается исполняемым) из файлов с расширением
\filename{.o}:

{\footnotesize
\begin{verbatim}
%: %.o
    $(LINK.o) $^ $(LOADLIBES) $(LDLIBS) -o $@
\end{verbatim}
}

Прежде, чем углубиться в детали синтаксиса, давайте внимательно
рассмотрим вывод \GNUmake{} и разберёмся, как он применяет встроенные
правила.

Если запустить \GNUmake{} с нашим исправленным \Makefile{}'ом, вывод
будет следующим:

{\footnotesize
\begin{alltt}
\$ \textbf{make}
gcc  -I include  -c -o count\_words.o src/count\_words.c
gcc  -I include  -c -o counter.o src/counter.c
flex  -t src/lexer.l > lexer.c
gcc  -I include  -c -o lexer.o lexer.c
gcc   count\_words.o counter.o lexer.o /lib/libfl.a -o count\_words
rm lexer.c
\end{alltt}
}

Сначала \GNUmake{} читает \Makefile{} и выставляет целью по умолчанию
файл \filename{count\_words}, поскольку аргументы командной строки не
заданы. Рассматривая цель по умолчанию, \GNUmake{} определяет
её реквизиты: \filename{count\_words.o} (этот реквизит не указан в
\Makefile{}, он следует из неявного правила), \filename{counter.o},
\filename{lexer.o} и \filename{-lfl}. Далее \GNUmake{} пытается
собрать каждый из этих реквизитов.

Когда \GNUmake{} рассматривает первый реквизит,
\filename{count\_words.o}, он не находит явного правила, однако
обнаруживает неявное. Не найдя исходного файла в текущем каталоге,
\index{Переменные!встроенные!VPATH@\variable{VPATH}}
он просматривает директории, указанные в \variable{VPATH}, и
обнаруживает исходный файл в каталоге \filename{src}. Поскольку файл
\filename{src/count\_words.c} не имеет реквизитов, \GNUmake{}
запускает сценарий компиляции \filename{count\_words.o},
ассоциированный с неявным правилом.  Таким же образом получается файл
\filename{counter.o}. Когда \GNUmake{} рассматривает
\filename{lexer.o}, соответствующий исходный файл не обнаруживается,
поэтому \GNUmake{} предполагает, что \filename{lexer.c} является
промежуточным файлом, и ищет подходящее правило для его получения.
Обнаружив правило создания файла с расширением \filename{.c} из файла
с расширением \filename{.l}, \GNUmake{} замечает, что файл
\filename{lexer.l} существует. Поскольку \filename{lexer.l} не имеет
реквизитов, вызывается сценарий генерации файла \filename{lexer.c},
содержащий вызов команды \utility{flex}. Затем происходит компиляция с
получением объектного файла. Использование последовательностей правил
\index{Цепочка правил}
для сборки цели, подобных предыдущим, называется \newword{цепочкой
правил}.

Далее, \GNUmake{} исследует спецификацию библиотеки \filename{-lfl}.
Осуществляется поиск в каталогах из списка стандартных каталогов
библиотек, результатом которого является файл \filename{/lib/libfl.a}.

Теперь \GNUmake{} имеет все реквизиты для получения
\filename{count\_words}, запускается команда компоновки \utility{gcc}.
Наконец, \GNUmake{} вспоминает, что был создан промежуточный файл,
который больше не нужно хранить, и осуществляет удаление этого файла.

Как мы убедились, использование правил в \Makefile{}'ах позволяет
избежать спецификации многих деталей. Правила могут иметь сложные
отношения, которые порождают чрезвычайно широкие возможности. В
частности, наличие встроенной базы данных общих правил значительно
упрощает спецификацию многих \Makefile{}'ов.

Встроенные правила могут быть настроены путём изменения значений
переменных, фигурирующих в сценариях сборки. Типичные сценарии имеют
множество переменных, начиная с имени команды, подлежащей выполнению,
и заканчивая большими группами опций командной строки, такими, как имя
выходного файла, флаги оптимизации, включение символьной информации и
т.д. Вы можете увидеть стандартную базу данных \GNUmake{} с помощью
команды \command{make --print\hyp{}data\hyp{}base}.

%---------------------------------------------------------------------
% The patterns
%---------------------------------------------------------------------
\subsection{Шаблоны}
\index{Шаблоны}
Символ процента в шаблонном правиле практически эквивалентен
специальному символу \command{*} командного интерпретатора \UNIX{}. Он
соответствует произвольному числу символов. Знак процента может
появиться в любом месте шаблона ровно один раз. Вот примеры
допустимого использования этого символа.

{\footnotesize
\begin{verbatim}
%,v
s%.o
wrapper_%
\end{verbatim}
}

Символы, входящие в шаблон и отличные от процента, соответствуют самим
себе.  Шаблон может содержать префикс, суффикс или и то, и другое.
Когда \GNUmake{} ищет соответствие шаблонному правилу, сначала
проверяется соответствие шаблону цели. Шаблон цели должен начинаться с
префикса и заканчиваться суффиксом (если они есть). Если соответствие
найдено, символы между префиксом и суффиксом становятся основой имени
файла. Затем \GNUmake{} производит поиск реквизитов шаблонного
правила, подставляя основу имени файла в шаблон реквизита. Если
результирующий файл существует или может быть получен с помощью
другого правила, соответствие считается установленным и правило
применяется. Основа файла должна содержать хотя бы один символ.

Допускается также иметь шаблоны с одним только символом процента.
Наиболее частое использование этого шаблона~--- сборка исполняемых
файлов \UNIX{}. Например, вот несколько встроенных шаблонных правил,
использующихся \GNUmake{} для сборки программ:

{\footnotesize
\begin{verbatim}
%: %.mod
    $(COMPILE.mod) -o $@ -e $@ $^

%: %.cpp

    $(LINK.cpp) $^ $(LOADLIBES) $(LDLIBS) -o $@

%: %.sh
    cat $< >$@
    chmod a+x $@
\end{verbatim}
}

Эти правила будут использованы для сборки программы из исходных файлов
Modula, исходных файлов \Clang{}, обработанных препроцессором, или
командных сценариев интерпретатора Bourne shell соответственно. Мы
рассмотрим другие неявные правила в разделе
<<\nameref{sec:implicit_rule_db}>>.

%---------------------------------------------------------------------
% Static pattern rules
%---------------------------------------------------------------------
\subsection{Статические шаблонные правила}
Статические шаблонные правила применяются только к определённому
списку целей.

{\footnotesize
\begin{verbatim}
$(OBJECTS): %.o: %c
        $(CC) -c $(CFLAGS) $< -o $@
\end{verbatim}
}

Единственная разница между этим правилом и обычным шаблонным
правилом~--- наличие спецификации \command{\$(OBJECTS):}. Эта
спецификация ограничивает область действия правила файлами,
перечисленными в переменной \variable{\$(OBJECTS)}. Каждый объектный
файл из этой переменной проверяется на соответствие шаблону
\command{\%.c}, извлекается основа его имени, которая затем
подставляется в шаблон \command{\%.c}, порождая имя реквизита цели.
Если файл из списка не соответствует шаблону цели, \GNUmake{} выдаст
предупреждение.

Используйте статические шаблонные правила везде, где легче перечислить
файлы целей явно, нежели определять их по расширению или шаблону.

%---------------------------------------------------------------------
% Suffix rules
%---------------------------------------------------------------------
\subsection{Суффиксные правила} \label{sec:suffix_rules}
\index{Правила!суффиксные}
Суффиксные правила~--- это первоначальный (и ныне устаревший) способ
определения неявных правил. Поскольку другие версии \GNUmake{} могут
не поддерживать синтаксис шаблонов GNU \GNUmake{}, вы ещё можете
встретить такие правила в \Makefile{}'ах дистрибутивов,
предназначенных для широкого распространения, поэтому важно
уметь читать и понимать их синтаксис. Итак, несмотря на то, что
компиляция GNU \GNUmake{} для целевой платформы является
предпочтительным методом переноса \Makefile{}'ов, при некоторых
обстоятельствах вам, возможно, придётся писать суффиксные правила.

Суффиксные правила выглядят как два написанных слитно суффикса,
выступающие в роли правила:

{\footnotesize
\begin{verbatim}
.c.o:
    $(COMPILE.c) $(OUTPUT_OPTION) $<
\end{verbatim}
}

Тот факт, что первым указывается суффикс реквизита, а вторым~---
суффикс цели, может внести некоторую путаницу. Предыдущему правилу
соответствуют в точности те же цели и реквизиты, что и следующему:

{\footnotesize
\begin{verbatim}
%.o: %.c
    $(COMPILE.c) $(OUTPUT_OPTION) $<
\end{verbatim}
}

Суффиксные правила формируют основу имени файла путём удаления
суффикса; имя файла реквизита получается в результате замены суффикса
цели суффиксом реквизита. Суффиксное правило распознаётся \GNUmake{}
только в том случае, если оба суффикса находятся в списке известных
суффиксов.

Предыдущее суффиксное правило называется также двусуффиксным,
поскольку оно содержит два суффикса. Существуют также односуффиксные
правила. Такие правила содержат только один суффикс~--- суффикс
реквизита. Эти правила используются для сборки исполняемых файлов
\UNIX{}, не имеющих расширения:

{\footnotesize
\begin{verbatim}
.p:
    $(LINK.p) $^ $(LOADLIBES) $(LDLIBS) -o $@
\end{verbatim}
}

Предыдущее правило описывает создание исполняемого файла из исходного
файла Pascal. Оно полностью эквивалентно следующему правилу:

{\footnotesize
\begin{verbatim}
%: %.p
    $(LINK.p) $^ $(LOADLIBES) $(LDLIBS) -o $@
\end{verbatim}
}

Список известных суффиксов является наиболее необычной частью
синтаксиса. Для определения этого списка используется специальная
цель \target{.SUFFIXES}. Вот первая часть стандартного определения
\target{.SUFFIXES}:

{\footnotesize
\begin{verbatim}
.SUFFIXES: .out .a .ln .o .c .cc .C .cpp .p .f .F .r .y .l
\end{verbatim}
}

Вы можете добавить собственные суффиксы, добавив правило
\target{.SUFFIXES} в ваш \Makefile{}:

{\footnotesize
\begin{verbatim}
.SUFFIXES: .pdf .fo .html .xml
\end{verbatim}
}

Если вы хотите удалить все известные суффиксы (например, они
перекрываются с вашими специальными суффиксами), определите цель
\target{.SUFFIXES} как не имеющую реквизитов:

{\footnotesize
\begin{verbatim}
.SUFFIXES:
\end{verbatim}
}

Вы также можете использовать опцию командной строки
\index{Опции!no-builtin-rules@\command{-{}-no\hyp{}builtin\hyp{}rules (-r)}}
\command{-{}-no\hyp{}builtin\hyp{}rules} (или \command{-r}).

Мы не будем использовать суффиксные правила в этой книге, поскольку
шаблонные правила GNU \GNUmake{} более выразительны и \'{о}бщны.

%%--------------------------------------------------------------------
%% Implicit rules database
%%--------------------------------------------------------------------
\section{База данных неявных правил}
\label{sec:implicit_rule_db}

\index{База данных!неявных правил}
GNU \GNUmake{} 3.80 содержит примерно 90 встроенных неявных правил
(шаблонных и суффиксных), предназначенных для работы с исходными
файлами \Clang{}, \Cplusplus{}, Pascal, \FORTRAN{}, ratfor, Modula,
Texinfo, \TeX{} (включая инструменты \utility{tangle} и
\utility{weave}), Emacs Lisp, RCS и SCCS, а также правила для
поддержки программ, предназначенных для работы с этими языками:
\utility{cpp}, \utility{as}, \utility{yacc},  \utility{lex},
\utility{tangle}, \utility{weave} и инструменты для работы с
\utility{dvi}.

Если вы используете одну из этих программ, то, возможно, вас вполне
устроят возможности, предоставляемые встроенными правилами. Если же вы
используете некоторые не поддерживаемые языки наподобие \Java{} или
XML, вам придётся писать правила самим. Впрочем, обычно для добавления
поддержки языка достаточно написать всего несколько простых правил.

Чтобы увидеть встроенные правила \GNUmake{}, нужно запустить его с
\index{Опции!print-data-base@\command{-{}-print-data-base (-p)}}
опцией \command{-{}-print\hyp{}data\hyp{}base} (или просто
\command{-p}). Вывод составит около тысячи строк текста: после номера
версии и текста лицензии \GNUmake{} выведет на экран определения всех
переменных с описанием их <<происхождения>>. Например, переменные
могут быть взяты из окружения, являться стандартными или
автоматическими. После описания переменных последуют правила. Текущий
формат правил GNU \GNUmake{} выглядит следующим образом:

{\footnotesize
\begin{verbatim}
%: %.C
#  Команды для выполнения (встроенные):
    $(LINK.C) $^ $(LOADLIBES) $(LDLIBS) -o $@
\end{verbatim}
}

Комментарии к правилам, определённым в \Makefile{}, будут содержать
имя файла и номер строки, в которых эти правила определены:

{\footnotesize
\begin{verbatim}
%.html: %.xml
#  Команды для выполнения (из `Makefile', строка 168):
    $(XMLTO) $(XMLTO\_FLAGS) html-nochunks $<
\end{verbatim}
}

%---------------------------------------------------------------------
% Working with implicit rules
%---------------------------------------------------------------------
\subsection{Работа с неявными правилами}
Встроенные правила применяются каждый раз, когда рассматривается цель,
для которой не указан сценарий сборки. Таким образом, использовать
встроенные правила очень просто~--- достаточно не указывать команд при
добавлении цели в \Makefile{}. Это будет сигналом для \GNUmake{},
призывающим к поиску подходящего правила в базе данных встроенных
правил. Обычно это приводит к нужному результату, но в редких случаях
ваша среда разработки может породить некоторые проблемы.  Предположим,
у вас имеется смешанная среда разработки, состоящая из исходных файлов
C и Lisp. Если файлы \filename{editor.l} и \filename{editor.c}
находятся в одном каталоге (например, один из них является
низкоуровневой реализацией, к которой имеет доступ другой), \GNUmake{}
будет считать, что файл с исходным кодом на Lisp на самом деле
является файлом \utility{flex} (повторим, \utility{flex} использует
файлы с расширением \filename{.l}), а исходный файл \Clang{}~---
результат запуска \utility{flex}. Если файл \filename{editor.o}
является целью, а \filename{editor.l} модифицировался позднее
\filename{editor.c}, \GNUmake{} попытается обновить исходный файл
\Clang{} выводом команды \utility{flex}, заместив ваш исходный код.

Чтобы избежать этой проблемы, вы можете удалить из базы данных
правила, вызывающие \utility{flex}, следующим образом:

{\footnotesize
\begin{verbatim}
%.o: %.l
%.c: %.l
\end{verbatim}
}

Шаблонное правило без сценария сборки удаляет правило из базы данных
\GNUmake{}. На практике ситуации, подобные описанной, встречаются
крайне редко. Тем не менее, важно помнить, что правила из встроенной
базы данных могут взаимодействовать с вашими \Makefile{}'ами
неожиданным для вас образом.

Мы уже рассмотрели несколько примеров того, как \GNUmake{} выстраивает
цепочки правил для сборки цели. Такое поведение может порождать
довольно сложные конструкции. Когда \GNUmake{} рассматривает варианты
сборки цели, происходит поиск неявных правил, шаблону цели которых
соответствует имя текущей цели. Для каждого подходящего шаблона цели
\GNUmake{} осуществляет проверку соответствия реквизитов. Таким
образом, проверка реквизитов осуществляется \emph{сразу} после
нахождения подходящего шаблона цели. Если реквизиты найдены, правило
применяется. Для некоторых шаблонов цели может быть несколько
возможных видов реквизитов. Например, файлы с расширением
\filename{.o} можно получить из файлов с расширением \filename{.c},
\filename{.cc}, \filename{.cpp}, \filename{.p}, \filename{.f},
\filename{.r}, \filename{.s} и \filename{.mod}. Если исходные файлы не
обнаружены при переборе всех возможных вариантов, \GNUmake{}
произведёт повторный поиск правил, рассматривая исходные файлы в
качестве новых целей. Повторяя этот поиск рекурсивно, \GNUmake{}
выстроит цепь правил, позволяющих произвести сборку цели. Мы уже имели
возможность убедиться в этом на примере файла \filename{lexer.o}:
\GNUmake{} смог получить \filename{lexer.o} из \filename{lexer.l} даже
при отсутствующем файле \filename{lexer.c}, вызвав сначала правило
\filename{.l}$\,\,\rightarrow{}$\filename{.c}, а затем правило
\filename{.c}$\,\,\rightarrow{}$\filename{.o}.

Давайте рассмотрим одну из самых впечатляющих цепочек, которую
\GNUmake{} может породить с помощью встроенной базы данных правил.
Сначала создадим пустой исходный файл \utility{yacc} и зарегистрируем
его в системе контроля ревизий RCS командой \utility{ci}:

\begin{alltt}
\footnotesize
\$ \textbf{touch foo.y}
\$ \textbf{ci foo.y}
foo.y,v  <--  foo.y
.
initial revision: 1.1
done
\end{alltt}

Теперь попросим \GNUmake{} создать исполняемый файл \filename{foo}.
\index{Опции!just-print@\command{-{}-just-print (-n)}}
Опция \command{-{}-just\hyp{}print} (или просто \command{-n})
означает, что от \GNUmake{} требуется лишь описать, какие действия
будут выполнены.  Заметим, что у нас нет \Makefile{}'а и исходного
кода, только RCS-файл.

\begin{alltt}
\footnotesize
\$ \textbf{make -n foo}
co  foo.y,v foo.y
foo.y,v  -->  foo.y
revision 1.1
done
bison -y  foo.y
mv -f y.tab.c foo.c
gcc -c -o foo.o foo.c
gcc foo.o -o foo
rm foo.c foo.o foo.y
\end{alltt}

Следуя по цепочке неявных правил и реквизитов, \GNUmake{} определяет,
что сборка исполняемого файла \filename{foo} возможна при наличии
объектного файла \filename{foo.o}, который можно получить из исходного
файла \filename{foo.c}. В свою очередь, \filename{foo.c} может быть
получен из файла \filename{yacc} \filename{foo.y}, доступного при
извлечении его из файла RCS \filename{foo.y,v}, имеющегося в наличии.
Составив план действий, \GNUmake{} выполняет извлечение
\filename{foo.y} с помощью команды \utility{co}, преобразует его в
исходный файл \filename{foo.c} командой \utility{bison}, компилирует
полученный исходный файл C в объектный файл \filename{foo.o} вызовом
\utility{gcc} и производит компоновку для получения исполняемого файла
\filename{foo} повторным вызовом \utility{gcc}. Всё это происходит с
использованием лишь встроенной базы данных правил. Впечатляет.

Файлы, порождаемые применением цепочки правил, называются
\emph{промежуточными} и обрабатываются \GNUmake{} особым образом.
Во-первых, поскольку промежуточные файлы не специфицируются в качестве
целей (иначе они бы не были промежуточными), \GNUmake{} никогда не
удовлетвориться просто сборкой этого файла. Во-вторых, поскольку
\GNUmake{} создаёт эти файлы самостоятельно как побочные эффекты
сборки цели, такие файлы должны быть удалены перед завершением работы.
Вы можете убедиться в этом, посмотрев на последнюю строку предыдущего
примера.

%---------------------------------------------------------------------
% Rule structure
%---------------------------------------------------------------------
\subsection{Структура правил}
Встроенные правила имеют стандартную структуру, направленную на
предоставление возможности простой настройки этих правил.
Рассмотрим сначала общую структуру, затем обсудим тонкости настройки.
Ниже представлено уже знакомое правило компиляции исходного файла
\Clang{} в объектный файл:

{\footnotesize
\begin{verbatim}
%.o: %.c
    $(COMPILE.c) $(OUTPUT_OPTION) $<
\end{verbatim}
}

Настройка этого правила осуществляется за счёт переменных, используемых
им. Мы видим две переменные, однако переменная
\variable{\$(COMPILE.c)}, например, определена через другие переменные:

{\footnotesize
\begin{verbatim}
COMPILE.c = $(CC) $(CFLAGS) $(CPPFLAGS) $(TARGET\_ARCH) -c
CC = gcc
OUTPUT_OPTION = -o $@
\end{verbatim}
}

Таким образом, замена компилятора языка \Clang{} может быть
произведена изменением значения переменной \variable{CC}. Другие
переменные используются для определения опций компиляции
(\variable{CFLAGS}), опций препроцессора (\variable{CPPFLAGS}) и
архитектурно\hyp{}зависимых опций (\variable{TARGET\_ARCH}).

Все переменные, использующиеся во встроенных правилах, нацелены на
максимально простую настройку правил. По этой причине очень важно
быть осторожным при определении значений этих переменных в вашем
\Makefile{}'е. Если вы будете определять переменные <<по наитию>>,
вы можете нарушить возможность настройки правил конечным
пользователем. Рассмотрим пример присваивания в \Makefile{}'е:

{\footnotesize
\begin{verbatim}
CPPFLAGS = -I project/include
\end{verbatim}
}

Если пользователь хочет самостоятельно определить значение переменной
в командной строке, он обычно поступает следующим образом:

\begin{alltt}
\footnotesize
\$ \textbf{make CPPFLAGS=-DDEBUG}
\end{alltt}

Однако такой вызов случайно удалит опцию \command{-I} (которая,
предположительно, необходима для компиляции) поскольку значения
переменных, определённые в командной строке, перекрывают все другие
присваивания этим переменным\footnote{Для более подробных сведений о
присваиваниях в командной строке следует обратиться к
разделу <<\nameref{sec:where_vars_come_from}>>.}. Таким образом,
ненадлежащее определение переменной \variable{CPPFLAGS} нарушило
возможность настройки, на наличие которой рассчитывает б\'{о}льшая
часть пользователей. Вместо использования простых присваиваний,
рассмотрим переопределение переменной компиляции с добавлением новых
переменных:

{\footnotesize
\begin{verbatim}
COMPILE.c = $(CC) $(CFLAGS) $(INCLUDES) $(CPPFLAGS) $(TARGET_ARCH) -c
INCLUDES = -I project/include
\end{verbatim}
}

Ещё одним выходом из ситуации является использование доопределения
переменных вместо присваивания, этот подход обсуждается в разделе
<<\nameref{sec:other_types_of_assign}>> главы~\ref{chap:vars}.

%---------------------------------------------------------------------
% Implicit rules for source control
%---------------------------------------------------------------------
\subsection{Неявные правила для управления ревизиями}
В \GNUmake{} реализована на уровне встроенных правил поддержка двух
\index{RCS} \index{SCCS}
систем контроля ревизий: RCS и SCCS. Складывается впечатление, что
искусство управления исходным кодом на текущем этапе своего развития
и современная компьютерная инженерия оставили \GNUmake{} далеко позади.
На практике поддержка управления ревизиями в \GNUmake{} практически
не используется. И на это есть ряд причин.

Во\hyp{}первых, инструменты контроля ревизий, поддерживаемые
\GNUmake{}, RCS и SCCS, в прошлом весьма значимые и ценные, были
\index{CVS}
повсеместно вытеснены CVS (Concurrent Version System) или
коммерческими инструментами. Несмотря на то, что CVS использует RCS
для внутреннего управления одиночными файлами, прямое использование
RCS порождает значительные проблемы, когда проект состоит из более чем
одного каталога, или в нём задействовано более одного разработчика. В
частности, CVS была разработана для заполнения пробелов в
функциональности RCS в упомянутых аспектах. Поддержки CVS в \GNUmake{}
никогда не было, и это, возможно, является правильным
решением\footnote{CVS, в свою очередь, постепенно вытесняется более
современными инструментами. На данный момент наиболее часто
встречается система управления исходным кодом
\index{Subversion}
Subversion (\filename{\url{http://subversion.tiqris.org}})
(прим. автора).}.

Общепризнано, что жизненный цикл разработки программного обеспечения
становится всё более сложным. Приложения редко плавно продвигаются от
одного релиза к другому. Как правило, одновременно может
использоваться (и, следовательно, требовать исправления ошибок)
несколько версий приложения, в то время как ещё несколько версий могут
находиться в активной разработке. CVS предоставляет богатые
возможности управления параллельной разработкой программного
обеспечения. Однако это требует от разработчика чёткого осознания
того, над какой версией исходного кода он работает. Если бы \GNUmake{}
автоматически извлекал из хранилища исходный код для компиляции, сразу
бы вставали вопросы актуальности извлечённой версии и совместимости
этой версии с кодом, расположенным в рабочих каталогах разработчика.
Во многих проектах разработчики работают с тремя и более различными
версиями программного продукта в течение одного дня. Проверка
целостности сложной системы достаточно сложна и без наличия
инструмента, безмолвно изменяющего ваш исходный код.

К тому же, одной из самых важных возможностей CVS является возможность
доступа к удалённому хранилищу. В большинстве проектов CVS
хранилище (база данных версионных файлов) располагается на сервере,
а не на машине разработчика. Несмотря на то, что удалённый доступ
достаточно быстр (по крайней мере, в локальных сетях), запуск
\GNUmake{}, проверяющего каждый файл на удалённом сервере, не является
хорошей идеей, поскольку производительность упадёт катастрофически.

Таким образом, можно использовать встроенные правила для довольно
прозрачного взаимодействия с RCS и SCCS, но правил для доступа к
хранилищам CVS для поиска файлов не существует. По большому счёту,
даже если бы такие правила существовали, им трудно было бы найти
разумное применение. С другой стороны, использование CVS для
управления версиями \Makefile{}'ов чрезвычайно полезно и порождает
много интересных применений, таких как проверка правильности помещения
файла в хранилище, управление нумерацией релизов и корректное
выполнение автоматического тестирования. Все это возможно при
использовании CVS авторами \Makefile{}'ов, а не за счёт интеграции
\GNUmake{} с CVS.

%---------------------------------------------------------------------
% A simple help command
%---------------------------------------------------------------------
\subsection{Пример простой справки}
Большие \Makefile{}'ы могут содержать много целей, трудных для
запоминания. Одним из способов устранения этой проблемы является
выбор в качестве цели по умолчанию абстрактной цели вывода краткой
справки. Однако поддержка текста этой справки в актуальном состоянии
является довольно утомительным занятием. К счастью, для составления
справки могут быть использованы команды из встроенной базы данных
правил \GNUmake{}. Ниже приведён пример цели, выводящей отсортированный
список доступных целей:

{\footnotesize
\begin{verbatim}
# help - цель по умолчанию
.PHONY: help
help:
    $(MAKE) --print-data-base --question |            \
    $(AWK) '/\^[\^.\%][-A-Za-z0-9\_]*:/               \
           { print substr($$1, 1, length($$1)-1) }' | \
    $(SORT) |                                         \
    $(PR) --omit-pagination --width=80 --columns=4
\end{verbatim}
}

Сценарий состоит из одного конвейера программ. Список правил \GNUmake{}
\index{Опции!print-data-base@\command{-{}-print-data-base (-p)}}
выводится при помощи ключа \command{-{}-print\hyp{}data\hyp{}base}, ключ
\index{Опции!question@\command{-{}-question (-q)}}
\command{-{}-question} исключает выполнение сценариев сборки. Затем
база данных пропускается через простой сценарий \utility{awk},
извлекающий из потока правил все цели, имена которых не начинаются
с символов процента или точки (то все шаблонные и суффиксные правила,
соответственно) и вырезающий всю прочую информацию в строке. Наконец,
цели сортируются по имени и выводятся в четыре колонки.

Другим возможным решением проблемы является применение специального
сценария \utility{awk} непосредственно к \Makefile{}'у. Это потребует
специальной обработки включения \Makefile{}'ов (см. раздел
<<\nameref{sec:include_directive}>> главы~\ref{chap:vars}) и сделает
невозможным обработку правил, созданных самим \GNUmake{} на основе
шаблонных и суффиксных правил. Версия, представленная выше, избавлена
от этих недостатков благодаря вызову \GNUmake{}, осуществляющему все
необходимые действия самостоятельно.

%%--------------------------------------------------------------------
%% Special targets
%%--------------------------------------------------------------------
\section{Специальные цели}

\index{Цели!специальные}
\newword{Специальной целью} называют встроенную абстрактную цель,
предназначенную для спецификации поведения \GNUmake{}. Например, уже
известная нам \target{.PHONY} является специальной целю, реквизиты
которой не связаны с реальными файлами и всегда требуют обновления.

Для специальных целей используется стандартный синтаксис
\ItalicMono{цель: реквизиты}, но \ItalicMono{цель} не является файлом
или обычной абстрактной целью. Более всего специальные цели похожи на
директивы, изменяющие внутренние алгоритмы \GNUmake{}.

Существует двенадцать специальных целей. Их можно разделить на три
категории: одни изменяют поведение \GNUmake{}, вторые являются просто
флагами, наконец, специальная цель \variable{.SUFFIXES} используется
для спецификации суффиксных правил (обсуждавшихся в разделе
<<\nameref{sec:suffix_rules}>>).

Наиболее полезны следующие специальные цели:

\begin{description}
\item[\target{.INTERMEDIATE}] \hfill\\
Реквизиты этой цели интерпретируются как промежуточные файлы. Если
\GNUmake{} создаст файл во время сборки другой цели, файл будет
автоматически удалён перед завершением работы \GNUmake{}. Если файл
уже существует в момент сборки цели, файл не будет удалён.

Такое поведение может быть очень полезным при построении цепочки
правил. Например, большинство Java утилит принимают списки файлов в
стиле Windows. Создание правил для сохранения списков
файлов и спецификация их как промежуточных позволяет \GNUmake{}
автоматически удалять множество временных файлов.

\item[\target{.SECONDARY}] \hfill\\
Реквизиты этой специальной цели интерпретируются как промежуточные
файлы, которые не удаляются автоматически. Наиболее часто
\target{.SECONDARY} используется для пометки объектных файлов,
хранимых в виде библиотек. Обычно такие объекты будут удалены сразу
после добавления их в архив. Иногда удобнее не удалять объектные
файлы во время разработки.

\item[\target{.PRECIOUS}] \hfill\\
Когда \GNUmake{} аварийно завершает выполнение, файл цели может быть
удалён, если он изменился с момента старта \GNUmake{}. Таким образом
избегается возможность сохранения частично собранных (и, возможно,
повреждённых) файлов. Иногда вы можете захотеть от \GNUmake{} другого
поведения, например, если файл велик и требует много вычислений для
своего создания. Если вы укажете имя такого файла в качестве
реквизитов цели \target{.PRECIOUS}, \GNUmake{} не будет удалять этот
файл в случае аварийного завершения. Эта цель используется довольно
редко, но если её применение действительно необходимо, её наличие
сохранит разработчику много времени и сил. Заметим, что \GNUmake{} не
осуществляет автоматического удаления в случае ошибки в выполняемой
команде, только в случае получения сигнала останова.


\item[\target{.DELETE\_ON\_ERROR}] \hfill\\
Эта цель~--- противоположность \target{.PRECIOUS}. Спецификация файла
как реквизита этой цели означает, что файл должен быть удалён в случае
любой ошибки в сценарии сборки, ассоциированном с соответствующим
правилом. Обычно \GNUmake{} удаляет цель только в случае получении
сигнала останова.
\end{description}

Остальные специальные цели будут рассмотрены в момент их
непосредственного использования. Цели, относящиеся к параллельному
выполнению, будут рассмотрены в
главе~\ref{chap:improving_the_performance}, цель
\target{.EXPORT\_ALL\_VARIABLES}~--- в главе~\ref{chap:vars}.

%%--------------------------------------------------------------------
%% Automatic dependency generation
%%--------------------------------------------------------------------
\section{Автоматическое определение зависимостей}
\label{sec:auto_dep_gen}
Когда мы изменили нашу программу подсчёта слов так, чтобы часть
объявлений была описана в заголовочных файлах, мы, сами того не
замечая, добавили новую проблему. Мы описали зависимости между
объектными и заголовочными файлами в наш \Makefile{} самостоятельно. В
нашем случае сделать это было нетрудно, но в реальных программах (а не
в игрушечных примерах) это может быть весьма утомительным и
порождающим ошибки процессом. На самом деле, в большинстве программ
указание зависимостей практически невозможно, поскольку заголовочные
файлы могут включать другие заголовочные файлы, образуя сложное дерево
включений.
\index{Заголовочный файл}
Например, в моей системе один заголовочный файл \filename{stdio.h}
(наиболее часто используемый заголовочный файл стандартной библиотеки
языка \Clang{}) в общем счёте включает 15 других заголовочных файлов.
Разрешение подобных зависимостей вручную является практически
безнадёжным занятием. Однако неудавшаяся компиляция ведёт к часам
потраченного на отладку времени или, что ещё хуже, к проблемам в уже
выпущенном программном обеспечении. Что же нам делать?

К счастью, компьютеры весьма хорошо справляются с задачами поиска и
нахождения соответствий шаблону. Давайте используем программу для
определения зависимостей между исходными файлами, и даже записи этих
зависимостей в соответствии со стандартным синтаксисом \GNUmake{}. Как
вы, возможно, уже догадались, такая программа уже существует, по крайней
мере, для исходных файлов на \Clang{}/\Cplusplus{}.
\index{gcc}
\index{Опции!компилятора}
Компилятор \utility{gcc}, как и многие другие компиляторы, имеет опцию
для чтения исходных файлов и составления зависимостей для \GNUmake{}.
Например, так мы можем определить зависимости для \filename{stdio.h}

\begin{alltt}
\footnotesize
\$ \textbf{echo "#include <stdio.h>" > stdio.c}
\$ \textbf{gcc -M stdio.c}
stdio.o: stdio.c /usr/include/stdio.h /usr/include/\_ansi.h \textbackslash{}
/usr/include/newlib.h /usr/include/sys/config.h \textbackslash{}
/usr/include/machine/ieeefp.h /usr/include/cygwin/config.h \textbackslash{}
/usr/lib/gcc-lib/i686-pc-cygwin/3.2/include/stddef.h \textbackslash{}
/usr/lib/gcc-lib/i686-pc-cygwin/3.2/include/stdarg.h \textbackslash{}
/usr/include/sys/reent.h /usr/include/sys/\_types.h \textbackslash{}
/usr/include/sys/types.h /usr/include/machine/types.h \textbackslash{}
/usr/include/sys/features.h /usr/include/cygwin/types.h \textbackslash{}
/usr/include/sys/sysmacros.h /usr/include/stdint.h \textbackslash{}
/usr/include/sys/stdio.h
\end{alltt}

<<Отлично,>>~--- скажете вы,~---<<Теперь мне придётся запускать gcc,
открывать текстовый редактор и вставлять результаты работы компилятора
с ключом \command{-M} в свой \Makefile{}. Какой ужас.>>. И вы были бы
правы, если бы это была вся правда. Существует два стандартных способа
включения автоматически составленных зависимостей в \Makefile{}.
Первый, он же самый старый, заключается в добавлении комментария
наподобие следующего:

{\footnotesize
\begin{verbatim}
# Далее следуют автоматически составленные зависимости:
# НЕ РЕДАКТИРОВАТЬ
\end{verbatim}
}

{\flushleft
в конец \Makefile{}'а и написании сценария командного интерпретатора
для автоматического обновления этого раздела. Это, безусловно, гораздо
лучше ручного обновления, но всё ещё довольно неудобно. Второй метод
заключается в добавлении директивы include. Б\'{о}льшая часть версий
\GNUmake{} поддерживает эту директиву, и, безусловно, GNU \GNUmake{} в
их числе. Идея заключается в спецификации цели, с которой
ассоциированы действия по запуску \utility{gcc} с ключом \command{-M},
сохранении результатов в файле зависимостей и повторный запуск
\GNUmake{} с включением составленного файла зависимостей в основной
\Makefile{}. До появления GNU \GNUmake{} это делалось правилом
следующего вида:
}

{\footnotesize
\begin{verbatim}
depend: count\_words.c lexer.c counter.c
    $(CC) -M $(CPPFLAGS) $^ > $@
include depend
\end{verbatim}
}

Сначала вы запускаете \GNUmake{} с целью составить файл зависимостей,
и только после этого производите повторный пуск для сборки программы.
На момент появления этой возможности она выглядела неплохо, однако
часто люди добавляли или удаляли зависимости из исходного кода, забыв
заново составить файл зависимостей. Это становилось причиной
неправильной компиляции со всеми вытекающими неприятностями. GNU
\GNUmake{} решил эту неприятную проблему с помощью мощной
функциональности и довольно простого алгоритма. Рассмотрим сначала
алгоритм. Если мы составим для каждого исходного файла собственный
файл зависимостей, скажем, файл с расширением \filename{.d}, и добавим
этот файл в качестве цели к соответствующему правилу, то сможем
сообщить \GNUmake{}, что \filename{.d} файл нуждается в обновлении
(наряду с объектным файлом) при изменении исходного файла:

{\footnotesize
\begin{verbatim}
counter.o counter.d: src/counter.c include/counter.h include/lexer.h
\end{verbatim}
}

Составление этого правила может быть завершено шаблонным правилом и
довольно неуклюжим сценарием (взятым прямо из руководства по GNU
\GNUmake{}) \footnote{Этот довольно выразительный сценарий, по моему
мнению, всё же требует некоторого объяснения. Сначала мы используем
компилятор \Clang{} с опцией \command{-M} для создания временного
файла, содержащего список зависимостей цели. Имя временного файла
получается из названия цели \command{\${}@} и добавочного уникального
числового суффикса \command{.\${}\${}\${}\${}}. В командном
интерпретаторе \utility{sh} переменная \command{\${}\${}} содержит
идентификатор текущего запущенного процесса командного интерпретатора.
Поскольку этот идентификатор является уникальным, имя нашего
временного файла также получается уникальным.  Затем мы используем
\utility{sed} для добавления файла с расширением \filename{.d} в
качестве цели правила. Выражение \utility{sed} состоит из шаблона
поиска
\command{\textbackslash{}(\${}\textbackslash{})\textbackslash{}1.o[
:]*} и подстановки \command{\textbackslash{}1.o \${}@ :}, разделённых
запятыми. Шаблон поиска состоит из основы имени цели \command{\${}*},
заключенной в группу регулярного выражения
\command{\textbackslash{}(\textbackslash{})}, за которой следует
суффикс \command{.o}. После имени цели могут следовать пробелы или
двоеточия (\command{[ :]*}). Подстановка восстанавливает
первоначальную цель с помощью ссылки на первую группу регулярного
выражения с добавлением суффикса (\command{\textbackslash{}1.o}) и
добавляет файл зависимостей в качестве второй цели правила
(\command{\${}@}).}:

{\footnotesize
\begin{verbatim}
%.d: %.c
    $(CC) -M $(CPPFLAGS) $< > $@.$$$$;                  \
    sed 's,\($*\)\.o[ :]*,\1.o $@ : ,g' < $@.$$$$ > $@; \
    rm -f $@.$$$$
\end{verbatim}
}

Теперь рассмотрим вышеупомянутую функциональность. GNU \GNUmake{}
будет рассматривать каждый включаемый файл в качестве цели,
нуждающейся в обновлении.  Таким образом, когда мы будем упоминать
\filename{.d} файлы, \GNUmake{} автоматически попытается создать эти
файлы во время чтения \Makefile{}'а. Ниже представлен наш пример с
добавлением автоматического управления зависимостями:

{\footnotesize
\begin{verbatim}
VPATH    = src include
CPPFLAGS = -I include
SOURCES  = count_words.c \
           counter.c     \
           lexer.c 
count_words: counter.o lexer.o -lfl
count_words.o: counter.h
counter.o: counter.h lexer.h
lexer.o: lexer.h

include $(subst .c,.d,$(SOURCES))

%.d: %.c
    $(CC) -M $(CPPFLAGS) $< > $@.$$$$;                  \
    sed 's,\($*\)\.o[ :]*,\1.o $@ : ,g' < $@.$$$$ > $@; \
    rm -f $@.$$$$
\end{verbatim}
}

Директива включения должна появляться только после записанных вручную
правил, чтобы не подменить цель по умолчанию целью из включаемого
\index{Директивы!include@\directive{include}}
файла. Директива \directive{include} принимает в качестве аргумента
список файлов (чьи имена могут включать шаблоны). В предыдущем примере
\index{Функции!встроенные!substr@\function{substr}}
мы использовали встроенную функцию \GNUmake{} \function{substr} для
трансформации списка исходных файлов в список файлов зависимостей (мы
рассмотрим \function{substr} более подробно в разделе
<<\nameref{sec:str_func}>> главы~\ref{chap:functions}). Пока просто
примите к сведению, что мы используем эту функцию для замены строки
\filename{.c} на строку \filename{.d} в каждом слове списка
\variable{\${}(SOURCES)}.

\index{Опции!just-print@\command{-{}-just-print (-n)}}
Если теперь мы запустим \GNUmake{} с опцией
\command{-{}-just\hyp{}print}, то получим следующее:

\begin{alltt}
\footnotesize
\$ \textbf{make --just-print}
Makefile:13: count\_words.d: No such file or directory
Makefile:13: lexer.d: No such file or directory
Makefile:13: counter.d: No such file or directory
\verb#gcc -M -I include src/counter.c > counter.d.$$;       \#
\verb#sed 's,\(counter\)\.o[ :]*,\1.o counter.d : ,g'       \#
\verb#< counter.d.$$ > counter.d;                           \#
rm -f counter.d.\$\$
flex -t src/lexer.l > lexer.c
\verb#gcc -M -I include lexer.c > lexer.d.$$;           \#
\verb#sed 's,\(lexer\)\.o[ :]*,\1.o lexer.d : ,g'       \#
\verb#< lexer.d.$$ > lexer.d;                           \#
rm -f lexer.d.\$\$
\verb#gcc -M -I include src/count_words.c > count_words.d.$$; \#
\verb#sed 's,\(count_words\)\.o[ :]*,\1.o count_words.d : ,g' \#
\verb#< count_words.d.$$ count_words.d;                       \#
rm -f count\_words.d.\$\$
rm lexer.c
gcc -I include -c -o count\_words.o src/count\_words.c
gcc -I include -c -o counter.o src/counter.c
gcc -I include -c -o lexer.o lexer.c
gcc count\_words.o counter.o lexer.o /lib/libfl.a -o count\_words
\end{alltt}

Сначала \GNUmake{} выводит несколько предупреждений, с виду
напоминающих ошибки. Не стоит волноваться, это всего лишь
предупреждения. \GNUmake{} производит поиск файлов, указанных в
директиве \index{Директивы!sinclude@\directive{-include}}
\directive{include}, не находит их, и перед началом поиска правила для
создания этих файлов выводит предупреждение \command{No such file or
directory}. Эти предупреждения могут быть подавлены при помощи символа
\command{-}, добавленного перед директивой \directive{include}.
Следующие строки демонстрируют вызов \utility{gcc} с опцией
\command{-M} и запуск команды \utility{sed}.  Обратите внимание на то,
что \GNUmake{} должен вызвать \utility{flex} для создания
\filename{lexer.c}, удаляемый перед началом сборки цели по умолчанию.

\index{Автоматическое определение зависимостей}
Теперь у вас есть представление об автоматическом определении
зависимостей. Эта тема содержит ещё много интересных вопросов,
например, построение зависимостей для других языков программирования,
или вывод зависимостей в виде дерева. Мы вернёмся к этим темам во
второй части книги.

%%--------------------------------------------------------------------
%% Managing libraries
%%--------------------------------------------------------------------
\section{Управление библиотеками}
\label{sec:managing_libs}

\index{Библиотечный архив} \index{archive!library}
\index{Элемент архива} \index{archive!member}
\newword{Библиотечный архив} (\newword{archive library}), обычно
называемый просто библиотекой или архивом,~--- это специальный файл,
содержащий в себе другие файлы, именуемые \newword{элементами архива}
(\newword{archive members}). Например, стандартная библиотека языка
\Clang{} \filename{libc.a} содержит низкоуровневые функции. Библиотеки
используются настолько часто, что \GNUmake{} имеет специализированную
функциональность для создания, поддержки и компоновки архивов. Архивы
\index{Программы!ar@\utility{ar}}
создаются и модифицируются при помощи программы \utility{ar}.

Давайте вернёмся к нашему примеру. Мы можем модифицировать нашу
программу подсчёта слов, упаковав все её компоненты, пригодные для
повторного использования, в библиотеку.  Наша библиотека будет
состоять из двух файлов: \filename{counter.o} и \filename{lexer.o}.
Для создания библиотеки вызовем команду \filename{ar}:

{\footnotesize
\begin{alltt}
\$ \textbf{ar rv libcounter.a counter.o lexer.o}
a - counter.o
a - lexer.
\end{alltt}
}

Опции \command{rv} означают, что мы хотим заменить элементы библиотеки
указанными объектными файлами, и что \utility{ar} должен выводить отчёт
о своих действиях. Мы можем использовать действие замены даже в том
случае, если указанная библиотека не существует. Первым аргументом
после опций является имя библиотеки, за ним следуют имена объектных
файлов (некоторые версии \filename{ar} требуют опции \utility{c} в
случае, если библиотека ещё не существует, но GNU \utility{ar} не
требует этого). Две строки, следующие за вызовом команды \utility{ar},
являются отчётом о том, что объектные файлы были добавлены в
библиотеку.

Использование опции замены позволяет создавать и изменять архив
последовательно:

{\footnotesize
\begin{alltt}
\$ \textbf{ar rv libcounter.a counter.o}
r - counter.o
\$ \textbf{ar rv libcounter.a lexer.o}
r - lexer.o
\end{alltt}
}

Теперь \utility{ar} предваряет имена файлов символом "r". Это значит,
что файлы в архиве были заменены.

Библиотека может быть скомпонована в исполняемый файл несколькими
способами. Самый простой способ~--- просто указать имя библиотеки в
списке аргументов компилятора.  В свою очередь, компилятор или
компоновщик будут использовать расширение для определения типа каждого
из указанных в командной строке файлов:

{\footnotesize
\begin{verbatim}
cc count_words.o libcounter.a /lib/libfl.a -o count_words
\end{verbatim}
}

Компилятор \utility{cc} распознает два файла \filename{libcounter.a} и
\filename{/lib/libfl.a} как библиотеки и будет искать в них
недостающие символы.  Ещё одним способом ссылки на библиотеку является
опция \command{-l}:

{\footnotesize
\begin{verbatim}
cc count_words.o -lcounter -lfl -o count_words
\end{verbatim}
}

Как вы можете видеть, при использовании этой опции опускается префикс
и суффикс имени библиотеки. Опция \command{-l} делает командную строку
более компактной и удобочитаемой, однако, при использовании этой опции
вы получаете гораздо более весомое преимущество. Когда компилятор
\utility{cc} видит опцию \command{-l}, он \emph{ищет} библиотеку в
стандартных каталогах системных библиотек. Это избавляет программиста
от необходимости знать точный путь к файлу библиотеки и делает команду
компоновки более переносимой. К тому же, в системах, поддерживающих
разделяемые библиотеки (библиотеки с расширением \filename{.so} на
системах семейства \UNIX{}), компоновщик будет искать сначала
разделяемые библиотеки, и только если подходящей не обнаружено, будет
осуществлён поиск библиотечного архива. Такой подход позволяет
программам пользоваться преимуществами разделяемых библиотек без их
явной спецификации. Таково стандартное поведение компилятора и
компоновщика GNU. Старые компоновщики и компиляторы могут не
осуществлять такой оптимизации.

Список каталогов, в которых компилятор должен осуществлять поиск
библиотек, может быть изменён с помощью опции \command{-L},
указывающей список и порядок каталогов, в которых нужно искать
библиотеки. Эти каталоги будут добавлены в список прямо перед
системными каталогами библиотеки и будут использоваться для всех опций
\command{-l} в командной строке. На самом деле, компиляции в
предыдущем примере не завершится успехом, поскольку текущий каталог не
входит в список каталогов библиотек \filename{cc}. Мы можем решить эту
проблему добавлением текущего каталога в список как показано ниже:

{\footnotesize
\begin{verbatim}
cc count_words.o -L. -lcounter -lfl -o count_words
\end{verbatim}
}

Библиотеки вносят некоторые трудности в процесс сборки программ. Какие
возможности предоставляет \GNUmake{} для упрощения этого процесса? GNU
\GNUmake{} включает функциональность как по созданию библиотек, так и
использованию библиотек при компоновке. Давайте посмотрим, как это
работает.

%---------------------------------------------------------------------
% Creating and updating libraries
%---------------------------------------------------------------------
\subsection{Создаём и изменяем библиотеки}

Библиотеки фигурируют в \Makefile{}'е в качестве обычных файлов. Ниже
представлено простое правило для создания нашей библиотеки:

{\footnotesize
\begin{verbatim}
libcounter.a: counter.o lexer.o
    $(AR) $(ARFLAGS) $@ $^
\end{verbatim}
}

Это правило использует встроенные переменные \variable{AR} и
\variable{ARFLAGS}, содержащие имя программы \utility{ar} и
стандартные опции \command{rv} соответственно. Для спецификации файла
архива используется автоматическая переменная \variable{\$@}, а для
спецификации реквизитов~--- автоматическая переменная \variable{\$\^}.

Теперь, если вы укажите файл \filename{libcounter.a} в качестве
реквизита цели \target{count\_words}, \GNUmake{} обновит нашу
библиотеку перед компоновкой исполняемого файла. Обратите внимание на
одну деталь. \emph{Все} элементы архива будут замещены, даже если
среди них есть не изменявшиеся с момента последнего обновления архива
элементы. Чтобы не терять время впустую, мы можем написать более
подходящее правило:

{\footnotesize
\begin{verbatim}
libcounter.a: counter.o lexer.o
    $(AR) $(ARFLAGS) $@ $?
\end{verbatim}
}

Если вы используете \variable{\$?} вместо \variable{\$\^},
\GNUmake{} будет подставлять в список аргументов только те объектные
файлы, которые имеют более позднюю дату модификации, чем цель.

Можем ли мы ещё улучшить это правило? Может быть да, а может и нет.
\GNUmake{} имеет встроенную поддержку обновления отдельных файлов в
архиве, но прежде, чем мы вдадимся в эти детали, стоит сделать
несколько важных замечаний относительно такого подхода к работе с
библиотеками. Одна из основных задач \GNUmake{} состоит в том, чтобы
эффективно использовать время процессора и собирать только те файлы,
которые действительно в этом нуждаются. К сожалению, вызов
\utility{ar} для каждого элемента архива по отдельности при наличии
несколько десятков файлов занимает настолько много времени, что
перевешивает преимущество элегантного синтаксиса, рассмотренного
далее. Используя простой метод, представленный выше, мы можем вызвать
\utility{ar} один раз для всех изменившихся файлов и избежать
множества ненужных системных вызовов \filename{fork/exec}. Кроме того,
на многих системах использование ключа \command{r} при вызове
\utility{ar} очень неэффективно. На моём компьютере 1.9 GHz Pentium 4
создание большого архива, содержащего 14216 элементов общим размером
55 MB, занимает 4 минуты 24 секунды, в то время как замена одного
элемента в этом архиве требует 28 секунд. Таким образом, создание
архива заново будет более быстрой альтернативой замене элементов при
наличии более 10 (из 14216!) изменившихся файлов. В такой ситуации
более разумным подходом будет единовременное обновление архива с
использованием автоматической переменной \variable{\$?}. Для небольших
библиотек и более быстрых компьютеров нет причин отказываться от
элегантного подхода, описанного ниже, в пользу более простого, но и
более быстрого.

В GNU \GNUmake{} элемент архива может быть специфицирован при помощи
следующей нотации:

{\footnotesize
\begin{verbatim}
libgraphics.a(bitblt.o): bitblt.o
    $(AR) $(ARFLAGS) $@ $<
\end{verbatim}
}

Здесь \filename{libgraphics.a}~--- это имя библиотеки, а
\filename{bitblt.o} (сокращение от \newword{bit block transfer,
передача битовых блоков})~--- имя её элемента. Синтаксис
\filename{libgraphics.a(bitblt.o)} означает модуль, содержащийся в
библиотеке. Реквизитом для цели является сам объектный файл, а
командой~--- добавление этого файла в архив. Автоматическая переменная
\variable{\$<} используется для получения первого реквизита. На самом
деле существует встроенное шаблонное правило, предоставляющее в
точности ту же функциональность.

Когда мы соединим всё это воедино, наш \Makefile{} будет выглядеть
следующим образом:

{\footnotesize
\begin{verbatim}
VPATH    = src include
CPPFLAGS = -I include

count_words: libcounter.a /lib/libfl.a

libcounter.a: libcounter.a(lexer.o) libcounter.a(counter.o)

libcounter.a(lexer.o): lexer.o
    $(AR) $(ARFLAGS) $@ $<

libcounter.a(counter.o): counter.o
    $(AR) $(ARFLAGS) $@ $<

count_words.o: counter.h

counter.o: counter.h lexer.h

lexer.o: lexer.h
\end{verbatim}
}

При запуске \GNUmake{} выводит следующее:

{\footnotesize
\begin{alltt}
\$ \textbf{make}
gcc -I include -c -o count\_words.o src/count\_words.c
flex -t src/lexer.l > lexer.c
gcc -I include -c -o lexer.o lexer.c
ar rv libcounter.a lexer.o
ar: creating libcounter.a
a - lexer.o
gcc -I include -c -o counter.o src/counter.c
ar rv libcounter.a counter.o
a - counter.o
gcc count\_words.o libcounter.a /lib/libfl.a -o count\_words
rm lexer.c
\end{alltt}
}

Обратите внимание на правило обновления архива. Автоматическая
переменная \variable{\$@} приняла значение имени библиотеки, несмотря
на то, что имя цели в \Makefile{}'е было
\filename{libcounter.a(lexer.o)}.

Наконец, нужно отметить, что библиотечный архив включает индекс всех
символов, содержащихся в нём. Новые программы архиваторов, такие как
GNU \utility{ar}, обновляют этот индекс автоматически при добавлении в
архив нового символа. Более старые версии архиваторов могут этого не
делать. Для создания и обновления индекса архива используется
\index{Программы!runlib}
программа \utility{ranlib}. В системах со старой версией архиватора
должно использоваться правило следующего вида:

{\footnotesize
\begin{verbatim}
libcounter.a: libcounter.a(lexer.o) libcounter.a(counter.o)
    $(RANLIB) $@
\end{verbatim}
}

Вы также можете использовать альтернативный подход для больших
архивов:

{\footnotesize
\begin{verbatim}
libcounter.a: counter.o lexer.o
    $(RM) $@
    $(AR) $(ARFLGS) $@ $^
    $(RANLIB) $@
\end{verbatim}
}

Конечно, синтаксис управления элементами архива может использоваться с
применением встроенных правил. GNU \GNUmake{} содержит встроенные
правила обновления архивов. Если мы используем эти правила, наш
\Makefile{} будет выглядеть следующим образом:

{\footnotesize
\begin{verbatim}
VPATH    = src include
CPPFLAGS = -I include

count_words: libcounter.a -lfl

libcounter.a: libcounter.a(lexer.o) libcounter.a(counter.o)

count_words.o: counter.h

counter.o: counter.h lexer.h

lexer.o: lexer.h
\end{verbatim}
}

%---------------------------------------------------------------------
% Using libraries as prerequisites
%---------------------------------------------------------------------
\subsection{Использование библиотек в качестве реквизитов}

Когда библиотеки появляются в качестве реквизитов, они могут быть
обозначены с помощью расширения файла или опции \command{-l}. Если
указать имя файла библиотеки:

{\footnotesize
\begin{verbatim}
xpong: $(OBJECTS) /lib/X11/libX11.a /lib/X11/libXaw.a
    $(LINK) $^ -o $@
\end{verbatim}
}

{\noindent то компоновщик просто прочитает библиотечные файлы из
командой строки.  При использовании опции \command{-l} реквизиты вовсе
не выглядят обычными файлами:}

{\footnotesize
\begin{verbatim}
xpong: $(OBJECTS) -lX11 -lXaw
    $(LINK) $^ -o $@
\end{verbatim}
}

Когда в реквизитах используется форма \command{-l}, \GNUmake{}
производит поиск библиотеки (предпочитая разделяемую версию) и
подставляет абсолютный путь в переменные \variable{\$\^} и
\variable{\$?}. Одно из преимуществ такого подхода состоит в
возможности производить автоматический поиск библиотек даже в том
случае, если компоновщик в вашей системе не поддерживает такой
возможности. Другим преимуществом является возможность настройки
путей поиска \GNUmake{}, что позволяет вам производить поиск
собственных библиотек наравне с системными. В приведённом примере
первая форма (с использованием абсолютных путей) будет игнорировать
разделяемые библиотеки. При использовании же второй формы \GNUmake{}
будет знать, что разделяемые библиотеки более предпочтительны, поэтому
сначала произведёт поиск разделяемой версии \filename{X11}, и только в
случае неудачи будет выбрана статическая библиотека. Шаблоны
для распознавания имён библиотек хранятся в виде реквизитов специальной
\index{Цели!специальные!LIBPATTERNS@\target{.LIBPATTERNS}}
цели \target{.LIBPATTERNS} и могут быть настроены для различных
форматов имён библиотек.

К сожалению, есть одна неприятная мелочь. Если в какая-либо цель в
\Makefile{}'е специфицирует библиотеку, на неё нельзя ссылаться в
реквизитах с помощью опции \command{-l}.  Например, запуск \GNUmake{}
с таким \Makefile{}'ом:

{\footnotesize
\begin{verbatim}
count_words: count_words.o -lcounter -lfl
    $(CC) $^ -o $@ libcounter.a: libcounter.a(lexer.o)

libcounter.a(counter.o)
\end{verbatim}
}

{\noindent завершится неудачей со следующей ошибкой:}

{\footnotesize
\begin{verbatim}
No rule to make target `-lcounter', needed by `count_words'
\end{verbatim}
}

Причиной ошибки является то, что \GNUmake{} не совершил подстановку
\filename{libcounter.a} вместо \filename{-lcounter} и поиск подходящей
цели. Вместо этого был осуществлён обычный поиск библиотеки. Таким
образом, для библиотек, собранных в \GNUmake{}, должно указывается
непосредственно имя файла.

Компоновка больших программ без возникновения ошибок подобна искусству
чёрной магии. Компоновщик производит поиск библиотек в том порядке, в
каком они указаны в командной строке. Таким образом, если библиотека
\filename{A} содержит неопределённый символ, например, \textit{open},
определённый в библиотеке \filename{B}, то \filename{A} должна быть
указана в командной строке \emph{перед} \filename{B} (именно так,
\filename{A} требует \filename{B}). Иначе, когда компоновщик прочитает
\filename{A} и не найдёт определения символа \filename{open}, будет
слишком поздно возвращаться назад к \filename{B}. Компоновщик никогда не
осуществляет поиск в уже просмотренных библиотеках. Таким образом,
порядок появления библиотек в командной строке играет фундаментальное
значение.

Когда реквизиты цели сохраняются в переменных \variable{\$\^} и
\variable{\$?}, порядок их следования также сохраняется. Это
справедливо даже для реквизитов, размещённых в нескольких правилах. В
этом случае реквизиты каждого правила добавляются к списку реквизитов
в том порядке, в котором они появляются.

Родственной проблемой является проблема перекрёстных ссылок между
\index{Циклические ссылки}
библиотеками,также известных как \newword{циклические ссылки}
\index{Зацикливания}
(\newword{circular references}) или \newword{зацикливания}
(\newword{circularities}). Предположим, что после некоторой
модификации библиотека \filename{B} использует символ из \filename{A}.
Мы уже знаем, что \filename{A} должна быть указана до \filename{B}, но
теперь ещё и \filename{B} должно быть указана до \filename{A}.
Решением является ссылка на \filename{A} и до, и после ссылки на
\filename{B}: \command{-lA -lB -lA}. В больших и сложных программах
библиотеки часто должны повторяться подобным образом, иногда более
одного раза.

Такая ситуация ставит небольшую проблему при использовании \GNUmake{},
поскольку автоматические переменные, как правило, не содержат
дубликатов. Например, предположим, что нам нужно повторить библиотеку
в реквизитах для устранения циклических ссылок:

{\footnotesize
\begin{verbatim}
xpong: xpong.o libui.a libdynamics.a libui.a -lX11
    $(CC) $^ -o $@
\end{verbatim}
}

Этот список реквизитов после подстановки переменных будет выглядеть
следующим образом:

{\footnotesize
\begin{verbatim}
gcc xpong.o libui.a libdynamics.a /usr/lib/X11R6/libX11.a -o xpong
\end{verbatim}
}

Для подавления последствий такого поведения переменной \variable{\$\^}
в \GNUmake{} была добавлена переменная \variable{\$+}. Эта переменная
идентична \variable{\$\^} с той лишь разницей, что в списке реквизитов
сохраняются дубликаты. Используем \variable{\$+}:

{\footnotesize
\begin{verbatim}
xpong: xpong.o libui.a libdynamics.a libui.a -lX11
    $(CC) $+ -o $@
\end{verbatim}
}

Теперь список реквизитов породит следующую команду компоновки:

\begin{verbatim}
gcc xpong.o libui.a libdynamics.a libui.a \
/usr/lib/X11R6/libX11.a -o xpong
\end{verbatim}

%---------------------------------------------------------------------
% Double-colon rules
%---------------------------------------------------------------------
\subsection{Правила с двойным двоеточием}

\index{Правила!с двойным двоеточием}
Правила с двойным двоеточием~--- это реализация функциональности,
позволяющей собирать одну и ту же цель с помощью разных сценариев, в
зависимости от того, какое из подмножеств реквизитов было
модифицировано. Обычно если цель появляется более одного раза, все её
реквизиты соединяются в один список, сценарий сборки же для одной цели
может быть указан только один раз. При использовании же правил с
двойным двоеточием каждое появление цели рассматривается как отдельное
правило и обрабатывается индивидуально. Это значит, что для какой-то
определённой цели все правила должны быть одного типа: либо с одним
двоеточием, либо с двумя.

По-настоящему полезные применения этой возможности придумать довольно
сложно, поэтому давайте рассмотрим следующий искусственный пример:

{\footnotesize
\begin{verbatim}
file-list:: generate-list-script
    chmod +x $<
    generate-list-script $(files) > file-list

file-list:: $(files)
    generate-list-script $(files) > file-list
\end{verbatim}
}

Мы можем создать цель \target{file-list} двумя способами. Если
сценарий составления списка файлов изменился, то добавим файлу
сценария права на запуск и выполним его. Если изменились исходные
файлы, мы просто запускаем сценарий.  Несмотря на свою надуманность,
пример наглядно демонстрирует, как можно использовать эту
функциональность.

Мы рассмотрели большую часть функциональности \GNUmake{}, связанной с
правилами, которые, наряду с переменными и сценариями, составляют
самую сущность \GNUmake{}. Мы фокусировали внимание главным образом на
специфику синтаксиса и поведение различных возможностей, практически
не останавливаясь на способах их применения в более сложных ситуациях.
Это будет главным объектом нашего внимания во второй части книги. А
сейчас продолжим обсуждение переменных и команд.


%%%-------------------------------------------------------------------
%%% Variables and Macros
%%%-------------------------------------------------------------------
\chapter{Переменные и макросы}
\label{chap:vars}

Мы уже видели переменные в \Makefile{}'ах и множество примеров их
использования как во встроенных, так и в определённых пользователем
правилах. Однако те примеры, которые мы видели, являются лишь вершиной
айсберга. Переменные и макросы могут быть гораздо более сложными.
Именно они придают GNU \GNUmake{} часть его невероятной мощи.

Прежде, чем мы продолжим, важно осознать, что \GNUmake{} является
смешением двух языков. Первый язык описывает граф зависимостей,
состоящий из целей и реквизитов (этот язык был подробно рассмотрен в
\index{Макроязык}
главе~\ref{chap:rules}). Второй язык является макроязыком для
осуществления текстовых подстановок. Быть может, вы знакомы и с
другими макроязыками: препроцессор \Clang{}, \utility{m4}, \TeX{} и
макроассемблеры.  Как и эти макроязыки, \GNUmake{} позволяет
определять условные обозначения для длинной последовательности
символов и использовать их в вашей программе. Макропроцессор
распознает их в тексте программы и заменит на соответствующую
последовательность. Несмотря на то, что удобно думать о переменных в
\Makefile{}'е как о переменных в традиционных языках программирования,
есть существенное отличие между макропеременными и <<традиционными>>
переменными. Значения макропеременных подставляются сразу при встрече
их имени в тексте программы, порождая строку, которая затем также
сканируется на наличие макросов. Это отличие станет более ясным, когда
мы рассмотрим переменные \GNUmake{} подробнее.

Имена переменных могут содержать почти любые символы, включая многие
знаки пунктуации. Разрешаются даже пробелы, но, если вы считаете себя
здравомыслящим человеком, избегайте их. Не разрешается использовать в
составе имени переменных следующие символы: \command{:}, \command{\#}
и \command{=}.

Регистр букв в имени переменных имеет значение, то есть переменные
\variable{cc} и \variable{CC} являются различными переменными. Для
получения значения переменной нужно заключить её имя внутрь круглых
скобок, предваряемых символом доллара (\variable{\$( )}). Из этого
правила есть одно исключение: если имя переменной состоит из одного
символа, то круглые скобки можно опустить и писать просто
\index{Переменные!автоматические}
\command{\${}\textit{символ}}. Вот почему автоматически переменные
могут использоваться без круглых скобок. Как правило вам стоит
предпочитать форму со скобками и избегать переменных, имя которых
состоит из одного символа.

Значение переменной также может быть получено с использованием
фигурных скобок, например, \variable{\${}\{CC\}}. Эта форма
встречается довольно часто, в частности, в старых \Makefile{}'ах.
Трудно найти причину, по которой использование одной из этих форм было
бы предпочтительным.  Выберите для себя какую-то одну и
придерживайтесь её. Некоторые люди используют фигурные скобки для
ссылки на переменную, а круглые~--- для вызовов функций, подражая
синтаксису командного интерпретатора Bourne shell. В современных
\Makefile{}'ах используются круглые скобки, вот почему мы будем
придерживаться именно этого стиля в этой книге.

Существуют определённые соглашения относительно имён переменных. Все
буквы имени переменных, представляющих значения, не изменяемые в ходе
работы \GNUmake{} (констант), и которые могут быть указаны
пользователем через интерфейсы командной строки или переменных
окружения, должны быть заглавными. Слова внутри имён таких переменных
обычно разделяются подчёркиваниями. Имена переменных, которые
встречаются только внутри \Makefile{}'а, содержат только прописные
буквы, слова в них также разделяются подчёркиваниями. Наконец, в этой
книге имена всех функций, определяемых пользователем с помощью
переменных или макросов, состоят из прописных букв, слова внутри имён
разделяются знаками тире.  Другие соглашения относительно имён будут
оглашаться тогда, когда в этом будет необходимость. Следующие примеры
используют функциональность, которую мы ещё не обсуждали. Поскольку
они иллюстрируют применение соглашений именования, не старайтесь
вникать в детали:

{\footnotesize
\begin{verbatim}
# Обычные константы.
CC    := gcc
MKDIR := mkdir -p

# Внутренние переменные.
sources = *.c
objects = $(subst .c,.o,$(sources))

# Пара функций.
maybe-make-dir  = $(if $(wildcard $1),,$(MKDIR) $1)
assert-not-null = $(if $1,,$(error Illegal null value.))
\end{verbatim}
}

Значение переменной состоит из всех слов, находящихся справа от знака
присваивания без учёта начального пробела. Пробелы в конце строки
также входят в состав значения. Иногда это может вызывать проблемы,
например, при использовании переменных, чьи значения оканчиваются
пробелами, в сценариях командного интерпретатора:

{\footnotesize
\begin{verbatim}
LIBRARY = libio.a # LIBRARY заканчивается пробелом

missing_file:
    touch $(LIBRARY)
    ls -l | grep '$(LIBRARY)'
\end{verbatim}
}

Присваивание переменной содержит пробел, который становится более
заметным за счёт комментария (однако на самом деле комментария может и
не быть). После запуска \GNUmake{} мы увидим следующий вывод:

{\footnotesize
\begin{alltt}
\$ \textbf{make}

touch libio.a 
ls -l | grep 'libio.a '
make: *** [missing\_file] Error 1
\end{alltt}
}

Поскольку шаблон поиска, переданный программе \utility{grep}, также
содержит пробел, поиск его вхождения в выводе команды \utility{ls}
закончился неудачей. Позднее мы обсудим проблемы, связанные с
пробелами, более детально. А пока давайте рассмотрим поближе
переменные \GNUmake{}.

%%--------------------------------------------------------------------
%% What variables are used for
%%--------------------------------------------------------------------
\section{Для чего используются переменные}

В целом использование переменных для представления внешних программ
является хорошей идеей, поскольку позволяет пользователю легко
адаптировать \Makefile{} под своё окружение. Например, в системе
часто бывает несколько версий \utility{awk}: \utility{awk},
\utility{nawk} и \utility{gawk}. Создавая переменную \utility{AWK} для
хранения имени программы \utility{awk}, вы облегчаете жизнь
пользователям вашего \Makefile{}'а. К тому же, если безопасность
критична для вашей среды, доступ к внешним программам с указанием
абсолютного пути к их исполняемым файлам является хорошей практикой
решения проблем с переменной \variable{PATH} пользователя. Абсолютные
пути также уменьшают вероятность запуска троянского коня под видом
системной утилиты. И, конечно, абсолютные пути делают ваши
\Makefile{}'ы менее переносимым. Здесь вы должны сделать выбор,
основываясь на ваших собственных требованиях.

Хотя наше первое использование переменных заключалось в хранении
простых констант, переменные также могут быть использованы для
хранения последовательностей команд, например, следующий пример
определяет функцию для вывода отчёта о количестве свободных блоков в
файловой системе\footnote{Команда \utility{df} возвращает список всех
смонтированных файловых систем, их ёмкость и объём занятого
пространства. Если задан аргумент, то выводится статистика для
указанной файловой системы. Первая строка вывода содержит заголовки
столбцов. Вывод этой команды подаётся на вход сценария \utility{awk},
извлекающего вторую строку и игнорирующего остальные. Четвёртый
столбец в выводе \utility{df}~--- число свободных блоков.}:

{\footnotesize
\begin{verbatim}
DF  = df
AWK = awk
free-space := $(DF) . | $(AWK) 'NR == 2 { print $$4 }'
\end{verbatim}
}

Переменные могут использоваться для обоих указанных целей. Как мы
увидим позднее, это не единственные их применения. 

%%--------------------------------------------------------------------
%% Variable types
%%--------------------------------------------------------------------
\section{Типы переменных}
\label{var_types}

\index{variables!simple expanded}
\index{variables!recursively expanded}
В \GNUmake{} существует два типа переменных: упрощённо вычисляемые
(simple expanded variables) и рекурсивно вычисляемые (recursively
expanded variables). \newword{Упрощённо вычисляемые} переменные (или
\index{Переменные!простые}
\newword{простые переменные}) определяются при помощи оператора
присваивания <<\command{:=}>>:

{\footnotesize
\begin{verbatim}
MAKE_DEPEND := $(CC) -M
\end{verbatim}
}

Такие переменные называются <<упрощённо вычисляемыми>> потому, что
правая часть присваивания вычисляется непосредственно при чтении
\Makefile{}'а. При этом подставляется значение всех переменных
\GNUmake{}, входящих в правую часть, и результирующий текст
сохраняется в качестве значения переменной. Это поведение идентично
поведению большинства языков программирования и командных сценариев.
Например, вычисление предыдущей переменной, скорее всего, породит
текст \command{gcc -M}. Однако если переменная \variable{CC} не
определена, то переменная \variable{MAKE\_DEPEND} примет значение
\command{<пробел>-M}. В этом случае выражение \command{\$(CC)}
вычисляется как пустая строка, поскольку переменная \variable{CC} ещё
не определена. Отсутствие определения переменной не является ошибкой.
На самом деле это очень удобно. Большинство неявных правил содержат
неопределённые переменные, необходимые для настройки поведения правил
пользователями. Если пользователь не определяют никаких настроек,
переменные просто содержат пустые строки. Теперь рассмотрим пробел в
начале полученного значения. Правая часть присваивания после
отбрасывания начального пробела выглядит следующим образом:
\command{\${CC} -M}. После того, как ссылка на переменную
вычисляется как пустая строка, \GNUmake{} не производит повторного
сканирования и не удаляет начальные пробелы.

\index{Переменные!рекурсивные}
\newword{Рекурсивно вычисляемые} переменные (или просто рекурсивные
переменные) определяются при помощи оператора присваивания
<<\command{=}>>:

{\footnotesize
\begin{verbatim}
MAKE_DEPEND = $(CC) -M
\end{verbatim}
}

Второй тип переменных называется <<рекурсивно вычисляемые переменные>>
потому, что правая часть присваивания просто копируется \GNUmake{} и
сохраняется как значение переменной без вычисления. Вместо этого
вычисление происходит каждый раз, когда переменная
\emph{используется}. Быть может, более подходящим названием для таких
переменных~--- \newword{лениво вычисляемые} переменные, поскольку
вычисления откладываются до тех пор, пока не потребуется их результат.
Одним из удивительных следствий такого рода вычислений заключается в
том, что присваивания могут осуществляться <<в неправильном порядке>>:

{\footnotesize
\begin{verbatim}
MAKE_DEPEND = $(CC) -M
...
# Чуть позже
CC = gcc
\end{verbatim}
}

Теперь значение \variable{MAKE\_DEPEND} будет равно \command{gcc -M},
несмотря на то что значение \variable{CC} не определено в момент
присваивания значения переменной \variable{MAKE\_DEPEND}.

На самом деле рекурсивные переменные не являются просто ленивыми
присваиваниями (по крайней мере, обычными ленивыми присваиваниями).
Каждый раз при обращении к рекурсивной переменной её значение
вычисляется заново. Для переменных, определённых в терминах простых
констант, таких как \variable{MAKE\_DEPEND}, эта разница бессмысленна,
поскольку все переменные справа от оператора присваивания также
являются простыми константами. Однако представим, что переменная в
присваиваемом выражении представляет собой результат выполнения
некоторой программы, например, \utility{date}.  Каждый раз, когда
подобная рекурсивная переменная будет вычисляться, будет происходить
запуск программы \utility{date} и переменная будет получать новое
значение (в предположении, что повторное вычисление происходит по
крайней мере через секунду). Иногда это чрезвычайно полезно. А
иногда весьма раздражает!

%---------------------------------------------------------------------
% Other types of assignment
%---------------------------------------------------------------------
\subsection*{Другие виды присваивания}
\label{sec:other_types_of_assign}

В предыдущем примере мы видели два типа присваивания: <<\command{=}>>
для определения рекурсивных переменных и <<\command{:=}>> для
определения простых переменных. \GNUmake{} имеет ещё два оператора
присваивания.

\index{Операторы!условного присваивания}
Оператор <<\command{?=}>> называется \newword{оператором условного
присваивания переменной}. Для краткости мы будем называть его просто
условным присваиванием. Этот оператор осуществляет присваивание
переменной только в том случае, если её значение ещё не определено.

{\footnotesize
\begin{verbatim}
# Положить полученные файлы в каталог $(PROJECT_DIR)/out.
OUTPUT_DIR ?= \(PROJECT_DIR)/out
\end{verbatim}
}

В этом примере мы присвоим значение переменной \variable{OUTPUT\_DIR}
только в том случае, если оно ещё не было определено. Такое поведение
очень удобно для работы с переменными окружения. Мы обсудим этот
вопрос более подробно в разделе
<<\nameref{sec:where_vars_come_from}>>.

Другой оператор присваивания, \command{+=}, обычно называют
\emph{добавлением}. Как можно предположить из названия, этот оператор
добавляет текст к значению переменной. Может быть не очевидно, что это
довольно важная функциональность, необходимая при использовании
рекурсивных переменных. Значение справа от этого оператора
присваивания добавляется к значению переменной, \emph{не изменяя
в переменной первоначального значения}. <<Подумаешь,>>~--- скажете
вы,~--- <<Разве не так обычно работает добавление?>>. Да, но здесь
есть одна маленькая хитрость.

Добавление текста к простой переменной реализуется очевидным образом.
Оператор \command{+=} может быть реализован так:

{\footnotesize
\begin{alltt}
простая\_переменная := \$(простая\_переменная) что-то ещё
\end{alltt}
}

Поскольку значение простой переменной уже было вычислено, \GNUmake{}
может просто вычислить выражение \command{\$(простая\_переменная)},
добавить требуемый текст и закончить присваивание. Но рекурсивные
переменные порождают проблему. Следующая реализация недопустима:

{\footnotesize
\begin{alltt}
рекурсивная\_переменная = \$(рекурсивная\_переменная) что-то ещё
\end{alltt}
}

Это выражение является ошибкой, потому что для \GNUmake{} не
существует корректного способа его обработки. Если \GNUmake{} сохранит
текущее значение рекурсивной переменной плюс текст <<что-то ещё>>, то
не сможет вычислить правильное значение позднее. Более того, попытка
вычислить рекурсивную переменную, содержащую ссылку на себя, приводит
к бесконечному циклу:

{\footnotesize
\begin{alltt}
\$ \textbf{make}

makefile:2: *** Recursive variable `recursive' references
itself (eventually).  Stop.
\end{alltt}
}

Таким образом, оператор <<\command{+=}>> был создан специально для
возможности добавления текста к рекурсивным переменным. В частности,
этот оператор полезен при инкрементом определении значения переменной.

%%--------------------------------------------------------------------
%% Macros
%%--------------------------------------------------------------------
\section{Макросы}

Переменные хороши для хранения одиночных строк текста, но что если мы
хотим сохранить многострочное значение, такое, как командный сценарий,
который мы хотим выполнять в нескольких правилах? Например, следующая
последовательность команд может быть использована для создания \Java{}
архива (\filename{jar}) из \filename{.class} файлов:

{\footnotesize
\begin{verbatim}
echo Creating $@...
$(RM) $(TMP_JAR_DIR)
$(MKDIR) $(TMP_JAR_DIR)
$(CP) -r $^ $(TMP_JAR_DIR)
cd $(TMP_JAR_DIR) && $(JAR) $(JARFLAGS) $@ .
$(JAR) -ufm $@ $(MANIFEST)
$(RM) $(TMP_JAR_DIR)
\end{verbatim}
}

В начале длинной последовательности (наподобие предыдущей) я
предпочитаю печатать диагностическое сообщение. Это значительно
упрощает чтение вывода \GNUmake{}. После вывода сообщения мы помещаем
наши \filename{.class} файлы в новый временный каталог, удалив сначала
этот каталог, если он существует\footnote{%
Для достижения наилучшего эффекта переменная \variable{RM} должна
иметь значение \command{rm -rf}. Как правило, по умолчанию она
определена как \command{rm -f}, что безопаснее, но не так эффективно.
Переменная \variable{MKDIR} должна иметь значение \command{mkdir -p}, и
так далее (прим. автора).}, затем создав новый. После этого мы
копируем в этот каталог каталоги\hyp{}реквизиты (и все их
подкаталоги). Затем мы переходим в него и создаём jar\hyp{}архив с
именем цели. Наконец, мы добавляем к архиву файл манифеста и
осуществляем удаление временного каталога. Естественно, мы не хотим
делать копий этой последовательности команд, поскольку в будущем это
может усложнить поддержку. Мы можем рассмотреть вариант
упаковки всех этих команд в рекурсивную переменную, но такой подход
неудобен при поддержке и труден для чтения при выводе \GNUmake{} (вся
последовательность команд будет выведена на экран как одна большая
строка текста).

Вместо этого мы можем использовать <<упакованную последовательность
команд>> GNU \GNUmake{}, созданную при помощи директивы
\directive{define}.  Термин <<упакованная последовательность>> немного
\index{Макрос}
неуклюж, поэтому мы будем называть это \newword{макросом}. На самом
деле макросы~--- это просто ещё один метод определения переменных в
\GNUmake{}, позволяющий помещать символы окончания строки внутрь
значения переменной. Руководство пользователя GNU \GNUmake{}
использует слова \emph{переменная} и \emph{макрос} как синонимы. В
этой книге мы будем использовать термин \emph{макрос} исключительно
для обозначения переменных, определённых с помощью директивы
\directive{define}, если же определение происходит при помощи
оператора присваивания, будет применяться термин \emph{переменная}.

{\footnotesize
\begin{verbatim}
define create-jar
  @echo Creating $@...
  $(RM) $(TMP_JAR_DIR)
  $(MKDIR) $(TMP_JAR_DIR)
  $(CP) -r $^ $(TMP_JAR_DIR)
  cd $(TMP_JAR_DIR) && $(JAR) $(JARFLAGS) $@ .
  $(JAR) -ufm $@ $(MANIFEST)
  $(RM) $(TMP_JAR_DIR)
endef
\end{verbatim}
}

За директивой \directive{define} следуют имя макроса и новая строка.
Тело макроса содержит весь текст вплоть до слова \directive{endef},
которое должно находиться в отдельной строке. Макросы вычисляются
практически также, как и другие переменные, за тем исключением, что в
контексте командного сценария в начало каждой строки добавляется
символ табуляции. Вот пример, использующий эту особенность:

{\footnotesize
\begin{verbatim}
$(UI_JAR): $(UI_CLASSES)
    $(create-jar)
\end{verbatim}
}

Обратите внимание на символ \command{@}, добавленный перед командой
\command{echo}. \GNUmake{} не выводит команды с таким префиксом перед
их выполнением. Когда мы запустим \GNUmake{}, мы не увидим в выводе
команды \command{echo}, только результат её выполнения. Если префикс
\command{@} применяется внутри макроса, то его действие
распространяется только на ту строку, которую он предваряет. Однако,
если применить его при вызове макроса, его действие будет
распространятся на всё тело макроса:

{\footnotesize
\begin{verbatim}
$(UI_JAR): $(UI_CLASSES)
    @$(create-jar)
\end{verbatim}
}

Одним из результатов выполнения предыдущего примера будет следующий
вывод:

{\footnotesize
\begin{alltt}
\$ \textbf{make}
Creating ui.jar...
\end{alltt}
}

Использование префикса \command{@} рассматривается более детально в
разделе <<\nameref{sec:command_modifiers}>> главы~\ref{chap:commands}.

%%--------------------------------------------------------------------
%% When variables are expanded
%%--------------------------------------------------------------------
\section{Когда переменные получают свои значения}
\label{sec:when_vars_are_expanded}

В предыдущем разделе мы начали рассматривать <<закулисье>> вычисления
значения переменных. Результат во многом зависит от того, что уже было
определено, и где это было определено. Вы можете получить неожиданный
результат, даже если \GNUmake{} не может найти ошибки в вашей
спецификации. Так каковы же правила вычисления переменных? Как на
самом деле всё это работает?

После запуска \GNUmake{} выполняет свою работу в две фазы. Во время
первой фазы \GNUmake{} читает \Makefile{} и все включаемые им файлы.
На этом этапе переменные и правила загружаются во внутреннюю базу
данных \GNUmake{}, после чего создаётся граф зависимостей. Во время
второй фазы \GNUmake{} анализирует граф зависимостей и определяет
цели, которые нуждаются в сборке, затем выполняет командные сценарии
для сборки реквизитов.

Когда \GNUmake{} встречает директиву \directive{define} или
определение рекурсивной переменной, строки значения переменной или
тела макроса сохраняются вместе с символами новой строки без
каких-либо вычислений. Самый последний символ новой строки в теле
макроса не сохраняется в тексте определения. Иначе при вычислении
макроса читался бы один лишний символ новой строки.

При вычислении макроса полученный текст сразу же сканируется на
содержание других макросов или переменных, подлежащих вычислению, этот
процесс продолжается рекурсивно. Если макрос вычисляется в контексте
командного сценария, в начало каждой строки добавляется символ
табуляции.

Итак, вычисление переменных и макросов \GNUmake{} подчиняется
следующим правилам:
\begin{itemize}
%---------------------------------------------------------------------
\item
В случае присваивания переменной значения левая часть присваивания
всегда вычисляется сразу на первой фазе работы \GNUmake{}.
%---------------------------------------------------------------------
\item Вычисление правой части операторов \command{=} и \command{?=}
откладывается до тех пор, пока не потребуется значение соответствующей
переменной.
%---------------------------------------------------------------------
\item Правая часть оператора \command{:=} вычисляется сразу.
%---------------------------------------------------------------------
\item Правая часть оператора \command{+=} вычисляется сразу, если
переменная в левой части изначально была определена как простая, иначе
вычисление откладывается.
%---------------------------------------------------------------------
\item В определении макроса (использующего директиву
\directive{define}) имя определяемого макроса вычисляется сразу,
вычисление тела макроса откладывается.
%---------------------------------------------------------------------
\item Имена целей и реквизитов всегда вычисляются сразу, вычисление
команд всегда откладывается.
%---------------------------------------------------------------------
\end{itemize}

Таблица~\ref{tab:rules_for_imm_and_der_exp} содержит правила порядка
вычисления выражений при определении переменных.

%\begin{center}
\begin{table}
\begin{tabular}{|l|l|l|}
\hline
\multicolumn{1}{|c|}{\textbf{Определение}} &
\multicolumn{1}{c}{\textbf{\emph{A} вычисляется}} &
\multicolumn{1}{|c|}{\textbf{\emph{B} вычисляется}} \\
\hline
\(A = B\) & Сразу & При использовании \\
\hline
\(A ?= B\) & Сразу & При использовании\\
\hline
\(A := B\) & Сразу & Сразу \\
\hline
\(A += B\) & Сразу & Сразу или при использовании\\
\hline
\parbox{2.3cm}{
\begin{alltt}
define \emph{A}\\
\emph{B} \ldots\\
\emph{B} \ldots\\
endef
\end{alltt}
}
& Сразу & При использовании \\
\hline
\end{tabular}
\caption{Правила для незамедлительного и отложенного вычислений}
\label{tab:rules_for_imm_and_der_exp}
\end{table}
%\end{center}

Примите за правило определять переменные и макросы перед их
использованием. В частности, требуется, чтобы переменная,
используемая в описании цели или реквизита была определена.

Думаю, пример многое прояснит. Предположим, мы решили переделать наш
макрос \variable{free-space}. Сначала рассмотрим отдельные части
примера, затем соберём всё вместе.

{\footnotesize
\begin{verbatim}
BIN    := /usr/bin
PRINTF := $(BIN)/printf
DF     := $(BIN)/df
AWK    := $(BIN)/awk
\end{verbatim}
}

Мы определяем три переменных, содержащие имена программ, которые будут
использоваться в нашем макросе. Поскольку все переменные являются
простыми, их значения должны быть вычислены во время чтения
\Makefile{}'а.  Так как переменная \variable{BIN} определена раньше
остальных, её значение может быть использовано в определениях других
переменных.

Далее определим макрос \variable{free-space}.

{\footnotesize
\begin{verbatim}
define free-space
  $(PRINTF) "Free disk space "
  $(DF) . | $(AWK) 'NR == 2 { print $$4 }'
endef
\end{verbatim}
}

За директивой \directive{define} следует имя переменной, которое сразу
вычисляется. В нашем случае вычисления не требуется. Тело макроса
считывается и сохраняется не вычисленным.

Наконец, используем наш макрос внутри правила.

{\footnotesize
\begin{verbatim}
OUTPUT_DIR := /tmp

$(OUTPUT_DIR)/very_big_file:
    $(free-space)
\end{verbatim}
}

Когда считывается цель \target{\$(OUTPUT\_DIR)/very\_big\_file},
происходит подстановка значений всех переменных. Значение выражения
\command{\$(OUTPUT\_DIR)} вычисляется как \command{/tmp}, формируя
цель \target{/tmp/very\_big\_file}. Затем считывается командный
сценарий, ассоциированный с этой целью. Строки с командами
распознаются благодаря наличию символа табуляции, считываются и
сохраняются, но подстановка значений переменных и макросов не
происходит.

Теперь соберём отрывки воедино. Изменим порядок их следования для
иллюстрации алгоритма вычисления \GNUmake{}:

{\footnotesize
\begin{verbatim}
OUTPUT_DIR := /tmp

$(OUTPUT_DIR)/very_big_file:
    $(free-space)

define free-space
  $(PRINTF) "Free disk space "
  $(DF) . | $(AWK) 'NR == 2 { print $$4 }'
endef

BIN    := /usr/bin
PRINTF := $(BIN)/printf
DF     := $(BIN)/df
AWK    := $(BIN)/awk
\end{verbatim}
}

Заметим, что несмотря на то, что порядок строк кажется обратным,
выполнение происходит успешно. В этом заключается один из
замечательных эффектов рекурсивных переменных. Они могут быть
невероятно полезны и совершенно непонятны одновременно. Наш
\Makefile{} работает как нужно благодаря тому, что вычисление
командных сценариев и тел макросов откладываются до тех пор, пока
результаты этих вычислений не потребуются. Таким образом, порядок, в
котором появляются определения, не влияет на выполнение.

На второй фазе выполнения, когда \Makefile{} уже прочитан, \GNUmake{}
определяет цели, анализирует граф зависимостей и выполняет действия,
ассоциированные с каждым правилом. Поскольку мы специфицировали только
одну цель, не имеющую реквизитов
(\command{\$(OUTPUT\_DIR)/very\_big\_file}), \GNUmake{} просто выполнит
действия, с ассоциированные с ней (предположим, такой файл не
существует)~--- команду \command{\$(free-space)}. После вычислений
\GNUmake{} получит следующее:

{\footnotesize
\begin{verbatim}
/tmp/very_big_file:
    /usr/bin/printf "Free disk space "
    /usr/bin/df . | /usr/bin/awk 'NR == 2 { print $$4 }'
\end{verbatim}
}

Как только значения всех переменных вычислены, \GNUmake{} выполняет
команды одну за другой. Давайте рассмотрим две части \Makefile{}'а, в
которых порядок имеет значение. Как упоминалось ранее, имя цели
\command{\$(OUTPUT\_DIR)/very\_big\_file} вычисляется сразу. Если бы
определение переменной \variable{OUTPUT\_DIR} находилось после
спецификации правила, результатом вычисления имени стала бы строка
\command{/very\_big\_file}. Скорее всего, это не то, чего хотел
пользователь. Если бы определение \variable{BIN} было помещено после
определения \variable{AWK}, наши переменные получили бы значения
\command{/printf}, \command{/df} и \command{/awk}, так как оператор
\command{:=} вызывает немедленное вычисление правой части
присваивания. Однако в этом случае мы можем избежать проблемы,
использовав для определения переменных \variable{PRINTF},
\variable{DF} и \variable{AWK} оператор \command{=} вместо оператора
\command{:=} и сделав тем самым эти переменные рекурсивными.

Наконец, обратите внимание на одну деталь. Объявление переменных
\variable{OUTPUT\_DIR} и \variable{BIN} как рекурсивных не решило бы
рассмотренных проблем порядка спецификаций. Важно здесь то, что в
момент вычисления значения переменной
\command{\$(OUTPUT\_DIR)/very\_big\_file} и правых частей определений
\variable{PRINTF}, \variable{DF} и \variable{AWK} значения переменных,
на которые ссылаются эти выражения, должны быть уже определены.

%%--------------------------------------------------------------------
%% Target- and pattern-specific variables
%%--------------------------------------------------------------------
\section{Переменные, зависящие от цели или шаблона}

Обычно переменные принимают только одно значение во время выполнения
\GNUmake{}. Это гарантируется двухэтапной обработкой \Makefile{}'а.
На первом этапе \GNUmake{} читает \Makefile{}, производит присваивание
и вычисление переменных и строит граф зависимостей. На втором этапе
производится анализ графа зависимостей. Таким образом, когда происходит
выполнение команд, обработка переменных уже закончена. Предположим,
однако, что нам нужно переопределить значение переменной только для
какой-то цели или шаблона.

В приведённом ниже примере файл, который мы компилируем, требует
использования дополнительной опции \command{-DUSE\_NEW\_MALLOC=1},
которая не должна быть использована при компиляции других файлов:

{\footnotesize
\begin{verbatim}
gui.o: gui.h
    $(COMPILE.c) -DUSE_NEW_MALLOC=1 $(OUTPUT_OPTION) $<
\end{verbatim}
}

Проблема решена добавлением дубликата правила компиляции, включающего
необходимую опцию. Такой подход неудовлетворителен по нескольким
причинам. Во-первых, мы повторяем код. Если правило когда-нибудь
изменится, или если мы решим заменить встроенное правило собственным,
этот код потребует изменения, о котором легко можно забыть. Во-вторых,
если множество файлов требует специализированных опций, копирование и
вставка кусков кода быстро становится утомительным и рискованным
занятием (представьте, что у вас сотни таких файлов).

Для решения таких проблем \GNUmake{} предоставляет функциональность
\index{Переменные!зависящие от цели}
\newword{переменных, зависящих от цели}. Определения этих переменных
привязаны к цели и действительны только во время обработки цели или её
реквизитов. Используя эту функциональность, мы можем переписать наш
пример следующим образом:

{\footnotesize
\begin{verbatim}
gui.o: CPPFLAGS += -DUSE_NEW_MALLOC=1
gui.o: gui.h
$(COMPILE.c) $(OUTPUT_OPTION) $<
\end{verbatim}
}

Переменная \variable{CPPFLAGS} встроена в стандартное правило
компиляции исходных файлов \Clang{} и предназначена для опций
препроцессора \Clang{}. При помощи оператора \command{+=} мы добавляем
новую опцию к уже существующим. Теперь сценарий компиляции может
быть удалён полностью:

{\footnotesize
\begin{verbatim}
gui.o: CPPFLAGS += -DUSE_NEW_MALLOC=1
gui.o: gui.h
\end{verbatim}
}

Пока происходит сборка цели \target{gui.o}, значение переменной
\variable{CPPFLAGS} вдобавок к оригинальному значению будет содержать
строку \command{-DUSE\_NEW\_MALLOC=1}. Когда цель \filename{gui.o}
будет собрана, значение \variable{CPPFLAGS} будет восстановлено.

Общий синтаксис определения переменных, зависящих от цели, таков:

{\footnotesize
\begin{alltt}
\emph{цель ...: переменная}  = \emph{значение}
\emph{цель ...: переменная} := \emph{значение}
\emph{цель ...: переменная} += \emph{значение}
\emph{цель ...: переменная} ?= \emph{значение}
\end{alltt}
}

Как вы могли заметить, для определения таких переменных применимы все
формы оператора присваивания. Переменная не обязательно должна
существовать до присваивания.

Более того, присваивание переменным значения не осуществляется до
начала сборки цели. Таким образом, правая часть присваивания может
содержать ссылку на значение другой переменной, зависящей от этой
цели. Переменная также действительна во время сборки всех реквизитов.

%%--------------------------------------------------------------------
%% Where variables come from
%%--------------------------------------------------------------------
\section{Где определяются переменные}
\label{sec:where_vars_come_from}

Пока все переменные в наших \Makefile{}'ах определялись явно. На самом
деле они могут иметь и более сложное происхождение. Например, мы уже
видели, что переменные могут определяться через интерфейс командной
строки \GNUmake{}. На самом деле переменные могут определяться в
следующих источниках:

\begin{description}
%---------------------------------------------------------------------
\item[\emph{Файл}] \hfill \\
Естественно, переменные могут быть определены в \Makefile{}'е или в
файле, им подключенным.
%---------------------------------------------------------------------
\item[\emph{Командная строка}] \hfill \\
Переменные могут быть определены или переопределены прямо из командной
строки \GNUmake{}:

{\footnotesize
\begin{alltt}
\$ \textbf{make CFLAGS=-g CPPFLAGS='-DBSD -DDEBUG'}
\end{alltt}
}

Аргументы командной строки, содержащие символ \command{=}, являются
определениями переменных. Каждое такое определение должно быть
отдельным аргументом. Если значение переменной (или, упаси вас Боже,
её имя) содержит пробелы, нужно либо экранировать пробел, либо
заключить этот аргумент в кавычки.

Значение, полученное переменной через интерфейс командной строки,
переопределяет любое другое определение, сделанное в \Makefile{}'е или
содержащееся в окружении. Присваивания в командной строке могут
определять как простые, так и рекурсивные переменные с помощью
операторов \command{:=} и \command{=} соответственно. Однако всё же
существует возможность повысить приоритет определения переменной в
\index{Директивы!override@\directive{override}}
\Makefile{}'е, для этого нужно использовать директиву
\directive{override}.

{\footnotesize
\begin{verbatim}
# Для успешной компоновки необходим порядок байтов от старшего к
# младшим.
override LDFLAGS = -EB
\end{verbatim}
}

Разумеется, игнорировать явные определения пользователя стоит только в
чрезвычайных обстоятельствах.
%---------------------------------------------------------------------
\item[\emph{Окружение}] \hfill \\
\index{Переменные!окружения}
После старта \GNUmake{} все переменные окружения автоматически
становятся переменными \GNUmake{}. Эти переменные имеют очень низкий
приоритет, поэтому присваивание значения этим переменным, сделанное в
\Makefile{}'е или через командную строку, переопределит переменную
окружения. Вы можете форсировать переопределение переменных в
\Makefile{}'е переменными окружения при помощи опции
\index{Опции!environment-overrides@\command{-{}-environment\hyp{}overrides (-e)}}
\command{-{}-environment\hyp{}overrides} (или просто \command{-e}).

Когда \GNUmake{} вызывается рекурсивно, некоторые переменные из
родительского процесса \GNUmake{} передаются через окружение дочернему
процессу. По умолчанию только переменные, изначально определённые в
окружении, попадают в окружение дочернего процесса. Однако любая
переменная может быть помещена в окружение с помощью директивы
\index{Директивы!export@\directive{export}}
\directive{export}:

{\footnotesize
\begin{verbatim}
export CLASSPATH := \$(HOME)/classes:\$(PROJECT)/classes
SHELLOPTS = -x
export SHELLOPTS
\end{verbatim}
}

Вы также можете экспортировать все переменные:

{\footnotesize
\begin{verbatim}
export
\end{verbatim}
}

Обратите внимание на то, что \GNUmake{} может экспортировать даже те
переменные, имена которых содержат недопустимые с точки зрения
командного интерпретатора символы.  Например, результатом запуска
следующего \Makefile{}'а:

{\footnotesize
\begin{verbatim}
export valid-variable-in-make = Neat!
show-vars:
    env | grep '^valid-'
    valid_variable_in_shell=Great
    invalid-variable-in-shell=Sorry
\end{verbatim}
}

{\flushleft будет вывод:}

{\footnotesize
\begin{alltt}
\$ \textbf{make}
env | grep '\^{}valid-'
valid-variable-in-make=Neat!
valid\_variable\_in\_shell=Great
invalid-variable-in-shell=Sorry
/bin/sh: line 1: invalid-variable-in-shell=Sorry: command not found
make: *** [show-vars] Error 127
\end{alltt}
}

<<Недопустимая>> переменная командного интерпретатора была создана при
помощи экспорта из \GNUmake{} переменной
\variable{valid-variable-in-make}. Эта переменная не будет доступна
стандартными средствами интерпретатора, только при помощи трюков
наподобие применения программы \utility{grep} ко всему окружению. Тем
не менее, эта переменная будет доступна в дочернем процессе
\GNUmake{}. Мы рассмотрим применение рекурсивных вызовов \GNUmake{} во
второй части книги.

Вы также можете запретить экспорт переменной в окружение дочернего
процесса:

{\footnotesize
\begin{verbatim}
unexport DISPLAY
\end{verbatim}
}

\index{Директивы!export@\directive{export}}
\index{Директивы!unexport@\directive{unexport}} Директивы
\directive{export} и \directive{unexport} работают так же, как и
их аналоги из интерпретатора \utility{sh}.

\index{Операторы!условного присваивания}
Оператор условного присваивания очень удобен при работе с переменными
окружения. Допустим, вы определили в своём \Makefile{}'е стандартный
каталог для хранения создаваемых файлов и хотите предоставить
пользователю возможность легко его изменить. Условное присваивание
идеально подходит для таких ситуаций:

{\footnotesize
\begin{verbatim}
# Пусть каталогом по умолчанию будет $(PROJECT_DIR)/out.
OUTPUT_DIR ?= $(PROJECT_DIR)/out
\end{verbatim}
}

В этом случае присваивание произойдёт только в том случае, если
значение переменной \variable{OUTPUT\_DIR} ещё не определено. Мы можем
добиться того же эффекта более <<многословным>> способом:

{\footnotesize
\begin{verbatim}
ifndef OUTPUT_DIR
  # Пусть каталогом по умолчанию будет $(PROJECT_DIR)/out.
  OUTPUT_DIR = $(PROJECT_DIR)/out
endif
\end{verbatim}
}

Разница между этими двумя определениями заключается в том, что
оператор условного присваивания не будет выполнен, если переменная
\variable{OUTPUT\_DIR} определена, пусть даже имеет пустое значение,
\index{Директивы!условной обработки!ifdef@\directive{ifdef}}
тогда как операторы \directive{ifdef} и \directive{ifndef} проверяют
свой аргумент на пустоту. Таким образом, выражение
\command{OUTPUT\_DIR=} с точки зрения оператора условного присваивания
будет считаться определением, а с точки зрения оператора
\directive{ifdef}~--- нет.

Следует отметить, что чрезмерное использование переменных окружения
делает ваши \Makefile{}'ы менее переносимыми, поскольку разные
пользователи, как правило, имеют разное окружение. На самом деле, я
редко использую переменные окружения именно по этой причине.
%---------------------------------------------------------------------
\item[\emph{Автоматические переменные}] \hfill \\
\index{Переменные!автоматические}
Наконец, \GNUmake{} создаёт автоматические переменные непосредственно
перед выполнением сценария, ассоциированного с правилом.
\end{description}

Традиционно переменные окружения используются для упрощения управления
разницей в настройке машин разработчиков. Например, общей практикой
является создание среды разработки (инструментов разработки, исходного
кода и дерева полученных компиляцией файлов) на основании переменных
окружения, используемых в \Makefile{}'е. При таком подходе \Makefile{}
определяет одну переменную для корня каждого дерева каталогов. Если
корень дерева исходных файлов содержится в переменной
\variable{PROJECT\_SRC}, корень дерева объектных файлов~--- в
переменной \variable{PROJECT\_BIN}, а каталог библиотек~--- в
переменной \variable{PROJECT\_LIB}, то разработчик вправе определить
эти каталоги так, как он считает нужным.

Потенциальной проблемой такого подхода (и вообще использования
переменных окружения) является ситуация, когда упомянутые выше
переменные не определены. Одним из решений является определение
значений по умолчанию в \Makefile{}'е с помощью оператора условного
присваивания:

{\footnotesize
\begin{verbatim}
PROJECT_SRC ?= /dev/$(USER)/src
PROJECT_BIN ?= $(patsubst %/src,%/bin,$(PROJECT_SRC))
PROJECT_LIB ?= /net/server/project/lib
\end{verbatim}
}

Используя эти переменные для доступа к компонентам проекта, вы можете
создать среду разработки, адаптируемую под различные настройки машин
разработчиков (мы увидим более развёрнутые примеры во второй части
книги). Однако опасайтесь чрезмерного использования переменных
окружения. Как правило, \Makefile{} следует составлять так, чтобы как
можно меньше использовать окружение разработчика, предоставлять
разумные настройки по умолчанию и проверять наличие критически важных
компонентов.

%%--------------------------------------------------------------------
%% Conditional and include processing
%%--------------------------------------------------------------------
\section{Условная обработка и включения}
\label{sec:cond_inc_processing}
Части \Makefile{}'а могут быть опущены или выбраны для обработки во
время его чтения с помощью директив \newword{условной обработки}.
\index{Директивы!условной обработки}
Условия, контролирующие обработку, могут принимать несколько форм,
таких как <<A определено>> или <<A равно B>>. Например:

{\footnotesize
\begin{verbatim}
# переменная COMSPEC определена только в Windows.
ifdef COMSPEC
  PATH_SEP := ;
  EXE_EXT  := .exe
else
  PATH_SEP := :
  EXE_EXT  :=
endif
\end{verbatim}
}

В предыдущем примере обработается первая ветвь условия только в том
случае, если переменная \variable{COMSPEC} определена. Синтаксис
директив условной обработки имеет две формы:

{\footnotesize
\begin{alltt}
\emph{if-условие}
  текст для обработки если условие выполнено
endif
\end{alltt}
}

{\flushleft и:}

{\footnotesize
\begin{alltt}
\emph{if-условие}
  текст для обработки если условие выполнено
else
  текст для обработки если условие не выполнено
endif
\end{alltt}
}

Значения шаблона \ItalicMono{if-условие} могут быть следующими:

{\footnotesize
\begin{alltt}
ifdef  \emph{имя-переменной}
ifndef \emph{имя-переменной}
ifeq  \emph{тест}
ifneq \emph{тест}
\end{alltt}
}

При использовании директив \directive{ifdef\textbackslash{}ifndef}
\index{Директивы!условной обработки!ifdef@\directive{ifdef}}
\index{Директивы!условной обработки!ifndef@\directive{ifndef}}
\ItalicMono{имя-переменной} не нужно заключать в скобки
(\command{\$( )}). Наконец, значением шаблона \ItalicMono{тест}
может быть одно из следующих выражений:

{\footnotesize
\begin{alltt}
"\emph{a}" "\emph{b}"
(\emph{a},\emph{b})
\end{alltt}
}

В выражениях могут использоваться двойные или одинарные кавычки по
желанию (однако тип открывающейся и закрывающейся кавычки должен
совпадать).

Директивы условной обработки могут быть использованы внутри тела
макросов или в командных сценариях:

{\footnotesize
\begin{verbatim}
libGui.a: $(gui_objects)
    $(AR) $(ARFLAGS) $@ $<
  ifdef RANLIB
    $(RANLIB) $@
  endif
\end{verbatim}
}

Я предпочитаю делать отступ перед директивами условной обработки,
однако неосторожное выравнивание может привести к ошибкам. В
предыдущем примере перед директивами сделан отступ в два пробела, а
заключённые в них команды предваряются символом табуляции. \GNUmake{}
не может распознать команды, не начинающиеся с символа табуляции. Если
директива условной обработки предваряется символом табуляции, она
рассматривается как команда и передаётся в командный интерпретатор.

\index{Директивы!условной обработки!ifeq@\directive{ifeq}}
\index{Директивы!условной обработки!ifneq@\directive{ifneq}}
Директивы \directive{ifeq} и \directive{ifneq} проверяют свои
аргументы на на равенство и неравенство соответственно. Пробелы в
условиях директив могут быть причиной трудноуловимых ошибок. Например,
когда используется форма теста с круглыми скобками, пробел после
запятой не учитывается, тогда как все остальные пробелы имеют
значение:

{\footnotesize
\begin{verbatim}
ifeq (a, a)
  # Равенство
endif

ifeq ( b, b )
  # Неравенство - ` b' != `b '
endif
\end{verbatim}
}

Лично я предпочитаю форму с кавычками:

{\footnotesize
\begin{verbatim}
ifeq "a" "a"
  # Равенство
endif

ifeq 'b' 'b'
  # Тоже равенство
endif
\end{verbatim}
}

Однако иногда случается так, что значение переменной содержит пробел в
начале или в конце. Это может быть источником ошибки, поскольку
сравнение учитывает все символы. Для создания более надёжных
\index{Функции!встроенные!strip@\function{strip}}
\Makefile{}'ов используйте функцию \function{strip}:

{\footnotesize
\begin{verbatim}
ifeq "$(strip $(OPTIONS))" "-d"
  COMPILATION_FLAGS += -DDEBUG
endif
\end{verbatim}
}

%---------------------------------------------------------------------
% include directive
%---------------------------------------------------------------------
\subsection{Директива \directive{include}}
\label{sec:include_directive}
\index{Директивы!include@\directive{include}}
Мы уже встречали директиву \directive{include} в
разделе~<<\nameref{sec:auto_dep_gen}>> главы~\ref{chap:rules}. Теперь
давайте рассмотрим её более детально.

Любой \Makefile{} может включать другие файлы. Наиболее частое
использование этой возможности~--- помещение общих определений
\GNUmake{} в заголовочный файл и включение автоматически составленных
файлов зависимостей. Директива \directive{include} используется
следующим образом:

\begin{alltt}
\footnotesize
include definitions.mk
\end{alltt}

Параметры директивы могут содержать произвольное число файлов, шаблоны
командного интерпретатора и переменные \GNUmake{}.

%---------------------------------------------------------------------
% include and dependencies
%---------------------------------------------------------------------
\subsection{Директива \directive{include} в контексте зависимостей}
Когда \GNUmake{} встречает директиву \directive{include}, он
производит раскрытие шаблонов и подстановку значений переменных, а
затем пытается прочитать подключенные файлы. Если файл существует,
выполнение продолжается. Если же указанный файл не существует,
\GNUmake{} выводит предупреждение и продолжает читать остаток
\Makefile{}'а. Когда весь \Makefile{} прочитан, \GNUmake{}
просматривает базу данных в поисках правила сборки подключаемых
файлов. Если соответствие найдено, \GNUmake{} следует найденному
правилу сборки цели. Если хотя бы один из включаемых файлов был
собран, \GNUmake{} очищает свою базу данных и производит повторное
чтение всего \Makefile{}'а. Если после завершения всего процесса
чтения, сборки и повторного чтения какая-то из директив
\directive{include} завершается неудачей ввиду отсутствующих файлов,
\GNUmake{} завершает своё выполнение с ненулевым кодом возврата.

Мы можем увидеть этот процесс в действии при помощи следующего
примера, состоящего из двух файлов. Мы используем встроенную функцию
\index{Функции!встроенные!warning@\function{warning}}
\function{warning}, для вывода сообщений из \GNUmake{} (эта и многие
другие функции рассмотрены в главе~\ref{chap:functions}). Вот
\Makefile{}:

{\footnotesize
\begin{alltt}
# Простой makefile, включающий файл.
include foo.mk
\$(warning Finished include)

foo.mk: bar.mk
    m4 --define=FILENAME=\${}@ bar.mk > \${}@
\end{alltt}
}

Ниже представлен \filename{bar.mk}, подключаемый в \Makefile{}:

{\footnotesize
\begin{alltt}
# bar.mk - выдать сообщение о чтении файла.
\$(warning Reading FILENAME)
\end{alltt}
}

После запуска \GNUmake{} мы увидим следующий вывод:

{\footnotesize
\begin{alltt}
\$ \textbf{make}
Makefile:2: foo.mk: No such file or directory
Makefile:3: Finished include
m4 --define=FILENAME=foo.mk bar.mk > foo.mk
foo.mk:2: Reading foo.mk
Makefile:3: Finished include
make: `foo.mk' is up to date.
\end{alltt}
}

Первая строка отражает тот факт, что \GNUmake{} не смог найти
включаемый файл, однако, вторая строка показывает, что чтение и
выполнение \Makefile{}'а продолжилось. По завершении чтения \GNUmake{}
обнаружил правило для  создания подключаемого файла,
\filename{foo.mk}, и выполнил соответствующий сценарий. Затем
\GNUmake{} начал весь процесс заново, в этот раз не встретив никаких
трудностей с чтением включаемого файла.

Теперь самое время заметить, что \GNUmake{} интерпретирует \Makefile{}
как потенциальную цель. После того, как \Makefile{} прочитан,
\GNUmake{} ищет правило для сборки текущего \Makefile{}'а.  Если такое
правило находится, \GNUmake{} выполняет соответствующий правилу
сценарий и проверяет, изменился ли текущий \Makefile{}. Если
\Makefile{} изменился, \GNUmake{} производит очистку своего состояния
и считывает \Makefile{} заново, повторяя анализ. Ниже приведён простой
пример бесконечного цикла, основанный на описанном поведении:

{\footnotesize
\begin{verbatim}
.PHONY: dummy

makefile: dummy
    touch $@
\end{verbatim}
}

Когда \GNUmake{} выполняет \Makefile{}, он видит, что файл нуждается в
сборке (поскольку цель \target{dummy} является абстрактной) и
выполняет команду \utility{touch}, которая изменяет время последней
модификации \Makefile{}'а. Затем \GNUmake{} читает файл и
обнаруживает, что он требует обновления\ldots{}. В общем, вы поняли.

Где \GNUmake{} ищет включаемые файлы? Если аргумент директивы
\directive{include} является абсолютным путём, то \GNUmake{} открывает
файл по указанному пути. Если указан относительный путь, \GNUmake{}
ищет файл относительно текущего рабочего каталога. Если файл не
найден, то осуществляется поиск в каталогах, указанных при помощи опции
\index{Опции!include-dir@\command{-{}-include-dir (-I)}}
\command{-{}-inc\-lu\-de\hyp{}dir} (или просто \command{-I}). Если
файл не найден и там, производится поиск в стандартных каталогах,
указанных при компиляции \GNUmake{}: \filename{/usr/local/include},
\filename{/usr/gnu/include}, \filename{/usr/include}. Пути могут
отличаться, поскольку они зависят от опций компиляции \GNUmake{}.

Если \GNUmake{} не может найти файл и не может собрать его с помощью
правила, происходит выход с ненулевым кодом возврата. Если вы хотите,
чтобы \GNUmake{} игнорировал включение несуществующих файлов, добавьте
знак дефиса перед директивой \directive{include}:

\begin{alltt}
\footnotesize
-include i-may-not-exist.mk
\end{alltt}

Для обратной совместимости с другими версиями \GNUmake{} слово
\index{Директивы!sinclude@\directive{sinclude}}
\index{Директивы!sinclude@\directive{-include}}
\directive{sinclude} является синонимом \directive{-include}.

%%--------------------------------------------------------------------
%% Standard make variables
%%--------------------------------------------------------------------
\section{Стандартные переменные \GNUmake{}}
\label{sec:std_make_vars}

\index{Переменные!стандартные}
Вдобавок к автоматическим переменным, \GNUmake{} содержит переменные,
предназначенные для получения текущего состояния \GNUmake{} и
настройки встроенных правил:

\begin{description}
%---------------------------------------------------------------------
% MAKE_VERSION
%---------------------------------------------------------------------
\item[\variable{MAKE\_VERSION}] \hfill \\
\index{Переменные!стандартные!MAKE\_VERSION@\variable{MAKE\_VERSION}}
Значением этой переменной является номер текущей версии GNU
\GNUmake{}.  Во время написания этой книги этим значением было
\command{3.80}, а CVS хранилище содержало версию \texttt{3.81rc1.}

Предыдущая версия \GNUmake{}, \command{3.79.1}, не поддерживающая
функций \function{eval} и \function{value}, имеет довольно широкое
распространение. Так что когда пишу свои \Makefile{}'ы, требующие
подобной функциональности, я использую эту переменную для проверки
версии \GNUmake{}. Мы увидим примеры такого использования в разделе
<<\nameref{sec:flow_ctrl_func}>> главы~\ref{chap:functions}.

%---------------------------------------------------------------------
% CURDIR
%---------------------------------------------------------------------
\item[\variable{CURDIR}] \hfill \\
\index{Переменные!стандартные!CURDIR@\variable{CURDIR}}
\index{Текущий каталог}
\index{current working directory}
Эта переменная содержит текущий рабочий каталог (current working
directory, cwd) исполняемого процесса \GNUmake{}. Это тот же самый
каталог, в котором была выполнена команда \GNUmake{} (и тот же самый
каталог, что хранится в переменной командного интерпретатора
\variable{PWD}), если только не использовалась опция
\index{Опции!directory@\command{-{}-directory (-C)}}
\command{-{}-di\-rec\-to\-ry} (\command{-C}). Опция
\command{-{}-di\-rec\-to\-ry} сообщает \GNUmake{}, что нужно изменить
текущий каталог перед тем, как начать искать \Makefile{}. Полная форма
опции выглядит следующим образом:
\command{-{}-di\-rec\-to\-ry=\emph{имя\hyp{}каталога}} или
\command{-C \emph{имя\hyp{}каталога}}. Если использовалась опция
\command{-{}-di\-rec\-to\-ry}, то переменная \variable{CURDIR} будет
содержать аргумент этой опции.

Обычно я вызываю \GNUmake{} из редактора \utility{emacs} во время
редактирования кода. Например, мой текущий проект написан на \Java{} и
использует единственный \Makefile{} в корневом каталоге проекта (не
обязательно в каталоге, содержащем исходный код). В моём случае
использование опции \command{-{}-di\-rec\-to\-ry} позволяет мне
запускать \GNUmake{} из любого каталога с исходными файлами и иметь
доступ к \Makefile{}'у. Внутри \Makefile{}'а все пути являются
относительными и вычисляются от каталога, в котором располагается
\Makefile{}. Абсолютные пути используются очень редко, доступ к ним
осуществляется при помощи переменной \variable{CURDIR}.

%---------------------------------------------------------------------
% MAKEFILE_LIST
%---------------------------------------------------------------------
\item[\variable{MAKEFILE\_LIST}] \hfill \\
\index{Переменные!стандартные!MAKEFEILE\_LIST@\variable{MAKEFEILE\_LIST}}
Значение этой переменной содержит список всех файлов, прочитанных
\GNUmake{}, включая стандартный \Makefile{}, все \Makefile{}'ы,
указанные в опциях командной строки, и файлы, подключённые директивой
\directive{include}. Перед чтением каждого файла \GNUmake{} добавляет
его имя к значению переменной \variable{MAKEFILE\_LIST}. Итак,
\Makefile{} всегда может определить собственное имя, прочитав
последнее слово в этом списке.

%---------------------------------------------------------------------
% MAKECMDGOALS
%---------------------------------------------------------------------
\item[\variable{MAKECMDGOALS}] \hfill \\
\index{Переменные!стандартные!MAKECMDGOALS@\variable{MAKECMDGOALS}}
Значение \variable{MAKECMDGOALS} содержит список всех целей, указанных
в командной строке для текущего процесса \GNUmake{}. Оно не включает
опций и определений переменных. Например:

{\footnotesize
\begin{alltt}
\$ \textbf{make -f- FOO=bar -k goal <<< 'goal:;\#{} \$(MAKECMDGOALS)'}
\#{} goal
\end{alltt}
}

Предыдущий пример использует возможность \GNUmake{} читать \Makefile{}
из стандартного потока ввода, включающуюся опцией \command{-f-}
\index{Опции!file@\command{-{}-file (-f-)}} (или
\command{-{}-fi\-le}). Происходит перенаправление стандартного потока
ввода на командную строку с помощью синтаксиса командного
интерпретатора \utility{bash} \verb|<<<|\footnote{ Если вы хотите
выполнить этот пример в другом интерпретаторе, наберите:

\begin{alltt}
\$ echo 'goal:;\#{} \$(MAKECMDGOALS)' | make -f- FOO=bar -k goal
\end{alltt}
}.

Сам \Makefile{} состоит из спецификации цели по умолчанию
\target{goal}. Командный сценарий указан на той же строке и
отделяется от определения цели точкой с запятой. Сценарий состоит из
одной строки:

{\footnotesize
\begin{verbatim}
# $(MAKECMDGOALS)
\end{verbatim}
}

Переменная \variable{MAKECMDGOALS} обычно используется в том случае,
когда цель требует особой обработки. Выразительным примером является
цель \target{clean}. Когда происходит сборка \target{clean},
\GNUmake{} не должен осуществлять стандартную проверку изменения
подключаемых файлов (обсуждавшуюся в разделе
<<\nameref{sec:cond_inc_processing}>> главы~\ref{chap:vars}).  Для
подавления этой проверки можно применить директиву \directive{ifneq} и
переменную \variable{MAKECMDGOALS}:

{\footnotesize
\begin{verbatim}
ifneq "$(MAKECMDGOALS)" "clean"
  -include $(subst .xml,.d,$(xml_src))
endif
\end{verbatim}
}

\item[\variable{.VARIABLES}] \hfill \\
\index{Переменные!стандартные!VARIABLES@\variable{.VARIABLES}}
Значение этой переменной содержит имена всех переменных, определённых
\index{Переменные!зависящие от цели}.
в \Makefile{}'е на данный момент, исключая переменные, зависящие от
цели. Эта переменная доступна только для чтения и любое её
переопределение игнорируется.

{\footnotesize
\begin{verbatim}
list:
    @echo "$(.VARIABLES)" | tr ' ' '\015' | grep MAKEF
\end{verbatim}
}

{\footnotesize
\begin{alltt}
\$ \textbf{make}
MAKEFLAGS
MAKEFILE\_LIST
MAKEFILES
\end{alltt}
}
\end{description}

Как вы уже видели, переменные также используются для настройки
встроенных неявных правил \GNUmake{}. Правила для компиляции и
компоновки исходных файлов \Clang{}/\Cplusplus{} демонстрируют
типичную форму, принимаемую переменными для произвольных языков
программирования.  Рисунок~\ref{fig:vars_for_c_comp} изображает
переменные, контролирующие трансформацию одного типа файлов в другой.

\begin{figure}[!ht]
\centering
\begin{picture}(320,220)
\tiny

\put(218,210){\texttt{.y}}
\put(250,210){\texttt{.l}}

\put(222,207){\vector(0,-1){26}}
\put(254,207){\vector(0,-1){26}}

\put(212,175){\texttt{YACC.y}}
\put(246,175){\texttt{LEX.l}}

\put(222,172){\vector(1,-2){14}}
\put(254,172){\vector(-1,-2){14}}

\put(14,140){\texttt{.S}}
\put(58,140){\texttt{.s}}
\put(108,140){\texttt{.S}}
\put(142,140){\texttt{.cpp}}
\put(186,140){\texttt{.C}}
\put(234,140){\texttt{.c}}
\put(272,140){\texttt{.y}}
\put(304,140){\texttt{.l}}

\put(18,137){\vector(0,-1){26}}
\put(62,137){\vector(0,-1){26}}
\put(112,137){\vector(0,-1){26}}
\put(148,137){\vector(0,-1){26}}
\put(190,137){\vector(0,-1){26}}
\put(238,137){\vector(0,-1){26}}
\put(276,137){\vector(0,-1){26}}
\put(308,137){\vector(0,-1){26}}

\put(0,105){\texttt{PREPROCESS.S}}
\put(50,105){\texttt{COMPILE.s}}
\put(98,105){\texttt{COMPILE.S}}
\put(134,105){\texttt{COMPILE.cpp}}
\put(176,105){\texttt{COMPILE.C}}
\put(224,105){\texttt{COMPILE.c}}
\put(266,105){\texttt{YACC.y}}
\put(300,105){\texttt{LEX.l}}

\put(18,102){\vector(0,-1){26}}
\put(62,102){\vector(3,-1){70}}
\put(112,102){\vector(1,-1){26}}
\put(148,102){\vector(0,-1){26}}
\put(184,102){\vector(-1,-1){26}}
\put(238,102){\vector(-3,-1){74}}
\put(276,102){\vector(1,-2){14}}
\put(308,102){\vector(-1,-2){14}}

\put(14,70){\texttt{.s}}
\put(58,70){\texttt{.S}}
\put(144,70){\texttt{.o}}
\put(184,70){\texttt{.cpp}}
\put(234,70){\texttt{.C}}
\put(288,70){\texttt{.c}}

\put(18,67){\vector(0,-1){26}}
\put(62,67){\vector(0,-1){26}}
\put(148,67){\vector(0,-1){26}}
\put(190,67){\vector(0,-1){26}}
\put(238,67){\vector(0,-1){26}}
\put(292,67){\vector(0,-1){26}}

\put(8,35){\texttt{LINK.s}}
\put(54,35){\texttt{LINK.S}}
\put(140,35){\texttt{LINK.o}}
\put(180,35){\texttt{LINK.cpp}}
\put(229,35){\texttt{LINK.C}}
\put(285,35){\texttt{LINK.c}}

\put(18,33){\vector(4,-1){110}}
\put(62,33){\vector(3,-1){78}}
\put(148,33){\vector(0,-1){26}}
\put(190,33){\vector(-3,-2){38}}
\put(238,33){\vector(-3,-1){78}}
\put(292,33){\vector(-4,-1){120}}

\put(134,0){\texttt{Executable}}
\end{picture}

\caption{Переменные для компиляции исходных файлов
\Clang{}/\Cplusplus{}.} \label{fig:vars_for_c_comp}
\end{figure}

Все эти переменные имеют одинаковую форму:
\ItalicMono{ДЕЙСТВИЕ.суффикс}. \ItalicMono{ДЕЙСТВИЕ} может быть
\variable{COMPILE} для создания объектного файла, \variable{LINK} для
создания исполняемого файла, или одной из <<специальных>> операций
\variable{PREPROCESS}, \variable{YACC} или \variable{LEX} (для запуска
препроцессора языка \Clang{}, \utility{yacc} и \utility{lex}
соответственно). Тип обрабатываемых файлов указывается с помощью части
\ItalicMono{суффикс}.

Стандартный <<путь>> через эти переменные, например, для \Cplusplus{},
проходит через два правила. Сначала происходит компиляция исходных
файлов \Cplusplus{} в объектные файлы, которые затем компонуются в
исполняемый файл.

{\footnotesize
\begin{verbatim}
%.o: %.C
    $(COMPILE.C) $(OUTPUT_OPTION) $<

%: %.o
    $(LINK.o) $\ $(LOADLIBES) $(LDLIBS) -o $@
\end{verbatim}
}

Первое правило использует следующие определения переменных:

{\footnotesize
\begin{verbatim}
COMPILE.C     = $(COMPILE.cc)
COMPILE.cc    = $(CXX) $(CXXFLAGS) $(CPPFLAGS) $(TARGET_ARCH) -c
CXX           = g++
OUTPUT_OPTION = -o $@
\end{verbatim}
}

GNU \GNUmake{} поддерживает два расширения исходных файлов
\Cplusplus{}: \filename{.C} и \filename{.cc}. Значением переменной
\variable{CXX} является имя компилятора \Cplusplus{}, по умолчанию это
\index{Переменные!стандартные!CXXFLAGS@\variable{CXXFLAGS}}
\index{Переменные!стандартные!CPPFLAGS@\variable{CPPFLAGS}}
\index{Переменные!стандартные!TARGETARCH@\variable{TARGET\_ARCH}}
\utility{g++}. Переменные \variable{CXXFLAGS}, \variable{CPPFLAGS} и
\variable{TARGET\_ARCH} (опции компилятора \Cplusplus{}, опции
препроцессора \Clang{} и опции, специфичные для архитектуры,
соответственно) не имеют значения по умолчанию. Они нужны для
настройки процесса сборки конечным пользователем. Переменная
\index{Переменные!стандартные!OUTPUT\_OPTION@\variable{OUTPUT\_OPTION}}
\variable{OUTPUT\_OPTION} содержит имя выходного файла.

Правило компоновки немного проще:

{\footnotesize
\begin{verbatim}
LINK.o = $(CC) $(LDFLAGS) $(TARGET_ARCH)
CC     = gcc
\end{verbatim}
}

\index{Переменные!стандартные!LDFLAGS@\variable{LDFLAGS}}
\index{Переменные!стандартные!TARGET\_ARCH@\variable{TARGET\_ARCH}}
\index{Переменные!стандартные!LDLIBS@\variable{LDLIBS}}
\index{Переменные!стандартные!LOADLIBES@\variable{LOADLIBES}}
Это правило использует компилятор языка \Clang{} для компоновки
объектных файлов в исполняемые. Компилятором языка \Clang{} по
умолчанию является \utility{gcc}. Переменные \variable{LDFLAGS} и
\variable{TARGET\_ARCH} не имеют значений по умолчанию. Значение
\variable{LDFLAGS} содержит опции компоновки, например, флаги
\command{-L}. Переменные \variable{LOADLIBES} и \variable{LDLIBS}
содержат список библиотек для компоновки. Такая избыточность была
введена из соображений переносимости.

Итак, мы завершили наш краткий обзор переменных \GNUmake{}. На самом
деле их гораздо больше, однако того, что мы рассмотрели, достаточно
для того, чтобы понять связь переменных и правил. Например, существует
группа переменных, предназначенных для работы с \TeX{}, и набор
соответствующих правил. Рекурсивный вызов \GNUmake{}~--- ещё одна
тема, для обсуждения которой нам понадобятся переменные. Мы вернёмся к
ней в главе~\ref{chap:managing_large_proj}.


%%%-------------------------------------------------------------------
%%% Functions
%%%-------------------------------------------------------------------
\chapter{Функции} \label{chap:functions}
GNU \GNUmake{} поддерживает как встроенные, так и определяемые
пользователем функции. Вызов функций во многом подобен обращению к
переменной, с той разницей, что содержит один или более параметров,
разделённых запятыми. Результат вычисления встроенных функций обычно
присваивается некоторой переменной или передаётся в дочерний процесс
командного интерпретатора. Функции, определяемые пользователем,
хранятся в виде значения переменной или тела макроса и ожидают
получить на вход один или более аргументов.

%%--------------------------------------------------------------------
%% User-defined functions
%%--------------------------------------------------------------------
\section{Функции, определяемые пользователем}
\label{sec:user_def_func}

\index{Функции!определяемые пользователем}
Хранение последовательностей команд в переменных открывает широкие
возможности для приложений. Например, вот небольшой макрос для
завершения процесса\footnote{
<<Зачем нам делать это в \Makefile{}'е?>>,~--- спросите вы. Например,
в Windows, открытие файла одним процессом делает его недоступным
для записи другому процессу. Пока я писал эту книгу, PDF файл часто
был заблокирован процессом программы Acrobat Reader, что не допускало
обновление PDF-файла с помощью \GNUmake{}. Поэтому я добавил эту
команду в несколько целей, чтобы завершать процесс Acrobat Reader
и открывать доступ к заблокированному файлу.
}:

{\footnotesize
\begin{verbatim}
AWK  := awk
KILL := kill

# $(kill-acroread)

define kill-acroread
  @ ps -W |                                           \
  $(AWK) 'BEGIN      { FIELDWIDTHS = "9 47 100" }     \
          /AcroRd32/ {                                \
                       print "Killing " $$3;          \
                       system( "$(KILL) -f " $$1 )    \
                     }'
endef
\end{verbatim}
}

\index{Cygwin}
Этот макрос был написан с использованием инструментов Cygwin\footnote{
Инструменты Cygwin~--- это порт многих стандартных утилит GNU и Linux
под Windows. Они включают набор компиляторов, X11R6, \utility{ssh} и
даже \utility{inetd}. Порт основывается на библиотеке, реализующей
системные вызовы \UNIX{} в терминах функций Win32 API. Это настоящий
инженерный шедевр, и я настоятельно рекомендую его вам. Загрузите
его с сайта \filename{\url{http://www.cygwin.com}}.
}, поэтому имена программ и опции команд \utility{ps} и \utility{kill}
нестандартны для \UNIX{}. Для завершения программы мы направляем вывод
программы \utility{ps} на вход \utility{awk}. Сценарий \utility{awk}
ищет процесс Acrobat Reader по его имени в системе Windows, и
завершает его в случае необходимости. Мы использовали возможности,
предоставляемые переменной \variable{FIELDWIDTH}, чтобы трактовать имя
команды со всеми аргументами как одно поле с точки зрения
\utility{awk}. Поэтому сценарий корректно напечатает полное имя
запущенной программы вместе с аргументами, даже если оно содержит
пробелы. Ссылки на поля в \utility{awk} обозначаются как
\variable{\${}1}, \variable{\${}2} и так далее.  Если обращения к
полям не экранировать, они могут быть рассмотрены как переменные
\GNUmake{}. Мы можем сообщить \GNUmake{}, что \variable{\${}n} не
нужно вычислять, с помощью экранирования символа доллара в
\variable{\${}n} вторым символом доллара, \variable{\${}\${}n}.  В
этом случае \GNUmake{} увидит два идущих подряд символа доллара,
уберёт лишний и передаст оставшийся в дочерний процесс командного
интерпретатора. 

Хороший макрос. К тому же, директива \directive{define} предохраняет
нас от дублирования кода в случае, если нам придётся использовать
макрос часто.  И всё же он далёк от совершенства. Что если мы захотим
завершить произвольный процесс, не только процесс Acrobat Reader?
Придётся ли нам определить ещё один макрос и написать сценарий заново?
Нет!

Переменным и макросам можно передавать аргументы, таким образом,
каждое их вычисление может отличаться от предыдущего. Параметры
макроса доступны в его теле через позиционные переменные
\variable{\${}1}, \variable{\${}2} и так далее. В качестве параметра
нашей функции \function{kill-acroread} подходит шаблон поиска:

{\footnotesize
\begin{verbatim}
AWK        := awk
KILL       := kill
KILL_FLAGS := -f
PS         := ps
PS_FLAGS   := -W
PS_FIELDS  := "9 47 100"

# $(call kill-program,awk-pattern)
define kill-program
  @ $(PS) $(PS_FLAGS) |                                  \
  $(AWK) 'BEGIN { FIELDWIDTHS = $(PS_FIELDS) }           \
          /$1/  {                                        \
                  print "Killing " $$3;                  \
                  system( "$(KILL) $(KILL_FLAGS) " $$1 ) \
                }'
endef
\end{verbatim}
}

Мы заменили шаблон поиска для \utility{awk}, \command{/AcroRd32/},
обращением к параметру, \variable{\${}1}. Заметим, что существует
тонкое различие между параметром макроса \variable{\${}1} и обращением
к полю \utility{awk} \variable{\${}\${}1}.  Очень важно помнить, какая
имена программа нуждается в получении значения переменной. Пока мы
улучшали нашу функцию, мы также присвоили ей более подходящее имя и
заменили жёстко закодированные значения (зависящие от Cygwin)
переменными. Теперь мы имеем довольно переносимый макрос для
завершения процесса.

Давайте испытаем его на практике:

{\footnotesize
\begin{verbatim}
FOP        := org.apache.fop.apps.Fop
FOP_FLAGS  := -q
FOP_OUTPUT := > /dev/null

%.pdf: %.fo
    $(call kill-program,AcroRd32)
    $(JAVA) $(FOP) $(FOP_FLAGS) $< $@ $(FOP_OUTPUT)
\end{verbatim}
}

Это шаблонное правило завершает процесс Acrobat Reader, если он
запущен на выполнение, и конвертирует \filename{fo}-файлы (Formatting
Objects) в \filename{pdf}-файл вызовом процессора \utility{fop}
(\filename{\url{http://xml.apache.org/fop}}). Вычисление макроса или
переменной имеет следующий синтаксис:

{\footnotesize
\begin{alltt}
\$(call \emph{имя-макроса}[, \emph{аргумент}\subi{1} . . . ])
\end{alltt}
}

\index{Функции!встроенные!call@\function{call}}
Встроенная функция \function{call} вычисляет свой первый аргумент,
заменяя в нём все вхождения позиционных аргументов (\variable{\${}1},
\variable{\${}2} и так далее) остальными своими аргументами.  (На
самом деле, она вовсе не <<вызывает>> макрос в смысле передачи
управления, а производит специальное вычисление макроса).  На месте
шаблона \ItalicMono{имя-макроса} может быть любая переменная или
макрос (вспомним, что макросы~--- это просто переменные, в значениях
которых допускаются символы новой строки). Макрос или переменная даже
не обязательно должны содержать позиционные аргументы
(\variable{\${}n}), однако в этом случае использование функции
\function{call} не имеет особого смысла. Аргументы макроса, следующие
за параметром \ItalicMono{имя-макроса}, разделяются запятыми.

Заметим, что первый аргумент функции \function{call} является именем,
а не значением переменной (то есть не начинается со знака доллара).
Это довольно необычно. Существуют и другие встроенные функции,
\index{Функции!встроенные!origin@\function{origin}}
принимающие на вход имена переменных, например, \function{origin}.
Если вы окружите первый аргумент функции \function{call} знаком
доллара и скобочками, его значение будет вычислено и передано на вход
\function{call}.

Проверка аргументов в функции \function{call} минимальна.  На вход
может быть передано любое количество аргументов.  Если макрос
использует позиционный аргумент \variable{\${}n}, для которого в
обращении к \function{call} нет фактического параметра, позиционный
аргумент заменяется пустой строкой.  Если аргументов \function{call}
больше, чем позиционных  аргументов макроса, они никогда не будут
подставлены в его тело.

Если вы вызываете один макрос из другого, вам стоит быть осторожным,
поскольку порой GNU \GNUmake{} 3.80 ведёт себя довольно странно.
Функция \function{call} во время подстановки определяет свои аргументы
как обычные переменные \GNUmake{}. Таким образом, если один макрос
вызывает другой, есть возможность, что аргументы вызывающего макроса
будут видны вызываемому макросу:

{\footnotesize
\begin{verbatim}
define parent
  echo "вызывающий макрос имеет два аргумента: $1, $2"
  $(call child,$1)
endef

define child
  echo "вызываемый макрос имеет один аргумент: $1"
  echo "однако он также имеет доступ к аргументу "
  echo "вызывающего макроса: $2!"
endef

scoping_issue:
    @$(call parent,one,two)
\end{verbatim}
}

После запуска \GNUmake{} мы сможем наблюдать проблему с областью
видимости:

{\footnotesize
\begin{alltt}
\$ \textbf{make}

вызывающий макрос имеет два аргумента: one, two
вызываемый макрос имеет один аргумент: one
однако он также имеет доступ к аргументу
вызывающего макроса: two!
\end{alltt}
}

В версии 3.81 это было исправлено, так что аргумент~\variable{\${}2}
вызываемого макроса вычислится как пустая строка.

Мы проведём ещё много времени, составляя собственные функции, однако
нам нужно ещё немного знаний, чтобы делать по-настоящему интересные
вещи!

%%--------------------------------------------------------------------
%% Build-in functions
%%--------------------------------------------------------------------
\section{Встроенные функции}
\label{sec:built_in_func}

\index{Функции!встроенные}
Как только вы начнёте использовать переменные для более сложных
манипуляций, нежели хранение констант, вы обнаружите потребность
манипулировать их содержимым. У вас есть такая возможность. GNU
\GNUmake{} имеет несколько десятков встроенных функций для работы с
переменными и их содержимым. Все функции попадают в одну из следующих
категорий: функции для анализа и подстановки строк; функции для
обработки имён файлов; функции управления выполнением; функции,
определяемые пользователем и вспомогательные функции.

Но сначала рассмотрим синтаксис обращения к функций. Вызов любой
функции имеет следующую форму:

{\footnotesize
\begin{alltt}
\$(\emph{имя-функции} \emph{аргумент\,\subi{1}} [, \emph{аргумент\,\subi{2}}, ...])
\end{alltt}
}

Сначала указывается \variable{\${}(}, затем следует имя функции, за
которым следуют аргументы. Начальный пробел у первого аргумента
вырезается, остальные аргументы подставляются со всеми начальными (и,
конечно, внутренними и оконечными) пробелами.  Аргументы функций
разделяются запятыми, таким образом, при вызове функции с одним
аргументом запятые не требуются, при вызове функции с двумя
аргументами используется одна запятая, и так далее. Многие функции
принимают один аргумент, интерпретируя его как последовательность
слов, разделённых пробелами. Для таких функций пробел считается
разделителем слов и попросту игнорируется, если таковым не является.

Мне нравится использовать пробелы. Они делают код более читабельным и
удобным для поддержки. Поэтому я буду использовать пробелы везде, где
это сойдёт мне с рук. Однако иногда пробелы в списке аргументов или
определении переменной могут мешать корректному выполнению кода.
Когда это случается, у вас не остаётся другого выбора, кроме как
убрать лишний пробел. Мы уже видели один пример, в котором лишний
пробел случайно попал в шаблон поиска программы \utility{grep}.  Мы
будем указывать, где пробелы могут вызвать проблемы, при разборе
конкретных примеров.

Многие функции \GNUmake{} принимают в качестве аргумента шаблон.
Этот шаблон следует тому же синтаксису, что и шаблоны, используемые в
шаблонных правилах (см. раздел <<\nameref{sec:pattern_rules}>>
главы~\ref{chap:rules}). Шаблон может содержать один символ
\command{\%{}} с некоторым суффиксом или префиксом (или и тем, и
другим). Символ \command{\%{}} соответствует нулю или более
произвольных символов. Шаблону может соответствовать только вся
строка, но не подмножество символов внутри строки.  Позже мы
проиллюстрируем это коротким примером. Символ \command{\%{}}
опционален и при желании его можно опустить.

%---------------------------------------------------------------------
% String functions
%---------------------------------------------------------------------
\subsection{Строковые функции}
\label{sec:str_func}

Б\'{о}льшая часть функций \GNUmake{} переводит текст из одной формы в
другую, однако в этом разделе мы рассмотрим только те из них, которые
предоставляют наиболее мощную функциональность по манипулированию
строками.

Одной из наиболее общих операций над строками в \GNUmake{} является
выделение подмножества файлов из списка. В командных сценариях
для этого обычно используется \utility{grep}. В \GNUmake{} мы
можем использовать функции \function{filter}, \function{filter-out}
и \function{findstring}.

\begin{description}
%---------------------------------------------------------------------
% filter
%---------------------------------------------------------------------
\item[\texttt{\$(filter \emph{шаблон}...,\emph{текст})}] \hfill \\
\index{Функции!встроенные!filter@\function{filter}}
Функция \function{filter} интерпретирует аргумент \ItalicMono{текст}
как последовательность слов, разделённых пробелами и возвращает список
тех из них, которые соответствуют \ItalicMono{шаблону}. Например, для
для сборки библиотеки пользовательского интерфейса, мы можем выбрать
все объектные файлы из подкаталога \filename{ui}. В следующем примере
мы извлечём из списка всех имён фалов проекта имена, начинающиеся с
\filename{ui/} и заканчивающиеся \filename{.o}. Символ \texttt{\%{}}
соответствует любому числу символов между префиксом и суффиксом:

{\footnotesize
\begin{verbatim}
$(ui_library): $(filter ui/%.o,$(objects))
    $(AR) $(ARFLAGS) $@ $^
\end{verbatim}
}

Функция \function{filter} может принимать несколько шаблонов,
разделённых пробелами. Как уже упоминалось ранее, для того, чтобы
некоторое слово было включено в выходной список, оно должно
соответствовать шаблону целиком. Рассмотрим следующий пример:

{\footnotesize
\begin{verbatim}
words := he the hen other the%

get-the:
    @echo he matches:   $(filter he,   $(words))
    @echo %he matches:  $(filter %he,  $(words))
    @echo he% matches:  $(filter he%,  $(words))
    @echo %he% matches: $(filter %he%, $(words))
\end{verbatim}
}

Когда мы запустим \GNUmake{}, вывод будет следующим:

{\footnotesize
\begin{alltt}
\$ \textbf{make}

he matches: he
\%he matches: he the
he\% matches: he hen
\%he\% matches: the\%
\end{alltt}
}

Как видите, первый шаблону соответствует только слово \command{he},
так как на выход \function{filter} попадают только слова,
соответствующие шаблону целиком. Остальным шаблонам соответствует
слово \command{he} и слова, содержащие \command{he} в нужной позиции.

Шаблон может содержать только один метасимвол \command{\%{}}.
Дополнительные символы \command{\%} воспринимаются как литералы.

Может показаться странным, что функция \function{filter} не может
находить подстроки внутри слова или принимать более одного
метасимвола.  Во многих случаях вам будет не хватать подобной
функциональности.  Однако вы можете реализовать нечто похожее с
помощью циклов и условных операторов. Позже мы рассмотрим, как это
можно сделать.

%---------------------------------------------------------------------
% filter-out
%---------------------------------------------------------------------
\item[\texttt{\$(filter-out \emph{шаблон}...,\emph{текст})}] \hfill \\
\index{Функции!встроенные!filter-out@\function{filter-out}}
Функция \function{filter-out} осуществляет действие, обратное действию
функции \function{filter}, отбирая слова, не соответствующие шаблону.
В следующем примере мы отсеим все заголовочные файлы \Clang{}:

{\footnotesize
\begin{verbatim}
all_source := count_words.c counter.c lexer.l \
              counter.h lexer.h
to_compile := $(filter-out %.h, $(all_source))
\end{verbatim}
}

%---------------------------------------------------------------------
% findstring
%---------------------------------------------------------------------
\item[\texttt{\$(findstring \emph{строка},\emph{текст})}] \hfill \\
\index{Функции!встроенные!findstring@\function{findstring}}
Эта функция ищет вхождение \ItalicMono{строки} в \ItalicMono{тексте}.
Если вхождение обнаружено, функция возвращает \ItalicMono{строку},
иначе возвращается пустая строка.

Поначалу может показаться, что эта функция подобна функциональности
поиска подстрок программы \utility{grep} (ранее мы выдвигали на эту
роль функцию \function{filter}), но это не так. Первое и самое важное
отличие заключается в том, что \function{findstring} возвращает только
искомую подстроку, а не всё слово, в которое эта подстрока входит.
Вторым отличием является отсутствие возможности использования
метасимвола \command{\%} (если поместить этот символ в
\ItalicMono{строку}, он будет восприниматься как литерал).

Наиболее часто эта функция используется совместно с функцией
\function{if}, обсуждаемой дальше в этой главе. Однако есть одна
ситуация, в которой функция \function{findstring} полезна сама по
себе.

Предположим, что у нас есть несколько деревьев каталогов в файловой
системе, имеющих сходную структуру, например, каталоги исходного кода
приложения, исходного кода исполняющей среды, бинарных файлов с
таблицей символов и бинарных файлов, скомпилированных с флагами
оптимизации.  Скорее всего, в ваших сценариях вам потребуется
узнавать, внутри какого именно дерева вы находитесь (не получая при
этом относительный путь от корневого каталога проекта).  Вот набросок
кода для получения этой информации:

{\footnotesize
\begin{verbatim}
find-tree:
    # PWD = $(PWD)
    # $(findstring /test/book/admin,$(PWD))
    # $(findstring /test/book/bin,$(PWD))
    # $(findstring /test/book/dblite_0.5,$(PWD))
    # $(findstring /test/book/examples,$(PWD))
    # $(findstring /test/book/out,$(PWD))
    # $(findstring /test/book/text,$(PWD))
\end{verbatim}
}

Каждая строка начинается с символа табуляции и символа комментария
\utility{shell}, поэтому <<выполняется>> в отдельном дочернем процессе
командного интерпретатора, как и обычные команды. Командный
интерпретатор <<Bourne Again Shell>>, \utility{bash}, и многие
родственные ему интерпретаторы просто проигнорируют эти строки. Это
более распространённый способ распечатки результатов вычисления
конструкций \GNUmake{}, чем использование команды \command{@echo}. Вы
можете добиться того же эффекта, если будете использовать более
переносимый оператор командного интерпретатора \command{:}, однако
оператор \command{:} осуществляет перенаправление потоков. Таким
образом, команда, содержащая подстроку \command{> слово}, в качестве
побочного эффекта создаст файл \command{слово}. После запуска
предыдущего примера мы получим следующий вывод:

{\footnotesize
\begin{alltt}
\${} \textbf{make}
\begin{verbatim}
# PWD = /test/book/out/ch03-findstring-1
# 
# 
# 
# 
# /test/book/out
#
\end{verbatim}
\end{alltt}
}

Как вы можете видеть, каждый тест переменной \variable{\${}(PWD)}
возвращал пустую строку, пока не встретился нужный каталог. Как уже
было замечено, этот код лишь демонстрирует возможности
\function{findstring} и может быть использован для определения
текущего дерева каталогов.
\end{description}

Существует две функции для поиска и замены строк:

\begin{description}
%---------------------------------------------------------------------
% subst
%---------------------------------------------------------------------
\item[\texttt{\${}(subst \emph{строка-поиска},\emph{строка-замены},\emph{текст})}] \hfill \\
\index{Функции!встроенные!subst@\function{subst}}
Это функция для простого (не содержащего шаблонов) поиска и замены.
Одно из наиболее частых применений этой функции~--- замена суффиксов в
списке имён файлов:

{\footnotesize
\begin{verbatim}
sources := count_words.c counter.c lexer.c
objects := $(subst .c,.o,$(sources))
\end{verbatim}
}

В предыдущем примере все вхождения строки <<\filename{.c}>> в значении
переменной \variable{\${}(sources)} будут заменены строкой
<<\filename{.o}>>, или, более \'{о}бщно, все вхождения
\ItalicMono{строки-поиска} будут заменены \ItalicMono{строкой-замены}.

Эта функция также является примером ситуации, когда пробелы в списке
аргументов функции имеют значение. Обратите внимание, что после
запятых нет пробелов. Если бы мы написали такой код:

{\footnotesize
\begin{verbatim}
sources := count\_words.c counter.c lexer.c
objects := $(subst .c, .o, $(sources))
\end{verbatim}
}

{\flushleft (обратите внимание на пробелы после запятых), то значением
переменной \variable{\${}(objects)} стала бы следующая строка:}

{\footnotesize
\begin{verbatim}
count_words .o counter .o lexer .o
\end{verbatim}
}

Это не совсем то, что мы хотим. Проблема заключается в том, что пробел
перед аргументом \command{.o} является частью
\ItalicMono{строки-замены}, и поэтому попадает в результирующий текст.
Пробел перед аргументом \command{.c} не вызовет проблем, потому что
\GNUmake{} пропускает пробелы перед первым аргументом. Пробел перед
\variable{\${}(sources)} также не вызовет проблемы, поскольку
переменная \variable{\${}(objects)} скорее всего будет использоваться
как простой аргумент командной строки.  Однако я стараюсь никогда не
смешивать различные стили расстановки пробелов при вызове функций,
даже если это приводит к корректным результатам:

{\footnotesize
\begin{verbatim}
# Пожалуй, не самая удачная расстановка пробелов.
objects := $(subst .c,.o, $(source))
\end{verbatim}
}

Заметим, что \function{subst} не понимает имён файлов или их
суффиксов, только последовательности символов. Если какой-то файл с
исходным кодом содержит внутри имени подстроку \command{.c}, то она
будет заменена.  Например, имя файла \filename{car.cdr.c} будет
преобразовано в \filename{car.odr.o}. Вполне вероятно, что это не то,
что вы ожидали.

В разделе <<\nameref{sec:auto_dep_gen}>> главы~\ref{chap:rules} мы
обсуждали составление зависимостей. Последний пример \Makefile{}'а
этого раздела использовал \function{subst} следующим образом:

{\footnotesize
\begin{verbatim}
VPATH = src include
CPPFLAGS = -I include
SOURCES = count_words.c \
          lexer.c       \
          counter.c
count_words: counter.o lexer.o -lfl
count_words.o: counter.h
counter.o: counter.h lexer.h
lexer.o: lexer.h
include $(subst .c,.d,$(SOURCES))

%.d: %.c
    $(CC) -M $(CPPFLAGS) $< > $@.$$$$;                  \
    sed 's,\($*\)\.o[ :]*,\1.o $@ : ,g' < $@.$$$$ > $@; \
    rm -f $@.$$$$
\end{verbatim}
}

Функция \function{subst} использована для преобразования списка файлов
с исходным кодом в список файлов зависимостей. Поскольку файлы
зависимостей появляются в качестве директивы \directive{include}, они
рассматриваются как реквизиты и создаются при помощи шаблонного
правила \target{\%{}.d}.

%---------------------------------------------------------------------
% patsubst
%---------------------------------------------------------------------
\item[\texttt{\${}(patsubst \emph{шаблон-поиска},\emph{шаблон-замены},\emph{текст})}] \hfill \\
\index{Функции!встроенные!patsubst@\function{patsubst}}
Эта функция отличается от предыдущей тем, что допускает использование
шаблонов для поиска и замены символов. Шаблон, как обычно, может
содержать один и только один метасимвол \verb|%|. Метасимвол \verb|%|
в \ItalicMono{строке-замены} заменяется соответствующим текстом. Важно
помнить, что \ItalicMono{текст} должен полностью соответствовать
\ItalicMono{шаблону-поиска}. Например, следующий пример удалит только
последний (не каждый) слэш в \ItalicMono{тексте}:

{\footnotesize
\begin{verbatim}
strip-trailing-slash = $(patsubst %/,%,$(directory-path))
\end{verbatim}
}

\index{Подстановочная ссылка}
Использование \newword{подстановочной ссылки (substitution reference)}
является переносимым способом осуществления такой замены. Синтаксис
использования подстановочной ссылки таков:

{\footnotesize
\begin{alltt}
\${}(\emph{имя-переменной}:\emph{шаблон-поиска}=\emph{шаблон-замены})
\end{alltt}
}

\ItalicMono{Шаблон-поиска} может быть простой строкой, в этом случае
он заменяется на \ItalicMono{шаблон-замены} в конце каждого слова, то
есть когда за ним следует пробел или конец значения переменной.
\ItalicMono{Шаблон-поиска} может также содержать символ \command{\%},
в этом случае поиск и замена осуществляется по тем же правилам, что и
для функции \function{patsubst}. Я нахожу этот синтаксис более трудным
для чтения и восприятия по сравнению с \function{patsubst}.
\end{description}

Как мы уже видели, переменные часто состоят из списка слов. Ниже
представлены функции для выбора слов из списка, подсчёта длины списка
и т.д. Как и в случае остальных функций \GNUmake{}, слова в списке
разделяются пробелами.

\begin{description}
%---------------------------------------------------------------------
% words
%---------------------------------------------------------------------
\item[\texttt{\${}(words \emph{список})}] \hfill \\
\index{Функции!встроенные!words@\function{words}}
Функция, возвращающая число слов в \ItalicMono{списке}.

{\footnotesize
\begin{verbatim}
CURRENT_PATH := $(subst /, ,$(HOME))
words:
    @echo Мой домашний каталог содержит \
	$(words $(CURRENT_PATH)) подкаталогов.
\end{verbatim}
}

Эта функция имеет множество применений, однако для того, чтобы
эффективно её использовать, нам нужно изучить ещё несколько функций.

%---------------------------------------------------------------------
% word 
%---------------------------------------------------------------------
\item[\texttt{\${}(word \emph{n},\emph{список})}] \hfill \\
\index{Функции!встроенные!words@\function{words}}
Эта функция возвращает слово, находящееся в \ItalicMono{списке} под
номером \ItalicMono{n}, первое слово списка имеет номер 1. Если
\ItalicMono{n} больше числа слов в \ItalicMono{тексте}, функция
возвращает пустую строку.

{\footnotesize
\begin{verbatim}
version_list  := $(subst ., ,$(MAKE_VERSION))
minor_version := $(word 2, $(version_list))
\end{verbatim}
}

Переменная \variable{MAKE\_VERSION} является стандартной переменной
(см. раздел <<\nameref{sec:std_make_vars}>> главы~\ref{chap:vars}). Вы
можете получить последнее слово в списке следующим образом:

{\footnotesize
\begin{verbatim}
current := $(word $(words $(MAKEFILE_LIST)),\
                          $(MAKEFILE_LIST))
\end{verbatim}
}

Такая конструкция вернёт последний прочитанный \Makefile{}.

%---------------------------------------------------------------------
% firstword
%---------------------------------------------------------------------
\item[\texttt{\${}(firstword \emph{список})}] \hfill \\
\index{Функции!встроенные!firstword@\function{firstword}}
Эта функция возвращает первое слово в \ItalicMono{списке}.
Её действие эквивалентно вызову \texttt{\${}(word 1,\emph{список})}.

{\footnotesize
\begin{verbatim}
version_list  := $(subst ., ,$(MAKE_VERSION))
major_version := $(firstword $(version_list))
\end{verbatim}
}

%---------------------------------------------------------------------
% lastword 
%---------------------------------------------------------------------
\item[\texttt{\${}(lastword \emph{список})}] \hfill \\
\index{Функции!встроенные!lastword@\function{lastword}}
Эта функция, доступная в версиях GNU \GNUmake{} 3.81 и выше,
представляет дополнение к функции \function{firstword}. Она возвращает
последнее слово в \ItalicMono{списке}, что функционально эквивалентно
следующему выражению:

{\footnotesize
\begin{alltt}
\${}(word \${}(words \emph{список}),\emph{список})
\end{alltt}
}

%---------------------------------------------------------------------
% wordlist
%---------------------------------------------------------------------
\item[\texttt{\${}(wordlist \emph{номер-начала},\emph{номер-конца},\emph{список})}] \hfill \\
\index{Функции!встроенные!wordlist@\function{wordlist}}
Эта функция возвращает слова, имеющие в \ItalicMono{списке} номера с
\ItalicMono{номер-начала} по \ItalicMono{номер-конца} включительно.
Как и в функции \function{word}, первое слово имеет номер 1. Если
\ItalicMono{номер-начала} больше числа слов в \ItalicMono{списке}, то
функция вернёт пустую строку. Если \ItalicMono{номер-конца} больше
числа слов в \ItalicMono{списке}, функция вернёт всё слова начиная с
\ItalicMono{номера-начала}.

{\footnotesize
\begin{verbatim}
# $(call uid_gid, user-name)
uid_gid = $(wordlist 3, 4,   \
            $(subst :, ,     \
              $(shell grep "^$1:" /etc/passwd)))
\end{verbatim}
}
\end{description}

%---------------------------------------------------------------------
% Important miscellaneous functions
%---------------------------------------------------------------------
\subsection{Некоторые важные функции}
\label{sec:imp_misc_func}

Перед тем, как мы рассмотрим функции для работы с именами файлов,
следует ввести две чрезвычайно полезные функции: \function{sort} и
\function{shell}.

\begin{description}
%---------------------------------------------------------------------
% sort
%---------------------------------------------------------------------
\item[\texttt{\${}(sort \emph{список})}] \hfill \\
\index{Функции!встроенные!sort@\function{sort}}
Эта функция сортирует слова в \ItalicMono{списке}, удаляя дубликаты.
Результирующий список содержит уникальные слова из \ItalicMono{списка}
в лексикографическом порядке, разделённые одним пробелом. В дополнение
функция \function{sort} удаляет начальные и конечные пробелы.

{\footnotesize
\begin{alltt}
\${} \textbf{make -f- <<< 'x:;@echo =\${}(sort d b s d t )='}
=b d s t=
\end{alltt}
}

Функция \function{sort}, разумеется, реализована непосредственно
\GNUmake{} и потому не поддерживает опций стандартной утилиты
\utility{sort}. Эта функция оперирует лишь своим аргументом, обычно
это переменная или результат выполнения другой функции.

%---------------------------------------------------------------------
% shell
%---------------------------------------------------------------------
\item[\texttt{\${}(shell \emph{команда})}] \hfill \\
\index{Функции!встроенные!shell@\function{shell}}
Функция \function{shell} принимает на вход один аргумент, который
подвергается вычислению (как аргументы всех функций) и передаётся в
дочерний процесс командного интерпретатора для выполнения.
стандартный вывод команды считывается и возвращается в качестве
значения функции. Последовательности символов новой строки заменяются
одним пробелом, начальные и конечные символы новой строки удаляются.
Данные из стандартного потока ошибок не возвращаются, так же как и код
возврата команды.

{\footnotesize
\begin{verbatim}
stdout := $(shell echo normal message)
stderr := $(shell echo error message 1>&2)
shell-value:
    # $(stdout)
    # $(stderr)
\end{verbatim}
}

Как вы можете видеть, сообщения из стандартного потока ошибок попадают
на терминал, и потому не включаются в вывод функции \function{shell}:

{\footnotesize
\begin{alltt}
\${} \textbf{make}
error message
# normal message
#
\end{alltt}
}

Ниже приведён пример использования этой функции для создания набора
каталогов:

{\footnotesize
\begin{verbatim}
REQUIRED_DIRS = ...
_MKDIRS := $(shell for d in $(REQUIRED_DIRS);  \
             do                                \
                 [[ -d $$d ]] || mkdir -p $$d; \
             done)
\end{verbatim}
}

Часто \Makefile{} можно сделать проще, если все каталоги, в которые
\GNUmake{} осуществляет запись файлов, гарантированно существуют до
выполнения первой команды. Предыдущая переменная содержит код,
создающий требуемые каталоги при помощи цикла \command{for} командного
интерпретатора \utility{bash}. Двойные квадратные скобки являются
стандартным синтаксисом \utility{bash}, они работают почти также, как
программа \utility{test}. Отличие заключается в том, что внутри
двойных квадратных скобок не осуществляется разделение на слова и
подстановка путей. Поэтому если имя файла содержит пробел, тест будет
выполнен успешно (причём имя переменной не придётся заключать в
кавычки). Поскольку мы поместили определение переменной
\variable{\_MKDIRS} в начале \Makefile{}'а, мы можем быть уверены в
том, что её вычисление произойдёт до того, как другие команды или
вычисления переменных обратятся к каталогам. Значение
\variable{\_MKDIRS} для нас не важно и никогда не будет
использоваться.

Поскольку функция \function{shell} может быть использована для запуска
произвольной внешней программы, вам следует использовать её осторожно.
В частности, вам следует учитывать различия между простыми и
рекурсивными переменными.

{\footnotesize
\begin{verbatim}
START_TIME  := $(shell date)
CURRENT_TIME = $(shell date)
\end{verbatim}
}

Определение переменной \variable{START\_TIME} приводит к выполнению
программы \utility{date}. Переменная \variable{CURRENT\_DATE} будет
выполнять \utility{date} каждый раз при необходимости использования
значения в \Makefile{}'е.
\end{description}

Теперь у нас достаточно инструментов для того, чтобы писать довольно
интересные функции. Ниже приведён пример функции, которая проверяет,
содержит ли её аргумент одинаковые слова.

{\footnotesize
\begin{verbatim}
# $(call has-duplicates, word-list)
has-duplicates = $(filter                \
                   $(words $1),          \
                   $(words $(sort $1)))
\end{verbatim}
}

Мы считаем слова в первоначальном списке и в списке, из которого были
удалены все дубликаты, а затем <<сравниваем>> эти два числа.
\GNUmake{} не содержит функций, принимающий числа в качестве
аргументов, все аргументы воспринимаются как строки. Поэтому для того,
чтобы сравнить два числа, мы должны сравнить их как строки. Наиболее
простым способом сделать это является применение функции
\function{filter}. Мы <<ищем>> одно число в другом. Функция
\function{has-duplicates} вернёт непустую строку только в том случае,
если её аргумент содержит одинаковые слово.

А вот простой способ создать файл с временн\'{о}й меткой в имени:

{\footnotesize
\begin{verbatim}
RELEASE_TAR := mpwm-$(shell date +%F).tar.gz
\end{verbatim}
}

Это выражение порождает следующую строку:

{\footnotesize
\begin{verbatim}
mpwm-2003-11-11.tar.gz
\end{verbatim}
}

Мы могли бы достигнуть того же эффекта, заставив \utility{date}
проделать немного больше работы:

{\footnotesize
\begin{verbatim}
RELEASE_TAR := $(shell date +mpwm-%F.tar.gz)
\end{verbatim}
}

Следующая функция может быть использована для получения полного имени
\Java{}-класса из относительного пути (возможно, начиная с каталога
\filename{com}).

{\footnotesize
\begin{verbatim}
\# $(call file-to-class-name, file-name)
file-to-class-name := $(subst /,.,$(patsubst %.java,%,$1))
\end{verbatim}
}

Этот же эффект может быть достигнут при помощи следующего кода:

{\footnotesize
\begin{verbatim}
# $(call file-to-class-name, file-name)
file-to-class-name := $(subst /,.,$(subst .java,,$1))
\end{verbatim}
}

Мы можем использовать предыдущий пример для вызова классов \Java{}:

{\footnotesize
\begin{verbatim}
CALIBRATE_ELEVATOR := com/wonka/CalibrateElevator.java

calibrate:
    $(JAVA) $(call file-to-class-name,$(CALIBRATE_ELEVATOR))
\end{verbatim}
}

Если переменная \variable{\${}(sources)} содержит имена каталогов,
содержащих каталог \filename{com}, их можно убрать при помощи
следующей функции (корень дерева каталогов должен быть передан в
качестве первого аргумента\footnote{%
Существуют соглашения относительно структуры каталогов исходного кода
\Java{}: все классы должны быть объявлены в пакете, содержащем
доменное имя разработчика, записанное в обратном порядке (сначала
следуют домены более высокого уровня), а структура каталогов отражает
структуру пакетов. Однако довольно часто можно встретить подобную
структуру каталогов: \filename{root-dir/com/company-name/dir} (прим.
автора).}):

{\footnotesize
\begin{verbatim}
# $(call file-to-class-name, root-dir, file-name)
file-to-class-name := $(subst /,.,                 \
                        $(subst .java,,            \
                          $(subst $1/,,$2)))
\end{verbatim}
}

Функции, подобные этой, обычно легче понять, если читать их изнутри.
Начиная с внутреннего вызова \function{subst}, функция удаляет строку
\texttt{\${}1/}, затем удаляет строку \command{.java} и, наконец,
заменяет все слэши точками.

%---------------------------------------------------------------------
% Filename functions
%---------------------------------------------------------------------
\subsection{Функции для работы с файлами}
\label{sec:file_func}

Создатели \Makefile{}'ов тратят значительную часть времени на
осуществление обработки файлов. Поэтому неудивительно, что \GNUmake{}
содержит так много функций для облегчения этой задачи.

\begin{description}
%---------------------------------------------------------------------
% wildcard
%---------------------------------------------------------------------
\item[\texttt{\${}(wildcard \emph{шаблон} ...)}] \hfill \\
\index{Функции!встроенные!wildcard@\function{wildcard}}
Шаблоны были рассмотрены в главе~\ref{chap:rules} в контексте целей,
реквизитов и командных сценариев. Но что, если мы хотим получить эту
функциональность в другом контексте, например, в контексте определения
переменной? Конечно, мы могли бы использовать функцию \function{shell}
для подстановки шаблона с помощью дочернего процесса командного
интерпретатора, однако это будет очень неэффективным решением, если
подобная операция будет осуществляться достаточно часто.
Альтернативным подходом является использование функции
\function{wildcard}:

{\footnotesize
\begin{verbatim}
sources := $(wildcard *.c *.h)
\end{verbatim}
}

Функция \function{wildcard} принимает список шаблонов в качестве
аргумента и производит поиск файлов, соответствующих хотя бы одному
шаблону\footnote{%
Руководство пользователя GNU \GNUmake{} 3.80 почему-то умалчивает о
возможности использования нескольких шаблонов (прим. автора).}.
Если не обнаруживается файлов, соответствующих хотя-бы одному
шаблону, функция возвращает пустую строку. Как и в случае целей и
реквизитов, в шаблонах могут использоваться стандартные метасимволы
\command{\~}, \command{*}, \command{?}, \command{[...]},
\command{[\^\,...]}.

Другим применением функции \function{wildcard} является проверка
существования файлов в условных выражениях. Довольно часто встречается
использование \function{wildcard} с аргументом, не содержащим
метасимволов, совместно с функцией \function{if}. Например, в
следующем отрывке переменная \variable{dot-emacs-exists} будет
содержать непустое значение только в том случае, если домашний каталог
текущего пользователя содержит файл \filename{.emacs}:

{\footnotesize
\begin{verbatim}
dot-emacs-exists := $(wildcard ~/.emacs)
\end{verbatim}
}

%---------------------------------------------------------------------
% dir
%---------------------------------------------------------------------
\item[\texttt{\${}(dir \emph{список} ...)}] \hfill \\
\index{Функции!встроенные!dir@\function{dir}}
Функция \function{dir} возвращает каталог для каждого файла из
\ItalicMono{списка}. Следующее выражение вернёт все каталоги,
содержащие исходные файлы \Clang{}:

{\footnotesize
\begin{verbatim}
source-dirs = $(sort                             \
                $(dir                            \
                  $(shell find . -name '*.c')))
\end{verbatim}
}

Программа \utility{find} вернёт список всех файлов с исходным кодом,
функция \function{dirs} отсечёт имя файла, оставив лишь название
соответствующего каталога, а функция \function{sort} удалит
повторяющиеся каталоги. Заметим, что переменная \variable{source-dirs}
объявлена как простая для избежания повторного вызова \utility{find}
при каждом обращении к ней (предполагается, что файлы с исходным кодом
не будут спонтанно появляться и исчезать во время выполнения
\Makefile{}'а).  Ниже приведён пример функции, требующей рекурсивной
переменной:

{\footnotesize
\begin{verbatim}
# $(call source-dirs, dir-list)
source-dirs = $(sort                             \
                $(dir                            \
                  $(shell find $1 -name '*.c')))
\end{verbatim}
}

Эта версия принимает в качестве аргумента список каталогов для поиска
с пробелом в качестве разделителя. Первым аргументом программы
\utility{find} является список каталогов для поиска. Окончания списка
каталогов распознаётся благодаря символу тире, содержащемуся в
названии следующего аргумента\footnote{ Это одна из возможностей
\utility{find}, о которых я не знал в течении десятилетий! (прим.
автора)}.

%---------------------------------------------------------------------
% notdir
%---------------------------------------------------------------------
\item[\texttt{\${}(notdir \emph{список} ...)}] \hfill \\
\index{Функции!встроенные!notdir@\function{notdir}}
Функция \function{notdir} возвращает имена файлов для заданного списка
путей. Ниже приведён пример вычисления имени \Java{}-класса по
названию файла с его исходным кодом:

{\footnotesize
\begin{verbatim}
# $(call get-java-class-name, file-name)
get-java-class-name = $(notdir $(subst .java,,$1))
\end{verbatim}
}

Существует много примеров совместного применения функций
\function{dir} и \function{notdir} для получения желаемого результата.
Предположим, что некоторый командный сценарий должен выполняться в том
же каталоге, что и файл, создаваемый этим сценарием.

{\footnotesize
\begin{verbatim}
$(OUT)/myfile.out: $(SRC)/source1.in $(SRC)/source2.in
    cd $(dir $@); \
    generate-myfile $^ > $(notdir $@)
\end{verbatim}
}

\index{Переменные!автоматические!\${}"@@\variable{\$"@}}
Автоматическая переменная \variable{\${}@}, представляющая имя цели,
может подвергаться декомпозиции для получения имени каталога цели и
названия файла цели по-отдельности. Если \variable{OUT} содержит
абсолютный путь, вовсе не обязательно использовать функцию
\function{notdir}, однако её применение сделает вывод \GNUmake{} более
удобным для чтения.

Ещё одним способом декомпозиции имени файла является использование
\index{Переменные!автоматические!\${}("@D)@\variable{\$("@D)}}
\index{Переменные!автоматические!\${}("@F)@\variable{\$("@F)}}
переменных \variable{\${}(@D)} и \variable{\${}(@F)}, рассмотренных в
разделе <<\nameref{sec:automatic_vars}>> главы~\ref{chap:rules}.

%---------------------------------------------------------------------
% suffix
%---------------------------------------------------------------------
\item[\texttt{\${}(suffix \emph{список} ...)}] \hfill \\
\index{Функции!встроенные!suffix@\function{suffix}}
Функция \function{suffix} возвращает суффикс (расширение) каждого
файла в \ItalicMono{списке}. Ниже приведён пример функции,
проверяющей, все ли слова в списке имеют одинаковый суффикс:

{\footnotesize
\begin{verbatim}
# $(call same-suffix, file-list)
same-suffix = $(filter 1,$(words $(sort $(suffix $1))))
\end{verbatim}
}

Наиболее часто функция \function{suffix} применяется в условных
выражениях совместно с функцией \function{findstring}.

%---------------------------------------------------------------------
% basename
%---------------------------------------------------------------------
\item[\texttt{\${}(basename \emph{список} ...)}] \hfill \\
\index{Функции!встроенные!basename@\function{basename}}
Функция \function{basename} является дополнением к \function{suffix}.
Она возвращает имя файла без суффикса, оставляя нетронутыми любые
начальные компоненты имени файла. Вот пример альтернативной реализации
функций \function{file-to-class} и \function{get-java-class-name} с
использованием функции \function{basename}:

{\footnotesize
\begin{verbatim}
# $(call file-to-class-name, root-directory, file-name)
file-to-class-name := $(subst /,.,           \
                        $(basename           \
                          $(subst $1/,,$2)))

# $(call get-java-class-name, file-name)
get-java-class-name = $(notdir $(basename $1))
\end{verbatim}
}

%---------------------------------------------------------------------
% addsuffix
%---------------------------------------------------------------------
\item[\texttt{\${}(addsuffix \emph{суффикс},\emph{список} ...)}] \hfill \\
\index{Функции!встроенные!addsuffix@\function{addsuffix}}
Функция \function{addsuffix} добавляет заданный \ItalicMono{суффикс} к
имени каждого файла в \ItalicMono{списке}. В качестве
\ItalicMono{суффикса} может выступать произвольная строка. Ниже
приведён пример функции для поиска всех файлов из \variable{PATH},
соответствующих заданному шаблону:

{\footnotesize
\begin{verbatim}
# $(call find-program, filter-pattern)
find-program = \
    $(filter $1,                   \
      $(wildcard                   \
        $(addsuffix /*,            \
          $(sort                   \
            $(subst :, ,           \
              $(subst ::,:.:,      \
                $(patsubst :%,.:%, \
                  $(patsubst %:,%:.,$(PATH)))))))))

find:
    @echo $(words $(call find-program, %))
\end{verbatim}
}

Три внутренних подстановки осуществляются для корректной работы в
особых случаях. Пустой компонент \variable{PATH} может быть
использован для обозначения текущего каталога. Для нормализации этого
синтаксиса мы ищем пустые компоненты \variable{PATH} в конце строки, в
начале строки и внутри строки (именно в таком порядке). Все найденные
компоненты заменяются символом <<.>>. Затем разделитель компонент
\variable{PATH} заменяется пробелом. Функция \function{sort}
используется для удаления повторяющихся компонентов. После этого к
каждому слову в полученном списке добавляется суффикс \command{/*},
затем вызывается функция \function{wildcard}, осуществляющая поиск
всех файлов, соответствующих шаблону. Наконец, функция
\function{filter} отбирает только те файлы, которые соответствуют
заданному шаблону.

Хотя может показаться, что эта функция будет выполняться очень
медленно (и может быть действительно так на многих системах), на моём
1.9GHz P4 с 512MB оперативной памяти эта функция выполняется за 0,20
секунды и находит 4335 программ. Можно повысить производительность,
переместив аргумент \texttt{\${}1} внутрь вызова \function{wildcard}.
Следующая версия устраняет необходимость вызова \function{filter} и
вызывает \function{addsuffix} с аргументом, переданным вызывающей
функцией:

{\footnotesize
\begin{verbatim}
# $(call find-program,wildcard-pattern)
find-program =
    $(wildcard                   \
      $(sort                     \
        $(addsuffix /$1,         \
          $(subst :, ,           \
            $(subst ::,:.:,      \
              $(patsubst :%,.:%, \
                $(patsubst %:,%:.,$(PATH))))))))

find:
    @echo $(words $(call find-program,*))
\end{verbatim}
}

Этот вариант выполняется за 0,17 секунды. Он работает быстрее потому,
что функция \function{wildcard} больше не возвращает все файлы
каталогов только для того, чтобы затем отсеить их с помощью
\function{filter}.  Подобный пример можно встретить в руководстве
пользователя GNU \GNUmake{}.  Заметим, что первая версия использует
шаблоны в стиле \function{filter} (содержащие только метасимвол
\command{\%}), а вторая~--- шаблоны в стиле \function{wildcard}
(\command{\~}, \command{*}, \command{?}, \command{[...]} и
\command{[\^\,...]}).

%---------------------------------------------------------------------
% addprefix
%---------------------------------------------------------------------
\item[\texttt{\${}(addprefix \emph{префикс},\emph{список} ...)}] \hfill \\
\index{Функции!встроенные!addprefix@\function{addprefix}}
Функция \function{addprefix} является дополнением к функции
\function{addsuffix}. Ниже приведён пример функции, проверяющей набор
файлов на существование и ненулевую длину:

{\footnotesize
\begin{verbatim}
# $(call valid-files, file-list)
valid-files = test -s . $(addprefix -a -s ,$1)
\end{verbatim}
}

Функция \function{valid-files} отличается от большинства рассмотренных
тем, что её значение должно быть выполнено в командном интерпретаторе.
Для осуществления проверки используется программа \utility{test} с
ключом \command{-s} (истина, если файл существует и не пуст).
Поскольку для корректной работы программы \utility{test} имена файлов
должны быть разделены ключом \command{-a}, мы используем
\function{addprefix} для добавления этого ключа к каждому имени файла
в списке. В начале цепочки используется каталог \filename{.}, для него
условие существования и непустоты всегда является истинным.

%---------------------------------------------------------------------
% join
%---------------------------------------------------------------------
\item[\texttt{\${}(join \emph{список-префиксов},\emph{список-суффиксов})}] \hfill \\
\index{Функции!встроенные!join@\function{join}}
Функция \function{join} является дополнением к функциям \function{dir}
и \function{notdir}. Она принимает в качестве аргументов два списка и
производит конкатенацию первого элемента из
\ItalicMono{списка-префиксов} с первым элементом из
\ItalicMono{списка-суффиксов}, второго элемента из
\ItalicMono{списка-префиксов} со вторым элементом из
\ItalicMono{списка-суффиксов} и так далее. Эта функция может
использоваться для реконструкции списков, декомпозированных при помощи
функций \function{dir} и \function{notdir}.

%---------------------------------------------------------------------
% abspath
%---------------------------------------------------------------------
\item[\texttt{\${}(abspath \emph{список}...)}] \hfill \\
\index{Функции!встроенные!abspath@\function{abspath}}
Для каждого файла из \ItalicMono{списка} возвращает абсолютный путь,
не содержащий компонентов \filename{.} и \filename{..}, а также
повторяющихся разделителей (\filename{/}). Заметим, что в отличие от
функции \function{realpath}, \function{abspath} не производит
разрешения символических ссылок и не требует существования файлов.
Для проверки существования файлов используйте функцию
\function{wildcard}.

Эта функция доступна в версиях GNU \GNUmake{} 3.81 и выше.

%---------------------------------------------------------------------
% realpath
%---------------------------------------------------------------------
\item[\texttt{\${}(realpath \emph{список}...)}] \hfill \\
\index{Функции!встроенные!realpath@\function{realpath}}
Функция \function{realpath} подобна функции \function{abspath} c тем
отличием, что осуществляет разрешение символических ссылок и проверку
существования файлов и каталогов.

Эта функция доступна в версиях GNU \GNUmake{} 3.81 и выше.
\end{description}

%---------------------------------------------------------------------
% Flow control functions
%---------------------------------------------------------------------
\subsection{Функции управления выполнением}
\label{sec:flow_ctrl_func}

Поскольку многие из рассмотренных нами функций разработаны для
выполнения операций над списками, они прекрасно выполняют свою работу
без необходимости применения циклов. Однако без настоящих операторов
цикла и условного выполнения макроязык \GNUmake{} был бы очень
ограничен.  К счастью, \GNUmake{} предоставляет возможность
использовать оба этих оператора. Я также добавил в этот раздел функцию
\function{error}, являющуюся довольно экстремальной формой управления
выполнением.

\begin{description}
%---------------------------------------------------------------------
% if
%---------------------------------------------------------------------
\item[\texttt{\${}(if \emph{условие},\emph{then-часть},\emph{else-часть})}] \hfill \\
\index{Функции!встроенные!if@\function{if}}
Функция \function{if} (не путайте с директивами условной обработки
\directive{ifeq}, \directive{ifneq}, \directive{ifdef} и
\directive{ifndef}, рассмотренными в главе~\ref{chap:vars}) выбирает
одну из двух подстановок в зависимости от <<значения>> условного
выражения.  Значение \ItalicMono{условия} считается истинным, если оно
содержит хотя бы один символ (например, пробел). В этом случае
подставляется \ItalicMono{then-часть}. В противном случае, если
значение \ItalicMono{условия} является пустой строкой, подставляется
\ItalicMono{else-часть}\footnote{ В главе~\ref{chap:vars} мы обсуждали
разницу между макроязыками и обычными языками программирования.
Макроязыки трансформируют исходный код в текст с помощью определения и
подстановки макросов. Это различие станет яснее, когда мы увидим, как
работает функция \function{if}. (прим. автора)}.

Ниже приведёт простой способ узнать, исполняется ли ваш \Makefile{}
из-под Windows. Проверим, есть ли в окружении переменная
\variable{COMSPEC}, определённая только в Windows:

{\footnotesize
\begin{verbatim}
PATH_SEP := $(if $(COMSPEC),;,:)
\end{verbatim}
}

\GNUmake{} проверяет \ItalicMono{условие}, предварительно убрав
начальные и конечные пробелы и вычислив выражение. Если результатом
вычисления будет непустая строка, \ItalicMono{условие} считается
истинным. В итоге \variable{PATH\_SEP} содержит правильный разделитель
путей в переменной \variable{PATH}, независимо от того, выполняется ли
ваш \Makefile{} в Windows или \UNIX{}.

В одной из предыдущих глав мы упоминали возможность проверки версии
\GNUmake{} с помощью последних возможностей \GNUmake{} (например,
\function{eval}). Функции \function{if} и \function{filter} также
часто используются вместе для проверки значения строк:

{\footnotesize
\begin{verbatim}
$(if $(filter $(MAKE_VERSION),3.80),,\
  $(error Этот makefile требует GNU make версии 3.80.))
\end{verbatim}
}

Теперь по мере выхода новых версий GNU \GNUmake{}, выражение может
быть расширено для поддержки большего числа версий.

{\footnotesize
\begin{verbatim}
$(if $(filter $(MAKE_VERSION),3.80 3.81 3.90 3.92),,\
  $(error Этот makefile требует одну из следующих версий \
    GNU make ....))
\end{verbatim}
}

Эта техника имеет свои недостатки: придётся изменять код после каждой
установки свежей версии \GNUmake{}. С другой стороны, это происходит
не так уж часто (например, версия 3.80 была выпущена в октябре 2002
года, а версию 3.81 пришлось ждать вплоть до апреля 2006 года).
Предыдущий тест может быть помещён в самое начало \Makefile{}'а,
поскольку \function{if} либо вычислится как пустая строка, либо
вызовет функцию \function{error}, завершив выполнение \GNUmake{}.

%-------------------------------------------------------------------
% error
%-------------------------------------------------------------------
\item[\texttt{\${}(error \emph{сообщение})}] \hfill \\
\index{Функции!встроенные!error@\function{error}}
Функция \function{error} предназначена для вывода сообщений о
фатальных ошибках. После того, как эта функция напечатает
\ItalicMono{сообщение}, \GNUmake{} завершит свою работу с кодом
возврата 2. Вывод предваряется указанием имени текущего \Makefile{}'а
и номером текущей строки.  Ниже приведена реализация общей конструкции
программирования \index{Функции!assert@\function{assert}}
\newword{assert} для \GNUmake{}:

{\footnotesize
\begin{verbatim}
# $(call assert,condition,message)
define assert
  $(if $1,,$(error Assertion failed: $2))
endef

# $(call assert-file-exists,wildcard-pattern)
define assert-file-exists
  $(call assert,$(wildcard $1),$1 does not exist)
endef

# $(call assert-not-null,make-variable)
define assert-not-null
  $(call assert,$($1),The variable "$1" is null)
endef

error-exit:
    $(call assert-not-null,NON_EXISTENT)
\end{verbatim}
}

Первая функция, \function{assert}, просто проверяет свой первый
аргумент на пустоту и выводит пользовательское сообщение об ошибке,
если тестируемое значение является пустой строкой. Вторая функция
основывается на первой и проверяет с помощью функции
\function{wildcard} существование файла. Отметим, что аргумент может
содержать любое число шаблонов.

Третья функция очень полезна и основывается на \newword{вычисляемых
именах переменных}. Переменная \GNUmake{} может содержать что угодно,
даже имя другой переменной \GNUmake{}.  Однако если одна переменная
содержит имя второй переменной, как вы сможете получить доступ к
значению второй переменной? Можно просто вычислить переменную дважды:

{\footnotesize
\begin{verbatim}
NO_SPACE_MSG := No space left on device.
NO_FILE_MSG  := File not found.
...
STATUS_MSG   := NO_SPACE_MSG

$(error $($(STATUS_MSG)))
\end{verbatim}
}

Пример выглядит немного надуманным, однако он достаточно прост и
прекрасно иллюстрирует основную идею. Переменная
\variable{STATUS\_MSG} может принимать одно из нескольких сообщений об
ошибках путём сохранения имени переменной, содержащей это сообщение.
Когда приходит время вывести сообщение об ошибке, сначала вычисляется
\variable{STATUS\_MSG} для доступа к имени переменной, содержащей
сообщение об ошибке, \texttt{\${}(STATUS\_MSG)}, затем полученное имя
подвергается повторному вычислению для получения текста сообщения,
\texttt{\${}(\${}(STATUS\_MSG))}. В нашей функции
\function{assert-not-null} мы предполагали, что аргументом функции
является имя переменной. После первого вычисления аргумента,
\texttt{\${}1}, мы получаем имя переменной, которое сразу же
подвергаем повторному вычислению, \texttt{\${}(\${}1)}, для
определения значения этой переменной. Если значение не определено, мы
можем использовать имя переменной, \variable{\${}1}, для вывода
сообщения об ошибке:

{\footnotesize
\begin{alltt}
\${} \textbf{make}
Makefile:14: *** Assertion failed: The variable \textbackslash{}
"NON\_EXISTENT" is null. Stop.
\end{alltt}
}

Существует также функция \function{warning} (см. раздел
<<\nameref{sec:misc_func}>> далее в этой главе), которая выводит
сообщение в том же формате, что и функция \function{error}, но не
прерывает выполнения \GNUmake{}.

%---------------------------------------------------------------------
% foreach
%---------------------------------------------------------------------
\item[\texttt{\${}(foreach \emph{переменная},\emph{список},\emph{тело})}] \hfill \\
\index{Функции!встроенные!foreach@\function{foreach}}
Функция \function{foreach} предоставляет возможность вычислять
выражения в цикле. Заметим, что это отличается от повторного
вычисления функции от разных аргументов (хотя это также может быть
осуществлено с помощью этой функции). Например:

{\footnotesize
\begin{alltt}
letters := \${}(foreach letter,a b c d,\${}(letter))
show-words:
    \#{} letters has \${}(words \${}(letters)) words: '\${}(letters)'

\${} \textbf{make}
\#{} letters has 4 words: 'a b c d'
\end{alltt}
}

Когда выполняется этот цикл, переменной цикла \variable{letter}
последовательно присваиваются значения \textit{a b c d}, а тело цикла,
\variable{\${}(letter)}, вычисляется один раз для каждого значения.
Вычисленный текст аккумулируется в строке, каждое значение тела
отделяется от предыдущего пробелами.

Ниже приведена функция для проверки того, был ли определен набор
переменных:

{\footnotesize
\begin{verbatim}
VARIABLE_LIST := SOURCES OBJECTS HOME
$(foreach i,$(VARIABLE_LIST), \
  $(if $($i),,                \
    $(shell echo $i has no value > /dev/stderr)))
\end{verbatim}
}

Использование псевдофайла \filename{/dev/stderr} требует, чтобы
переменная окружения \variable{SHELL} имела значение \utility{bash}.
Этот цикл присваивает переменной \variable{i} поочерёдно каждое слово
из переменной \variable{VARIABLE\_LIST}. Условное выражение внутри
\function{if} сначала вычисляет \variable{\${}i} для получения имени
переменной, а затем находит значение \variable{\${}(\${}i)} для
проверки значения этой переменной на пустоту. Если вычисленная строка
не пуста, \ItalicMono{then-часть} ничего не делает; в противном случае
\ItalicMono{else-часть} выводит предупреждение. Заметим, что если мы
не перенаправим вывод \utility{echo}, вывод команды \function{shell}
будет подставлен в \Makefile{}, вызвав синтаксическую ошибку.
Результатом вычисления цикла \function{foreach} является пустая
строка.

Как и было обещано, ниже представлена функция для соединения в список
всех слов, содержащих заданную подстроку:

{\footnotesize
\begin{verbatim}
# $(call grep-string, search-string, word-list)
define grep-string
  $(strip                         \
    $(foreach w, $2,              \
      $(if $(findstring $1, $w),  \
        $w)))
endef

words := count_words.c counter.c lexer.l lexer.h counter.h

find-words:
    @echo $(call grep-string,un,$(words))
\end{verbatim}
}

К сожалению, эта функция не понимает шаблоны, однако нахождение
простых подстрок вполне ей под силу:

{\footnotesize
\begin{alltt}
\${} \textbf{make}
count\_words.c counter.c counter.h
\end{alltt}
}

%---------------------------------------------------------------------
% and
%---------------------------------------------------------------------
\item[\texttt{\${}(and \emph{условие}\subi{1}[,...\emph{условие}\subi{n}])}] \hfill \\
\index{Функции!встроенные!and@\function{and}}
Функция \function{and} представляет операцию \command{AND}. Она
производит вычисление всех своих аргументов по порядку. Если
результатом вычисления одного из аргументов является пустая строка,
процесс вычисления останавливается и функция возвращает пустую строку.
Если ни один из аргументов не порождает пустую строку, возвращается
результат вычисления последнего аргумента.

Эта функция доступна в GNU \GNUmake{}  версии 3.81 и выше.

%---------------------------------------------------------------------
% or
%---------------------------------------------------------------------
\item[\texttt{\${}(or \emph{условие}\subi{1}[,...\emph{условие}\subi{n}])}] \hfill \\
\index{Функции!встроенные!or@\function{or}}
Функция \function{or} представляет операцию \texttt{OR}. Она производит
вычисление своих аргументов по порядку. Если результатом вычисления
хотя бы одного из аргументов является непустая строка, это строка
возвращается в качестве значения функции и дальнейшее вычисление не
производится. Иначе возвращается пустая строка.

Эта функция доступна в GNU \GNUmake{}  версии 3.81 и выше.

\end{description}

%---------------------------------------------------------------------
% Style note concerning variables and parantheses
%---------------------------------------------------------------------
\subsubsection*{Замечание о скобках в переменных}

Как у же было замечено, если имя переменной \GNUmake{} состоит из
одного символа, то его не обязательно заключать в скобки. Например,
все основные автоматические переменные состоят из одного символа.
Написание автоматических переменных без скобок встречается
повсеместно, даже в руководстве по GNU \GNUmake{}. Однако практически
все остальные переменные, даже состоящие из одного символа, в
руководстве по GNU \GNUmake{} заключаются в скобки.  Это подчёркивает
особую природу переменных \GNUmake{}, поскольку практически во всех
языках программирования, в которых для работы с переменными
используется символ доллара (таких как командные интерпретаторы,
\utility{perl}, \utility{awk} и \utility{yacc}), применение скобок не
требуется. Одна из самых распространённых ошибок программирования на
\GNUmake{}~--- пропущенные скобки. Вот пример использования функции
\function{foreach}, содержащий ошибку:

{\footnotesize
\begin{verbatim}
INCLUDE_DIRS := ...
INCLUDES     := $(foreach i,$INCLUDE_DIRS,-I $i)
# INCLUDES теперь имеет значение "-I NCLUDE_DIRS"
\end{verbatim}
}

Однако я нахожу, что читать макросы гораздо легче, если использовать
переменные из одного символа и опускать ненужные скобки. Например, я
считаю, что функцию \function{has-duplicates} легче прочитать, если не
использовать лишние скобки:

{\footnotesize
\begin{verbatim}
# $(call has-duplicates, word-list)
has-duplicates = $(filter          \
                   $(words $1)     \
                     $(words $(sort $1))))
\end{verbatim}
}

Сравните с таким вариантом:

{\footnotesize
\begin{verbatim}
# $(call has-duplicates, word-list)
has-duplicates = $(filter          \
                   $(words $(1))   \
                     $(words $(sort $(1)))))
\end{verbatim}
}

Однако функция \function{kill-program} лишь выигрывает от применения
дополнительных скобок, поскольку они помогают отличить переменные
\GNUmake{} от переменных командного интерпретатора или переменных
других программ:

{\footnotesize
\begin{verbatim}
PS        := ps
PS_FIELDS := "9 47 100"
PS_FLAGS  := -W
define kill-program
  @$(PS) $(PS_FLAGS) |                              \
  $(AWK) 'BEGIN { FIELDWIDTHS = $(PS_FIELDS) }      \
          /$(1)/{                                   \
              print "Killing " $$3;                 \
              system( "$(KILL) $(KILLFLAGS) " $$1 ) \
          }'
endef
\end{verbatim}
}

Строка поиска содержит первый аргумент макроса, \variable{\${}(1)}.
Выражения \variable{\${}\${}3} и \variable{\${}\${}1} относятся к
переменным \utility{awk}.

Я опускаю скобки только в том случае, если это сделает код более
удобным для чтения. Обычно это касается параметров макросов и
переменных цикла \function{foreach}. Вы должны использовать тот стиль,
который наиболее подходит в вашей ситуации. Если у вас есть малейшие
сомнения относительно удобства сопровождения вашего \Makefile{}'а,
следуйте руководству пользователя GNU \GNUmake{} и заключайте
переменные в круглые скобки. Помните, программа \GNUmake{} создана для
решения проблем, связанных с поддержкой программного обеспечения. Если
вы будете иметь это в виду во время написания своих \Makefile{}'ов, вы
сможете избежать многих неприятностей.

%---------------------------------------------------------------------
% Important miscellaneous functions
%---------------------------------------------------------------------
\subsection{Различные вспомогательные функции}
\label{sec:misc_func}

Наконец, мы рассмотрим несколько полезных (но не очень важных) функций
для манипулирования строками. Скорее всего, вы будете использовать их
довольно часто, но не так часто, как \function{foreach} или
\function{call}.

\begin{description}
%---------------------------------------------------------------------
% strip
%---------------------------------------------------------------------
\item[\texttt{\${}(strip \emph{текст})}] \hfill \\
\index{Функции!встроенные!strip@\function{strip}}
Функция \function{strip} удаляет из \ItalicMono{текста} все начальные
и конечные пробелы, а также заменяет все повторяющиеся внутренние
пробелы одним. Наиболее часто эта функция используется для обработки
переменных, использующихся в условных выражениях.

Я часто использую эту функцию для удаления нежелательных пробелов из
определений переменных и макросов. Я использовал многострочное
форматирование, однако использование функции \function{strip} для
обёртки аргументов \variable{\${}1}, \variable{\${}2} и т.д. является
неплохой идеей, особенно если вызываемая функция чувствительна к
начальным пробелам. Часто программисты, не знающие тонкостей
\GNUmake{}, добавляют пробел после запятой в списке аргументов функции
\function{call}.

%---------------------------------------------------------------------
% origin
%---------------------------------------------------------------------
\item[\texttt{\${}(origin \emph{имя-переменной})}] \hfill \\
\index{Функции!встроенные!origin@\function{origin}}
Функция \function{origin} возвращает строку, описывающую происхождение
переменной. Это может быть очень полезно, чтобы принять решение о том,
как использовать значение переменной. Например, вы можете захотеть
проигнорировать значение переменной, если она получена из окружения, и
использовать её, если она была определена в командной строке.  В
качестве конкретного примера приведём функцию
\function{assert-defined}, определяющую, определена ли переменная:

{\footnotesize
\begin{verbatim}
# $(call assert-defined,variable-name)
define assert-defined
  $(call assert,                          \
    $(filter-out undefined,$(origin $1)), \
    Переменная '$1' не определена)
endef
\end{verbatim}
}

Функция \function{origin} может возвращать одно из следующих значений:
\begin{itemize}
%---------------------------------------------------------------------
\item \texttt{undefined} \hfill \\
Переменная никогда не была определена.

%---------------------------------------------------------------------
\item \texttt{default} \hfill \\
Переменная определена в стандартной базе данных \GNUmake{}. Если вы
измените значение такой переменной, \function{origin} вернёт
происхождение самого последнего определения.

%---------------------------------------------------------------------
\item \texttt{environment} \hfill \\
Переменная была определена в окружнии (и опция
\command{-{}-environment\hyp{}overrides} не была включена).

%---------------------------------------------------------------------
\item \texttt{environment override} \hfill \\
Переменная была определена в окружнии (и опция
\command{-{}-environment\hyp{}overrides} была включена).

%---------------------------------------------------------------------
\item \texttt{file} \hfill \\
Переменная была определена в \Makefile{}'е.

%---------------------------------------------------------------------
\item \texttt{file} \hfill \\
Переменная была определена в командной строке.

%---------------------------------------------------------------------
\item \texttt{override} \hfill \\
Для определения переменной была использована директива
\directive{override}.
%---------------------------------------------------------------------

\item \texttt{automatic} \hfill \\ Переменная является автоматической
переменной \GNUmake{}.
%---------------------------------------------------------------------
\end{itemize}

%---------------------------------------------------------------------
% warning
%---------------------------------------------------------------------
\item[\texttt{\${}(warning \emph{сообщение})}] \hfill \\
\index{Функции!встроенные!warning@\function{warning}}
Функция \function{warning} похожа на функцию \function{error}, но не
вызывает завершение выполнения \GNUmake{}.  Вывод
\ItalicMono{сообщения} предваряется именем текущего \Makefile{}'а и
номером текущей строки. Функция \function{warning} возвращает пустую
строку, и потому может использоваться практически везде:

{\footnotesize
\begin{verbatim}
$(if $(wildcard $(JAVAC)),,                           \
  $(warning Переменная, определяющая компилятор java, \
            JAVAC ($(JAVAC)), не определена должным   \
            образом.))
\end{verbatim}
}

%---------------------------------------------------------------------
% info
%---------------------------------------------------------------------
\item[\texttt{\${}(info \emph{сообщение})}] \hfill \\
\index{Функции!встроенные!info@\function{info}}
Функция \function{info} выводит \ItalicMono{сообщение} на стандартный
поток вывода, не предваряя его именем \Makefile{}'а или номером
строки. Результатом вычисления этой функции является пустая строка.

Эта функция доступна в GNU \GNUmake{} версии 3.81 и выше.

%---------------------------------------------------------------------
% value
%---------------------------------------------------------------------
\item[\texttt{\${}(value \emph{имя-переменной})}] \hfill \\
\index{Функции!встроенные!value@\function{value}}
Функция \function{value} предоставляет способ использовать значение
переменной, \emph{не вычисляя} его. Заметим, что откат уже
произведённых вычислений не может быть осуществлён. Например, если вы
определяете простую переменную (не рекурсивную), её значение
вычисляется в момент определения. В этом случае функция
\function{value} вернёт тот же результат, что и обычное вычисление
переменной.

%---------------------------------------------------------------------
% flavor
%---------------------------------------------------------------------
\item[\texttt{\${}(flavor \emph{имя-переменной})}] \hfill \\
\index{Функции!встроенные!flavor@\function{flavor}}
Функция \function{flavor}, в отличие от многих других функций, не
оперирует значением переменной, а возвращает некоторую информацию о
переменной. Значением этой функции является строка, определяющая тип
переменной, имя которой передано в качестве аргумента:

\begin{itemize}
\item \texttt{undefined} \hfill \\
Переменная не определена.
\item \texttt{recursive} \hfill \\
Переменная является рекурсивно-вычисляемой.
\item \texttt{simple} \hfill \\
Переменная является простой.
\end{itemize}

Эта функция доступна в GNU \GNUmake{} версии 3.81 и выше.
\end{description}

%%--------------------------------------------------------------------
%% Advanced user-defined functions
%%--------------------------------------------------------------------
\section{Специальные функции}
\label{sec:advanced_user_def_func}

Мы потратим много времени на написание макросов-функций. К сожалению,
\GNUmake{} имеет довольно скудные возможности для их отладки. Давайте
попробуем написать простую функцию для вывода отладочной информации.

Как уже было замечено, функция \function{call} связывает свои
параметры с позиционными параметрами \variable{\$1}, \variable{\$2} и
т.д. На вход этой функции может быть передано произвольное число
переменных. Особым случаем является имя вызываемой функции (т.е. имя
переменной), доступ к которому может быть осуществлён с помощью
переменной \variable{\$0}. Используя эту информацию, мы можем
написан пару отладочных функций для вывода результатов подстановки
макросов:

{\footnotesize
\begin{verbatim}
# $(debug-enter)
debug-enter = \
    $(if $(debug_trace),\
      $(warning Entering $0($(echo-args))))

# $(debug-leave)
debug-leave = \
    $(if $(debug_trace),$(warning Leaving $0))

comma := ,

echo-args   = \
    $(subst ' ','$(comma) ',\
      $(foreach a,1 2 3 4 5 6 7 8 9,'$($a)'))
\end{verbatim}
}

Если мы хотим узнать, как вызываются функции \function{a} и
\function{b}, мы можем использовать описанные выше функции следующим
образом:

{\footnotesize
\begin{verbatim}
debug_trace = 1

define a
  $(debug-enter)
  @echo $1 $2 $3
  $(debug-leave)
endef

define b
  $(debug-enter)
  $(call a,$1,$2,hi)
  $(debug-leave)
endef

trace-macro:
    $(call b,5,$(MAKE))
\end{verbatim}
}

Разместив переменные \function{debug-enter} и \function{debug-leave}
в начале и конце ваших собственных функций, вы сможете получать
отладочную информацию о их вызове. Однако эти функции далеки от
совершенства. Функция \function{echo-args} выводит только первые
девять аргументов, хуже того, она не может определить фактическое
число переданных аргументов (конечно, ведь \GNUmake{} тоже не может этого
сделать!). Тем не менее, я использовал эти макросы <<как есть>>
при отладке собственных \Makefile{}'ов. После выполнения предыдущего
примера мы сможем увидеть следующий вывод:

{\footnotesize
\begin{alltt}
\${} \textbf{make}
makefile:14: Entering b( '5', 'make', '', ...)
makefile:14: Entering a( '5', 'make', 'hi', '', ...)
makefile:14: Leaving a
makefile:14: Leaving b
5 make hi
\end{alltt}
}
Как сказал недавно один мой друг, <<я никогда раньше не думал о make
как о языке программирования>>. GNU \GNUmake{}~--- это не тот
\GNUmake{}, которым пользовались наши прадеды.

%-------------------------------------------------------------------
% eval and value
%-------------------------------------------------------------------
\subsection{Функции \function{eval} и \function{value}}

Функция \function{eval} принципиально отличается от остальных
встроенных функций. Её назначением является передача текста
синтаксическому арализатору \GNUmake{}. Вот пример её использования:

{\footnotesize
\begin{verbatim}
$(eval sources := foo.c bar.c)
\end{verbatim}
}

Аргумент функции \function{eval} сначала проверяется на наличие
переменных и при необходимости осуществляется подстановка (тоже самое
осуществляется для аргументов всех функций), затем полученный текст
разбирается и выполняется так, будто он был написан в файле
спецификации.  Приведённый пример настолько прост, что вы могли
удивиться, зачем вам вообще нужна эта функция. Однако давайте
рассмотрим более сложный пример. Предположим, у вас есть один
\Makefile{} для сборки десятка различных программ, и вы хотите
определить несколько переменных для каждой из этих программ, например,
\variable{sources}, \variable{headers} и \variable{objects}. Вместо
повторения определений этих переменных снова и снова:

{\footnotesize
\begin{verbatim}
ls_sources := ls.c glob.c
ls_headers := ls.h glob.h
ls_objects := ls.o glob.o
...
\end{verbatim}
}

{\flushleft мы можем попробовать определить макрос для выполнения
рутинной работы:}

{\footnotesize
\begin{verbatim}
# $(call program-variables, variable-prefix, file-list)
define program-variables
  $1_sources = $(filter %.c,$2)
  $1_headers = $(filter %.h,$2)
  $1_objects = $(subst .c,.o,$(filter %.c,$2))
endef

$(call program-variables, ls, ls.c ls.h glob.c glob.h)
show-variables:
    # $(ls_sources)
    # $(ls_headers)
    # $(ls_objects)
\end{verbatim}
}

Макрос \function{program-variables} принимает два аргумента: префикс
для имени трёх переменных и список файлов, из которого макрос выбирает
файлы для определения каждой переменной. Но если мы попробуем
использовать этот макрос, мы получим ошибку:

{\footnotesize
\begin{alltt}
\$ \textbf{make}
Makefile:7: *** missing separator. Stop.
\end{alltt}
}

Этот способ не работает так, как ожидалось, из-за алгоритма работы
синтаксического анализатора \GNUmake{}. Макрос, вычисляемый на верхнем
уровне, результат подстановки которого занимает больше одной строки,
считается некорректным и приводит к синтаксической ошибке. В нашем
случае анализатор ожидает, что строка является правилом или частью
сценария сборки, но не обнаруживает подходящего разделителя. Довольно
непонятное сообщение об ошибке.  Функция \function{eval} была создана
для решения этой проблемы. Если мы заменим нашу строку, содержащую
\function{call}, на следующую:

{\footnotesize
\begin{verbatim}
$(eval $(call program-variables, ls, ls.c ls.h glob.c glob.h))
\end{verbatim}
}

{\flushleft то получим именно то, что ожидали:}

{\footnotesize
\begin{alltt}
\${} \textbf{make}
# ls.c glob.c
# ls.h glob.h
# ls.o glob.o
\end{alltt}
}

Использование \function{eval} решает проблему с синтаксическим
анализатором, так как функция \function{eval} обрабатывает
многострочные значения макросов и возвращает пустую строку.

Теперь у нас есть макрос, который определяет три переменных и не
требует набора большого количества текста вручную. Заметим, что имена
определяемых переменных в макросе составляются с помощью префикса,
переданного в качестве аргумента, и фиксированного суффикса,
\variable{\$1\_sources}. Это напоминает технику переменных с
вычисляемыми именами, описанную ранее.

Продолжая этот пример, можно заметить, что правила также могут быть
заключены в макрос:

{\footnotesize
\begin{verbatim}
# $(call program-variables,variable-prefix,file-list)
define program-variables
  $1_sources = $(filter %.c,$2)
  $1_headers = $(filter %.h,$2)
  $1_objects = $(subst .c,.o,$(filter %.c,$2))

  $($1_objects): $($1_headers)
endef

ls: $(ls_objects)

$(eval $(call program-variables,ls,ls.c ls.h glob.c glob.h))
\end{verbatim}
}

Обратите внимание, как две версии функции \function{program-variables}
иллюстрируют проблему пробелов в аргументах функции. В предыдущей
версии использование аргументов функции было устойчивым к начальным
пробелам, то есть поведение кода не изменялось при добавлении
начальных пробелов к переменным \variable{\$1} и \variable{\$2}.
Однако в новой версии были введены вычисляемые имена переменных
\variable{\$(\$1\_objects)} и \variable{\$(\$1\_headers)}.
Теперь добавление пробелов к первому аргументу функции (\command{ls})
приведёт к тому, что имена переменных будут вычислены неправильно,
вследствие чего вместо списка файлов будет подставлена пустая строка.
Обычно обнаружить подобную ошибку довольно сложно.

Когда мы запустим на выполнение наш \Makefile{}, обнаружится, что
\GNUmake{} игнорирует некоторые \filename{.h} реквизиты.  Чтобы понять
причину этой проблемы, исследуем внутреннюю базу данных \GNUmake{},
использовав ключ \command{-{}-print\hyp{}data\hyp{}base}:

{\footnotesize
\begin{alltt}
\${} \textbf{make --print-database | grep \^{}ls}
ls\_headers = ls.h glob.h
ls\_sources = ls.c glob.c
ls\_objects = ls.o glob.o
ls.c:
ls.o: ls.c
ls: ls.o
\end{alltt}
}

Все \filename{.h} реквизиты для \filename{ls.o} пропущены! В правило,
использующее вычисляемые имена переменных, вкралась какая-то ошибка.

Когда \GNUmake{} производит синтаксический анализ вызова функции
\function{call}, сначала производится вычисление функции, определённой
пользователем, \function{program-variables}. После подстановки первая
строка макроса выглядит так:

{\footnotesize
\begin{verbatim}
ls_sources = ls.c glob.c
\end{verbatim}
}

Заметим, что каждая строка макроса вычисляется сразу же, как и
ожидалось.  Остальные определения переменных обрабатываются также.
Затем происходит обработка правила:

{\footnotesize
\begin{verbatim}
$($1_objects): $($1_headers)
\end{verbatim}
}

Сначала вычисляются имена переменных:

{\footnotesize
\begin{verbatim}
$(ls_objects): $(ls_headers)
\end{verbatim}
}

Затем происходит подстановка значений переменных, что порождает
следующее выражение:

{\footnotesize
\begin{verbatim}
:
\end{verbatim}
}

Подождите! Куда делись значения наших переменных? Дело в том, что
предыдущие три определения были правильно вычислены, но не были
интерпретированы \GNUmake{}. Давайте посмотрим, как это работает.  Как
только был произведён вызов \function{program-variables}, \GNUmake{}
<<видит>> следующее:

{\footnotesize
\begin{verbatim}
$(eval ls_sources = ls.c glob.c
ls_headers = ls.h glob.h
ls_objects = ls.o glob.o

:)
\end{verbatim}
}

Затем выполняется функция \function{eval}, после чего определяются три
переменные. Итак, причина найдена: переменные в правиле используются
до того, как были определены.

Мы можем решить эту проблему, указав явно на необходимость отложить
вычисление переменных, используемых в правиле, до момента их
определения.  Это можно сделать, экранировав знак доллара в вызове
переменных с вычисляемым именем:

{\footnotesize
\begin{verbatim}
$$($1_objects): $$($1_headers)
\end{verbatim}
}

На этот раз база данных \GNUmake{} содержит те реквизиты, которые мы
ожидали:

{\footnotesize
\begin{alltt}
\${} \textbf{make -p | grep \^{}ls}
ls\_headers = ls.h glob.h
ls\_sources = ls.c glob.c
ls\_objects = ls.o glob.o
ls.c:
ls.o: ls.c ls.h glob.h
ls: ls.o
\end{alltt}
}

Итак, аргумент функции \function{eval} вычисляется дважды: сначала при
подготовке списка аргументов для \function{eval}, затем ещё раз
синтаксическим анализатором.

Мы решили предыдущую проблему, отложив вычисление переменных. Вторым
способом решения этой проблемы является обёртка каждого определения
переменной дополнительным вызовом функции \function{eval}:

{\footnotesize
\begin{verbatim}
# $(call program-variables,variable-prefix,file-list)
define program-variables
  $(eval $1_sources = $(filter %.c,$2))
  $(eval $1_headers = $(filter %.h,$2))
  $(eval $1_objects = $(subst .c,.o,$(filter %.c,$2)))

  $($1_objects): $($1_headers)
endef

ls: $(ls_objects)
    $(eval $(call program-variables,ls,ls.c ls.h\
	glob.c glob.h))
\end{verbatim}
}

Заключая каждое определене переменной в собственный вызов
\function{eval}, мы форсируем их вычисление в процессе подстановки
макроса \function{program-variables}, поэтому их можно использовать
непосредственно в теле макроса.

Лишь только мы улучшили наш \Makefile{}, нашлось ещё одно правило,
которое можно добавить в макрос. Исполняемый файл программы зависит от
объектных файлов. Поэтому, в завершение нашего параметризованного
\Makefile{}'а, мы добавим цель \target{all} и переменную, содержащую
список программ, сборкой которых может управлять наш \Makefile{}:

{\footnotesize
\begin{verbatim}
#$(call program-variables,variable-prefix,file-list)
define program-variables
  $(eval $1_sources = $(filter %.c,$2))
  $(eval $1_headers = $(filter %.h,$2))
  $(eval $1_objects = $(subst .c,.o,$(filter %.c,$2)))
  
  programs += $1

  $1: $($1_objects)

  ($1_objects): $($1_headers)
endef

# Объявляем цель по умолчанию
all:

$(eval $(call program-variables,ls,ls.c ls.h glob.c glob.h))
$(eval $(call program-variables,cp,...))
$(eval $(call program-variables,mv,...))
$(eval $(call program-variables,ln,...))
$(eval $(call program-variables,rm,...))

# Определяем реквизиты цели по умолчанию
all: $(programs)
\end{verbatim}
}

Обратите внимание на расположение цели \target{all} и её реквизитов.
Переменная \variable{programs} не имеет правильного значения до
осуществления пяти вызовов \function{eval}, но мы хотели бы объявить
\target{all} целью по умолчанию, поместив её объявление первым в
\Makefile{}'е. Мы можем удовлетворить всем ограничениям, объявив цель
\target{all} первой и добавив реквизиты позже.

Функция \function{program-variables} имеет недостатки, поскольку
некоторые переменные вычисляются слишком рано. Для решения этой
проблемы \GNUmake{} предоставляет функцию \function{value}, которая
получает имя переменной в качестве аргумента и возвращает её значение,
не осуществляя подстановок. Результат выполнения \function{value}
может быть передан для обработки функции \function{eval}. Возвращая
значение переменной, в котором не осуществлены подстановки, мы можем
избежать проблем, из-за которых нам пришлось экранировать знаки
доллара в нашем макросе.

К несчастью, функцию \function{value} нельзя использовать в макросе
\function{program-variables}, поскольку она следует принципу <<всё или
ничего>>, не осуществляя подстановки ни одной переменной в аргументе.
Более того, \function{value} не принимает параметров (а при
обнаружении не производит с ними никаких действий), поэтому имя нашей
программы и список файлов останутся нетронутыми.

Из-за этих ограничений вы не очень часто будете видеть функцию
\function{value} на страницах этой книги.

%---------------------------------------------------------------------
% Hooking functions
%---------------------------------------------------------------------
\subsection{Триггеры}

Функции, определяемые пользователем~--- это просто переменные,
содержащие текст. Функция \function{call} заменяет вхождения
позиционных параметров \variable{\${}1}, \variable{\${}2} и т.д.
значениями фактических параметров, то есть разрешает ссылки в значении
переменной, если они существуют. Если переменная не содержит
формальных параметров, \function{call} не обратит на это внимания.
Более того, если переменная вообще не содержит текста, \function{call}
не произведёт никаких действий.  Не будет вывода предупреждений или
сообщений об ошибках. Это может разочаровать, если вы случайно
допустили опечатку в имени вызываемой функции, однако иногда это очень
полезно.

По своей сути функции~--- воплощение идеи повторного использования
кода. Чем чаще вы используете какую-то функцию, тем более тщательно
стоит подойти к её реализации. Функции можно сделать более удобными
для повторного использования, если добавить к ним триггеры.
\index{Триггеры}
\newword{Триггер} (\newword{hook})~--- это ссылка на функцию, которая
может быть определена пользователем для осуществления его собственных
задач в процессе выполнения общих операций. 

Предположим, ваш \Makefile{} служит для сборки множества библиотек.
На одних платформах вы предпочли бы запуск \utility{ranlib}, на других
больше подходит \utility{chmod}. Вместо явного указания команд для
этих операций вы можете написать функцию и добавить к ней триггер:

{\footnotesize
\begin{verbatim}
# $(call build-library, object-files)
define build-library
  $(AR) $(ARFLAGS) $@ $1
  $(call build-library-hook,$@)
endef
\end{verbatim}
}

Для использования триггера объявите функцию
\function{build-library-hook}:

{\footnotesize
\begin{verbatim}
$(foo_lib): build-library-hook = $(RANLIB) $1
$(foo_lib): $(foo_objects)
    $(call build-library,$^)
$(bar_lib): build-library-hook = $(CHMOD) 444 $1
$(bar_lib): $(bar_objects)
    $(call build-library,$^)
\end{verbatim}
}

%---------------------------------------------------------------------
% Passing paramenters
%---------------------------------------------------------------------
\subsection{Передача аргументов}

Функция может получать информацию из четырёх <<источников>>:
параметры, переданные с вызовом \function{call}, глобальные
переменные, автоматические переменные и переменные, зависящие от цели.
Наиболее отвечающим принципу модульности способом передачи информации
являются формальные параметры, поскольку они предохраняют функцию от
изменений глобальных данных, однако иногда модульность~--- не самый
важный критерий.

Предположим, у вас есть несколько проектов, использующих общий набор
функций \GNUmake{}. Каждый проект может идентифицироваться префиксом
переменных, допустим, \variable{PROJECT1\_}, и имена всех критические
важных переменных составляются из этого префикс и общих для всех
проектов суффиксов. В предыдущих примерах вместо переменных наподобие
\variable{PROJECT\_SRC} могли бы использоваться переменные
\variable{PROJECT1\_SRC}, \variable{PROJECT1\_BIN} и
\variable{PROJECT1\_LIB}. Вместо написания функции, которая требует
наличия всех этих переменных мы могли бы использовать переменные с
вычисляемым именем и передавать дополнительный аргумент~--- префикс:

{\footnotesize
\begin{verbatim}
# $(call process-xml,project-prefix,file-name)
define process-xml
  $($1_LIB)/xmlto -o $($1_BIN)/xml/$2 $($1_SRC)/xml/$2
endef
\end{verbatim}
}

\index{Переменные!зависящие от цели}
Другим подходом передачи аргументов является использование переменных,
зависящих от цели. Это особенно удобно, когда б\'{о}льшая часть
вызовов использует стандартные значения, и лишь в некоторых случаях
требуется уделить этому особое внимание. Переменные, зависящие от
цели, предоставляют также гибкость в том случае, когда правило
определено в подключаемом файле, а выполнение правила осуществляется в
\Makefile{}'е, в котором определены переменные.

{\footnotesize
\begin{verbatim}
release: MAKING_RELEASE = 1
release: libraries executables
...
$(foo_lib):
$(call build-library,$^)
...
# $(call build-library, file-list)
define build-library
  $(AR) $(ARFLAGS) $@           \
    $(if $(MAKING_RELEASE),     \
      $(filter-out debug/%,$1), \
	  $1)
endef
\end{verbatim}
}

Предыдущий код использует переменные, зависящие от цели, чтобы
сообщить вызываемым функциям, что должна быть выполнена чистовая
сборка библиотеки.  В этом случае функция сборки не будет включать в
библиотеку отладочные модули.


%%--------------------------------------------------------------------
%% Commands
%%--------------------------------------------------------------------
\chapter{Команды}
\label{chap:commands}

Мы уже рассмотрели б\'{о}льшую часть базовых концепций команд
\GNUmake{}, однако давайте всё же проведём их краткий обзор.

Команды являются однострочными сценариями командного интерпретатора.
\GNUmake{} считывает каждую команду и передаёт её в дочерний процесс
командного интерпретатора для выполнения. На самом деле \GNUmake{}
может осуществлять оптимизацию этого (относительно) ресурсоёмкого
fork/exec алгоритма, если это гарантированно не изменит поведения
программы. Для этого каждая команда сканируется на наличие специальных
символов командного интерпретатора, таких как шаблоны или перенаправления
ввода-вывода. Если эти символы не найдены, \GNUmake{} выполняет команду
самостоятельно без порождения дочернего процесса.

В качестве командного интерпретатора по умолчанию используется
\filename{/bin/sh}. За используемый интерпретатор отвечает переменная
\variable{SHELL}, не наследуемая из окружения. При старте \GNUmake{}
импортирует все переменные, кроме \variable{SHELL}, из окружения
пользователя, преобразуя их в переменные \GNUmake{}. Это сделано для
того, чтобы пользовательский выбор интерпретатора не вызвал ошибки
выполнения \Makefile{}'а (возможно, входящего в загруженный из сети
пакет программного обеспечения). Если пользователь действительно хочет
поменять интерпретатор по умолчанию, он должен явно присвоить
переменной \variable{SHELL} нужное значение прямо в \Makefile{}'е. Мы
вернёмся к этой теме в разделе <<\nameref{sec:which_shell_to_use}>> этой
главы.

%%--------------------------------------------------------------------
%% Parsing commands
%%--------------------------------------------------------------------
\section{Синтаксический анализ команд}
\label{sec:parsing_commands}

Строки, начинающиеся с символа табуляции и следующие за спецификацией
цели, считаются командами (если только предыдущая строка не
завершалась символом обратного слэша). Встретив символ табуляции в
другом контексте, GNU \GNUmake{} старается догадаться, что вы имели
ввиду. Например, присваивания переменных, комментарии и директивы
включения могут начинаться с символа табуляции, если это не приводит к
неоднозначности. Если \GNUmake{} обнаруживает команду, которая не
следует за спецификацией цели, выводится сообщение об ошибке:

{\footnotesize
\begin{verbatim}
makefile:20: *** commands commence before first target. Stop.
\end{verbatim}
}

Формулировка этого сообщения немного удивляет, ведь оно часто
появляется в середине \Makefile{}'а, много позже спецификации
<<первой>> цели, однако теперь мы в состоянии понять это сообщение без
особых проблем. Пожалуй, лучшей формулировкой была бы следующая:
<<обнаружена команда вне контекста определения правила>>.

Когда синтаксический анализатор обнаруживает команду в подходящем
контексте, он включает <<режим разбора команды>>, строя сценарий
сборки по строке за раз. Добавление текста к сценарию прекращается,
когда обнаруживается строка, не могущая быть частью сценария.  В
сценарии могут появляться следующие строки:

\begin{itemize}
%---------------------------------------------------------------------
\item Строки, начинающиеся с символа табуляции, являются командами,
предназначенными для выполнения в командном интерпретаторе. Даже
строки, которые могли бы быть интерпретированы как конструкции
\GNUmake{} (например, директивы \directive{ifdef}, \directive{include}
или комментарии), в режиме разбора команд трактуются как команды.
%---------------------------------------------------------------------
\item Пустые строки игнорируются, их выполнение не осуществляется.
%---------------------------------------------------------------------
\item Строки, начинающиеся с символа \command{\#}, возможно, с
начальными пробелами (не символами табуляции!), воспринимаются как
комментарии \GNUmake{} и игнорируются.
%---------------------------------------------------------------------
\item Директивы условной обработки, такие как \directive{ifdef} и
\directive{ifeq}, распознаются и правильно обрабатываются в контексте
сценария сборки.
%---------------------------------------------------------------------
\end{itemize}

Встроенные функции \GNUmake{}, не предваряемые символом табуляции, не
приводят к выходу из режима разбора команд. Это значит, что их
значением должны быть допустимые команды интерпретатора или пустая
строка.  Значением функций \function{warning} или \function{eval}
является пустая строка.

Тот факт, что пустые строки или комментарии \GNUmake{} допустимы в
сценариях сборки, поначалу может удивлять. Следующий пример
показывает, как это работает:

{\footnotesize
\begin{verbatim}
long-command:
    @echo Строка 2:  далее пустая строка

    @echo Строка 4:  далее комментарий shell
    # Комментарий командного интерпретатора \
      (начинается с символа табуляции)
    @echo Строка 6:  далее комментарий make
# комментарий make в начале строки
    @echo Строка 8:  далее комментарий make
  # выровненный пробелами комментарий make
    # ещё один выровненный пробелами комментарий make
    @echo Строка 11: далее директива make
  ifdef COMSPEC
    @echo мы работаем под Windows
  endif
    @echo Строка 15: далее <<команда>> warning
    $(warning предупреждение)
    @echo Строка 17: далее <<команда>> eval
    $(eval $(shell echo Вывод shell 1>&2))
\end{verbatim}
}

Заметим, что строки 5 и 10 выглядят очень похоже, но по сути они очень
разные. Строка 5 содержит комментарий командного интерпретатора,
начинающийся с символа табуляции, в то время как строка 10 содержит
комментарий \GNUmake{}, начинающийся с отступа в 4 пробела.  Очевидно,
что форматирование своих \Makefile{}'ов подобным образом~--- не самая
лучшая идея (если только вы не собираетесь участвовать в конкурсе
самых запутанных \Makefile{}'ов). Как вы можете видеть из следующего
листинга, комментарии \GNUmake{} не выполняются и не выводятся, даже
если они появляются в контексте сценария сборки:

{\footnotesize
\begin{verbatim}
$ make
makefile:2: предупреждение
Вывод shell
Строка 2:  далее пустая строка
Строка 4:  далее коментарий shell
# комментарий командного интерпретатора (начинается...)
Строка 6:  далее комментарий make
Строка 8:  далее комментарий make
Строка 11: далее директива make
мы работаем под Windows
Строка 15: далее <<команда>> warning
Строка 17: далее <<команда>> eval
\end{verbatim}
}

Сначала может показаться, что вывод функций \function{warning} и
\function{eval} появился слишком рано, однако это не так (мы обсудим
порядок вычисления в этой главе в разделе
<<\nameref{sec:evaluating_commands})>>. Тот факт, что сценарий сборки
может содержать произвольное число пустых строк и комментариев, может
быть источником трудно находимых ошибок.  Предположим, вы случайно
добавили строку с начальным символом табуляции.  Если выше неё
располагается определение цели, за которым следуют только комментарии
и пустые строки, \GNUmake{} будет трактовать вашу строку со случайным
символом табуляции как команду, ассоциированную с предыдущей целью.
Как вы уже видели, это вполне допустимо и не вызовет ошибки или
предупреждения, пока та же цель не встретится в составе другого
правила в другом месте \Makefile{}'а (или во включаемом файле).

Если вам повезёт, ваш \Makefile{} будет содержать непустую строку, не
содержащую комментария, между вашей ошибочной строкой и предыдущим
сценарием сборки. В этом случае вы получите сообщение <<commands
commence before first target>>.

Сейчас самое время упомянуть об инструментальных средствах. Я думаю,
теперь все согласны с тем, что использование символа табуляции для
обозначения команды было не самым удачным решением, однако теперь уже
поздно что-либо менять. Использование современного, понимающего
синтаксис текстового редактора может помочь вам предотвратить
потенциальные проблемы с помощью цветового выделения сомнительных
конструкций. GNU \utility{emacs} имеет довольно удобный режим для
редактирования \Makefile{}'ов. Этот режим осуществляет подсветку
синтаксиса и находит простые синтаксические ошибки, такие как пробелы
после символа переноса строки или смешанные символы табуляции и
пробелы. Мы вернёмся к теме совместного использования редактора
\utility{emacs} и \GNUmake{} немного позже.

%---------------------------------------------------------------------
% Continuing long commands
%---------------------------------------------------------------------
\subsection{Продолжение длинных команд}

Поскольку каждая команда выполняется в отдельном процессе командного
интерпретатора, последовательности выражений, которые должны
выполняться совместно, должны обрабатываться особым образом.
Предположим для примера, что нам нужно создать файл, содержащий список
файлов. Компилятор \Java{} принимает такие файлы в случае, если нужно
скомпилировать много исходного кода. Для этой цели мы можем написать
следующее правило:

{\footnotesize
\begin{verbatim}
.INTERMEDIATE: file_list

file_list:
    for d in logic ui
    do
      echo $d/*.java
    done > $@
\end{verbatim}
}

Очевидно, что этот пример не будет работать и вызовет ошибку при
запуске:

{\footnotesize
\begin{alltt}
  \${} \textbf{make}
  for d in logic ui
  /bin/sh: -c: line 2: syntax error: unexpected end of file
  make: *** [file\_{}list] Error 2
\end{alltt}
}

В качестве решения можно попробовать добавить символы продолжения в
конце каждой строки:

{\footnotesize
\begin{verbatim}
.INTERMEDIATE: file_list

file_list:
    for d in logic ui \
    do                \
      echo $d/*.java  \
    done > $@
\end{verbatim}
}

Однако теперь при запуске появляется другая ошибка:

{\footnotesize
\begin{alltt}
\${} \textbf{make}
for d in logic ui \textbackslash{}
do                \textbackslash{}
  echo /*.java    \textbackslash{}
done > file\_list
/bin/sh: -c: line 1: syntax error near unexpected token `>'
/bin/sh: -c: line 1: `for d in logic ui do       echo /*.java
make: *** [file\_list] Error 2
\end{alltt}
}

Что же произошло? В коде есть две ошибки. Во-первых, ссылка на
переменную цикла, \variable{d}, должна быть экранирована. Во-вторых,
поскольку цикл передаётся в интерпретатор одной строкой, мы должны
добавить точку с запятой после списка файлов и выражения в теле цикла:

{\footnotesize
\begin{verbatim}
.INTERMEDIATE: file_list

file_list:
    for d in logic ui; \
    do                 \
      echo $d/*.java;  \
    done > $@
\end{verbatim}
}

Теперь мы получим именно то, что ожидали. Поскольку цель
\target{file\_list} помечена как \variable{.INTERMEDIATE}, \GNUmake{}
удалит её после завершения компиляции.

В более реалистичном примере список файлов будет храниться в
переменной \GNUmake{}. Если есть уверенность в том, что этот список
достаточно мал, мы можем осуществить ту же операцию без использования
цикла, используя только встроенные функции \GNUmake{}:

{\footnotesize
\begin{verbatim}
.INTERMEDIATE: file_list

file_list:
    echo $(addsuffix /*.java,$(COMPILATION_DIRS)) > $@
\end{verbatim}
}

Однако у версии с циклом меньше шансов столкнуться с проблемой
конечности длины командной строки, если список каталогов будет расти
со временем.

Ещё одной общей проблемой является смена текущего каталога:

{\footnotesize
\begin{verbatim}
TAGS:
    cd src
    ctags --recurse
\end{verbatim}
}

Очевидно, что предыдущий пример не выполнит программу \utility{ctags}
в каталоге \filename{src}. Чтобы добиться желаемого эффекта, мы должны
либо разместить оба выражения в одной строке, либо экранировать символ
новой строки (разделив выражения точкой с запятой):

{\footnotesize
\begin{verbatim}
TAGS:
    cd src; \
    ctags --recurse
\end{verbatim}
}

Ещё более разумно проверять статус выполнения программы \utility{cd}
перед выполнением \utility{ctags}:

{\footnotesize
\begin{verbatim}
TAGS:
    cd src && \
    ctags --recurse
\end{verbatim}
}

Заметим, что при некоторых обстоятельствах можно опустить точку с
запятой, не вызвав ошибки командного интерпретатора или \GNUmake{}:

{\footnotesize
\begin{verbatim}
disk-free = echo "Проверяем размер дискового пространства..." \
    df . | awk '{ print $$4 }'
\end{verbatim}
}

Этот пример выводит простое сообщение, за которым следует число
свободных блоков на текущем устройстве. Или нет? Мы случайно забыли
поставить точку с запятой после команды \utility{echo}, в результате
чего программа \utility{df} никогда не будет запущена. Вместо этого мы
направим сообщение <<\command{Проверяем размер дискового
пространства...  df .}>> на вход программы \utility{awk}, которая
напечатает четвёртое поле строки, т.е. \command{пространства...}.

Возможно, вам приходилось использовать директиву \directive{define},
предназначенную для создания многострочных последовательностей. К
сожалению, она не решает проблемы переносов строк. Когда происходит
подстановка макроса, каждая его строка вставляется в сценарий сборки с
начальным символом табуляции, и \GNUmake{} работает с каждой строкой
независимо. Разные строки макроса выполняются в разных экземплярах
командного интерпретатора, поэтому вам следуюет обращать внимание на
перенос строк даже в определениях макросов.

%---------------------------------------------------------------------
% Command modifiers
%---------------------------------------------------------------------
\subsection{Модификаторы команд}
\label{sec:command_modifiers}

\index{Модификаторы команд}
Команды могут быть модифицированы при помощи нескольких префиксов.  Мы
уже встречались с <<молчаливым>> префиксом, \command{@}, ниже приведён
полный список возможных префиксов с некоторыми комментариями:

\begin{description}
%---------------------------------------------------------------------
\item[\command{@}] \hfill \\
Подавляет вывод команды. Для исторической совместимости вы можете
поместить вашу команду в реквизиты специальной цели \target{.SILENT},
если хотите, чтобы все команды, ассоциированные с вашей целью, были
скрыты. Однако использование \texttt{@} предпочтительней, поскольку
этот модификатор может быть применён к отдельным командам в сценарии.
Если вы хотите применить модификатор ко всем целям, используйте
\index{Опции!silent@\command{-{}-silent (-s)}}
опцию \command{-{}-si\-lent} (или просто \command{-s}).

Сокрытие команд может сделать вывод \GNUmake{} приятнее для глаза,
однако это также ведёт к затруднениям при отладке. Если вы обнаружите,
что часто удаляете модификатор \command{@}, а затем возвращаете его на
место, разумно завести переменную (к примеру, \variable{QUIET}),
содержащую модификатор, и использовать её в командах:

{\footnotesize
\begin{verbatim}
QUIET = @
hairy_script:
    $(QUIET) сложный сценарий ...
\end{verbatim}
}

В этом случае при необходимости увидеть сложный сценарий в том виде, в
котором его выполняет \GNUmake{}, просто сбросьте значение переменной
\variable{QUIET} в командной строке:

{\footnotesize
\begin{alltt}
\${} \textbf{make QUIET= hairy\_script}
сложный сценарий ...
\end{alltt}
}

%---------------------------------------------------------------------
\item[\command{-}] \hfill \\
Этот префикс сообщает \GNUmake{}, что ошибки, возникающие при
выполнении помеченной команды, следует игнорировать. По умолчанию
\GNUmake{} проверяет код возврата каждой программы или конвейера
программ. Если какая-то программа завершается с ненулевым кодом,
\GNUmake{} прерывает выполнение оставшейся части сценария и завершает
своё выполнение. Этот модификатор заставляет \GNUmake{} игнорировать
код возврата модифицированной строки и продолжать выполнение в обычном
режиме. Мы обсудим эту тему подробнее в следующем разделе.

Для исторической совместимости вы можете игнорировать ошибки в части
сценария сборки, поместив соответствующую цель в реквизиты специальной
цели \target{.IGNORE}. Вы можете игнорировать все ошибки в
\index{Опции!ignore-errors@\command{-{}-ignore\hyp{}errors (-i)}}
\Makefile{}'е используя опцию \command{-{}-ig\-no\-re\hyp{}errors}
(или \command{-i}).  Однако полезность этой возможности вызывает
сомнения.

%---------------------------------------------------------------------
\item[\texttt{+}] \hfill \\
Этот модификатор сообщает \GNUmake{}, что помеченная им команда должна
выполняться даже в том случае, если в опциях командной строки
встречается \command{-{}-just\hyp{}print} (или \command{-n}). Этот
модификатор используется при составлении рекурсивных \Makefile{}'ов.
Мы обсудим эту тему в разделе <<\nameref{sec:recursive_make}>>
главы~\ref{chap:managing_large_proj}.
%---------------------------------------------------------------------
\end{description}

Все модификаторы должны встречаться в строке только один раз.
Очевидно, перед выполнением команд модификаторы вырезаются.

%-------------------------------------------------------------------
% Errors and interrupts
%-------------------------------------------------------------------
\subsection{Ошибки и прерывания}

Каждая команда, выполняемая \GNUmake{}, возвращает код ошибки. Нулевой
код соответствует успешному выполнению команды, ненулевой~--- ошибочной
ситуации. Некоторые программы используют код возврата для передачи
более полезной информации, чем просто факта возникновения ошибки.
Например, программа \utility{grep} возвращает 0 в случае обнаружения
соответствия шаблону, 1 в случае отсутствия соответствий и 2 в случае
возникновения ошибки какого-либо рода.

Обычно при ошибке выполнения команды (т.е. при возвращении ненулевого
кода ошибки) \GNUmake{} прекращает выполнение команд и завершает
выполнение с ненулевым кодом возврата. Иногда нужно, чтобы \GNUmake{}
продолжал работу, собрав столько целей, сколько возможно. Например,
вам может понадобиться скомпилировать максимально возможное число
исходных файлов, чтобы увидеть все ошибки компиляции в один проход.
Вы можете добиться такого поведения при помощи опции
\index{Опции!keep-going@\command{-{}-keep\hyp{}going (-k)}}
\command{-{}-keep\hyp{}going} (или \command{-k}).

Поскольку модификатор \command{-} заставляет \GNUmake{} игнорировать
ошибки в отдельных командах, я стараюсь избегать его использования,
поскольку это усложняет автоматическую обработку ошибок и привносит в
код небрежность.

Когда \GNUmake{} игнорирует ошибку, выводится сообщение об ошибке с
именем цели в квадратных скобках. Ниже приведён вывод, полученный при
попытке удаления несуществующего файла:

{\footnotesize
\begin{verbatim}
rm non-existent-file
rm: cannot remove `non-existent-file': No such file or directory
make: [clean] Error 1 (ignored)
\end{verbatim}
}

Некоторые команды (например, \utility{rm}) имеют опции для подавления
ошибочных кодов возврата. Опция \command{-f} заставит программу
\utility{rm} вернуть нулевой код ошибки и подавить вывод
предупреждений. Использование этой опции лучше, чем зависимость от
модификатора.

Бывают случаи, когда возврат программой нулевого кода ошибки считается
неудачей и наоборот. В таких случаях можно просто инвертировать код
возврата программы:

{\footnotesize
\begin{verbatim}
# Убедимся, что в коде не осталось отладочных сообщений.
.PHONY: no_debug_printf

no_debug_printf: $(sources)
    ! grep --line-number '"debug:' $^
\end{verbatim}
}

К сожалению, версия GNU \GNUmake{} 3.80 содержит ошибку, мешающую
непосредственному использованию этой возможности: \GNUmake{} не
распознаёт символ \command{!} как часть команды, требующей вызова
командного интерпретатора, и выполняет оставшуюся часть строки
самостоятельно, вызывая ошибку. В качестве простого обхода проблемы
можно использовать в команде специальные символы как намёк для
\GNUmake{}:

{\footnotesize
\begin{verbatim}
# Убедимся, что в коде не осталось отладочных сообщений.
.PHONY: no_debug_printf

no_debug_printf: $(sources)
    ! grep --line-number '"debug:' $^ < /dev/null
\end{verbatim}
}

Другим источником неожиданных ошибок в командах является оператор
\command{if} без ветки \command{else}:

{\footnotesize
\begin{verbatim}
$(config): $(config_template)
    if [ ! -d $(dir $@) ];  \
    then                    \
        $(MKDIR) $(dir $@); \
    fi
    $(M4) $^ > $@
\end{verbatim}
}

Первая команда проверяет, существует ли целевой каталог и в случае
необходимости вызывает программу \utility{mkdir} для его создания. К
сожалению, если каталог существует, команда \command{if} вернёт
ненулевой код ошибки (код возврата программы \utility{test}), что
приведёт к завершению работы сценария. Одним из решений этой проблемы
является добавление ветви \command{else}:

{\footnotesize
\begin{verbatim}
$(config): $(config_template)
    if [ ! -d $(dir $@) ];  \
    then                    \
        $(MKDIR) $(dir $@); \
    else                    \
        true;               \
    fi
    $(M4) $^ > $@
\end{verbatim}
}

Двоеточие (\command{:})~--- это команда интерпретатора, которая всегда
возвращает истину, поэтому её можно использовать вместо \command{true}.
Альтернативной рабочей реализацией является следующая:

{\footnotesize
\begin{verbatim}
$(config): $(config_template)
    [[ -d $(dir $@) ]] || $(MKDIR) $(dir $@)
    $(M4) $^ > $@
\end{verbatim}
}

 Теперь первое выражение истинно, если целевой каталог существует или
 выполнение программы \utility{mkdir} завершилось успешно. Другой
 альтернативой является использование ключа \command{-t} программы
 \utility{mkdir}. Это приведёт к успешному завершению \utility{mkdir}
 даже в том случае, если требуемый каталог уже сужествует.

Все предыдущие реализации вызывали интерпретатор даже в том случае,
если каталог уже существовал. Использование функции
\function{wildcard} позволяет избежать выполнения команд в случае
наличия каталога:

{\footnotesize
\begin{verbatim}
# $(call make-dir, directory)
make-dir = $(if $(wildcard $1),,$(MKDIR) -p $1)

$(config): $(config_template)
    $(call make-dir, $(dir $@))
    $(M4) $^ > $@
\end{verbatim}
}

Поскольку каждая команда выполняется в отдельном экземпляре командного
интерпретатора, общей практикой является использование многострочных
команд, разделённых точками с запятой. Остерегайтесь случаев, когда
ошибка в таких сценариях может не привести к завершению выполнения
сборки:

{\footnotesize
\begin{verbatim}
target:
    rm rm-неудачен; echo но следующая команда выполняется 
\end{verbatim}
}

Лучшей практикой является минимизация длины сценария, что даёт
\GNUmake{} шанс обработать код возврата самостоятельно. Например:

{\footnotesize
\begin{verbatim}
path-fixup = \
    -e "s;[a-zA-Z:/]*/src/;$(SOURCE_DIR)/;g" \
    -e "s;[a-zA-Z:/]*/bin/;$(OUTPUT_DIR)/;g"

# хорошая версия
define fix-project-paths
  sed $(path-fixup) $1 > $2.fixed && \
  mv $2.fixed $2
endef

# отличная версия
define fix-project-paths
  sed $(path-fixup) $1 > $2.fixed
  mv $2.fixed $2
endef
\end{verbatim}
}

Этот макрос преобразует пути в стиле DOS (с прямыми слэшами) в пути
назначения для определённой структуры каталогов исходного кода и
бинарных файлов. Макрос принимает имена двух файлов: входного и
выходного. Кроме того, принимаются дополнительные действия, чтобы
выходной файл был перезаписан только в том случае, если \utility{sed}
завершится корректно. В <<хорошей>> версии это достигается за счёт
соединения программ \utility{sed} и \utility{mv} операцией
\command{\&\&}, в результате чего они выполняются в одном экземпляре
командного интерпретатора. <<Лучшая>> версия выполняет их как две
отдельных команды, позволяя \GNUmake{} завершить выполнение сценария,
если программа \utility{sed} завершится неудачей. Однако <<лучшая>>
версия не является менее производительной (программа \utility{mv} не
требует вызова командного интерпретатора и выполняется непосредственно
\GNUmake{}), к тому же её легче понять, а в случае возникновения
ошибки полученное сообщение будет более информативным (поскольку
\GNUmake{} укажет, какая именно команда закончилась неудачей).

Заметим, что предыдущая ситуация не имеет отношения к общей проблеме
команды \utility{cd}:

{\footnotesize
\begin{verbatim}
TAGS:
    cd src && \
    ctags --recurse
\end{verbatim}
}

В этом случае оба выражения должны выполняться в одном процессе
командного интерпретатора, поэтому должен использоваться разделитель
команд, например, \command{;} или \command{\&\&}.

%---------------------------------------------------------------------
% Deleting and preserving target files
%---------------------------------------------------------------------
\subsubsection*{Удаление и сохранение целевых файлов}
Если происходит ошибка, \GNUmake{} подразумевает, что повторная сборка
цели не может быть осуществлена. В этом случае любая другая цель,
имеющая текущую в качестве реквизита, также не может быть собрана,
поэтому \GNUmake{} не будет даже пытаться осуществить её сборку. Если
использована опция \command{-{}-keep\hyp{}going} (\command{-k}), будет
произведена попытка сборки следующей цели, иначе \GNUmake{} закончит
своё выполнение. Если текущей целью является файл, его содержимое
может быть повреждено, если команда из сценария завершится, не
закончив своей работы. К сожалению, \GNUmake{} оставит этот
потенциально повреждённый файл на диске из соображений исторической
совместимости. Поскольку время последнего изменения файла будет
изменено, все последующие вызовы \GNUmake{} не смогут поместить в этот
файл корректные данные. Вы можете избежать этой проблемы и заставить
\GNUmake{} удалять эти подозрительные файлы в случае возникновения
ошибки, указав целевой файл реквизитом специальной цели
\index{Цели!специальные!.DELETE\_ON\_ERROR@\target{.DELETE\_ON\_ERROR}}
\target{.DELETE\_ON\_ERROR}. Если цель \target{.DELETE\_ON\_ERROR}
используется без реквизитов, ошибка при сборке любого файла приведёт к
его удалению.

Дополнительные проблемы связаны с ситуацией, когда выполнение
\GNUmake{} прерывается сигналом, например, по нажатию клавиш Ctrl-C. В
этом случае \GNUmake{} удалит текущий целевой файл, если он был
модифицирован. Иногда удаление целевого файла не является желаемой
реакцией. Возможно, создание целевого файла~--- чрезвычайно затратная
операция, или получение части его содержимого желательней полного
его отсутствия. В таком случае вы можете защитить целевой файл, сделав
\index{Цели!специальные!.PRECIOUS@\target{.PRECIOUS}}
его реквизитом специальной цели \target{.PRECIOUS}.

%%-------------------------------------------------------------------
%% Which shell to use
%%-------------------------------------------------------------------
\section{Выбор командного интерпретатора}
\label{sec:which_shell_to_use}

Когда \GNUmake{} требуется передать команду интерпретатору, он
использует \utility{/bin/sh}. Вы можете изменить интерпретатор,
выставив соответствующим образом значение переменной \variable{SHELL}.
Однако хорошенько подумайте перед тем, как это сделать. Обычно
назначением \GNUmake{} является предоставление для команды
разработчиков инструмента сборки системы из исходного кода. Довольно
легко создать \Makefile{}, не соответствующий этому назначению,
используя инструменты, не доступные для других участников процесса
разработки или строя предположения, для них не справедливые.
Использование любых интерпретаторов, отличных от \utility{/bin/sh},
считается дурным тоном для любого широко распространённого приложения
(доступного через ftp или открытый репозиторий cvs). Мы обсудим
вопросы переносимости более детально в
главе~\ref{chap:portable_makefiles}.

Однако и есть другой контекст использования \GNUmake{}. В закрытых
средах разработки часто все участники проекта работают на ограниченном
множестве машин и операционных систем. На самом деле это именно та
среда, в которой мне чаще всего приходилось работать. В этой ситуации
вы имеете полное право настроить среду, в которой будет работать
\GNUmake{}, по своему усмотрению. Следует лишь инструктировать всех
разработчиков в вопросах настройки среды и работы со сборками.

В подобных средах я предпочитаю открыто жертвовать переносимостью в
некоторых аспектах. Я уверен, что это может сделать процесс разработки
гораздо более гладким. Одной их таких жертв является замена
стандартного значения переменной \variable{SHELL} на
\utility{/usr/bin/bash}. \utility{bash}~--- это переносимый,
\POSIX{}-совместимый командный интерпретатор (отсюда следует, что он
включает в себя все возможности \utility{sh}), являющийся
интерпретатором по умолчанию для GNU/Linux. Причиной многих проблем с
переносимостью \Makefile{}'ов является использование непереносимых
конструкций в сценариях сборки. Решением этих проблем является явное
использование одного стандартного интерпретатора вместо употребления
лишь переносимого подмножества команд \utility{sh}. У Пола Смита,
разработчика, занимающегося поддержкой GNU \GNUmake{}, ести
веб-страница <<Правила Пола для \Makefile{}'ов>> (<<Paul's Rules of
Makefiles>>,
\filename{\url{http://make.paulandlesley.org/rules.html}}), на которой
он делает следующее замечание: <<Не тратьте силы, пытаясь написать
переносимые \Makefile{}'ы, используйте переносимую версию
\GNUmake{}!>> (<<Don't hassle with writing portable makefiles, use
portable make instead!>>). Я могу добавить следующее: <<Когда есть
возможность, не тратьте силы, пытаясь написать переносимый сценарий,
используйте переносимый командный интерпретатор (bash)>>.
\utility{bash} работает на большинстве операционных систем, включая
практически все варианты \UNIX{}, Windows, BeOS, Amiga и OS/2.

В оставшейся части книги я буду явно указывать на случаи использования
специфичных возможностей \utility{bash}.

%%--------------------------------------------------------------------
%% Empty commands
%%--------------------------------------------------------------------
\section{Пустые команды}
\label{sec:empty_commands}

\index{Команды!пустые}
\newword{Пустая команда}~-- это команда, которая не производит никаких
действий:

{\footnotesize
\begin{verbatim}
header.h: ;
\end{verbatim}
}

Вспомним, что за списком реквизитов цели может следовать точка с
запятой и команда. Здесь используется только точка с запятой, что
означает, что команды не предполагаются. Вместо этого вы можете
поместить после определения цели строку, содержащую только один символ
табуляции, однако это будет невозможно прочитать. Пустые команды чаще
всего используются для предотвращения соответствия цели шаблонному
правилу и выполнения нежелательных команд.

Заметим, что в других версиях \GNUmake{} пустые цели иногда
используются в качестве абстрактных. В GNU \GNUmake{} следует
использовать специальную цель \target{.PHONY}, это безопасней и яснее.

%%-------------------------------------------------------------------
%% Command enviromnent
%%-------------------------------------------------------------------
\section{Команды и окружение}
\label{sec:command_environment}

Команды, выполняемые \GNUmake{}, наследуют окружение процесса
\GNUmake{}. Это окружение включает текущий каталог, файловые
дескрипторы и переменные окружения, передаваемые \GNUmake{}.

Когда создаётся дочерний процесс командного интерпретатора, \GNUmake{}
добавляется в окружение несколько переменных:

{\footnotesize
\begin{verbatim}
MAKEFLAGS
MFLAGS
MAKELEVEL
\end{verbatim}
}

Переменная \variable{MAKEFLAGS} включает опции командной строки,
переданные \GNUmake{}. Переменная \variable{MFLAGS} дублирует
содержимое \variable{MAKEFLAGS} и существует по историческим причинам.
Переменная \variable{MAKELEVEL} содержит число вложенных вызовов
\GNUmake{}. Таким образом, когда \GNUmake{} рекурсивно вызывает
\GNUmake{}, переменная \variable{MAKELEVEL} увеличивается на единицу.
Подпроцесс родительского процесса \GNUmake{} будет иметь переменную
\variable{MAKELEVEL}, значением которой будет единица. Все эти
переменные обычно используются для управления рекурсивным \GNUmake{}.
Мы обсудим эту тему в разделе <<\nameref{sec:recursive_make}>>
главы~\ref{chap:managing_large_proj}.

Конечно, пользователь может передать в окружение дочернего процесса
любую переменную по своему усмотрению, используя директиву
\index{Директивы!export@\directive{export}}
\directive{export}.

Текущий рабочий каталог исполняемой команды совпадает с рабочим
каталогом родителького процесса \GNUmake{}. Обычно это тот же каталог,
из которого была вызвана программа \GNUmake{}, однако его можно
заменить при помощи опции \command{--directory=\emph{каталог}} (или
\command{-C}). Заметим, что спецификация \Makefile{}'а при помощи
опции \command{-{}-fi\-le} не изменяет рабочий каталог, только
устанавливает \Makefile{}, который нужно прочитать.

Каждый подпроцесс, порождаемый \GNUmake{}, наследует три стандартных
файловых дескриптора: \filename{stdin}, \filename{stdout} и
\filename{stderr}. Здесь нет ничего особенного, за исключением одного
следствия: сценарий сборки может считывать данные из стандартного
потока ввода. Как только сценарий считает все данные из потока,
оставшиеся команды выполняются в обычном порядке. Однако ожидается,
что \Makefile{}'ы должны работать корректно без этого типа
взаимодействия. Пользователь часто расчитывает на возможность просто
запустить \GNUmake{} и далее не принимать никакого участия в процессе
сборки, проверив лишь результаты по завершению. И, конечно, сложно
придумать полезное применение чтению стандартного потока ввода в
контексте автоматизированных сборок, основанных на использовании
\utility{cron}.

Общей ошибкой является случайное чтение стандартного потока ввода:

{\footnotesize
\begin{verbatim}
$(DATA_FILE): $(RAW_DATA)
    grep pattern $(RAW_DATA_FILES) > $@
\end{verbatim}
}

Здесь входные файлы для \utility{grep} хранятся в переменной (при
использовании которой произошла опечатка). Если вместо значения
переменной подставится пустая строка, \utility{grep} останется только
читать данные со стандартного потока ввода, без каких либо объяснений
причины <<зависания>> \GNUmake{}. Простым способом избежать такой
проблемы является включение в команду дополнительного файла устройства
\filename{/dev/null}:

{\footnotesize
\begin{verbatim}
$(DATA_FILE): $(RAW_DATA)
    grep pattern $(RAW_DATA_FILES) /dev/null > $@
\end{verbatim}
}

Такая команда никогда не примет попытки чтения стандартного потока
ввода. Естественно, отладка \Makefile{}'ов также помогает избежать
неприятностей.

%%--------------------------------------------------------------------
%% Evaluating commands
%%--------------------------------------------------------------------
\section{Выполнение команд}
\label{sec:evaluating_commands}

Обработка командного сценария происходит в четыре этапа: чтение кода,
подстановка переменных, вычисление выражений \GNUmake{} и выполнение
команд. Давайте посмотрим, как все эти этапы применяются к сложному
сценарию. Рассмотрим следующий (немного надуманный) \Makefile{}.
Приложение компонуется, затем от полученного исполняемого файла
отделяется таблица символов, после чего он сжимается при помощи
компрессора исполняемых файлов \utility{upx}:

{\footnotesize
\begin{verbatim}
# $(call strip-program, file)
define strip-program
  strip $1
endef

complex_script:
    $(CC) $^ -o $@
  ifdef STRIP
    $(call strip-program, $@)
  endif
    $(if $(PACK), upx --best $@)
    $(warning Final size: $(shell ls -s $@))
\end{verbatim}
}

Вычисление командных сценариев откладывается до того момента, когда
оно действительно потребуется, однако директивы \directive{ifdef}
обрабатывается сразу после их обнаружения. Поэтому \GNUmake{}
считывает команды сценария, игнорируя их содержимое и сохраняя каждую
строку, пока не обнаружит строку \command{ifdef STRIP}. \GNUmake{}
выполняет тест, и если переменная \variable{STRIP} не определена,
\GNUmake{} считывает и отбрасывает весь текст сценария, пока не
натолкнётся на закрывающую директиву \directive{endif}. После этого
\GNUmake{} считывает и сохраняет оставшуюся часть сценария.

Когда приходит время выполнения сценария, \GNUmake{} сначала сканирует
команды на наличие конструкций \GNUmake{}, требующих подстановки. После
подстановки макросов каждая строка сценария начинается с символа
табуляции. Вычисление макросов \emph{перед} выполнением команд может
привести к неожиданным результатам. Последняя строка в нашем сценарии
некорректна. Функции \function{shell} и \function{warning} выполняются
\emph{до} компоновки приложения. Поэтому команда \utility{ls} будет
выполнена до того, как целевой файл будет собран. Это объясняет
<<неправильный>> порядок выполнения, который мы наблюдали в разделе
<<\nameref{sec:parsing_commands}>>.

Также заметим, что строка \command{ifdef STRIP} выполняется во время
чтения \Makefile{}'а, однако строка \command{\$(if...)} вычисляется
непосредственно перед выполнения сценария сборки цели
\target{complex\_script}. Использование функции \function{if}
допускает написание более гибких сценариев, поскольку предоставляет
больше возможностей для контроля определения переменных, однако такой
подход не очень хорошо приспособлен для управления большими блоками
текста.

Как показывает наш пример, всегда очень важно обращать внимание на то,
какая программа вычисляет выражение (т.е. \GNUmake{} или командный
интерпретатор), и когда именно это вычисление происходит:

{\footnotesize
\begin{verbatim}
$(LINK.c) $(shell find                              \
            $(if $(ALL),$(wildcard core ext*),core) \
              -name '*.o')
\end{verbatim}
}

Это запутанный командный сценарий компоновки множества объектных
файлов. Порядок вычисления операций таков (в скобках указана
программа, выполняющая соответствующую операцию):

\begin{enumerate}
\item Вычисление \variable{\$ALL} (\GNUmake{}).
\item Вычисление \function{if}  (\GNUmake{}).
\item Вычисление \function{wildcard} в предположении, что
  \variable{ALL} содержит непустое значение (\GNUmake{}).
\item Вычисление \function{shell} (\GNUmake{}).
\item Вычисление \utility{find} (\utility{sh}).
\item После завершения подстановок и вычисления конструкций
  \GNUmake{}, происходит выполнение команды компоновки
  (\utility{sh}).
\end{enumerate}

%%--------------------------------------------------------------------
%% Command-line limits
%%--------------------------------------------------------------------
\section{Ограничения командной строки}
\label{sec:command_line_limits}

Во время работы с крупными проектами вы можете столкнуться с
ограничениями на длину команд, которые \GNUmake{} пытается выполнить.
Ограничения на длину командной строки варьируются в зависимости от
операционной системы. Red Hat 9 GNU/Linux позволяет выполнять команды
длиной не более 128 Кб, а Windows XP ограничивает длину 32 Кб.
Сообщения об ошибке также варьируются. Если вы вызовите команду
\utility{ls} со слишком большим списком параметров, в Cygwin под
Windows, то получите следующее сообщение:

{\footnotesize
\begin{verbatim}
C:\usr\cygwin\bin\bash: /usr/bin/ls: Invalid argument
\end{verbatim}
}

На Red Hat 9 сообщение выглядит иначе:

{\footnotesize
\begin{verbatim}
/bin/ls: argument list too long
\end{verbatim}
}

Даже 32 Кб выглядит как довольно большой объём данных для командной
строки, однако когда ваш проект содержит 3000 файлов и 100
подкаталогов, и вы хотите манипулировать ими всеми, это ограничение
может быть довольно существенным.

Существует два основных пути, которые ведут к неприятностям с
ограничениями на длину строки: вычисление базовых значений при помощи
инструментов командного интерпретатора или использование \GNUmake{}
для присваивания переменной значения очень большой длины. Предположим
для примера, что мы хотим скомпилировать все исходные файлы одной
командой:

{\footnotesize
\begin{verbatim}
ompile_all:
$(JAVAC) $(wildcard $(addsuffix /*.java,$(source_dirs)))
\end{verbatim}
}

Переменная \GNUmake{} \variable{source\_dirs} может содержать всего
несколько сотен слов, однако после добавления шаблона исходных файлов
\Java{} и применения функции \function{wildcard} этот список может
превысить предельную длину командной строки вашей системы. Кстати,
\GNUmake{} не имеет собственных ограничений на длину строки, позволяя
вам хранить столько данных, сколько может вместить виртуальная память.

Когда сталкиваешься с подобной ситуацией, возникает ощущение, что
играешь в старую игру <<Приключение>> (Adventure): <<Вы находитесь в
лабиринте из одинаковых извилистых коридоров>>. Например, вы можете
попробовать использовать \utility{xargs} для решения проблемы, так как
\utility{xargs} разделяет строки на части в соответствии с
ограничениями текущей системы:

{\footnotesize
\begin{verbatim}
compile_all:
    echo $(wildcard
           $(addsuffix /*.java,$(source_dirs))) | \
    xargs $(JAVAC)
\end{verbatim}
}

К сожалению, так мы просто переместили проблему ограничений из команды
\utility{javac} в команду \utility{echo}. Мы также не можем
использовать \utility{echo} или \utility{printf} для записи данных в
файл (предполагается, что компилятор может читать список файлов из
файла).

Нет, для решения этой проблемы нужно в первую очередь избежать
создания одного большого списка файлов. Вместо этого мы можем
просматривать по одному каталогу за раз, используя шаблоны командного
интерпретатора:

{\footnotesize
\begin{verbatim}
compile_all:
    for d in $(source_dirs); \
    do                       \
        $(JAVAC) $$d/*.java; \
    done
\end{verbatim}
}

Также можно использовать канал в \utility{xargs}, чтобы достигнуть
желаемого результата за меньшее количество вызовов компилятора:

{\footnotesize
\begin{verbatim}
compile_all:
    for d in $(source_dirs); \
    do                       \
        echo $$d/*.java;     \
    done |                   \
    xargs $(JAVAC)
\end{verbatim}
}

К сожалению, ни один из этих сценариев не обрабатывает должным образом
ошибки компиляции. Лучшим подходом является сохранение полного списка
файлов и последующая передача его компилятору, если, конечно,
компилятор поддерживает чтение аргументов из файла. Компилятор \Java{}
поддерживает эту возможность:

{\footnotesize
\begin{verbatim}
compile_all: $(FILE_LIST)
    $(JAVA) @$<

.INTERMEDIATE: $(FILE_LIST)
$(FILE_LIST):
    for d in $(source_dirs); \
    do                       \
        echo $$d/*.java;     \
    done > $@

\end{verbatim}
}

Обратите внимание на тонкую ошибку в цикле \command{for}. Если
какой-либо из каталогов не содержит исходных файлов \Java{}, строка
\filename{*.java} будет включена в список файлов и компилятор \Java{}
выдаст сообщение об ошибке: <<File not found>> (файл не найден). Мы
можем приказать \utility{bash} подставлять пустые строки на место
шаблонов, которым не соответствует ни один файл, использовав опцию
\command{nullglob}:

{\footnotesize
\begin{verbatim}
compile_all: $(FILE_LIST)
    $(JAVA) @$<

.INTERMEDIATE: $(FILE_LIST)
$(FILE_LIST):
    shopt -s nullglob;       \
    for d in $(source_dirs); \
    do                       \
        echo $$d/*.java;     \
    done > $@
\end{verbatim}
}

Многим проектам приходится создавать список файлов. Ниже представлен
макрос, содержащий сценарий \utility{bash}, создающий список файлов.
Первым аргументом является корневой каталог, пути всех найденных
файлов будут указываться относительно этого каталога. Вторым
параметром является список каталогов, в которых нужно искать файлы,
соответствующие шаблону. Третий и четвёртый аргументы опциональны и
содержат расширения интересующих файлов.

{\footnotesize
\begin{verbatim}
# $(call collect-names,root-dir,dir-list,
#        suffix1-opt,suffix2-opt)
define collect-names
  echo Making $@ from directory list...             
  cd $1;                                              \
  shopt -s nullglob;                                  \
  for f in $(foreach file,$2,'$(file)'); do           \
    files=( $$f$(if $3,/*.{$3$(if $4,$(comma)$4)}) ); \
    if (( $${#files[@]} > 0 ));                       \
    then                                              \
      printf '"%s"\n' $${files[@]};                   \
    else :; fi;                                       \
  done
endef
\end{verbatim}
}

Так выглядит шаблонное правило создания списка файлов изображений:

{\footnotesize
\begin{verbatim}
%.images:
    @$(call collect-names,$(SOURCE_DIR),$^,gif,jpeg) > $@
\end{verbatim}
}

Вычисление макроса скрыто с помощью модификатора \command{@},
поскольку сценарий достаточно велик, а причину для копирования и
вставки полученного кода найти трудно. Список каталогов указан в
реквизитах. После смены текущего каталога сценарий включает опцию
\texttt{nullglob}. Остаток макроса~--- цикл \emph{for}, осуществляющий
проход по всем каталогам, которые нужно обработать. Первое выражение
поиска файлов~--- это список слов, переданный в качестве второго
параметра (\variable{\${}2}). Сценарий экранирует слова в списке
файлов с помощью апострофа, так как они могут содержать символы,
имеющие для командного интерпретатора специальный смысл. В частности,
имена файлов в некоторых языках программирования (например, \Java{})
могут содержать символы доллара:

{\footnotesize
\begin{verbatim}
for f in $(foreach file,$2,'$(file)'); do
\end{verbatim}
}

Мы производим поиск файлов в каталоге, заполняя массив
\variable{files} результатами вычислений подстановок. Если полученный
массив содержит элементы, мы используем \function{printf} для того,
чтобы напечатать каждое слово на новой строке. Использование массива
позволяет макросу правильно обрабатывать пути, содержащие
пробелы. Возможность наличия пробелов в путях~--- это ещё одна причина,
по которой аргумент \function{printf} окружён кавычками.

Список файлов создаётся при помощи следующей строки:

{\footnotesize
\begin{verbatim}
files=( $$f$(if $3,/*.{$3$(if $4,$(comma)$4)}) );
\end{verbatim}
}

Переменная \variable{\$\$f}~--- это каталог или файл, переданный
макросу в составе аргумента. Следующее выражение~--- это функция
\function{if}, проверяющая третий аргумент на непустоту. Это один из
путей, который можно использовать для реализации необязательных
аргументов. Если третий аргумент пуст, четвёртый также подразумевается
пустым. В этом случае файл, переданный пользователем, должен быть
включён в список как есть. Это позволяет макросу строить списки
обычных файлов, для которых использование шаблонов не подходит. Если
третий аргумент не пуст, функция \function{if} добавляет к корневому
каталогу строку \texttt{/*\{\$3\}}. Если передан четвёртый аргумент,
после \variable{\$3} происходит вставка \variable{\$4}. Обратите
внимание на то, как происходит вставка запятой в шаблон. Поместив
символ запятой в переменную \GNUmake{}, мы можем незаметно передать её
в выражение, явное использование запятой было бы воспринято как
отделение \emph{then} части от \emph{else} части функции
\function{if}. Определение переменной \variable{comma} очевидно:

{\footnotesize
\begin{verbatim}
comma = ,
\end{verbatim}
}
 
Все рассмотренные циклы \function{for} зависели от пределов длины
командной строки, поскольку использовали шаблонные выражения. Разница
в том, что результат применения шаблона для поиска файлов в одном
каталоге имеет гораздо меньше шансов превысить предел.

Что будет, если какая-то переменная \GNUmake{} содержит длинный список
файлов? Чтож, тогда мы столкнулись с настоящей неприятностью. Я нашёл
лишь два пути передать длинную переменную \GNUmake{} в интерпретатор.
Первый подход~--- передавать содержимое переменной по частям,
используя фильтры, основанные на применении функции
\function{wordlist}:

{\footnotesize
\begin{verbatim}
compile_all:
    $(JAVAC) $(wordlist 1, 499, $(all-source-files))
    $(JAVAC) $(wordlist 500, 999, $(all-source-files))
    $(JAVAC) $(wordlist 1000, 1499, $(all-source-files))
\end{verbatim}
}

Второй путь~--- использовать функцию \function{filter}, однако в этом
случае результаты менее предсказуемы, поскольку число отбираемых
фильтром файлов может зависеть от числа слов, соответствующему
каждому из выбранных шаблонов. В следующем примере используются
шаблоны, основанные на алфавитном порядке:

{\footnotesize
\begin{verbatim}
compile_all:
    $(JAVAC) $(filter a%, $(all-source-files))
    $(JAVAC) $(filter b%, $(all-source-files))
\end{verbatim}
}

Ваши шаблоны могут использовать специальные свойства имён файлов.

Обратите внимание на то, как сложно автоматизировать этот процесс. Мы
могли бы попробовать использовать алфавитный подход совместно с циклом
\function{foreach}:

{\footnotesize
\begin{verbatim}
compile_all:
    $(foreach l,a b c d e ...,                 \
      $(if $(filter $l%, $(all-source-files)), \
        $(JAVAC) $(filter $l%, $(all-source-files));))
\end{verbatim}
}

Однако такой подход не работает. \GNUmake{} превратит этот сценарий в
одну строку текста, что только усугубит проблемы с длиной команд.
Вместо этого можно использовать \function{eval}:

{\footnotesize
\begin{verbatim}
compile_all:
    $(foreach l,a b c d e ...,                 \
      $(if $(filter $l%, $(all-source-files)), \
        $(eval                                 \
          $(shell                              \
            $(JAVAC) $(filter $l%, $(all-source-files));))))
\end{verbatim}
}

Этот вариант будет работать правильно, потому что функция
\function{eval} выполняет команду \function{shell} незамедлительно, и
результатом её вычисления является пустая строка. Таким образом,
результатом вычисления функции \function{foreach} является также
пустая строка. Проблема заключается в том, что проверка ошибок в этом
контексте не происходит, поэтому ошибки компиляции не будут переданы
\GNUmake{} напрямую.

Подход, основанный на использовании \function{wordlist} значительно
хуже. Из-за ограниченных возможностей \GNUmake{} в области численных
операций, применить эту технику в цикле не получится. В общем,
сколь-нибудь удовлетворительных техник обращения с огромными списками
файлов практически не существует.


