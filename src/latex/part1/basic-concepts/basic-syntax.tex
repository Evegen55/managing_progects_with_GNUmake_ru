%%--------------------------------------------------------------------
%% Basic makefile syntax
%%--------------------------------------------------------------------
\section{Основы синтаксиса \Makefile{}}
Теперь, когда вы уже получили базовое представление о \GNUmake{},
можно приступить к составлению собственных \Makefile{}'ов. В этом
разделе мы рассмотрим синтаксис и структуру \Makefile{}'а, чтобы вы
могли начать использование \GNUmake{}. 

Как правило, \Makefile{}'ы пишутся в манере <<сверху вниз>>, то есть
сначала описывается наиболее общая цель (как правило, она имеет имя
\index{Цели!по умолчанию} \index{Цели!стандартные!all}
\target{all}), которая будет целью по умолчанию. Далее следуют всё
более детализированные цели, и в самом конце описываются цели,
используемые для поддержки программного продукта (такие, например, как
\index{Цели!стандартные!clean}
\target{clean}, используемая для удаления ненужных вр'{е}менных
файлов). Именами целей вовсе не обязательно должны быть настоящие
файлы, можно использовать любое имя.

\index{Правило}
В предыдущем примере мы видели упрощённую форму правила. Более полной
(но всё ещё не законченной) формой правила является следующая:

{\footnotesize
\begin{alltt}
\emph{цель\subi{1} цель\subi{2} цель\subi{3} : реквизит\subi{1} реквизит\subi{2}
    команда\subi{1}
    команда\subi{2}
    команда\subi{3}
}
\end{alltt}
}

Одна или более целей указываются слева от двоеточия, справа от него
следуют реквизиты. Если реквизиты не указаны, то собираются только те
цели, которые ещё не существуют. Последовательность команд, которые
необходимо выполнить для сборки цели, иногда называют
\index{Сценарий сборки}
\newword{сценарием сборки}, но чаще просто \newword{командами}.

Каждая команда должна начинаться с символа табуляции. Такой синтаксис
сообщает \GNUmake{} о том, что следующие за табуляцией символы должны
быть переданы в командный интерпретатор для последующего выполнения.
Если вы случайно поставите символ табуляции в начале строки, не
являющейся командой, то в большинстве случаев \GNUmake{} будет
интерпретировать последующий текст в этой строке как команду. Однако
может случиться и так, что \GNUmake{} сможет распознать ваш символ
табуляции как синтаксическую ошибку, в этом случае вы увидите
сообщение, подобное следующему:

{\footnotesize
\begin{alltt}
\$ \textbf{make}
Makefile:6: *** commands commence before first target.  Stop.
\end{alltt}
}

Мы вернёмся к аспектам использования символа табуляции в
главе~\ref{chap:rules}.
