%%--------------------------------------------------------------------
%% Minimazing rebuilds
%%--------------------------------------------------------------------
\section{Минимизируем число действий}

Если мы запустим нашу программу, то обнаружим, что помимо вывода числа
вхождений слов fee, fies, foes и fums она также выводит содержимое
входного файла. Это совсем не то, что от неё ожидалось. Проблема в
том, что мы забыли добавить несколько правил в наш лексический
\index{flex}
анализатор, и \utility{flex} выдаёт нераспознанный текст на
стандартный поток вывода. Для решения этой проблемы мы просто добавим
правило для <<любого символа>>:

{\footnotesize
\begin{verbatim}
        int fee_count = 0;
        int fie_count = 0;
        int foe_count = 0;
        int fum_count = 0;
%%
fee     fee_count++;
fie     fie_count++;
foe     foe_count++;
fum     fum_count++;
.
\n
\end{verbatim}
}

После редактирования файла с исходным кодом анализатора нам нужно
собрать наше приложение заново и проверить его работу:

{\footnotesize
\begin{alltt}
\$ \textbf{make}

flex -t lexer.l > lexer.c
gcc -c lexer.c
gcc count\_words.o lexer.o -lfl -ocount\_words
\end{alltt}
}

На этот раз файл \filename{count\_words.c} не подвергся компиляции.
При анализе правил \GNUmake{} обнаружил, что файл
\filename{count\_words.o} существует и имеет более позднюю дату
модификации, чем цель, потому не было предпринято никаких действий по
сборке этой цели. Однако при анализе цели \filename{lexer.c} было
обнаружено, что реквизит \filename{lexer.l} имеет более позднюю дату
модификации, поэтому произошла повторная сборка цели
\filename{lexer.c}. Это, в свою очередь, вызвало повторную сборку
\filename{lexer.o}, а затем и \filename{count\_words}. Теперь наша
программа подсчёта слов работает правильно:

{\footnotesize
\begin{alltt}
\$ \textbf{count\_words < lexer.l}
3 3 3 3
\end{alltt}
}
