%%--------------------------------------------------------------------
%% Special targets
%%--------------------------------------------------------------------
\section{Специальные цели}

\index{Цели!специальные}
\newword{Специальной целью} называют встроенную абстрактную цель,
предназначенную для спецификации поведения \GNUmake{}. Например, уже
известная нам \target{.PHONY} является специальной целю, реквизиты
которой не связаны с реальными файлами и всегда требуют обновления.

Для специальных целей используется стандартный синтаксис
\ItalicMono{цель: реквизиты}, но \ItalicMono{цель} не является файлом
или обычной абстрактной целью. Более всего специальные цели похожи на
директивы, изменяющие внутренние алгоритмы \GNUmake{}.

Существует двенадцать специальных целей. Их можно разделить на три
категории: одни изменяют поведение \GNUmake{}, вторые являются просто
флагами, наконец, специальная цель \variable{.SUFFIXES} используется
для спецификации суффиксных правил (обсуждавшихся в разделе
<<\nameref{sec:suffix_rules}>>).

Наиболее полезны следующие специальные цели:

\begin{description}
\item[\target{.INTERMEDIATE}] \hfill\\
Реквизиты этой цели интерпретируются как промежуточные файлы. Если
\GNUmake{} создаст файл во время сборки другой цели, файл будет
автоматически удалён перед завершением работы \GNUmake{}. Если файл
уже существует в момент сборки цели, файл не будет удалён.

Такое поведение может быть очень полезным при построении цепочки
правил. Например, большинство Java утилит принимают списки файлов в
стиле Windows. Создание правил для сохранения списков
файлов и спецификация их как промежуточных позволяет \GNUmake{}
автоматически удалять множество временных файлов.

\item[\target{.SECONDARY}] \hfill\\
Реквизиты этой специальной цели интерпретируются как промежуточные
файлы, которые не удаляются автоматически. Наиболее часто
\target{.SECONDARY} используется для пометки объектных файлов,
хранимых в виде библиотек. Обычно такие объекты будут удалены сразу
после добавления их в архив. Иногда удобнее не удалять объектные
файлы во время разработки.

\item[\target{.PRECIOUS}] \hfill\\
Когда \GNUmake{} аварийно завершает выполнение, файл цели может быть
удалён, если он изменился с момента старта \GNUmake{}. Таким образом
избегается возможность сохранения частично собранных (и, возможно,
повреждённых) файлов. Иногда вы можете захотеть от \GNUmake{} другого
поведения, например, если файл велик и требует много вычислений для
своего создания. Если вы укажете имя такого файла в качестве
реквизитов цели \target{.PRECIOUS}, \GNUmake{} не будет удалять этот
файл в случае аварийного завершения. Эта цель используется довольно
редко, но если её применение действительно необходимо, её наличие
сохранит разработчику много времени и сил. Заметим, что \GNUmake{} не
осуществляет автоматического удаления в случае ошибки в выполняемой
команде, только в случае получения сигнала останова.


\item[\target{.DELETE\_ON\_ERROR}] \hfill\\
Эта цель~--- противоположность \target{.PRECIOUS}. Спецификация файла
как реквизита этой цели означает, что файл должен быть удалён в случае
любой ошибки в сценарии сборки, ассоциированном с соответствующим
правилом. Обычно \GNUmake{} удаляет цель только в случае получении
сигнала останова.
\end{description}

Остальные специальные цели будут рассмотрены в момент их
непосредственного использования. Цели, относящиеся к параллельному
выполнению, будут рассмотрены в
главе~\ref{chap:improving_the_performance}, цель
\target{.EXPORT\_ALL\_VARIABLES}~--- в главе~\ref{chap:vars}.
