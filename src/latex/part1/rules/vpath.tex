%%--------------------------------------------------------------------
%% Finding files with VPATH and vpath
%%--------------------------------------------------------------------
\section{Поиск файлов с помощью VPATH и vpath}
\label{sec:vpath}

Наши примеры были довольно просты, поэтому \Makefile{} и исходные
файлы находились в одном каталоге. Реальные программы гораздо сложнее
(когда в последний раз вы работали над проектом, все файлы которого
размещались в одном каталоге?). Давайте изменим наш пример, разместив
файлы более реалистичным образом. Мы может модифицировать нашу
программу подсчёта слов, перенеся часть работы, выполняемой функцией
\function{main}, в функцию \function{counter}:

{\footnotesize
\begin{verbatim}
#include <lexer.h>
#include <counter.h>

void counter( int counts[4] )
{
    while ( yylex( ) )
        ;

    counts[0] = fee_count;
    counts[1] = fie_count;
    counts[2] = foe_count;
    counts[3] = fum_count;
}
\end{verbatim}
}

Поместим объявление этой функции в заголовочный файл
\filename{counter.h}:

{\footnotesize
\begin{verbatim}
#ifdef COUNTER_H_
#define COUNTER_H_

extern void
counter( int counts[4] );

#endif
\end{verbatim}
}

Мы также можем поместить объявления функций из \filename{lexer.l} в
файл \filename{lexer.h}:

{\footnotesize
\begin{verbatim}
#ifndef LEXER_H_
#define LEXER_H_

extern int fee_count, fie_count, foe_count, fum_count;

extern int yylex( void );

#endif
\end{verbatim}
}

В соответствии с негласным правилами размещения исходного кода в
файловой системе, все заголовочные файлы помещаются в каталог
\filename{include}, а исходные файлы~--- в каталог \filename{src}.
Разместим наши файлы подобным образом, оставив \Makefile{} в корневом
каталоге проекта. Результирующее дерево изображено на
рисунке~\ref{fig:ex_src_tree}.

\begin{figure}
{\footnotesize
\begin{verbatim}
.
|--include
|  |--counter.h
|  `--lexer.h
|--src
|  |--count\_words.c
|  |--counter.c
|  `--lexer.l
`--Makefile
\end{verbatim}
}
\caption{Структура каталогов проекта программы подсчёта слов.}
\label{fig:ex_src_tree}
\end{figure}

Поскольку теперь наши файлы с исходным кодом включают заголовочные
файлы, эта новая зависимость должна быть отражена в нашем \Makefile{},
то есть если модифицируется заголовочный файл, то соответствующий
объектный файл должен быть заново скомпилирован из исходного кода.

{\footnotesize
\begin{verbatim}
count_words: count_words.o counter.o lexer.o -lfl
    gcc $^ -o $@

count_words.o: count_words.c include/counter.h
    gcc -c $<

counter.o: counter.c include/counter.h include/lexer.h
    gcc -c $<

lexer.o: lexer.c include/lexer.h
    gcc -c $<

lexer.c: lexer.l
    flex -t $< > $@
\end{verbatim}
}

После запуска \GNUmake{} получаем следующий результат:

{\footnotesize
\begin{alltt}
\$ \textbf{make}
make:  *** No rule to make target `count\_words.c',
       needed by `count\_words.o'.  Stop.
\end{alltt}
}

Что же произошло? \GNUmake{} попытался собрать исходный файл
\filename{count\_words.c}! Давайте проследим за работой \GNUmake{}.
Первым реквизитом является файл \filename{count\_words.o}. Такой файл
не существует, следовательно, его нужно создать. Явное правило для
создания \filename{count\_words.o} указывает на
\filename{count\_words.c}. Но \GNUmake{} не сможет найти этот
исходный файл, ведь он находится не в текущем каталоге, а в каталоге
\filename{src}. Пока вы не укажите другого, \GNUmake{} будет искать
цели и реквизиты в текущем каталоге. Как сообщить \GNUmake{}, что
исходные файлы нужно искать в каталоге \filename{src}, или, если
поставить вопрос более \'{о}бщно, где нужно искать файлы с исходным
кодом?

Вы можете указать \GNUmake{} каталоги, в которых нужно искать
файлы, используя переменную \variable{VPATH} и директиву
\directive{vpath}. Для того, чтобы решить нашу проблему, мы может
добавить определение \variable{VPATH} в наш \Makefile{}:

{\footnotesize
\begin{verbatim}
VPATH = src
\end{verbatim}
}

Это означает, что \GNUmake{} должен искать в каталоге \filename{src}
те файлы, которые не обнаружены в текущем каталоге. После запустка
\GNUmake{}, мы получим следующий вывод:

{\footnotesize
\begin{alltt}
\$ \textbf{make}
gcc -c src/count\_words.c -o count\_words.o
src/count\_words.c:2:21: counter.h: No such file or directory
make: *** [count\_words.o] Error 1
\end{alltt}
}

Заметим, что в этот раз \GNUmake{} успешно запустил команду
компиляции, правильно подставив относительный путь к исходному файлу.
В этом заключается ещё один плюс использования автоматических
переменных: \GNUmake{} не сможет использовать подходящий путь к
исходному файлу, если вы явно укажите его имя. К несчастью, процесс
компиляции окончился неудачей, так как \utility{gcc} не смог найти
заголовочного файла. Эту проблему можно решить, настроив неявное
правило компиляции подходящей опцией \command{-I}:

{\footnotesize
\begin{verbatim}
CPPFLAGS = -I include
\end{verbatim}
}

Теперь сборка проходит успешно:

{\footnotesize
\begin{alltt}
\$ \textbf{make}

gcc -I include -c src/count\_words.c -o count\_words.o
gcc -I include -c src/counter.c -o counter.o
flex -t src/lexer.l > lexer.c
gcc -I include -c lexer.c -o lexer.o
gcc count\_words.o counter.o lexer.o /lib/libfl.a -o count\_words
\end{alltt}
}

Переменная \variable{VPATH} содержит список всех каталогов, в которых
\GNUmake{} будет искать недостающие файлы. В этих каталогах будет
происходить поиск как целей, так и реквизитов, но не файлов, указанных
в сценариях сборки. Элементы списка могут разделяться пробелами или
двоеточиями в \UNIX{} и пробелами или точками с запятой в Windows.
Предпочтительней пользоваться пробелами, поскольку такой формат
подходит для любой операционной системы и позволяет избежать путаницы
с точками с запятой и двоеточиями. Кроме того, имена каталогов,
разделённые пробелами, легче читать.

Переменная \variable{VAPTH} решает нашу проблему поиска, но всё же она
является слишком мощным инструментом. \GNUmake{} ищет в каждом
указанном в \variable{VPATH} каталоге \emph{каждый} недостающий
файл. Если в разных каталогах существуют файлы с одинаковыми
именами, \GNUmake{} возьмёт первый найденный. Иногда это может быть
причиной неприятностей.

Директива \directive{vpath} больше подходит для наших нужд. Синтаксис
вызова этой директивы представлен ниже:

{\footnotesize
\begin{alltt}
vpath \emph{шаблон список-каталогов}
\end{alltt}
}

Теперь мы можем исключить использование переменной \variable{VPATH},
добавив следующие строки в наш \Makefile{}:

{\footnotesize
\begin{verbatim}
vpath %.c src
vpath %.h include
\end{verbatim}
}

Теперь мы сообщили \GNUmake{}, что искать все файлы с расширением
\filename{.c} нужно в каталоге \filename{src}, а файлы с расширением
\filename{.h}~--- в каталоге \filename{include} (теперь мы можем
удалить префикс \filename{include} у заголовочных файлов в списке
реквизитов). В более сложных приложениях такой подход позволяет
сохранить нервы и время, потраченные на отладку.

В нашем примере мы использовали \directive{vpath} для решения проблемы
поиска файлов, разнесённых по разным директориям. Существует также
родственная проблема сборки приложения таким образом, чтобы объектные
файлы помещались в отдельное <<бинарное>> дерево каталогов, в то время
как исходные файлы хранились дереве каталогов <<исходного кода>>.
Правильное использование \directive{vpath} помогает решить и эту
проблему, однако эта задача быстро становиться очень сложной, так что
использование одной лишь директивы \directive{vpath} становится
неэффективным.  Мы обсудим эту проблему более детально в следующих
разделах.
