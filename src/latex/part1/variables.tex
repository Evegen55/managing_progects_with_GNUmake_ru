%%%-------------------------------------------------------------------
%%% Variables and Macros
%%%-------------------------------------------------------------------
\chapter{Переменные и макросы}
\label{chap:vars}

Мы уже видели переменные в \Makefile{}'ах и множество примеров их
использования как во встроенных, так и в определённых пользователем
правилах. Однако те примеры, которые мы видели, являются лишь вершиной
айсберга. Переменные и макросы могут быть гораздо более сложными.
Именно они придают GNU \GNUmake{} часть его невероятной мощи.

Прежде, чем мы продолжим, важно осознать, что \GNUmake{} является
смешением двух языков. Первый язык описывает граф зависимостей,
состоящий из целей и реквизитов (этот язык был подробно рассмотрен в
\index{Макроязык}
главе~\ref{chap:rules}). Второй язык является макроязыком для
осуществления текстовых подстановок. Быть может, вы знакомы и с
другими макроязыками: препроцессор \Clang{}, \utility{m4}, \TeX{} и
макроассемблеры.  Как и эти макроязыки, \GNUmake{} позволяет
определять условные обозначения для длинной последовательности
символов и использовать их в вашей программе. Макропроцессор
распознает их в тексте программы и заменит на соответствующую
последовательность. Несмотря на то, что удобно думать о переменных в
\Makefile{}'е как о переменных в традиционных языках программирования,
есть существенное отличие между макропеременными и <<традиционными>>
переменными. Значения макропеременных подставляются сразу при встрече
их имени в тексте программы, порождая строку, которая затем также
сканируется на наличие макросов. Это отличие станет более ясным, когда
мы рассмотрим переменные \GNUmake{} подробнее.

Имена переменных могут содержать почти любые символы, включая многие
знаки пунктуации. Разрешаются даже пробелы, но, если вы считаете себя
здравомыслящим человеком, избегайте их. Не разрешается использовать в
составе имени переменных следующие символы: \command{:}, \command{\#}
и \command{=}.

Регистр букв в имени переменных имеет значение, то есть переменные
\variable{cc} и \variable{CC} являются различными переменными. Для
получения значения переменной нужно заключить её имя внутрь круглых
скобок, предваряемых символом доллара (\variable{\$( )}). Из этого
правила есть одно исключение: если имя переменной состоит из одного
символа, то круглые скобки можно опустить и писать просто
\index{Переменные!автоматические}
\command{\${}\textit{символ}}. Вот почему автоматически переменные
могут использоваться без круглых скобок. Как правило вам стоит
предпочитать форму со скобками и избегать переменных, имя которых
состоит из одного символа.

Значение переменной также может быть получено с использованием
фигурных скобок, например, \variable{\${}\{CC\}}. Эта форма
встречается довольно часто, в частности, в старых \Makefile{}'ах.
Трудно найти причину, по которой использование одной из этих форм было
бы предпочтительным.  Выберите для себя какую-то одну и
придерживайтесь её. Некоторые люди используют фигурные скобки для
ссылки на переменную, а круглые~--- для вызовов функций, подражая
синтаксису командного интерпретатора Bourne shell. В современных
\Makefile{}'ах используются круглые скобки, вот почему мы будем
придерживаться именно этого стиля в этой книге.

Существуют определённые соглашения относительно имён переменных. Все
буквы имени переменных, представляющих значения, не изменяемые в ходе
работы \GNUmake{} (констант), и которые могут быть указаны
пользователем через интерфейсы командной строки или переменных
окружения, должны быть заглавными. Слова внутри имён таких переменных
обычно разделяются подчёркиваниями. Имена переменных, которые
встречаются только внутри \Makefile{}'а, содержат только прописные
буквы, слова в них также разделяются подчёркиваниями. Наконец, в этой
книге имена всех функций, определяемых пользователем с помощью
переменных или макросов, состоят из прописных букв, слова внутри имён
разделяются знаками тире.  Другие соглашения относительно имён будут
оглашаться тогда, когда в этом будет необходимость. Следующие примеры
используют функциональность, которую мы ещё не обсуждали. Поскольку
они иллюстрируют применение соглашений именования, не старайтесь
вникать в детали:

{\footnotesize
\begin{verbatim}
# Обычные константы.
CC    := gcc
MKDIR := mkdir -p

# Внутренние переменные.
sources = *.c
objects = $(subst .c,.o,$(sources))

# Пара функций.
maybe-make-dir  = $(if $(wildcard $1),,$(MKDIR) $1)
assert-not-null = $(if $1,,$(error Illegal null value.))
\end{verbatim}
}

Значение переменной состоит из всех слов, находящихся справа от знака
присваивания без учёта начального пробела. Пробелы в конце строки
также входят в состав значения. Иногда это может вызывать проблемы,
например, при использовании переменных, чьи значения оканчиваются
пробелами, в сценариях командного интерпретатора:

{\footnotesize
\begin{verbatim}
LIBRARY = libio.a # LIBRARY заканчивается пробелом

missing_file:
    touch $(LIBRARY)
    ls -l | grep '$(LIBRARY)'
\end{verbatim}
}

Присваивание переменной содержит пробел, который становится более
заметным за счёт комментария (однако на самом деле комментария может и
не быть). После запуска \GNUmake{} мы увидим следующий вывод:

{\footnotesize
\begin{alltt}
\$ \textbf{make}

touch libio.a 
ls -l | grep 'libio.a '
make: *** [missing\_file] Error 1
\end{alltt}
}

Поскольку шаблон поиска, переданный программе \utility{grep}, также
содержит пробел, поиск его вхождения в выводе команды \utility{ls}
закончился неудачей. Позднее мы обсудим проблемы, связанные с
пробелами, более детально. А пока давайте рассмотрим поближе
переменные \GNUmake{}.

%%--------------------------------------------------------------------
%% What variables are used for
%%--------------------------------------------------------------------
\section{Для чего используются переменные}

В целом использование переменных для представления внешних программ
является хорошей идеей, поскольку позволяет пользователю легко
адаптировать \Makefile{} под своё окружение. Например, в системе
часто бывает несколько версий \utility{awk}: \utility{awk},
\utility{nawk} и \utility{gawk}. Создавая переменную \utility{AWK} для
хранения имени программы \utility{awk}, вы облегчаете жизнь
пользователям вашего \Makefile{}'а. К тому же, если безопасность
критична для вашей среды, доступ к внешним программам с указанием
абсолютного пути к их исполняемым файлам является хорошей практикой
решения проблем с переменной \variable{PATH} пользователя. Абсолютные
пути также уменьшают вероятность запуска троянского коня под видом
системной утилиты. И, конечно, абсолютные пути делают ваши
\Makefile{}'ы менее переносимым. Здесь вы должны сделать выбор,
основываясь на ваших собственных требованиях.

Хотя наше первое использование переменных заключалось в хранении
простых констант, переменные также могут быть использованы для
хранения последовательностей команд, например, следующий пример
определяет функцию для вывода отчёта о количестве свободных блоков в
файловой системе\footnote{Команда \utility{df} возвращает список всех
смонтированных файловых систем, их ёмкость и объём занятого
пространства. Если задан аргумент, то выводится статистика для
указанной файловой системы. Первая строка вывода содержит заголовки
столбцов. Вывод этой команды подаётся на вход сценария \utility{awk},
извлекающего вторую строку и игнорирующего остальные. Четвёртый
столбец в выводе \utility{df}~--- число свободных блоков.}:

{\footnotesize
\begin{verbatim}
DF  = df
AWK = awk
free-space := $(DF) . | $(AWK) 'NR == 2 { print $$4 }'
\end{verbatim}
}

Переменные могут использоваться для обоих указанных целей. Как мы
увидим позднее, это не единственные их применения. 

%%--------------------------------------------------------------------
%% Variable types
%%--------------------------------------------------------------------
\section{Типы переменных}
\label{var_types}

\index{variables!simple expanded}
\index{variables!recursively expanded}
В \GNUmake{} существует два типа переменных: упрощённо вычисляемые
(simple expanded variables) и рекурсивно вычисляемые (recursively
expanded variables). \newword{Упрощённо вычисляемые} переменные (или
\index{Переменные!простые}
\newword{простые переменные}) определяются при помощи оператора
присваивания <<\command{:=}>>:

{\footnotesize
\begin{verbatim}
MAKE_DEPEND := $(CC) -M
\end{verbatim}
}

Такие переменные называются <<упрощённо вычисляемыми>> потому, что
правая часть присваивания вычисляется непосредственно при чтении
\Makefile{}'а. При этом подставляется значение всех переменных
\GNUmake{}, входящих в правую часть, и результирующий текст
сохраняется в качестве значения переменной. Это поведение идентично
поведению большинства языков программирования и командных сценариев.
Например, вычисление предыдущей переменной, скорее всего, породит
текст \command{gcc -M}. Однако если переменная \variable{CC} не
определена, то переменная \variable{MAKE\_DEPEND} примет значение
\command{<пробел>-M}. В этом случае выражение \command{\$(CC)}
вычисляется как пустая строка, поскольку переменная \variable{CC} ещё
не определена. Отсутствие определения переменной не является ошибкой.
На самом деле это очень удобно. Большинство неявных правил содержат
неопределённые переменные, необходимые для настройки поведения правил
пользователями. Если пользователь не определяют никаких настроек,
переменные просто содержат пустые строки. Теперь рассмотрим пробел в
начале полученного значения. Правая часть присваивания после
отбрасывания начального пробела выглядит следующим образом:
\command{\${CC} -M}. После того, как ссылка на переменную
вычисляется как пустая строка, \GNUmake{} не производит повторного
сканирования и не удаляет начальные пробелы.

\index{Переменные!рекурсивные}
\newword{Рекурсивно вычисляемые} переменные (или просто рекурсивные
переменные) определяются при помощи оператора присваивания
<<\command{=}>>:

{\footnotesize
\begin{verbatim}
MAKE_DEPEND = $(CC) -M
\end{verbatim}
}

Второй тип переменных называется <<рекурсивно вычисляемые переменные>>
потому, что правая часть присваивания просто копируется \GNUmake{} и
сохраняется как значение переменной без вычисления. Вместо этого
вычисление происходит каждый раз, когда переменная
\emph{используется}. Быть может, более подходящим названием для таких
переменных~--- \newword{лениво вычисляемые} переменные, поскольку
вычисления откладываются до тех пор, пока не потребуется их результат.
Одним из удивительных следствий такого рода вычислений заключается в
том, что присваивания могут осуществляться <<в неправильном порядке>>:

{\footnotesize
\begin{verbatim}
MAKE_DEPEND = $(CC) -M
...
# Чуть позже
CC = gcc
\end{verbatim}
}

Теперь значение \variable{MAKE\_DEPEND} будет равно \command{gcc -M},
несмотря на то что значение \variable{CC} не определено в момент
присваивания значения переменной \variable{MAKE\_DEPEND}.

На самом деле рекурсивные переменные не являются просто ленивыми
присваиваниями (по крайней мере, обычными ленивыми присваиваниями).
Каждый раз при обращении к рекурсивной переменной её значение
вычисляется заново. Для переменных, определённых в терминах простых
констант, таких как \variable{MAKE\_DEPEND}, эта разница бессмысленна,
поскольку все переменные справа от оператора присваивания также
являются простыми константами. Однако представим, что переменная в
присваиваемом выражении представляет собой результат выполнения
некоторой программы, например, \utility{date}.  Каждый раз, когда
подобная рекурсивная переменная будет вычисляться, будет происходить
запуск программы \utility{date} и переменная будет получать новое
значение (в предположении, что повторное вычисление происходит по
крайней мере через секунду). Иногда это чрезвычайно полезно. А
иногда весьма раздражает!

%---------------------------------------------------------------------
% Other types of assignment
%---------------------------------------------------------------------
\subsection*{Другие виды присваивания}
\label{sec:other_types_of_assign}

В предыдущем примере мы видели два типа присваивания: <<\command{=}>>
для определения рекурсивных переменных и <<\command{:=}>> для
определения простых переменных. \GNUmake{} имеет ещё два оператора
присваивания.

\index{Операторы!условного присваивания}
Оператор <<\command{?=}>> называется \newword{оператором условного
присваивания переменной}. Для краткости мы будем называть его просто
условным присваиванием. Этот оператор осуществляет присваивание
переменной только в том случае, если её значение ещё не определено.

{\footnotesize
\begin{verbatim}
# Положить полученные файлы в каталог $(PROJECT_DIR)/out.
OUTPUT_DIR ?= \(PROJECT_DIR)/out
\end{verbatim}
}

В этом примере мы присвоим значение переменной \variable{OUTPUT\_DIR}
только в том случае, если оно ещё не было определено. Такое поведение
очень удобно для работы с переменными окружения. Мы обсудим этот
вопрос более подробно в разделе
<<\nameref{sec:where_vars_come_from}>>.

Другой оператор присваивания, \command{+=}, обычно называют
\emph{добавлением}. Как можно предположить из названия, этот оператор
добавляет текст к значению переменной. Может быть не очевидно, что это
довольно важная функциональность, необходимая при использовании
рекурсивных переменных. Значение справа от этого оператора
присваивания добавляется к значению переменной, \emph{не изменяя
в переменной первоначального значения}. <<Подумаешь,>>~--- скажете
вы,~--- <<Разве не так обычно работает добавление?>>. Да, но здесь
есть одна маленькая хитрость.

Добавление текста к простой переменной реализуется очевидным образом.
Оператор \command{+=} может быть реализован так:

{\footnotesize
\begin{alltt}
простая\_переменная := \$(простая\_переменная) что-то ещё
\end{alltt}
}

Поскольку значение простой переменной уже было вычислено, \GNUmake{}
может просто вычислить выражение \command{\$(простая\_переменная)},
добавить требуемый текст и закончить присваивание. Но рекурсивные
переменные порождают проблему. Следующая реализация недопустима:

{\footnotesize
\begin{alltt}
рекурсивная\_переменная = \$(рекурсивная\_переменная) что-то ещё
\end{alltt}
}

Это выражение является ошибкой, потому что для \GNUmake{} не
существует корректного способа его обработки. Если \GNUmake{} сохранит
текущее значение рекурсивной переменной плюс текст <<что-то ещё>>, то
не сможет вычислить правильное значение позднее. Более того, попытка
вычислить рекурсивную переменную, содержащую ссылку на себя, приводит
к бесконечному циклу:

{\footnotesize
\begin{alltt}
\$ \textbf{make}

makefile:2: *** Recursive variable `recursive' references
itself (eventually).  Stop.
\end{alltt}
}

Таким образом, оператор <<\command{+=}>> был создан специально для
возможности добавления текста к рекурсивным переменным. В частности,
этот оператор полезен при инкрементом определении значения переменной.

%%--------------------------------------------------------------------
%% Macros
%%--------------------------------------------------------------------
\section{Макросы}

Переменные хороши для хранения одиночных строк текста, но что если мы
хотим сохранить многострочное значение, такое, как командный сценарий,
который мы хотим выполнять в нескольких правилах? Например, следующая
последовательность команд может быть использована для создания \Java{}
архива (\filename{jar}) из \filename{.class} файлов:

{\footnotesize
\begin{verbatim}
echo Creating $@...
$(RM) $(TMP_JAR_DIR)
$(MKDIR) $(TMP_JAR_DIR)
$(CP) -r $^ $(TMP_JAR_DIR)
cd $(TMP_JAR_DIR) && $(JAR) $(JARFLAGS) $@ .
$(JAR) -ufm $@ $(MANIFEST)
$(RM) $(TMP_JAR_DIR)
\end{verbatim}
}

В начале длинной последовательности (наподобие предыдущей) я
предпочитаю печатать диагностическое сообщение. Это значительно
упрощает чтение вывода \GNUmake{}. После вывода сообщения мы помещаем
наши \filename{.class} файлы в новый временный каталог, удалив сначала
этот каталог, если он существует\footnote{%
Для достижения наилучшего эффекта переменная \variable{RM} должна
иметь значение \command{rm -rf}. Как правило, по умолчанию она
определена как \command{rm -f}, что безопаснее, но не так эффективно.
Переменная \variable{MKDIR} должна иметь значение \command{mkdir -p}, и
так далее (прим. автора).}, затем создав новый. После этого мы
копируем в этот каталог каталоги\hyp{}реквизиты (и все их
подкаталоги). Затем мы переходим в него и создаём jar\hyp{}архив с
именем цели. Наконец, мы добавляем к архиву файл манифеста и
осуществляем удаление временного каталога. Естественно, мы не хотим
делать копий этой последовательности команд, поскольку в будущем это
может усложнить поддержку. Мы можем рассмотреть вариант
упаковки всех этих команд в рекурсивную переменную, но такой подход
неудобен при поддержке и труден для чтения при выводе \GNUmake{} (вся
последовательность команд будет выведена на экран как одна большая
строка текста).

Вместо этого мы можем использовать <<упакованную последовательность
команд>> GNU \GNUmake{}, созданную при помощи директивы
\directive{define}.  Термин <<упакованная последовательность>> немного
\index{Макрос}
неуклюж, поэтому мы будем называть это \newword{макросом}. На самом
деле макросы~--- это просто ещё один метод определения переменных в
\GNUmake{}, позволяющий помещать символы окончания строки внутрь
значения переменной. Руководство пользователя GNU \GNUmake{}
использует слова \emph{переменная} и \emph{макрос} как синонимы. В
этой книге мы будем использовать термин \emph{макрос} исключительно
для обозначения переменных, определённых с помощью директивы
\directive{define}, если же определение происходит при помощи
оператора присваивания, будет применяться термин \emph{переменная}.

{\footnotesize
\begin{verbatim}
define create-jar
  @echo Creating $@...
  $(RM) $(TMP_JAR_DIR)
  $(MKDIR) $(TMP_JAR_DIR)
  $(CP) -r $^ $(TMP_JAR_DIR)
  cd $(TMP_JAR_DIR) && $(JAR) $(JARFLAGS) $@ .
  $(JAR) -ufm $@ $(MANIFEST)
  $(RM) $(TMP_JAR_DIR)
endef
\end{verbatim}
}

За директивой \directive{define} следуют имя макроса и новая строка.
Тело макроса содержит весь текст вплоть до слова \directive{endef},
которое должно находиться в отдельной строке. Макросы вычисляются
практически также, как и другие переменные, за тем исключением, что в
контексте командного сценария в начало каждой строки добавляется
символ табуляции. Вот пример, использующий эту особенность:

{\footnotesize
\begin{verbatim}
$(UI_JAR): $(UI_CLASSES)
    $(create-jar)
\end{verbatim}
}

Обратите внимание на символ \command{@}, добавленный перед командой
\command{echo}. \GNUmake{} не выводит команды с таким префиксом перед
их выполнением. Когда мы запустим \GNUmake{}, мы не увидим в выводе
команды \command{echo}, только результат её выполнения. Если префикс
\command{@} применяется внутри макроса, то его действие
распространяется только на ту строку, которую он предваряет. Однако,
если применить его при вызове макроса, его действие будет
распространятся на всё тело макроса:

{\footnotesize
\begin{verbatim}
$(UI_JAR): $(UI_CLASSES)
    @$(create-jar)
\end{verbatim}
}

Одним из результатов выполнения предыдущего примера будет следующий
вывод:

{\footnotesize
\begin{alltt}
\$ \textbf{make}
Creating ui.jar...
\end{alltt}
}

Использование префикса \command{@} рассматривается более детально в
разделе <<\nameref{sec:command_modifiers}>> главы~\ref{chap:commands}.

%%--------------------------------------------------------------------
%% When variables are expanded
%%--------------------------------------------------------------------
\section{Когда переменные получают свои значения}
\label{sec:when_vars_are_expanded}

В предыдущем разделе мы начали рассматривать <<закулисье>> вычисления
значения переменных. Результат во многом зависит от того, что уже было
определено, и где это было определено. Вы можете получить неожиданный
результат, даже если \GNUmake{} не может найти ошибки в вашей
спецификации. Так каковы же правила вычисления переменных? Как на
самом деле всё это работает?

После запуска \GNUmake{} выполняет свою работу в две фазы. Во время
первой фазы \GNUmake{} читает \Makefile{} и все включаемые им файлы.
На этом этапе переменные и правила загружаются во внутреннюю базу
данных \GNUmake{}, после чего создаётся граф зависимостей. Во время
второй фазы \GNUmake{} анализирует граф зависимостей и определяет
цели, которые нуждаются в сборке, затем выполняет командные сценарии
для сборки реквизитов.

Когда \GNUmake{} встречает директиву \directive{define} или
определение рекурсивной переменной, строки значения переменной или
тела макроса сохраняются вместе с символами новой строки без
каких-либо вычислений. Самый последний символ новой строки в теле
макроса не сохраняется в тексте определения. Иначе при вычислении
макроса читался бы один лишний символ новой строки.

При вычислении макроса полученный текст сразу же сканируется на
содержание других макросов или переменных, подлежащих вычислению, этот
процесс продолжается рекурсивно. Если макрос вычисляется в контексте
командного сценария, в начало каждой строки добавляется символ
табуляции.

Итак, вычисление переменных и макросов \GNUmake{} подчиняется
следующим правилам:
\begin{itemize}
%---------------------------------------------------------------------
\item
В случае присваивания переменной значения левая часть присваивания
всегда вычисляется сразу на первой фазе работы \GNUmake{}.
%---------------------------------------------------------------------
\item Вычисление правой части операторов \command{=} и \command{?=}
откладывается до тех пор, пока не потребуется значение соответствующей
переменной.
%---------------------------------------------------------------------
\item Правая часть оператора \command{:=} вычисляется сразу.
%---------------------------------------------------------------------
\item Правая часть оператора \command{+=} вычисляется сразу, если
переменная в левой части изначально была определена как простая, иначе
вычисление откладывается.
%---------------------------------------------------------------------
\item В определении макроса (использующего директиву
\directive{define}) имя определяемого макроса вычисляется сразу,
вычисление тела макроса откладывается.
%---------------------------------------------------------------------
\item Имена целей и реквизитов всегда вычисляются сразу, вычисление
команд всегда откладывается.
%---------------------------------------------------------------------
\end{itemize}

Таблица~\ref{tab:rules_for_imm_and_der_exp} содержит правила порядка
вычисления выражений при определении переменных.

%\begin{center}
\begin{table}
\begin{tabular}{|l|l|l|}
\hline
\multicolumn{1}{|c|}{\textbf{Определение}} &
\multicolumn{1}{c}{\textbf{\emph{A} вычисляется}} &
\multicolumn{1}{|c|}{\textbf{\emph{B} вычисляется}} \\
\hline
\(A = B\) & Сразу & При использовании \\
\hline
\(A ?= B\) & Сразу & При использовании\\
\hline
\(A := B\) & Сразу & Сразу \\
\hline
\(A += B\) & Сразу & Сразу или при использовании\\
\hline
\parbox{2.3cm}{
\begin{alltt}
define \emph{A}\\
\emph{B} \ldots\\
\emph{B} \ldots\\
endef
\end{alltt}
}
& Сразу & При использовании \\
\hline
\end{tabular}
\caption{Правила для незамедлительного и отложенного вычислений}
\label{tab:rules_for_imm_and_der_exp}
\end{table}
%\end{center}

Примите за правило определять переменные и макросы перед их
использованием. В частности, требуется, чтобы переменная,
используемая в описании цели или реквизита была определена.

Думаю, пример многое прояснит. Предположим, мы решили переделать наш
макрос \variable{free-space}. Сначала рассмотрим отдельные части
примера, затем соберём всё вместе.

{\footnotesize
\begin{verbatim}
BIN    := /usr/bin
PRINTF := $(BIN)/printf
DF     := $(BIN)/df
AWK    := $(BIN)/awk
\end{verbatim}
}

Мы определяем три переменных, содержащие имена программ, которые будут
использоваться в нашем макросе. Поскольку все переменные являются
простыми, их значения должны быть вычислены во время чтения
\Makefile{}'а.  Так как переменная \variable{BIN} определена раньше
остальных, её значение может быть использовано в определениях других
переменных.

Далее определим макрос \variable{free-space}.

{\footnotesize
\begin{verbatim}
define free-space
  $(PRINTF) "Free disk space "
  $(DF) . | $(AWK) 'NR == 2 { print $$4 }'
endef
\end{verbatim}
}

За директивой \directive{define} следует имя переменной, которое сразу
вычисляется. В нашем случае вычисления не требуется. Тело макроса
считывается и сохраняется не вычисленным.

Наконец, используем наш макрос внутри правила.

{\footnotesize
\begin{verbatim}
OUTPUT_DIR := /tmp

$(OUTPUT_DIR)/very_big_file:
    $(free-space)
\end{verbatim}
}

Когда считывается цель \target{\$(OUTPUT\_DIR)/very\_big\_file},
происходит подстановка значений всех переменных. Значение выражения
\command{\$(OUTPUT\_DIR)} вычисляется как \command{/tmp}, формируя
цель \target{/tmp/very\_big\_file}. Затем считывается командный
сценарий, ассоциированный с этой целью. Строки с командами
распознаются благодаря наличию символа табуляции, считываются и
сохраняются, но подстановка значений переменных и макросов не
происходит.

Теперь соберём отрывки воедино. Изменим порядок их следования для
иллюстрации алгоритма вычисления \GNUmake{}:

{\footnotesize
\begin{verbatim}
OUTPUT_DIR := /tmp

$(OUTPUT_DIR)/very_big_file:
    $(free-space)

define free-space
  $(PRINTF) "Free disk space "
  $(DF) . | $(AWK) 'NR == 2 { print $$4 }'
endef

BIN    := /usr/bin
PRINTF := $(BIN)/printf
DF     := $(BIN)/df
AWK    := $(BIN)/awk
\end{verbatim}
}

Заметим, что несмотря на то, что порядок строк кажется обратным,
выполнение происходит успешно. В этом заключается один из
замечательных эффектов рекурсивных переменных. Они могут быть
невероятно полезны и совершенно непонятны одновременно. Наш
\Makefile{} работает как нужно благодаря тому, что вычисление
командных сценариев и тел макросов откладываются до тех пор, пока
результаты этих вычислений не потребуются. Таким образом, порядок, в
котором появляются определения, не влияет на выполнение.

На второй фазе выполнения, когда \Makefile{} уже прочитан, \GNUmake{}
определяет цели, анализирует граф зависимостей и выполняет действия,
ассоциированные с каждым правилом. Поскольку мы специфицировали только
одну цель, не имеющую реквизитов
(\command{\$(OUTPUT\_DIR)/very\_big\_file}), \GNUmake{} просто выполнит
действия, с ассоциированные с ней (предположим, такой файл не
существует)~--- команду \command{\$(free-space)}. После вычислений
\GNUmake{} получит следующее:

{\footnotesize
\begin{verbatim}
/tmp/very_big_file:
    /usr/bin/printf "Free disk space "
    /usr/bin/df . | /usr/bin/awk 'NR == 2 { print $$4 }'
\end{verbatim}
}

Как только значения всех переменных вычислены, \GNUmake{} выполняет
команды одну за другой. Давайте рассмотрим две части \Makefile{}'а, в
которых порядок имеет значение. Как упоминалось ранее, имя цели
\command{\$(OUTPUT\_DIR)/very\_big\_file} вычисляется сразу. Если бы
определение переменной \variable{OUTPUT\_DIR} находилось после
спецификации правила, результатом вычисления имени стала бы строка
\command{/very\_big\_file}. Скорее всего, это не то, чего хотел
пользователь. Если бы определение \variable{BIN} было помещено после
определения \variable{AWK}, наши переменные получили бы значения
\command{/printf}, \command{/df} и \command{/awk}, так как оператор
\command{:=} вызывает немедленное вычисление правой части
присваивания. Однако в этом случае мы можем избежать проблемы,
использовав для определения переменных \variable{PRINTF},
\variable{DF} и \variable{AWK} оператор \command{=} вместо оператора
\command{:=} и сделав тем самым эти переменные рекурсивными.

Наконец, обратите внимание на одну деталь. Объявление переменных
\variable{OUTPUT\_DIR} и \variable{BIN} как рекурсивных не решило бы
рассмотренных проблем порядка спецификаций. Важно здесь то, что в
момент вычисления значения переменной
\command{\$(OUTPUT\_DIR)/very\_big\_file} и правых частей определений
\variable{PRINTF}, \variable{DF} и \variable{AWK} значения переменных,
на которые ссылаются эти выражения, должны быть уже определены.

%%--------------------------------------------------------------------
%% Target- and pattern-specific variables
%%--------------------------------------------------------------------
\section{Переменные, зависящие от цели или шаблона}

Обычно переменные принимают только одно значение во время выполнения
\GNUmake{}. Это гарантируется двухэтапной обработкой \Makefile{}'а.
На первом этапе \GNUmake{} читает \Makefile{}, производит присваивание
и вычисление переменных и строит граф зависимостей. На втором этапе
производится анализ графа зависимостей. Таким образом, когда происходит
выполнение команд, обработка переменных уже закончена. Предположим,
однако, что нам нужно переопределить значение переменной только для
какой-то цели или шаблона.

В приведённом ниже примере файл, который мы компилируем, требует
использования дополнительной опции \command{-DUSE\_NEW\_MALLOC=1},
которая не должна быть использована при компиляции других файлов:

{\footnotesize
\begin{verbatim}
gui.o: gui.h
    $(COMPILE.c) -DUSE_NEW_MALLOC=1 $(OUTPUT_OPTION) $<
\end{verbatim}
}

Проблема решена добавлением дубликата правила компиляции, включающего
необходимую опцию. Такой подход неудовлетворителен по нескольким
причинам. Во-первых, мы повторяем код. Если правило когда-нибудь
изменится, или если мы решим заменить встроенное правило собственным,
этот код потребует изменения, о котором легко можно забыть. Во-вторых,
если множество файлов требует специализированных опций, копирование и
вставка кусков кода быстро становится утомительным и рискованным
занятием (представьте, что у вас сотни таких файлов).

Для решения таких проблем \GNUmake{} предоставляет функциональность
\index{Переменные!зависящие от цели}
\newword{переменных, зависящих от цели}. Определения этих переменных
привязаны к цели и действительны только во время обработки цели или её
реквизитов. Используя эту функциональность, мы можем переписать наш
пример следующим образом:

{\footnotesize
\begin{verbatim}
gui.o: CPPFLAGS += -DUSE_NEW_MALLOC=1
gui.o: gui.h
$(COMPILE.c) $(OUTPUT_OPTION) $<
\end{verbatim}
}

Переменная \variable{CPPFLAGS} встроена в стандартное правило
компиляции исходных файлов \Clang{} и предназначена для опций
препроцессора \Clang{}. При помощи оператора \command{+=} мы добавляем
новую опцию к уже существующим. Теперь сценарий компиляции может
быть удалён полностью:

{\footnotesize
\begin{verbatim}
gui.o: CPPFLAGS += -DUSE_NEW_MALLOC=1
gui.o: gui.h
\end{verbatim}
}

Пока происходит сборка цели \target{gui.o}, значение переменной
\variable{CPPFLAGS} вдобавок к оригинальному значению будет содержать
строку \command{-DUSE\_NEW\_MALLOC=1}. Когда цель \filename{gui.o}
будет собрана, значение \variable{CPPFLAGS} будет восстановлено.

Общий синтаксис определения переменных, зависящих от цели, таков:

{\footnotesize
\begin{alltt}
\emph{цель ...: переменная}  = \emph{значение}
\emph{цель ...: переменная} := \emph{значение}
\emph{цель ...: переменная} += \emph{значение}
\emph{цель ...: переменная} ?= \emph{значение}
\end{alltt}
}

Как вы могли заметить, для определения таких переменных применимы все
формы оператора присваивания. Переменная не обязательно должна
существовать до присваивания.

Более того, присваивание переменным значения не осуществляется до
начала сборки цели. Таким образом, правая часть присваивания может
содержать ссылку на значение другой переменной, зависящей от этой
цели. Переменная также действительна во время сборки всех реквизитов.

%%--------------------------------------------------------------------
%% Where variables come from
%%--------------------------------------------------------------------
\section{Где определяются переменные}
\label{sec:where_vars_come_from}

Пока все переменные в наших \Makefile{}'ах определялись явно. На самом
деле они могут иметь и более сложное происхождение. Например, мы уже
видели, что переменные могут определяться через интерфейс командной
строки \GNUmake{}. На самом деле переменные могут определяться в
следующих источниках:

\begin{description}
%---------------------------------------------------------------------
\item[\emph{Файл}] \hfill \\
Естественно, переменные могут быть определены в \Makefile{}'е или в
файле, им подключенным.
%---------------------------------------------------------------------
\item[\emph{Командная строка}] \hfill \\
Переменные могут быть определены или переопределены прямо из командной
строки \GNUmake{}:

{\footnotesize
\begin{alltt}
\$ \textbf{make CFLAGS=-g CPPFLAGS='-DBSD -DDEBUG'}
\end{alltt}
}

Аргументы командной строки, содержащие символ \command{=}, являются
определениями переменных. Каждое такое определение должно быть
отдельным аргументом. Если значение переменной (или, упаси вас Боже,
её имя) содержит пробелы, нужно либо экранировать пробел, либо
заключить этот аргумент в кавычки.

Значение, полученное переменной через интерфейс командной строки,
переопределяет любое другое определение, сделанное в \Makefile{}'е или
содержащееся в окружении. Присваивания в командной строке могут
определять как простые, так и рекурсивные переменные с помощью
операторов \command{:=} и \command{=} соответственно. Однако всё же
существует возможность повысить приоритет определения переменной в
\index{Директивы!override@\directive{override}}
\Makefile{}'е, для этого нужно использовать директиву
\directive{override}.

{\footnotesize
\begin{verbatim}
# Для успешной компоновки необходим порядок байтов от старшего к
# младшим.
override LDFLAGS = -EB
\end{verbatim}
}

Разумеется, игнорировать явные определения пользователя стоит только в
чрезвычайных обстоятельствах.
%---------------------------------------------------------------------
\item[\emph{Окружение}] \hfill \\
\index{Переменные!окружения}
После старта \GNUmake{} все переменные окружения автоматически
становятся переменными \GNUmake{}. Эти переменные имеют очень низкий
приоритет, поэтому присваивание значения этим переменным, сделанное в
\Makefile{}'е или через командную строку, переопределит переменную
окружения. Вы можете форсировать переопределение переменных в
\Makefile{}'е переменными окружения при помощи опции
\index{Опции!environment-overrides@\command{-{}-environment\hyp{}overrides (-e)}}
\command{-{}-environment\hyp{}overrides} (или просто \command{-e}).

Когда \GNUmake{} вызывается рекурсивно, некоторые переменные из
родительского процесса \GNUmake{} передаются через окружение дочернему
процессу. По умолчанию только переменные, изначально определённые в
окружении, попадают в окружение дочернего процесса. Однако любая
переменная может быть помещена в окружение с помощью директивы
\index{Директивы!export@\directive{export}}
\directive{export}:

{\footnotesize
\begin{verbatim}
export CLASSPATH := \$(HOME)/classes:\$(PROJECT)/classes
SHELLOPTS = -x
export SHELLOPTS
\end{verbatim}
}

Вы также можете экспортировать все переменные:

{\footnotesize
\begin{verbatim}
export
\end{verbatim}
}

Обратите внимание на то, что \GNUmake{} может экспортировать даже те
переменные, имена которых содержат недопустимые с точки зрения
командного интерпретатора символы.  Например, результатом запуска
следующего \Makefile{}'а:

{\footnotesize
\begin{verbatim}
export valid-variable-in-make = Neat!
show-vars:
    env | grep '^valid-'
    valid_variable_in_shell=Great
    invalid-variable-in-shell=Sorry
\end{verbatim}
}

{\flushleft будет вывод:}

{\footnotesize
\begin{alltt}
\$ \textbf{make}
env | grep '\^{}valid-'
valid-variable-in-make=Neat!
valid\_variable\_in\_shell=Great
invalid-variable-in-shell=Sorry
/bin/sh: line 1: invalid-variable-in-shell=Sorry: command not found
make: *** [show-vars] Error 127
\end{alltt}
}

<<Недопустимая>> переменная командного интерпретатора была создана при
помощи экспорта из \GNUmake{} переменной
\variable{valid-variable-in-make}. Эта переменная не будет доступна
стандартными средствами интерпретатора, только при помощи трюков
наподобие применения программы \utility{grep} ко всему окружению. Тем
не менее, эта переменная будет доступна в дочернем процессе
\GNUmake{}. Мы рассмотрим применение рекурсивных вызовов \GNUmake{} во
второй части книги.

Вы также можете запретить экспорт переменной в окружение дочернего
процесса:

{\footnotesize
\begin{verbatim}
unexport DISPLAY
\end{verbatim}
}

\index{Директивы!export@\directive{export}}
\index{Директивы!unexport@\directive{unexport}} Директивы
\directive{export} и \directive{unexport} работают так же, как и
их аналоги из интерпретатора \utility{sh}.

\index{Операторы!условного присваивания}
Оператор условного присваивания очень удобен при работе с переменными
окружения. Допустим, вы определили в своём \Makefile{}'е стандартный
каталог для хранения создаваемых файлов и хотите предоставить
пользователю возможность легко его изменить. Условное присваивание
идеально подходит для таких ситуаций:

{\footnotesize
\begin{verbatim}
# Пусть каталогом по умолчанию будет $(PROJECT_DIR)/out.
OUTPUT_DIR ?= $(PROJECT_DIR)/out
\end{verbatim}
}

В этом случае присваивание произойдёт только в том случае, если
значение переменной \variable{OUTPUT\_DIR} ещё не определено. Мы можем
добиться того же эффекта более <<многословным>> способом:

{\footnotesize
\begin{verbatim}
ifndef OUTPUT_DIR
  # Пусть каталогом по умолчанию будет $(PROJECT_DIR)/out.
  OUTPUT_DIR = $(PROJECT_DIR)/out
endif
\end{verbatim}
}

Разница между этими двумя определениями заключается в том, что
оператор условного присваивания не будет выполнен, если переменная
\variable{OUTPUT\_DIR} определена, пусть даже имеет пустое значение,
\index{Директивы!условной обработки!ifdef@\directive{ifdef}}
тогда как операторы \directive{ifdef} и \directive{ifndef} проверяют
свой аргумент на пустоту. Таким образом, выражение
\command{OUTPUT\_DIR=} с точки зрения оператора условного присваивания
будет считаться определением, а с точки зрения оператора
\directive{ifdef}~--- нет.

Следует отметить, что чрезмерное использование переменных окружения
делает ваши \Makefile{}'ы менее переносимыми, поскольку разные
пользователи, как правило, имеют разное окружение. На самом деле, я
редко использую переменные окружения именно по этой причине.
%---------------------------------------------------------------------
\item[\emph{Автоматические переменные}] \hfill \\
\index{Переменные!автоматические}
Наконец, \GNUmake{} создаёт автоматические переменные непосредственно
перед выполнением сценария, ассоциированного с правилом.
\end{description}

Традиционно переменные окружения используются для упрощения управления
разницей в настройке машин разработчиков. Например, общей практикой
является создание среды разработки (инструментов разработки, исходного
кода и дерева полученных компиляцией файлов) на основании переменных
окружения, используемых в \Makefile{}'е. При таком подходе \Makefile{}
определяет одну переменную для корня каждого дерева каталогов. Если
корень дерева исходных файлов содержится в переменной
\variable{PROJECT\_SRC}, корень дерева объектных файлов~--- в
переменной \variable{PROJECT\_BIN}, а каталог библиотек~--- в
переменной \variable{PROJECT\_LIB}, то разработчик вправе определить
эти каталоги так, как он считает нужным.

Потенциальной проблемой такого подхода (и вообще использования
переменных окружения) является ситуация, когда упомянутые выше
переменные не определены. Одним из решений является определение
значений по умолчанию в \Makefile{}'е с помощью оператора условного
присваивания:

{\footnotesize
\begin{verbatim}
PROJECT_SRC ?= /dev/$(USER)/src
PROJECT_BIN ?= $(patsubst %/src,%/bin,$(PROJECT_SRC))
PROJECT_LIB ?= /net/server/project/lib
\end{verbatim}
}

Используя эти переменные для доступа к компонентам проекта, вы можете
создать среду разработки, адаптируемую под различные настройки машин
разработчиков (мы увидим более развёрнутые примеры во второй части
книги). Однако опасайтесь чрезмерного использования переменных
окружения. Как правило, \Makefile{} следует составлять так, чтобы как
можно меньше использовать окружение разработчика, предоставлять
разумные настройки по умолчанию и проверять наличие критически важных
компонентов.

%%--------------------------------------------------------------------
%% Conditional and include processing
%%--------------------------------------------------------------------
\section{Условная обработка и включения}
\label{sec:cond_inc_processing}
Части \Makefile{}'а могут быть опущены или выбраны для обработки во
время его чтения с помощью директив \newword{условной обработки}.
\index{Директивы!условной обработки}
Условия, контролирующие обработку, могут принимать несколько форм,
таких как <<A определено>> или <<A равно B>>. Например:

{\footnotesize
\begin{verbatim}
# переменная COMSPEC определена только в Windows.
ifdef COMSPEC
  PATH_SEP := ;
  EXE_EXT  := .exe
else
  PATH_SEP := :
  EXE_EXT  :=
endif
\end{verbatim}
}

В предыдущем примере обработается первая ветвь условия только в том
случае, если переменная \variable{COMSPEC} определена. Синтаксис
директив условной обработки имеет две формы:

{\footnotesize
\begin{alltt}
\emph{if-условие}
  текст для обработки если условие выполнено
endif
\end{alltt}
}

{\flushleft и:}

{\footnotesize
\begin{alltt}
\emph{if-условие}
  текст для обработки если условие выполнено
else
  текст для обработки если условие не выполнено
endif
\end{alltt}
}

Значения шаблона \ItalicMono{if-условие} могут быть следующими:

{\footnotesize
\begin{alltt}
ifdef  \emph{имя-переменной}
ifndef \emph{имя-переменной}
ifeq  \emph{тест}
ifneq \emph{тест}
\end{alltt}
}

При использовании директив \directive{ifdef\textbackslash{}ifndef}
\index{Директивы!условной обработки!ifdef@\directive{ifdef}}
\index{Директивы!условной обработки!ifndef@\directive{ifndef}}
\ItalicMono{имя-переменной} не нужно заключать в скобки
(\command{\$( )}). Наконец, значением шаблона \ItalicMono{тест}
может быть одно из следующих выражений:

{\footnotesize
\begin{alltt}
"\emph{a}" "\emph{b}"
(\emph{a},\emph{b})
\end{alltt}
}

В выражениях могут использоваться двойные или одинарные кавычки по
желанию (однако тип открывающейся и закрывающейся кавычки должен
совпадать).

Директивы условной обработки могут быть использованы внутри тела
макросов или в командных сценариях:

{\footnotesize
\begin{verbatim}
libGui.a: $(gui_objects)
    $(AR) $(ARFLAGS) $@ $<
  ifdef RANLIB
    $(RANLIB) $@
  endif
\end{verbatim}
}

Я предпочитаю делать отступ перед директивами условной обработки,
однако неосторожное выравнивание может привести к ошибкам. В
предыдущем примере перед директивами сделан отступ в два пробела, а
заключённые в них команды предваряются символом табуляции. \GNUmake{}
не может распознать команды, не начинающиеся с символа табуляции. Если
директива условной обработки предваряется символом табуляции, она
рассматривается как команда и передаётся в командный интерпретатор.

\index{Директивы!условной обработки!ifeq@\directive{ifeq}}
\index{Директивы!условной обработки!ifneq@\directive{ifneq}}
Директивы \directive{ifeq} и \directive{ifneq} проверяют свои
аргументы на на равенство и неравенство соответственно. Пробелы в
условиях директив могут быть причиной трудноуловимых ошибок. Например,
когда используется форма теста с круглыми скобками, пробел после
запятой не учитывается, тогда как все остальные пробелы имеют
значение:

{\footnotesize
\begin{verbatim}
ifeq (a, a)
  # Равенство
endif

ifeq ( b, b )
  # Неравенство - ` b' != `b '
endif
\end{verbatim}
}

Лично я предпочитаю форму с кавычками:

{\footnotesize
\begin{verbatim}
ifeq "a" "a"
  # Равенство
endif

ifeq 'b' 'b'
  # Тоже равенство
endif
\end{verbatim}
}

Однако иногда случается так, что значение переменной содержит пробел в
начале или в конце. Это может быть источником ошибки, поскольку
сравнение учитывает все символы. Для создания более надёжных
\index{Функции!встроенные!strip@\function{strip}}
\Makefile{}'ов используйте функцию \function{strip}:

{\footnotesize
\begin{verbatim}
ifeq "$(strip $(OPTIONS))" "-d"
  COMPILATION_FLAGS += -DDEBUG
endif
\end{verbatim}
}

%---------------------------------------------------------------------
% include directive
%---------------------------------------------------------------------
\subsection{Директива \directive{include}}
\label{sec:include_directive}
\index{Директивы!include@\directive{include}}
Мы уже встречали директиву \directive{include} в
разделе~<<\nameref{sec:auto_dep_gen}>> главы~\ref{chap:rules}. Теперь
давайте рассмотрим её более детально.

Любой \Makefile{} может включать другие файлы. Наиболее частое
использование этой возможности~--- помещение общих определений
\GNUmake{} в заголовочный файл и включение автоматически составленных
файлов зависимостей. Директива \directive{include} используется
следующим образом:

\begin{alltt}
\footnotesize
include definitions.mk
\end{alltt}

Параметры директивы могут содержать произвольное число файлов, шаблоны
командного интерпретатора и переменные \GNUmake{}.

%---------------------------------------------------------------------
% include and dependencies
%---------------------------------------------------------------------
\subsection{Директива \directive{include} в контексте зависимостей}
Когда \GNUmake{} встречает директиву \directive{include}, он
производит раскрытие шаблонов и подстановку значений переменных, а
затем пытается прочитать подключенные файлы. Если файл существует,
выполнение продолжается. Если же указанный файл не существует,
\GNUmake{} выводит предупреждение и продолжает читать остаток
\Makefile{}'а. Когда весь \Makefile{} прочитан, \GNUmake{}
просматривает базу данных в поисках правила сборки подключаемых
файлов. Если соответствие найдено, \GNUmake{} следует найденному
правилу сборки цели. Если хотя бы один из включаемых файлов был
собран, \GNUmake{} очищает свою базу данных и производит повторное
чтение всего \Makefile{}'а. Если после завершения всего процесса
чтения, сборки и повторного чтения какая-то из директив
\directive{include} завершается неудачей ввиду отсутствующих файлов,
\GNUmake{} завершает своё выполнение с ненулевым кодом возврата.

Мы можем увидеть этот процесс в действии при помощи следующего
примера, состоящего из двух файлов. Мы используем встроенную функцию
\index{Функции!встроенные!warning@\function{warning}}
\function{warning}, для вывода сообщений из \GNUmake{} (эта и многие
другие функции рассмотрены в главе~\ref{chap:functions}). Вот
\Makefile{}:

{\footnotesize
\begin{alltt}
# Простой makefile, включающий файл.
include foo.mk
\$(warning Finished include)

foo.mk: bar.mk
    m4 --define=FILENAME=\${}@ bar.mk > \${}@
\end{alltt}
}

Ниже представлен \filename{bar.mk}, подключаемый в \Makefile{}:

{\footnotesize
\begin{alltt}
# bar.mk - выдать сообщение о чтении файла.
\$(warning Reading FILENAME)
\end{alltt}
}

После запуска \GNUmake{} мы увидим следующий вывод:

{\footnotesize
\begin{alltt}
\$ \textbf{make}
Makefile:2: foo.mk: No such file or directory
Makefile:3: Finished include
m4 --define=FILENAME=foo.mk bar.mk > foo.mk
foo.mk:2: Reading foo.mk
Makefile:3: Finished include
make: `foo.mk' is up to date.
\end{alltt}
}

Первая строка отражает тот факт, что \GNUmake{} не смог найти
включаемый файл, однако, вторая строка показывает, что чтение и
выполнение \Makefile{}'а продолжилось. По завершении чтения \GNUmake{}
обнаружил правило для  создания подключаемого файла,
\filename{foo.mk}, и выполнил соответствующий сценарий. Затем
\GNUmake{} начал весь процесс заново, в этот раз не встретив никаких
трудностей с чтением включаемого файла.

Теперь самое время заметить, что \GNUmake{} интерпретирует \Makefile{}
как потенциальную цель. После того, как \Makefile{} прочитан,
\GNUmake{} ищет правило для сборки текущего \Makefile{}'а.  Если такое
правило находится, \GNUmake{} выполняет соответствующий правилу
сценарий и проверяет, изменился ли текущий \Makefile{}. Если
\Makefile{} изменился, \GNUmake{} производит очистку своего состояния
и считывает \Makefile{} заново, повторяя анализ. Ниже приведён простой
пример бесконечного цикла, основанный на описанном поведении:

{\footnotesize
\begin{verbatim}
.PHONY: dummy

makefile: dummy
    touch $@
\end{verbatim}
}

Когда \GNUmake{} выполняет \Makefile{}, он видит, что файл нуждается в
сборке (поскольку цель \target{dummy} является абстрактной) и
выполняет команду \utility{touch}, которая изменяет время последней
модификации \Makefile{}'а. Затем \GNUmake{} читает файл и
обнаруживает, что он требует обновления\ldots{}. В общем, вы поняли.

Где \GNUmake{} ищет включаемые файлы? Если аргумент директивы
\directive{include} является абсолютным путём, то \GNUmake{} открывает
файл по указанному пути. Если указан относительный путь, \GNUmake{}
ищет файл относительно текущего рабочего каталога. Если файл не
найден, то осуществляется поиск в каталогах, указанных при помощи опции
\index{Опции!include-dir@\command{-{}-include-dir (-I)}}
\command{-{}-inc\-lu\-de\hyp{}dir} (или просто \command{-I}). Если
файл не найден и там, производится поиск в стандартных каталогах,
указанных при компиляции \GNUmake{}: \filename{/usr/local/include},
\filename{/usr/gnu/include}, \filename{/usr/include}. Пути могут
отличаться, поскольку они зависят от опций компиляции \GNUmake{}.

Если \GNUmake{} не может найти файл и не может собрать его с помощью
правила, происходит выход с ненулевым кодом возврата. Если вы хотите,
чтобы \GNUmake{} игнорировал включение несуществующих файлов, добавьте
знак дефиса перед директивой \directive{include}:

\begin{alltt}
\footnotesize
-include i-may-not-exist.mk
\end{alltt}

Для обратной совместимости с другими версиями \GNUmake{} слово
\index{Директивы!sinclude@\directive{sinclude}}
\index{Директивы!sinclude@\directive{-include}}
\directive{sinclude} является синонимом \directive{-include}.

%%--------------------------------------------------------------------
%% Standard make variables
%%--------------------------------------------------------------------
\section{Стандартные переменные \GNUmake{}}
\label{sec:std_make_vars}

\index{Переменные!стандартные}
Вдобавок к автоматическим переменным, \GNUmake{} содержит переменные,
предназначенные для получения текущего состояния \GNUmake{} и
настройки встроенных правил:

\begin{description}
%---------------------------------------------------------------------
% MAKE_VERSION
%---------------------------------------------------------------------
\item[\variable{MAKE\_VERSION}] \hfill \\
\index{Переменные!стандартные!MAKE\_VERSION@\variable{MAKE\_VERSION}}
Значением этой переменной является номер текущей версии GNU
\GNUmake{}.  Во время написания этой книги этим значением было
\command{3.80}, а CVS хранилище содержало версию \texttt{3.81rc1.}

Предыдущая версия \GNUmake{}, \command{3.79.1}, не поддерживающая
функций \function{eval} и \function{value}, имеет довольно широкое
распространение. Так что когда пишу свои \Makefile{}'ы, требующие
подобной функциональности, я использую эту переменную для проверки
версии \GNUmake{}. Мы увидим примеры такого использования в разделе
<<\nameref{sec:flow_ctrl_func}>> главы~\ref{chap:functions}.

%---------------------------------------------------------------------
% CURDIR
%---------------------------------------------------------------------
\item[\variable{CURDIR}] \hfill \\
\index{Переменные!стандартные!CURDIR@\variable{CURDIR}}
\index{Текущий каталог}
\index{current working directory}
Эта переменная содержит текущий рабочий каталог (current working
directory, cwd) исполняемого процесса \GNUmake{}. Это тот же самый
каталог, в котором была выполнена команда \GNUmake{} (и тот же самый
каталог, что хранится в переменной командного интерпретатора
\variable{PWD}), если только не использовалась опция
\index{Опции!directory@\command{-{}-directory (-C)}}
\command{-{}-di\-rec\-to\-ry} (\command{-C}). Опция
\command{-{}-di\-rec\-to\-ry} сообщает \GNUmake{}, что нужно изменить
текущий каталог перед тем, как начать искать \Makefile{}. Полная форма
опции выглядит следующим образом:
\command{-{}-di\-rec\-to\-ry=\emph{имя\hyp{}каталога}} или
\command{-C \emph{имя\hyp{}каталога}}. Если использовалась опция
\command{-{}-di\-rec\-to\-ry}, то переменная \variable{CURDIR} будет
содержать аргумент этой опции.

Обычно я вызываю \GNUmake{} из редактора \utility{emacs} во время
редактирования кода. Например, мой текущий проект написан на \Java{} и
использует единственный \Makefile{} в корневом каталоге проекта (не
обязательно в каталоге, содержащем исходный код). В моём случае
использование опции \command{-{}-di\-rec\-to\-ry} позволяет мне
запускать \GNUmake{} из любого каталога с исходными файлами и иметь
доступ к \Makefile{}'у. Внутри \Makefile{}'а все пути являются
относительными и вычисляются от каталога, в котором располагается
\Makefile{}. Абсолютные пути используются очень редко, доступ к ним
осуществляется при помощи переменной \variable{CURDIR}.

%---------------------------------------------------------------------
% MAKEFILE_LIST
%---------------------------------------------------------------------
\item[\variable{MAKEFILE\_LIST}] \hfill \\
\index{Переменные!стандартные!MAKEFEILE\_LIST@\variable{MAKEFEILE\_LIST}}
Значение этой переменной содержит список всех файлов, прочитанных
\GNUmake{}, включая стандартный \Makefile{}, все \Makefile{}'ы,
указанные в опциях командной строки, и файлы, подключённые директивой
\directive{include}. Перед чтением каждого файла \GNUmake{} добавляет
его имя к значению переменной \variable{MAKEFILE\_LIST}. Итак,
\Makefile{} всегда может определить собственное имя, прочитав
последнее слово в этом списке.

%---------------------------------------------------------------------
% MAKECMDGOALS
%---------------------------------------------------------------------
\item[\variable{MAKECMDGOALS}] \hfill \\
\index{Переменные!стандартные!MAKECMDGOALS@\variable{MAKECMDGOALS}}
Значение \variable{MAKECMDGOALS} содержит список всех целей, указанных
в командной строке для текущего процесса \GNUmake{}. Оно не включает
опций и определений переменных. Например:

{\footnotesize
\begin{alltt}
\$ \textbf{make -f- FOO=bar -k goal <<< 'goal:;\#{} \$(MAKECMDGOALS)'}
\#{} goal
\end{alltt}
}

Предыдущий пример использует возможность \GNUmake{} читать \Makefile{}
из стандартного потока ввода, включающуюся опцией \command{-f-}
\index{Опции!file@\command{-{}-file (-f-)}} (или
\command{-{}-fi\-le}). Происходит перенаправление стандартного потока
ввода на командную строку с помощью синтаксиса командного
интерпретатора \utility{bash} \verb|<<<|\footnote{ Если вы хотите
выполнить этот пример в другом интерпретаторе, наберите:

\begin{alltt}
\$ echo 'goal:;\#{} \$(MAKECMDGOALS)' | make -f- FOO=bar -k goal
\end{alltt}
}.

Сам \Makefile{} состоит из спецификации цели по умолчанию
\target{goal}. Командный сценарий указан на той же строке и
отделяется от определения цели точкой с запятой. Сценарий состоит из
одной строки:

{\footnotesize
\begin{verbatim}
# $(MAKECMDGOALS)
\end{verbatim}
}

Переменная \variable{MAKECMDGOALS} обычно используется в том случае,
когда цель требует особой обработки. Выразительным примером является
цель \target{clean}. Когда происходит сборка \target{clean},
\GNUmake{} не должен осуществлять стандартную проверку изменения
подключаемых файлов (обсуждавшуюся в разделе
<<\nameref{sec:cond_inc_processing}>> главы~\ref{chap:vars}).  Для
подавления этой проверки можно применить директиву \directive{ifneq} и
переменную \variable{MAKECMDGOALS}:

{\footnotesize
\begin{verbatim}
ifneq "$(MAKECMDGOALS)" "clean"
  -include $(subst .xml,.d,$(xml_src))
endif
\end{verbatim}
}

\item[\variable{.VARIABLES}] \hfill \\
\index{Переменные!стандартные!VARIABLES@\variable{.VARIABLES}}
Значение этой переменной содержит имена всех переменных, определённых
\index{Переменные!зависящие от цели}.
в \Makefile{}'е на данный момент, исключая переменные, зависящие от
цели. Эта переменная доступна только для чтения и любое её
переопределение игнорируется.

{\footnotesize
\begin{verbatim}
list:
    @echo "$(.VARIABLES)" | tr ' ' '\015' | grep MAKEF
\end{verbatim}
}

{\footnotesize
\begin{alltt}
\$ \textbf{make}
MAKEFLAGS
MAKEFILE\_LIST
MAKEFILES
\end{alltt}
}
\end{description}

Как вы уже видели, переменные также используются для настройки
встроенных неявных правил \GNUmake{}. Правила для компиляции и
компоновки исходных файлов \Clang{}/\Cplusplus{} демонстрируют
типичную форму, принимаемую переменными для произвольных языков
программирования.  Рисунок~\ref{fig:vars_for_c_comp} изображает
переменные, контролирующие трансформацию одного типа файлов в другой.

\begin{figure}[!ht]
\centering
\begin{picture}(320,220)
\tiny

\put(218,210){\texttt{.y}}
\put(250,210){\texttt{.l}}

\put(222,207){\vector(0,-1){26}}
\put(254,207){\vector(0,-1){26}}

\put(212,175){\texttt{YACC.y}}
\put(246,175){\texttt{LEX.l}}

\put(222,172){\vector(1,-2){14}}
\put(254,172){\vector(-1,-2){14}}

\put(14,140){\texttt{.S}}
\put(58,140){\texttt{.s}}
\put(108,140){\texttt{.S}}
\put(142,140){\texttt{.cpp}}
\put(186,140){\texttt{.C}}
\put(234,140){\texttt{.c}}
\put(272,140){\texttt{.y}}
\put(304,140){\texttt{.l}}

\put(18,137){\vector(0,-1){26}}
\put(62,137){\vector(0,-1){26}}
\put(112,137){\vector(0,-1){26}}
\put(148,137){\vector(0,-1){26}}
\put(190,137){\vector(0,-1){26}}
\put(238,137){\vector(0,-1){26}}
\put(276,137){\vector(0,-1){26}}
\put(308,137){\vector(0,-1){26}}

\put(0,105){\texttt{PREPROCESS.S}}
\put(50,105){\texttt{COMPILE.s}}
\put(98,105){\texttt{COMPILE.S}}
\put(134,105){\texttt{COMPILE.cpp}}
\put(176,105){\texttt{COMPILE.C}}
\put(224,105){\texttt{COMPILE.c}}
\put(266,105){\texttt{YACC.y}}
\put(300,105){\texttt{LEX.l}}

\put(18,102){\vector(0,-1){26}}
\put(62,102){\vector(3,-1){70}}
\put(112,102){\vector(1,-1){26}}
\put(148,102){\vector(0,-1){26}}
\put(184,102){\vector(-1,-1){26}}
\put(238,102){\vector(-3,-1){74}}
\put(276,102){\vector(1,-2){14}}
\put(308,102){\vector(-1,-2){14}}

\put(14,70){\texttt{.s}}
\put(58,70){\texttt{.S}}
\put(144,70){\texttt{.o}}
\put(184,70){\texttt{.cpp}}
\put(234,70){\texttt{.C}}
\put(288,70){\texttt{.c}}

\put(18,67){\vector(0,-1){26}}
\put(62,67){\vector(0,-1){26}}
\put(148,67){\vector(0,-1){26}}
\put(190,67){\vector(0,-1){26}}
\put(238,67){\vector(0,-1){26}}
\put(292,67){\vector(0,-1){26}}

\put(8,35){\texttt{LINK.s}}
\put(54,35){\texttt{LINK.S}}
\put(140,35){\texttt{LINK.o}}
\put(180,35){\texttt{LINK.cpp}}
\put(229,35){\texttt{LINK.C}}
\put(285,35){\texttt{LINK.c}}

\put(18,33){\vector(4,-1){110}}
\put(62,33){\vector(3,-1){78}}
\put(148,33){\vector(0,-1){26}}
\put(190,33){\vector(-3,-2){38}}
\put(238,33){\vector(-3,-1){78}}
\put(292,33){\vector(-4,-1){120}}

\put(134,0){\texttt{Executable}}
\end{picture}

\caption{Переменные для компиляции исходных файлов
\Clang{}/\Cplusplus{}.} \label{fig:vars_for_c_comp}
\end{figure}

Все эти переменные имеют одинаковую форму:
\ItalicMono{ДЕЙСТВИЕ.суффикс}. \ItalicMono{ДЕЙСТВИЕ} может быть
\variable{COMPILE} для создания объектного файла, \variable{LINK} для
создания исполняемого файла, или одной из <<специальных>> операций
\variable{PREPROCESS}, \variable{YACC} или \variable{LEX} (для запуска
препроцессора языка \Clang{}, \utility{yacc} и \utility{lex}
соответственно). Тип обрабатываемых файлов указывается с помощью части
\ItalicMono{суффикс}.

Стандартный <<путь>> через эти переменные, например, для \Cplusplus{},
проходит через два правила. Сначала происходит компиляция исходных
файлов \Cplusplus{} в объектные файлы, которые затем компонуются в
исполняемый файл.

{\footnotesize
\begin{verbatim}
%.o: %.C
    $(COMPILE.C) $(OUTPUT_OPTION) $<

%: %.o
    $(LINK.o) $\ $(LOADLIBES) $(LDLIBS) -o $@
\end{verbatim}
}

Первое правило использует следующие определения переменных:

{\footnotesize
\begin{verbatim}
COMPILE.C     = $(COMPILE.cc)
COMPILE.cc    = $(CXX) $(CXXFLAGS) $(CPPFLAGS) $(TARGET_ARCH) -c
CXX           = g++
OUTPUT_OPTION = -o $@
\end{verbatim}
}

GNU \GNUmake{} поддерживает два расширения исходных файлов
\Cplusplus{}: \filename{.C} и \filename{.cc}. Значением переменной
\variable{CXX} является имя компилятора \Cplusplus{}, по умолчанию это
\index{Переменные!стандартные!CXXFLAGS@\variable{CXXFLAGS}}
\index{Переменные!стандартные!CPPFLAGS@\variable{CPPFLAGS}}
\index{Переменные!стандартные!TARGETARCH@\variable{TARGET\_ARCH}}
\utility{g++}. Переменные \variable{CXXFLAGS}, \variable{CPPFLAGS} и
\variable{TARGET\_ARCH} (опции компилятора \Cplusplus{}, опции
препроцессора \Clang{} и опции, специфичные для архитектуры,
соответственно) не имеют значения по умолчанию. Они нужны для
настройки процесса сборки конечным пользователем. Переменная
\index{Переменные!стандартные!OUTPUT\_OPTION@\variable{OUTPUT\_OPTION}}
\variable{OUTPUT\_OPTION} содержит имя выходного файла.

Правило компоновки немного проще:

{\footnotesize
\begin{verbatim}
LINK.o = $(CC) $(LDFLAGS) $(TARGET_ARCH)
CC     = gcc
\end{verbatim}
}

\index{Переменные!стандартные!LDFLAGS@\variable{LDFLAGS}}
\index{Переменные!стандартные!TARGET\_ARCH@\variable{TARGET\_ARCH}}
\index{Переменные!стандартные!LDLIBS@\variable{LDLIBS}}
\index{Переменные!стандартные!LOADLIBES@\variable{LOADLIBES}}
Это правило использует компилятор языка \Clang{} для компоновки
объектных файлов в исполняемые. Компилятором языка \Clang{} по
умолчанию является \utility{gcc}. Переменные \variable{LDFLAGS} и
\variable{TARGET\_ARCH} не имеют значений по умолчанию. Значение
\variable{LDFLAGS} содержит опции компоновки, например, флаги
\command{-L}. Переменные \variable{LOADLIBES} и \variable{LDLIBS}
содержат список библиотек для компоновки. Такая избыточность была
введена из соображений переносимости.

Итак, мы завершили наш краткий обзор переменных \GNUmake{}. На самом
деле их гораздо больше, однако того, что мы рассмотрели, достаточно
для того, чтобы понять связь переменных и правил. Например, существует
группа переменных, предназначенных для работы с \TeX{}, и набор
соответствующих правил. Рекурсивный вызов \GNUmake{}~--- ещё одна
тема, для обсуждения которой нам понадобятся переменные. Мы вернёмся к
ней в главе~\ref{chap:managing_large_proj}.

