%%%%------------------------------------------------------------------
%%%% Advanced and specialized topics
%%%%------------------------------------------------------------------
\part{Специализированные вопросы}
\label{part:advanced_topics}

Во второй части мы получим проблемно-ориентированный взгляд на
\GNUmake{}. Далеко не всегда бывает очевидно, как применить \GNUmake{}
к проблемам реального мира, таким как сборки в нескольких каталогах,
новые языки программирования, переносимость, проблемы
производительности и отладка. В этой части обсуждается каждая из этих
проблем, в добавок здесь вы найдёте главу, в которой рассматривается
несколько сложных примеров.

%%%-------------------------------------------------------------------
%%% Managing large projects
%%%-------------------------------------------------------------------
\chapter{Управление большими проектами}
\label{chap:managing_large_proj}

Какой проект можно назвать большим? Для наших целей мы назовём большим
проект, требующий команды разработчиков, поддерживающий несколько
архитектур, предполагающий несколько релизов и нуждающийся в
поддержке. Конечно, проект не должен иметь все эти признаки, чтобы
называться большим. Миллион строк кода на \Cplusplus{} в
предварительном релизе, предназначенных для одной платформы~--- это
тоже большой проект. Однако программное обеспечение редко остаётся в
стадии предварительного релиза навечно. Если оно успешно, в конечном
итоге кто-то попросит перенести его на другую платформу. Поэтому все
крупные системы программного обеспечения на определённом этапе
становятся похожими.

Большие проекты обычно упрощаются при помощи декомпозиции на отдельные
компоненты, которые обычно собираются в самостоятельные программы или
библиотеки (или и то, и другое). Эти компоненты часто хранятся в
отдельных каталогах файловой системы и управляются при помощи
собственных \Makefile{}'ов. Один из способов сборки всей системы
компонентов подразумевает наличие главного \Makefile{}'а,
вызывающего \Makefile{}'ы компонентов в нужном порядке. Этот подход
называется \newword{рекурсивный \GNUmake{}} (\newword{recursive
  make}), потому что главный \Makefile{} вызывает \GNUmake{}
рекурсивно для обработки \Makefile{}'а каждого компонента. Рекурсивный
\GNUmake{}~--- это общая техника для компонентных
сборок. Альтернатива, предложенная Питером Миллером (Peter Miller) в
1988 году, лишена многих недостатков, присущих рекурсивному
\GNUmake{}, и основана на использовании единственного \Makefile{}'а,
включающего информацию из каталогов компонентов\footnote{
Miller, P.A., \emph{Recursive Make Considered Harmful}, AUUGN Journal
of AUUG Inc., 19(1), pp. 14-25(1998). Эта статья также доступна по
адресу \filename{\url{http://aegis.sourceforge.net/auug97.pdf}} (прим.
автора).}.

Как только проект проходит этап сборки компонентов, обычно на его пути
встают более серьёзные организационные проблемы управления
сборками, включающие управление разработкой нескольких версий проекта,
поддержку нескольких платформ, предоставление эффективного доступа к
исходному коду и исполняемым файлам и осуществление автоматических
сборок. Мы обсудим эти проблемы во второй части этой главы.

%%--------------------------------------------------------------------
%% Recursive make
%%--------------------------------------------------------------------
\section{Рекурсивный \GNUmake{}}
\label{sec:recursive_make}

\index{Рекурсивный \GNUmake{}} Мотивация использования рекурсивного
\GNUmake{} довольно проста: \GNUmake{} отлично работает в рамках
одного каталога (или небольшого набора каталогов), однако его
использование заметно усложняется с ростом числа каталогов. Таким
образом, мы можем использовать \GNUmake{} для сборки большого проекта,
написав для каждого каталога свой простой самодостаточный \Makefile{},
а затем выполнив полученные \Makefile{}'ы по очереди. Мы могли бы
использовать для этой цели сценарий, однако наиболее эффективным
подходом является использование \GNUmake{}, поскольку между
компонентами обычно существуют зависимости более высокого уровня.

Предположим, что мы ведём разработку приложения для воспроизведения
mp3 файлов. Логически его можно разделить на несколько компонентов:
пользовательский интерфейс, кодеки и система управления базой
данных. Эти компоненты могут быть представлены трёмя библиотеками:
\filename{libui.a}, \filename{libcodec.a} и \filename{libdb.a}. Само
приложение является <<клеем>>, связывающим эти три части
воедино. Наиболее простое отображение этих компонентов в структуру
каталогов представлено на рисунке~\ref{fig:file_layout_mp3}.

\begin{figure}
{\footnotesize
\begin{verbatim}
.
|
|--makefile
|
|--include
|  |--db
|  |--codec
|  `--ui
|
|--lib
|  |--db
|  |--codec
|  `--ui
|
|--app
|  `--player
|
`--doc
\end{verbatim}
}
\caption{Структура каталогов проекта mp3 плеера}
\label{fig:file_layout_mp3}
\end{figure}

Более традиционная структура каталогов подразумевает помещение функции
main и <<клея>> в корневой каталог, а не в подкаталог
\filename{app/player}. Я предпочитаю помещать код приложения в
собсвенный каталог, поскольку это делает структуру корневого каталога
более ясной и позволяет легко добавлять в систему новые
модули. Например, если мы решим добавить отдельное приложение для
управления музыкальными каталогами, мы можем аккуратно поместить его в
\filename{app/catalog}.

Если каждый каталог из \filename{lib/db}, \filename{lib/codec},
\filename{lib/ui} и \filename{app/player} содержит собственный
\Makefile{}, то работой головного \Makefile{}'а станет их
последовательный вызов:

{\footnotesize
\begin{verbatim}
lib_codec := lib/codec
lib_db := lib/db
lib_ui := lib/ui
libraries := $(lib_ui) $(lib_db) $(lib_codec)
player := app/player

.PHONY: all $(player) $(libraries)
all: $(player)

$(player) $(libraries):
    $(MAKE) --directory=$@

$(player): $(libraries)
$(lib_ui): $(lib_db) $(lib_codec)
\end{verbatim}
}

Главный \Makefile{} вызывает \GNUmake{} в каждом подкаталоге в рамках
правила, перечисляющего все подкаталоги как цели и выполняющего вызов
\GNUmake{}:

{\footnotesize
\begin{verbatim}
$(player) $(libraries):
    $(MAKE) --directory=$@
\end{verbatim}
}

Переменная \variable{MAKE} должна использоваться всегда при вызове
\GNUmake{} из \Makefile{}'а. \GNUmake{} распознаёт эту переменную и
подставляет на её место реальный путь к исполняемому файлу \GNUmake{},
чтобы все рекурсивные вызовы \GNUmake{} использовали один исполняемый
файл. К тому же, строки, содержащие переменную \variable{MAKE},
обрабатываются особым образом, если используются опции \command{-{}-touch}
(\command{-t}), \command{-{}-just\hyp{}print} (\command{-n}) или
\command{-{}-question} (\command{-q}). Мы обсудим эти детали в
разделе <<\nameref{sec:command_line_options}>> далее в этой главе.

Целевые каталоги помечены как \variable{.PHONY}, поэтому правила
выполняются даже в том случае, когда пересборка целей не
\index{Опции!directory@\command{-{}-directory (-C)}}
требуется. Опция \command{-{}-di\-rec\-to\-ry} (\command{-C})
используется для того, чтобы заставить \GNUmake{} поменять текущий
каталог перед чтением \Makefile{}'а.

Это правило, также довольно тонкое, помогает избежать нескольких
проблем, возникающих при использовании более <<очевидного>> командного
сценария:

{\footnotesize
\begin{verbatim}
all:
    for d in $(player) $(libraries); \
    do                               \ 
        $(MAKE) --directory=$$d;     \
    done
\end{verbatim}
}

Этот командный сценарий не сможет правильно передать ошибки
родительскому \GNUmake{}. Он также не позволит \GNUmake{} выполнять
сборки в разных подкаталогах параллельно. Мы обсудим эту возможность
\GNUmake{} в главе~\ref{chap:improving_the_performance}.

Когда \GNUmake{} планирует выполнение графа зависимостей, реквизиты
цели выглядят для него независимыми. В добавок к этому, две
цели, не связанные зависимостью от третьей цели, также
независимы. Например, библиотеки не  имеют непосредственной
зависимости от цели \target{app/player} или друг от друга. Это
позволяет \GNUmake{} выполнять \Makefile{} для \filename{app/player}
перед сборкой библиотек. Естественно, это вызовет ошибку сборки, так
как компоновка приложения требует наличия библиотек. Для решения этой
проблемы мы добавили дополнительную информацию о зависимостях:

{\footnotesize
\begin{verbatim}
$(player): $(libraries)
$(lib_ui): $(lib_db) $(lib_codec)
\end{verbatim}
}

Этот отрывок сообщает \GNUmake{}, что \Makefile{}'ы в подкаталогах
библиотек должны быть выполнены раньше \Makefile{}'а в каталоге
\filename{player}. Точно так же код библиотеки \filename{lib/ui}
требует, чтобы библиотеки \filename{lib/db} и \filename{lib/codec}
были уже собраны. Это позволяет быть уверенным, что любой код,
требующий генерации (например, исходные файлы на yacc/lex), будет
сгенерирован до того, как начнётся компиляция кода \filename{ui}.

Существует также одна тонкость в отношении порядка сборки
реквизитов. Как и в случае остальных зависимостей, порядок сборки
определяется на основании анализа графа зависимостей, однако когда
реквизиты цели перечисляются в одной строке, GNU \GNUmake{} иногда
собирает их слева направо. Рассмотрим пример:

{\footnotesize
\begin{verbatim}
all: a b c
all: d e f
\end{verbatim}
}

Если нет других зависимостей, требующих рассмотрения, шесть реквизитов
могут быть собраны в любом порядке (например, <<d b a c e f>>), однако
\GNUmake{} в рамках одной строки использует порядок слева направо,
порождая один из следующих результатов: <<a b c d e f>> \emph{или} <<d
e f a b c>>. Несмотря на то, что этот порядок является случайностью
реализации, порядок выполнения будет выглядеть правильным. Легко
забыть, что правильный порядок сборки является счастливой
случайностью, и не предоставить \GNUmake{} полную информацию о
зависимостях компонентов. Рано или поздно анализ зависимостей породит
другой порядок сборки, став причиной ошибок. Таким образом, если набор
целей должен быть собран в определённом порядке, укажите этот порядок
явно при помощи соответствующих реквизитов.

Когда будет выполнен головной \Makefile{}, мы увидим следующий вывод:

{\footnotesize
\begin{verbatim}
make --directory=lib/db
make[1]: Entering directory `/test/lib/db'
Update db library...
make[1]: Leaving directory `/test/lib/db'
make --directory=lib/codec
make[1]: Entering directory `/test/lib/codec'
Update codec library...
make[1]: Leaving directory `/test/lib/codec'
make --directory=lib/ui
make[1]: Entering directory `/test/lib/ui'
Update ui library...
make[1]: Leaving directory `/test/lib/ui'
make --directory=app/player
make[1]: Entering directory `/test/app/player'
Update player application...
make[1]: Leaving directory `/test/app/player'
\end{verbatim}
}

Когда \GNUmake{} определяет, что происходит рекурсивный вызов
\index{Опции!print-directory@\command{-{}-print-directory (-w)}}
\GNUmake{}, он автоматически включает опцию
\command{-{}-print\hyp{}directory} (\command{-w}), руководствуясь
которой, \GNUmake{} печатает сообщения при входе в каталог или выходе
из него. Эта опция также автоматически включается при использовании
опции \command{-{}-directory} (\command{-C}). В добавок ко всему на
каждой строке в квадратных скобках печатается значение переменной
\variable{MAKELEVEL}. В нашем простом примере \Makefile{} каждого
компонента печатает только сообщение о сборке соответствующего
компонента.

%%--------------------------------------------------------------------
%% Command line options
%%--------------------------------------------------------------------
\subsection{Опции командной строки}
\label{sec:command_line_options}

Рекурсивный \GNUmake{}~--- это простая идея, которая очень быстро
становится сложной. Идеальная реализация рекурсивного \GNUmake{} ведёт
себя так, будто множество \Makefile{}'ов системы является одним целым.
Такой уровень координации практически не достижим, поэтому в
реальности всегда приходится идти на компромисы. Тонкие проблемы
станут яснее, когда мы рассмотрим, как должны обрабатываться опции
командной строки.

Предположим, что мы добавили комментарии в заголовочный файл нашего
mp3 плеера. Вместо перекомпиляции всего исходного кода, зависящего от
модифицированного заголовочного файла, мы можем выполнить команду
\index{Опции!touch@\command{-{}-touch (-t)}}
\command{make -{}-touch}, чтобы обновить время модификации всех
файлов. Выполнив эту команду в каталоге с главным \Makefile{}'ом, мы
хотели бы, чтобы \GNUmake{} обновил временные метки всех файлов,
управляемыми дочерними экземплярами \GNUmake{}. Посмотрим, как это
работает.

Когда используется опция \command{-{}-touch}, обычно нормальный
процесс выполнения правил отменяется. Вместо этого \GNUmake{}
производит обход графа зависимостей, обновляя дату модификацию всех
запрошенных неабстракных целей и их реквизитов при помощи команды
\utility{touch}.  Поскольку все наши каталоги помечены как
\variable{.PHONY}, при нормальном ходе событий они будут
проигнорированы (поскольку обновление даты модификации для них смысла
не имеет). Однако мы не хотим, чтобы эти цели игнорировались, нам
требуется выполнение ассоциированных с ними правил. Чтобы обеспечить
правильное поведение, \GNUmake{} автоматически помечает все строки в
сценариях, содержащие переменную \variable{MAKE}, модификатором
\command{+}, в результате чего \GNUmake{} запускает дочерние процессы
\GNUmake{}, несмотря на опцию \command{-{}-touch}.

Когда \GNUmake{} запускает дочерние процессы \GNUmake{}, он должен
позаботиться о передаче им флага \command{-{}-touch}. Это достигается при
помощи переменной \variable{MAKEFLAGS}. Когда \GNUmake{} стартует,
происходит автоматическое добавление большей части опций к переменной
\variable{MAKEFLAGS}. Исключениями являются опции
\command{-{}-directory} (\command{-C}), \command{-{}-file}
(\command{-f}), \command{-{}-old\hyp{}file} (\command{-o}) и
\command{-{}-new\hyp{}file} (\command{-w}). Переменная
\variable{MAKEFLAGS} экпортируется в окружение и считывается дочерними
процессами \GNUmake{} при старте.

Благодаря этой функцональности дочерние процессы \GNUmake{} по
большей части ведут себя так, как вы ожидаете. Рекурсивное выполнение
\variable{\$(MAKE)} и специальная обработка переменной
\variable{MAKEFLAGS}, применяемая к опции \command{-{}-touch}, также
применяется к опциям \command{-{}-just\hyp{}print} (\command{-n}) и
\command{-{}-question} (\command{-q}).


%%--------------------------------------------------------------------
%% Passing variables
%%--------------------------------------------------------------------
\subsection{Передача переменных}

Как уже было замечено, переменные передаются в дочерние процессы
\GNUmake{} через окружение и контролируются при помощи директив
\index{Директивы!export@\directive{export}}
\index{Директивы!unexport@\directive{unexport}}
\directive{export} и \directive{unexport}. Значения переменных, переданные
через окружение, принимаются как значения по умалчанию, однако любое
присваивание изменит их значение. Для того, чтобы разрешить переменным
окружения переопределять локальные присваивания, используйте опцию
\index{Опции!environment-overrides@\command{-{}-environment\hyp{}overrides (-e)}}
\command{-{}-en\-vi\-ron\-ment\hyp{}overrides} (\command{-e}). Вы можете явно
переопределить переменную окружения (даже при включённой опции
\index{Директивы!override@\directive{override}}
\command{-{}-en\-vi\-ron\-ment\hyp{}overrides}) при помощи директивы
\directive{override}:

{\footnotesize
\begin{verbatim}
override TMPDIR = ~/tmp
\end{verbatim}
}

Переменные, определённые в командной строке, автоматически
экспортируются в окружение, если их имена удовлетворяют синтаксису
командного интерпретатора, то есть содержат только буквы, цифры и
подчёркивания. Присваивания переменных в командной строке сохраняются
в переменной \variable{MAKEFLAGS} наряду с другими опциями.

%%--------------------------------------------------------------------
%% Error handling
%%--------------------------------------------------------------------
\subsection{Обработка ошибок}

Что происходит, когда рекурсивный вызов \GNUmake{} обнаруживает
ошибку? На самом деле ничего особенного. Процесс \GNUmake{},
обнаруживший ошибку, завершается с кодом возврата 2. После этого
происходит выход из родительского процесса \GNUmake{}, и ошибка
передаётся вверх по дереву рекурсивных вызовов. Если первый вызов
\GNUmake{} содержал опцию \command{-{}-keep\hyp{}going}
(\command{-k}), она передаётся в дочерние процессы. В этом случае
дочерний процесс \GNUmake{} продолжает нормальное выполнение,
отбрасывает текущую цель и переходит к следующей, не используя цель,
вызвавшую ошибку, в качестве реквизита.

Например, если во время сборки нашего mp3 плеера обнаружится ошибка
компиляции в компоненте \filename{lib/db}, \GNUmake{} закончит
выполнение, вернув код ошибки 2 родительскому процессу. Если мы
\index{Опции!keep-going@\command{-{}-keep\hyp{}going (-k)}}
использовали опцию \command{-{}-keep\hyp{}going} (\command{-k}),
главный процесс \GNUmake{} начнёт обработку следующей независимой
цели, \filename{lib/codec}. Когда сборка этой цели будет закончена,
\GNUmake{} завершит выполнение с кодом возврата 2, поскольку сборка
остальных целей не может быть осуществлена по причине ошибки в
\filename{lib/db}.

\index{Опции!question@\command{-{}-question (-q)}}
Опция \command{-{}-question} (\command{-q}) приводит к похожему
поведению. При включении этой опции \GNUmake{} возвращает код ошибки 1
в случае, если какая-то цель требует повторной сборки, и 0 в противном
случае. Если применить эту опцию к дереву \Makefile{}'ов, \GNUmake{}
будет рекурсивно выполнять \Makefile{}'ы, пока не определит, требует
ли проект сборки. Как только обнаружится файл, требующий сборки,
\GNUmake{} завершит выполняемый в данный момент процесс \GNUmake{} и
<<размотает>> рекурсию.

%%--------------------------------------------------------------------
%% Building other targets
%%--------------------------------------------------------------------
\subsection{Сборка других целей}

Базовые цели для сборки естественны для большинства систем сборок,
однако нам нужны и другие вспомогательные цели, от которых мы зависим,
такие как \target{clean}, \target{install}, \target{print} и так
далее. Поскольку это абстрактные цели, описанная выше техника работает
не очень хорошо.

Например, ниже представлены несколько неработающих подходов:

{\footnotesize
\begin{verbatim}
clean: $(player) $(libraries)
    $(MAKE) --directory=$@ clean
\end{verbatim}
}
или:
{\footnotesize
\begin{verbatim}
$(player) $(libraries):
    $(MAKE) --directory=$@ clean
\end{verbatim}
}

Первый пример не работает потому, что реквизиты цели \target{clean}
вызовут сборку целей по умолчанию в \Makefile{}'ах
\variable{\$(player)} и \variable{\$(libraries)}, а не сборку цели
\target{clean}. Второй пример неверен потому, что для этих целей уже
определён другой командный сценарий.

Один из рабочих подходов основывается на использование цикла
\texttt{for}:

{\footnotesize
\begin{verbatim}
clean:
    for d in $(player) $(libraries); \
    do                               \
      $(MAKE) --directory=$$f clean; \
    done
\end{verbatim}
}

Цикл \texttt{for} не очень хорошо отвечает всем доводам, приведённым
ранее, однако он (вместе с предыдущим неверным примером) приводит нас
к следующему решению:

{\footnotesize
\begin{verbatim}
$(player) $(libraries):
    $(MAKE) --directory=$@ $(TARGET)
\end{verbatim}
}

Добавив к строке с рекурсивным вызовом \GNUmake{} переменную
\variable{TARGET} и выставляя значение этой переменной через командную
стоку, мы можем собирать в дочерних процессах \GNUmake{} произвольные
цели:

{\footnotesize
\begin{alltt}
\$ \textbf{make TARGET=clean}
\end{alltt}
}

К сожалению, это не приведёт к сборке цели \variable{\$(TARGET)} в
головном \Makefile{}'е. Часто это неважно, поскольку головной
\Makefile{} не делает ничего, однако в случае необходимости мы можем
добавить ещё один вызов \GNUmake{}, защищённый функцией \function{if}:

{\footnotesize
\begin{verbatim}
$(player) $(libraries):
    $(MAKE) --directory=$@ $(TARGET)
    $(if $(TARGET), $(MAKE) $(TARGET))
\end{verbatim}
}

Теперь мы можем собрать цель \target{clean} (или любую другую), просто
присвоив соответствующее значение переменной \variable{TARGET} в
командной строке.

%%--------------------------------------------------------------------
%% Cross-Makefile dependencies
%%--------------------------------------------------------------------
\subsection{Общие зависимости}

Специальная поддержка \GNUmake{} переменных командной строки и
коммуникация через переменные окружения подразумевают, что механизм
рекурсивного \GNUmake{} хорошо отлажен. Так в чём же заключаются
упомянутые ранее сложности?

Разделённые \Makefile{}'ы, соединяемые воедино командами
\variable{\$(MAKE)} описывают только наиболее поверхностные
высокоуровневые связи. К сожалению, часто бывают более тонкие
зависимости, скрытые в некоторых каталогах.

Предположим для примера, что модуль \filename{db} включает анализатор,
основанный на \utility{yacc}, для импорта и экспорта мызукальных
данных. Если модуль \filename{ui}, \filename{ui.c}, включает
сгенерированный \utility{yacc} заголовочный файл, на лицо связи между
этими двумя модулями. Если зависимости смоделированы правильно,
\GNUmake{} должен знать, что модуль \filename{ui} требует пересборки в
случае изменения заголовочного файла грамматики. Это нетрудно
организовать, используя технику автоматической генерации зависимостей,
описанную ранее. Однако что если исполняемый файл \utility{yacc} также
изменился? В этом случае после запуска \Makefile{}'а модуля
\filename{ui} корректный \Makefile{} определит, что сначала должна
быть выполнена команда \utility{yacc} для генерации анализатора и
заголовочного файла, и только после этого должна быть осуществлена
компиляция \filename{ui.c}. При нашей декомпозиции этого не случится,
потому что правила для запуска \utility{yacc} находятся в
\Makefile{}'е \filename{db}, а не \filename{ui}.

В этом случае лучшее, что мы можем сделать~--- это убедиться в том,
что \Makefile{} модуля \filename{db} запускается всегда раньше
\Makefile{}'а модуля \filename{ui}. Эта высокоуровневая зависимость
должна быть указана вручную. Мы были достаточно проницательны, чтобы
указать эту зависимость в первой версии нашего \Makefile{}'а, однако в
целом это может стать серьёзной проблемой при поддержке. Поскольку код
добавляется и модифицируется, головной \Makefile{} в какой-то момент
будет неправильно описывать зависимости между модулями.

В продолжение примера предположим, что грамматика \utility{yacc} в
модуле \filename{db} была изменена, и \Makefile{} модуля \filename{ui}
был выполнен до \Makefile{}'а модуля \filename{db} (вручную в обход
головного \Makefile{}'а). \Makefile{} модуля \filename{ui} не содержит
информации о неудовлетворённой зависимости в \Makefile{}'е модуля
\filename{db} и о необходимости запуска программы \utility{yacc} для
изменения головного файла. Вместо этого \Makefile{} модуля
\filename{ui} компилирует программу с устаревшим заголовочным
файлом. Если при модификации были добавлены новые символы, будет
обнаружена ошибка компиляции. Поэтому рекурсивный подход изначально
более хрупок по сравнению с монолитным \Makefile{}'ом.

Ситуация ухудшается с повышением интенсивности использования
генераторов исходного кода. Предположим, в реализации модуля
\filename{ui} был использован генератор заглушек RPC\footnote{
RPC (remote procedure call, вызов удалённых процедур)~--- класс
технологий, позволяющий программному обеспечению вызывать функции,
находящиеся в другом адресном пространстве (прим. переводчика).
},
заголовочные файлы которых используются в модуле
\filename{db}. Теперь нам придётся бороться с перекрёстными ссылками
между модулями. Для решения этой проблемы нам придётся сначала
посетить модуль \filename{db} и сгенерировать заголовочные файлы
\utility{yacc}, затем посетить модуль \filename{ui} и сгенерировать
заглушки RPC, затем вернуться в \filename{db} и произвести компиляцию,
и, наконец, посетить \filename{ui} и завершить процесс компиляции. Число
проходов, требуемое для создания и компиляции исходного кода проекта
зависит от структуры кода и инструментов, при помощи которых он
создаётся. Такой вид перекрёстных зависимостей встречается в сложных
системах довольно часто.

Стандартные решения в настоящих \Makefile{}'ах как правило являются
уловками. Для того, чтобы убедиться, что обновлены все файлы, каждый
\Makefile{} выполняется по команде головного \Makefile{}'а. Заметьте,
что это именно тот подход, который мы использовали в примере с mp3
плеером. Когда происходит запуск головного \Makefile{}'а, каждый из
четырёх дочерних \Makefile{}'ов запускается по очереди. В более
сложных случаях для проверки того, что весь код сначала сгенерирован,
и только затем скомпилирован, дочерние \Makefile{}'ы запускаются по
несколько раз. Чаще всего такие итерации являются напрасной тратой
времени, однако иногда они действительно необходимы.

%%--------------------------------------------------------------------
%% Avoiding duplicate code
%%--------------------------------------------------------------------
\subsection{Избегаем дублирования кода}

Структура каталогов нашего приложения включает три
библиотеки. \Makefile{}'ы этих библиотек очень похожи. Несмотря на то,
что эти библиотеки служат разным целям, все они собираются похожими
командами. Этот тип декомпозиции типичен для больших проектов и ведёт
к большому количеству похожих \Makefile{}'ов и дублированию сценариев.

Дублирование кода~--- это плохо, даже если оно происходит в
\Makefile{}'е. Оно увеличивает стоимость поддержки программного
обеспечения и ведёт к росту количества ошибок. Оно также затрудняет
понимание алгоритмов и определение небольших их вариаций. Поэтому
желательно избежать дублирования кода \Makefile{}'ов, настолько это
возможно. Легче всего достигнуть этого выносом общих частей в
отдельный включаемый файл.

Например, \Makefile{} модуля \filename{codec} содержит следущее:

{\footnotesize
\begin{verbatim}
lib_codec := libcodec.a
sources := codec.c
objects := $(subst .c,.o,$(sources))
dependencies := $(subst .c,.d,$(sources))

include_dirs := .. ../../include
CPPFLAGS     += $(addprefix -I ,$(include_dirs))
vpath %.h $(include_dirs)

all: $(lib_codec)

$(lib_codec): $(objects)
    $(AR) $(ARFLAGS) $@ $^

.PHONY: clean
clean:
    $(RM) $(lib_codec) $(objects) $(dependencies)

ifneq "$(MAKECMDGOALS)" "clean"
  include $(dependencies)
endif

%.d: %.c
    $(CC) $(CFLAGS) $(CPPFLAGS) $(TARGET_ARCH) -M $< | \
    sed 's,\($*\.o\) *:,\1 $@: ,' > $@.tmp
    mv $@.tmp $@
\end{verbatim}
}

Почти весь этот код дублицируется в \Makefile{}'ах модулей
\filename{db} и \filename{ui}. Единственное, что изменяется~--- это
имя библиотеки и исходные файлы. После того, как весь дублированный
код вынесен в файл \filename{common.mk}, мы можем сократить предыдущий
\Makefile{} следующим образом:

{\footnotesize
\begin{verbatim}
library := libcodec.a
sources := codec.c

include ../../common.mk
\end{verbatim}
}

Посмотрим, что вынесено в единственный общий включаемый файл:

{\footnotesize
\begin{verbatim}
MV := mv -f
RM := rm -f
SED := sed

objects      := $(subst .c,.o,$(sources))
dependencies := $(subst .c,.d,$(sources))
include_dirs := .. ../../include
CPPFLAGS     += $(addprefix -I ,$(include_dirs))

vpath %.h $(include_dirs)

.PHONY: library
library: $(library)

$(library): $(objects)
    $(AR) $(ARFLAGS) $@ $^

.PHONY: clean
clean:
    $(RM) $(objects) $(program) $(library) \
          $(dependencies) $(extra_clean)

ifneq "$(MAKECMDGOALS)" "clean"
  -include $(dependencies)
endif

%.c %.h: %.y
    $(YACC.y) --defines $<
    $(MV) y.tab.c $*.c
    $(MV) y.tab.h $*.h

%.d: %.c
    $(CC) $(CFLAGS) $(CPPFLAGS) $(TARGET_ARCH) -M $< | \
    $(SED) 's,\($*\.o\) *:,\1 $@: ,' > $@.tmp
    $(MV) $@.tmp $@
\end{verbatim}
}

Переменная \variable{include\_dirs}, которая раньше была разной для
каждого \Makefile{}'а, теперь одинакова во всех \Makefile{}'ах. Это
достигнуто благодаря переработке пути, используемого для поиска
заголовочных файлов при компиляции: теперь все библиотеки используют
один и тот же путь.

Файл \filename{common.mk} включает также цель по умолчанию для файлов
библиотек. Исходные \Makefile{}'ы использовали в качестве цели по
умолчанию \target{all}. Это вызвало бы проблемы в \Makefile{}'ах
программ, которым требуется указать различные наборы реквизитов для
своих целей по умолчанию. Поэтому включаемая версия кода использует
цель по умолчанию \target{library}.

Заметим, что поскольку общий файл содержит цели, в \Makefile{}'ы
программ он должен включаться \emph{после} цели по умолчанию. Заметим
также, что команда сценария \target{clean} содержит ссылки на
переменные \variable{program}, \variable{library} и
\variable{extra\_clean}. Для \Makefile{}'ов библиотек переменная
\variable{program} содержит пустую строку, в \Makefile{}'ах программ
пустую строку содержит переменная \variable{library}. Переменная
\variable{extra\_clean} добавлена специально для \Makefile{}'а модуля
\filename{db}. Этот \Makefile{} использует переменную для обозначения
кода, сгенерированного программой \utility{yacc}. Код \Makefile{}'а
представлен ниже:

{\footnotesize
\begin{verbatim}
library := libdb.a
sources := scanner.c playlist.c
extra_clean := $(sources) playlist.h

.SECONDARY: playlist.c playlist.h scanner.c

include ../../common.mk
\end{verbatim}
}

При использовании этой техники дублирование кода может быть сведено к
минимуму. Поскольку б\'{о}льшая часть кода вынесена во включаемый
\Makefile{}, со временем он эволюционирует в общий \Makefile{} всего
проекта. Для настройки используются переменные \GNUmake{} и функции,
определяемые пользователем, позволяющие модифицировать общий
\Makefile{} для каждого конкретного каталога.

%%-------------------------------------------------------------------
%% Nonrecursive make
%%-------------------------------------------------------------------
\section{Нерекурсивный \GNUmake{}}

Проекты, содержащие множество каталогов, могут управляться и без
рекурсивного \GNUmake{}. Разница заключается в том, что исходные
файлы, которыми манипулирует \Makefile{}, находятся более чем в одном
каталоге.  Чтобы отразить этот факт, все ссылки на файлы
должны использовать абсолютные или относительные пути к файлам.

Часто \Makefile{} большого проекта содержит множество целей, по одной
для каждого модуля системы. Например, в нашем проекте mp3 плеера нам
понадобились цели для каждой библиотеки и каждого приложения. Также
полезным может быть включение абстрактных целей для групп компонентов,
таких, например, как группа всех библиотек. Цель по умолчанию, как
правило, производит сборку всех этих целей. Часто цель по умолчанию
также производит составление документации и запуск процедур
автоматического тестирования.

Наиболее простой способ использования нерекурсивного \GNUmake{}~---
включение всех целей, ссылок на объектные файлы и зависимостей в один
\Makefile{}. Это часто не устраивает разработчиков, знакомых с
рекурсивным \GNUmake{}, поскольку в этом случае вся информация о
файлах и каталогах сосредоточена в одном файле, в то время как сами
файлы рассредоточены по файловой системе. Для решения этой проблемы
Миллер в своей статье о нерекурсивном \GNUmake{} предлагает добавлять
в каждый каталог включаемый файл, содержащий список файлов модуля и
правила, специфичные для него. Головной \Makefile{} включает все
дочерние \Makefile{}'ы.

Следующий пример демонстрирует \Makefile{} нашего проекта mp3 плеера,
включающий \Makefile{}'ы модулей из соответствующих каталогов.
{\footnotesize
\begin{verbatim}
# Информация о каждом модуле хранится в следующих четырёх
# переменных. Инициализируем их как простые переменные.
programs    :=
sources     :=
libraries   :=
extra_clean :=

objects      = $(subst .c,.o,$(sources))
dependencies = $(subst .c,.d,$(sources))

include_dirs := lib include
CPPFLAGS     += $(addprefix -I ,$(include_dirs))
vpath %.h $(include_dirs)

MV  := mv -f
RM  := rm -f
SED := sed

all:

include lib/codec/module.mk
include lib/db/module.mk
include lib/ui/module.mk
include app/player/module.mk

.PHONY: all
all: $(programs)

.PHONY: libraries
libraries: $(libraries)

.PHONY: clean
clean:
    $(RM) $(objects) $(programs) $(libraries) \
          $(dependencies) $(extra_clean)

  ifneq "$(MAKECMDGOALS)" "clean"
    include $(dependencies)
  endif

%.c %.h: %.y
    $(YACC.y) --defines $<
    $(MV) y.tab.c $*.c
    $(MV) y.tab.h $*.h

%.d: %.c
    $(CC) $(CFLAGS) $(CPPFLAGS) $(TARGET_ARCH) -M $< | \
    $(SED) 's,\($(notdir $*)\.o\) *:,$(dir $@)\1 $@: ,' > $@.tmp
    $(MV) $@.tmp $@
\end{verbatim}
}

Далее приведён пример \Makefile{}'а модуля \filename{/lib/codec}
(\filename{module.mk}):
{\footnotesize
\begin{verbatim}
local_dir  := lib/codec
local_lib  := $(local_dir)/libcodec.a
local_src  := $(addprefix $(local_dir)/,codec.c) 
local_objs := $(subst .c,.o,$(local_src))

libraries += $(local_lib)
sources   += $(local_src)

$(local_lib): $(local_objs)
    $(AR) $(ARFLAGS) $@ $^
\end{verbatim}
}

Таким образом, информация, специфичная для модуля, хранится во
включаемом файле в каталоге соответствующего модуля. Головной
\Makefile{} содержит только список модулей и директивы
\directive{include}. Давайте рассмотрим файл \filename{module.mk}
более детально.

Каждый файл \filename{module.mk} добавляет к переменной
\variable{libraries} имя текущей библиотеки, а к переменной
\variable{sources}~--- пути к исходным файлам. Переменные с префиксом
\variable{local\_} используются для хранения констант или для
предотвращения повторного вычисления значений. Обратите внимание на
то, что каждый модуль использует одинаковые имена \variable{local\_}
переменных. Именно поэтому вместо рекурсивных переменных используются
простые (объявляемые при помощи оператора \texttt{:=}): так сборки,
затрагивающие несколько \Makefile{}'ов, не подвержены риску
повреждения значений переменных в отдельных \Makefile{}'ах. Как уже
упоминалось, имена библиотек и списки исходных файлов используют
относительные пути. Наконец, включаемый файл содержит правила для
сборки текущей библиотеки. Использование \variable{local\_} переменных
в правилах вполне допустимо, так как цели и реквизиты правил
вычисляются при чтении файла.

Первые четыре строки головного \Makefile{}'а определяют переменные,
дополняемые информацией о каждом отдельном модуле. Эти переменные
должны быть простыми, поскольку каждый модуль будет добавлять к ним
данные из локальных переменных:
 
{\footnotesize
\begin{verbatim}
local_src := $(addprefix $(local_dir)/,codec.c)
...
sources   += $(local_src)
\end{verbatim}
}

Если бы переменная \variable{sources} была рекурсивной, финальное её
значение содержало бы просто последнее значение \variable{local\_src},
повторяющееся снова и снова. Поскольку по умолчанию переменные
являются рекурсивными, применяется явная инициализация пустым
значением.

Следующий раздел содержит вычисление списка объектных файлов,
\variable{objects}, и списка файлов зависимостей при помощи значения
переменной \variable{sources}. Эти переменные являются рекурсивными,
поскольку на данном этапе обработки \Makefile{}'а переменная
\variable{sources} содержит пустое значение. Это значение не будет
использоваться до тех пор, пока не будут прочитаны включаемые
\Makefile{}'ы. В нашем случае наиболее разумно было бы поместить
определение этих переменных после директив включения и объявить эти
переменные как простые, однако расположение переменных,
хранящих списки файлов (\variable{sources}, \variable{libraries},
\variable{objects}), рядом друг с другом упрощает понимание
\Makefile{}'а в целом и является хорошей практикой. К тому же, в
более сложных ситуациях перекрёстные ссылки между переменными
потребовали бы использования рекурсивных переменных.

Далее мы специфицируем обработку заголовочных файлов \Clang{}, указывая
значение переменной \variable{CPPFLAGS}. Это позволяет компилятору
находить заголовочные файлы. Для этой цели используется дополнение
значения (оператор \texttt{+=}), поскольку заранее нельзя сказать, что
значение переменной не определено: опции командной строки, переменные
окружения или конструкции \GNUmake{} могли уже придать ей какое-то
значение. Директива \directive{vpath} позволяет \GNUmake{} находить
заголовочные файлы, располагающиеся в других каталогах. Переменная
\variable{include\_dirs} используется для того, чтобы избежать
повторного вычисления списка включаемых каталогов.

Переменные \variable{MV}, \variable{RM} и \variable{SED} используются
для того, чтобы избежать жёсткой привязки к конкретным программам.
Обратите внимание на регистр имён переменных. Здесь мы следовали
соглашениям, принятым в руководству по \GNUmake{}. Имена переменных,
используемых только внутри \Makefile{}'а, состоят из прописных букв,
имена переменных, значение которых можно задать из командной
строки~--- из заглавных.

В следующей секции \Makefile{}'а всё ещё интереснее. Мы начинаем
раздел явных правил с указания цели по умолчанию, \target{all}. К
сожалению, реквизитом цели \target{all} является переменная
\variable{programs}.  Эта переменная будет вычислена незамедлительно,
однако её значение будет известно только после чтения включаемых
файлов. Таким образом, нам требуется прочитать включаемые файлы перед
тем, как определить цель \target{all}. Однако включаемые файлы
содержат цели, первая из которых станет целью по умолчанию. Чтобы
разрешить эту дилему, мы можем указать цель \target{all} без
реквизитов, прочитать включаемые файлы и добавить реквизиты к цели
\target{all} позднее.

Оставшаяся часть \Makefile{}'а уже знакома вам по предыдущим примерам,
однако всё же стоит обратить внимание на то, как \GNUmake{} применяет
неявные правила. Исходные файлы располагаются в подкаталогах. Когда
\GNUmake{} пытается применить стандартное правило \texttt{\%.o: \%.c},
реквизитом будет файл с относительным путём, например,
\filename{lib/ui/ui.c}. \GNUmake{} автоматически распространит
относительный путь на файл цели и попробует собрать
\filename{lib/ui/ui.o}. Таким образом, \GNUmake{} автомагически
(automagically) делает именно то, что нужно.

Есть один неприятный сбой. Несмотря на то, что \GNUmake{} обрабатывает
пути должным образом, не все инструменты, используемые им, делают тоже
самое. В частности, при использовании \utility{gcc} для генерации
зависимостей, результирующий файл не будет содержать относительного
пути к целевому объектному файлу. Вывод команды \texttt{gcc -M} будет
следующим:

{\footnotesize
\begin{verbatim}
ui.o: lib/ui/ui.c include/ui/ui.h lib/db/playlist.h
\end{verbatim}
}

в то время как мы ожидаем увидеть другое:


{\footnotesize
\begin{verbatim}
lib/ui/ui.o: lib/ui/ui.c include/ui/ui.h lib/db/playlist.h
\end{verbatim}
}

Это нарушает обработку файлов реквизитов. Для решения этой проблемы мы
можем настроить команду \utility{sed} так, чтобы она добавляла
информацию об относительных путях:

{\footnotesize
\begin{verbatim}
$(SED) 's,\($(notdir $*)\.o\) *:,$(dir $@)\1 $@: ,'
\end{verbatim}
}

Тонкая настройка \Makefile{}'а для обхода причуд различных
инструментов является естественной частью работы с \GNUmake{}.
Код переносимых \Makefile{}'ов часто бывают очень сложным из-за
капризов различных наборов инструментов, на которые приходится
полагаться.

Теперь у нас есть добротный нерекурсивный \Makefile{}, однако при
поддержке могут возникнуть проблемы. Дело в том, что включаемые файлы
\filename{module.mk} во многом схожи. Изменения в одном из них скорее
всего приведут к необходимости менять другие файлы. Для небольшого
проекта наподобие нашего mp3 плеера это неприятно. Для большого
проекта, содержащего несколько сотен включаемых файлов, это может быть
фатально. При разумном выборе имён переменных и регуляризации
содержимого включаемых файлов эта болезнь поддаётся лечению. Ниже
приводится включаемый файл \filename{lib/codec} после рефакторинга:

{\footnotesize
\begin{verbatim}
local_src := $(wildcard $(subdirectory)/*.c)
$(eval $(call make-library,
         $(subdirectory)/libcodec.a,
		 $(local_src)))
\end{verbatim}
}

Вместо того, чтобы перечислять исходные файлы явно, мы используем
предположение, согласно которому в сборке нуждаются все исходные файлы
в каталоге. Функция \function{make-library} осуществляет набор
операций, общих для всех включаемых файлов. Эта функция определяется в
начале головного \Makefile{}'а нашего проекта:

{\footnotesize
\begin{verbatim}
# $(call make-library, library-name, source-file-list)
define make-library
  libraries += $1
  sources   += $2

  $1: $(call source-to-object,$2)
    $(AR) $(ARFLAGS) $$@ $$^
endef
\end{verbatim}
}

Функция добавляет исходные файлы и имя библиотеки к соответствующим
переменным, затем определяет явные правила для сборки
библиотеки. Обратите внимание на то, что автоматические переменные
используются с двумя знаками доллара, чтобы отложить их вычисление до
выполнения правила. Функция \function{source-to-object} трансформирует
список исходных файлов в список соответствующих объектных файлов:

{\footnotesize
\begin{verbatim}
source-to-object = $(subst .c,.o,$(filter %.c,$1)) \
                   $(subst .y,.o,$(filter %.y,$1)) \
                   $(subst .l,.o,$(filter %.l,$1))
\end{verbatim}
}

В предыдущей версии \Makefile{}'а мы затушевали тот факт, что
настоящими исходными файлами являются \filename{playlist.y} и
\filename{scanner.l}. Вместо этого в списке файлов мы указывали
сгенерированные \filename{.c} файлы. Из-за этого нам приходилось
указывать их явно и включать дополнительную переменную,
\variable{extra\_clean}. Мы решили эту проблему, позволив переменной
\variable{sources} содержать файлы \filename{.y} и \filename{.l} и
возложив на функцию \function{source-to-object} работу по переводу
имён этих файлов в имена соответствующих объектных файлов.

В дополнение к модификации функции \function{source-to-object} нам
нужно добавить ещё одну функцию, вычисляющую имена выходных файлов
\utility{yacc} и \utility{lex}, чтобы позволить цели \target{clean}
должным образом выполнять свою работу. Функция
\function{generated-source} принимает на вход список файлов и
возвращает список промежуточных файлов:

{\footnotesize
\begin{verbatim}
# $(call generated-source, source-file-list)
generated-source = $(subst .y,.c,$(filter %.y,$1)) \
                   $(subst .y,.h,$(filter %.y,$1)) \
                   $(subst .l,.c,$(filter %.l,$1))
\end{verbatim}
}

Другая полезная функция, \function{subdirectory}, помогает избавиться
от локальной переменной \variable{local\_dir}.

{\footnotesize
\begin{verbatim}
subdirectory = $(patsubst %/makefile,%, \
                 $(word                 \
                   $(words $(MAKEFILE_LIST)),$(MAKEFILE_LIST)))
\end{verbatim}
}

Как уже было замечено в разделе <<\nameref{sec:str_func}>>
главы~\ref{chap:functions}, мы можем получить имя текущего
\Makefile{}'а из переменной \variable{MAKEFILE\_LIST}. Использовав
функцию \function{patsubst}, мы можем извлечь относительный путь из
имени последнего прочитанного \Makefile{}'а. Это помогает устранить
одну переменную и уменьшить разницу между включаемыми файлами.

Нашей последней оптимизацией (по крайней мере, в этом примере)
является использование функции \function{wildcard} для получения
списка исходных файлов. Это прекрасно работает в большинстве сред,
поддерживающих чистоту в каталогах с исходными файлами. Однако мне
приходилось работать в проекте, в котором это было не принято. Старый
код хранился в каталогах с исходными файлами <<на всякий случай>>.
Это влекло реальные затраты, выраженные во времени и нервах
программистов, поскольку средства поиска и замены находили символы в
старом коде, и новые программисты (или старые, не знакомые с модулем)
пытались откомпилировать и отладить код, который никогда не
использовался. Если вы используете современную систему контроля версий
(например, CVS), хранение старого кода в каталогах с исходным кодом
совершенно бессмысленно (поскольку весь код уже хранится в
репозитории), и использование \function{wildcard} становится
оправданным.

Директивы \directive{include} также могут быть оптимизированы:

{\footnotesize
\begin{verbatim}
modules := lib/codec lib/db lib/ui app/player
...
include $(addsuffix /module.mk,$(modules))
\end{verbatim}
}

Для больших проектов даже этот код может стать проблемой при
поддержке, поскольку список модулей может вырасти до сотен или даже
тысяч. При некоторых обстоятельствах более предпочтительным является
автоматическое определение модулей при помощи команды \utility{find}:

{\footnotesize
\begin{verbatim}
modules := $(subst /module.mk,,
             $(shell find . -name module.mk))
...
include $(addsuffix /module.mk,$(modules))
\end{verbatim}
}

Мы обрезаем имена файлов, обнаруженных командой \utility{find}, делая
переменную \variable{modules} более полезной как список модулей. Если
вам этого не требуется, тогда, конечно, можно опустить вызовы
\function{subst} и \function{addsuffix} и просто сохранить вывод
команды \utility{find} в переменной \variable{modules}.  Следующий
пример демонстрирует результирующий \Makefile{}.

{\footnotesize
\begin{verbatim}
# $(call source-to-object, source-file-list)
source-to-object = $(subst .c,.o,$(filter %.c,$1)) \
                   $(subst .y,.o,$(filter %.y,$1)) \
                   $(subst .l,.o,$(filter %.l,$1))

# $(subdirectory)
subdirectory = $(patsubst %/module.mk,%, \
                 $(word                  \
                   $(words $(MAKEFILE_LIST)),
				           $(MAKEFILE_LIST)))

# $(call make-library, library-name, source-file-list)
define make-library
  libraries += $1
  sources   += $2
  $1: $(call source-to-object,$2)
      $(AR) $(ARFLAGS) $$@ $$^
endef

# $(call generated-source, source-file-list)
generated-source = $(subst .y,.c,$(filter %.y,$1)) \
                   $(subst .y,.h,$(filter %.y,$1)) \
                   $(subst .l,.c,$(filter %.l,$1)) 

# Информация о каждом модуле хранится в следующих четырёх
# переменных. Инициализируем их как простые переменные.
modules      := lib/codec lib/db lib/ui app/player
programs     :=
libraries    :=
sources      :=
objects      = $(call source-to-object,$(sources))
dependencies = $(subst .o,.d,$(objects))

include_dirs := lib include
CPPFLAGS     += $(addprefix -I ,$(include_dirs))
vpath %.h $(include_dirs)

MV  := mv -f
RM  := rm -f
SED := sed

all:

include $(addsuffix /module.mk,$(modules))

.PHONY: all
all: $(programs)

.PHONY: libraries
libraries: $(libraries)

.PHONY: clean
clean:
    $(RM) $(objects) $(programs) $(libraries) $(dependencies) \
          $(call generated-source, $(sources))

  ifneq "$(MAKECMDGOALS)" "clean"
    include $(dependencies)
  endif

%.c %.h: %.y
    $(YACC.y) --defines $<
    $(MV) y.tab.c $*.c
    $(MV) y.tab.h $*.h

%.d: %.c
    $(CC) $(CFLAGS) $(CPPFLAGS) $(TARGET_ARCH) -M $< | \
    $(SED) 's,\($(notdir $*)\.o\) *:,$(dir $@)\1 $@: ,' > $@.tmp
    $(MV) $@.tmp $@
\end{verbatim}
}

Использование одного включаемого файла для каждого модуля является
весьма работоспособным подходом и имеет свои преимущества, однако я не
могу с уверенностью сказать, что он является наилучшим. Мой
собственный опыт работы с проектом на \Java{} показывает, что
использование головного \Makefile{}'а, эффективно включающего файлы
модулей, является разумным решением. Этот проект включал 997 отдельных
модулей, около двух десятков библиотек и полдюжины приложений.
Для обработки несвязанных подмножеств кода использовались различные
\Makefile{}'ы. Эти файлы в совокупности содержали примерно 2500 строк.
Общий включаемый файл, содержащий глобальные переменные, функции,
определяемые пользователем, и шаблонные правила, содержал ещё примерно
2500 строк.

Выберите ли вы один единственный \Makefile{}, или же поместите информацию
о модулях в отдельные включаемые файлы, нерекурсивный \GNUmake{}
является хорошим подходом к сборке крупных проектов. Он также решает
много традиционных проблем, связанных с использованием рекурсивного
\GNUmake{}. Единственный недостаток этого подхода, о котором стоит
предупредить~-- для разработчиков, использовавших рекурсивный
\GNUmake{}, потребуется смена парадигмы.

%%--------------------------------------------------------------------
%% Components of large systems
%%--------------------------------------------------------------------
\section{Компоненты больших систем}

Мы рассмотрим две популярных на сегодняшний день модели разработки:
модель свободного программного обеспечения и коммерческую модель.

В модели свободного программного обеспечения каждый разработчик в
основном может полагаться только на себя. Проект имеет \Makefile{} и
\filename{README} файл, и ожидается, что разработчикам потребуется
лишь немного помощи для начала работы. Приоритетами таких проектов, как
правило, являются качество и привлечение к участию всего сообщества,
однако наибольшую ценность имеет участие умелых и хорошо
мотивированных членов сообщества. Это не критика. С этой точки
зрения программное обеспечение должно быть высокого качества,
соблюдение временных ограничений имеет меньший приоритет.

В коммерческой модели разработки разработчики могут иметь различный
уровень подготовки, и все из них должны быть способны выполнить работу
к назначенному сроку. Разработчик, который не может разобраться, как
выполнить свою работу, расходует деньги понапрасну. Если система не
компилируется или не запускается должным образом, вся команда
разработчиков может простаивать: этот сценарий является наиболее
затратным из всех возможных. Для решения подобных проблем процесс
разработки управляется командой поддержки, координирующей процесс
сборки, конфигурацию инструментов разработки, поддержку и разработку,
а также менеджмент новых релизов. В подобной среде процессом управляют 
критерии эффективности.

Как правило, для коммерческой модели характерны более продуманные
системы сборки. Основной причиной такого перевеса является стремление
сократить стоимость разработки программного обеспечения за счёт
повышения эффективности труда программистов. Это, в свою очередь,
должно вести к увеличению прибыли. Это именно та модель, которая
требует от \GNUmake{} наибольшего функционала. Тем не менее, техники,
которые мы обсудим, применимы в случае необходиости и к модели
свободного программного обеспечения.

Этот раздел содержит много общей информации, чуть-чуть специфики
и совсем не содержит примеров. Причина этого заключается в том, что
очень многое зависит от языка разработки и используемой операционной
среды. В главах~\ref{chap:c_and_cpp} и \ref{chap:java} мы рассмотрим
примеры реализации многих возможностей, рассмотренных в этом разделе.

%---------------------------------------------------------------------
% Requirements
%---------------------------------------------------------------------
\subsection*{Требования}

Разумеется, требования различны для каждого проекта и каждой рабочей
среды. Здесь мы рассмотрим основные требования, считающиеся важными во
многих коммерческих средах разработки.

Наиболее общей потребностью команд разработчиков является отделение
исходного кода от бинарных файлов. То есть объектные файлы,
полученные в результате компиляции, должны располагаться в отдельном
дереве каталогов. Это, в свою очередь, позволяет включать ещё
множество возможностей. Отделение дерева каталогов бинарных файлов
сулит множество преимуществ:

\begin{itemize}
%---------------------------------------------------------------------
\item Когда расположение дерева каталогов бинарных файлов определено,
гораздо легче управлять дисковым пространством.
%---------------------------------------------------------------------
\item Различные версии деревьев бинарных файлов могут управляться
параллельно. Например, единственному дереву исходных файлов могут
соответствовать оптимизированное, отладочное, и профилировочное
деревья бинарных файлов.
%---------------------------------------------------------------------
\item Существует возможность одновременной поддержки различных
платформ. Правильным образом реализованное дерево исходных файлов
может быть использовано для параллельной компиляции исполняемых файлов
для различных платформ.
%---------------------------------------------------------------------
\item Разработчики могут взять небольшую часть исходного кода и
позволить системе сборки самостоятельно <<заполнять>> недостающие
файлы с помощью зависимостей исходных файлов и деревьев каталогов
объектных файлов. Это не обязательно требует отделения исходных файлов
от объектных, однако без отделения больше вероятность того, что
система сборки не сможет правильно определить, где следует искать
бинарные файлы.
%---------------------------------------------------------------------
\item Дерево исходного кода может быть сделано доступно только для
чтения. Это даёт дополнительную уверенность в том, что сборка отражает
реальное состояние репозитория.
%---------------------------------------------------------------------
\item Некоторые цели, подобные \target{clean}, можно реализовать
тривиальным образом (и выполнять с колоссальным выигрышем в
производительности), если всё дерево каталогов может быть рассмотрено
как отдельная единица, не требующая поиска и манипуляций файлами.
%---------------------------------------------------------------------
\end{itemize}

Б\'{о}льшая часть этих пунктов является важными преимуществами системы
сборки и может быть проектным требованием.

Возможность управления историей сборок проекта часто является важным
качеством системы сборки. Основная идея заключается в том, что сборка
исходного кода осуществляется по ночам, обычно при помощи задачи
\utility{cron}.
\index{Справочные деревья каталогов}
Поскольку результирующие деревья каталогов, содержащие
исходный код и бинарные файлы, являются не модифицируемыми с точки
зрения CVS, я буду называть их справочными. Эта идея имеет множество
применений.

Во-первых, справочное дерево каталогов исходного кода может
использоваться программистами и менеджерами, которым нужно просмотреть
исходный код. Это может показаться банальным, однако когда число
файлов и релизов растёт, извлечение всего исходного кода из
репозитория ради просмотра одного файла может быть не очень разумно. К
тому же, хоть инструменты для просмотра CVS репозиториев достаточно
распространены, они обычно не предоставляют средств для простого
поиска по всему исходному коду проекта. Для этих целей больше подходят
таблицы символов или даже команды \utility{find}/\utility{grep} (или
\utility{grep -R}).

Во-вторых, справочные деревья бинарных файлов являются индикатором
того, что соответствующая сборка прошла успешно. Когда разработчики
начинают утром свою работу, они уже знают, является ли система
работоспособной. Если проект использует систему автоматического
тестирования, свежие сборки могут использоваться для запуска
автоматических тестов. Каждый день разработчики могут проверять отчёты
с результатами тестов, чтобы определить жизнеспособность системы, не
тратя время на запуск тестов. Если же разработчик имеет на руках
только модифицированную версию исходного кода, имеет место дополнительное
сокращение затрат проекта, поскольку в этом случает разработчику не
нужно терять время на получение исходной версии и сборку. Наконец,
справочные сборки могут запускаться разработчиками для тестирования и
сравнения функциональности определённых компонентов.

Есть и другие способы использования справочных сборок. Для проекта,
состоящего из множества библиотек, прекомпилированные библиотеки,
полученные в результате ночных сборок, могут использоваться
программистами для компоновки собственных приложений с теми
библиотеками, которые они не модифицируют. Это позволяет сократить
цикл разработки за счёт исключения необходимости компиляции большей
части исходного кода при запуске собственных сборок. Разумеется,
лёгкий доступ к исходному коду проекта, располагающегося на локальном
файловом сервере, чрезвычайно удобен, если разработчикам, не имеющим
полной рабочей копии проекта, нужно просмотреть исходный код.

При таком разнообразии применений справочных деревьев каталогов
поддержание целостности их структуры становится чрезвычайно важным.
Одним из простых и эффективных способов повышения надёжности является
объявление дерева каталогов с исходным кодом доступным только для
чтения. Это гарантирует, что дерево отражает состояние репозитория в
момент сборки. Этот аспект может потребовать особого внимания,
поскольку во многих случаях система сборки может принимать попытки
записи в дерево каталогов, в частности, при генерации исходного кода
или создании временных файлов. Объявления дерева каталогов с исходным
кодом доступным только для чтения также предотвращает случайное его
повреждение обычными пользователями, что случается наиболее
часто.

Ещё одним общим требованием к проектной системе сборки является
возможность лёгкого управления различными конфигурациями компиляции,
компоновки и развёртывания системы. Система сборки обычно должна
оперировать различными версиями проекта (которые могут быть различными
ветками в репозитории).

Множество крупных проектов зависят от программного обеспечения третих
разработчиков, представленном в форме библиотек или инструментов
разработки. Если нет других инструментов для управления конфигурацией
программного обеспечения (а обычно их нет), использование
\Makefile{}'а и системы сборки для этих целей часто является разумным
выбором.

Наконец, когда программное обеспечение доставляется заказчику, оно часто
упаковывается на базе текущей рабочей версии. Это может быть также
сложно, как конструирование \filename{setup.exe} файла для Windows или
также просто, как редактирование HTML файла и связывание его с
\filename{jar} архивом. Иногда операция инсталляции сочетается с
обычным процессом сборки. Я предпочитаю разделять сборку и генерацию
инсталлятора на два независимых шага, поскольку, как правило, они
используют совершенно разные процессы. В любом случае, скорее всего,
обе эти операции будут влиять на систему сборки.

%%--------------------------------------------------------------------
%% Filesystem layout
%%--------------------------------------------------------------------
\section{Структура файловой системы}
\label{sec:filesystem_layout}

Как только вы решите поддерживать несколько деревьев каталогов,
содержащих бинарные файлы, встаёт вопрос о структуре файловой системы.
В средах, требующих использования нескольких деревьев каталогов, часто
содержится \emph{много} таких деревьев. Чтобы найти способ
поддержания порядка в таких средах требуется немного подумать.

Наиболее общим способом структурирования этих данных является
выделение большого жёсткого диска как хранилища деревьев каталогов с
бинарными файлами. Корневой (или близкий к корню) каталог содержит
один подкаталог для каждого дерева. Одним из разумных способов
идетнификации этих деревьев является включение в имя каждого каталога
именования поставщика, платформы, операционной системы и параметров
сборки бинарного дерева:

{\footnotesize
\begin{alltt}
\$ \textbf{ls}
hp-386-windows-optimized
hp-386-windows-debug
sgi-irix-optimized
sgi-irix-debug
sun-solaris8-profiled
sun-solaris8-debug
\end{alltt}
}

Если требуется хранить множество сборок, произведённых в разные
моменты времени, для их идентификации обычно используются временные
метки, включенные в имя каталога. Из-за удобства сортировки часто
используются форматы \texttt{гг-мм-дд} и \texttt{гг-мм-дд-чч-мм}:

{\footnotesize
\begin{alltt}
\$ \textbf{ls}
hp-386-windows-optimized-040123
hp-386-windows-debug-040123
sgi-irix-optimzed-040127
sgi-irix-debug-040127
sun-solaris8-profiled-040127
sun-solaris8-debug-040127
\end{alltt}
}

Конечно, способ упорядочивания имён компонентов целиком зависит от
ваших потребностей. Каталог верхнего уровня этих деревьев хорошо
подходит для хранения \Makefile{}'а и отчётов о результатах тестов.

Предыдущая структура хорошо подходит для хранения множества сборок,
осуществляемых разработчиками параллельно. Если команда разработчиков
выпускает <<релизы>>, возможно, для внутренних потребителей, вам стоит
рассмотреть возможность добавления хранилища релизов,
структурированное как множество продуктов, каждый из которых может
иметь номер ревизии и временную метку, как показано на
рисунке~\ref{fig:release_tree_layout}.

\begin{figure}
{\footnotesize
\begin{verbatim}
release
|
|--product1
|  |--1.0
|  |  |--040101
|  |  `--040112
|  |
|  `--1.4
|     `--040121
|
`--product1
   `--1.4
      `--031212
\end{verbatim}
}
\caption{Пример структуры дерева каталогов релиза}
\label{fig:release_tree_layout}
\end{figure}

Продукты могут быть библиотеками, производимыми командой разработчиков
для нужд других разработчиков. Конечно, это могут быть и продукты в
привычном их понимании.

Каковы бы ни были структура файловой системы и среда разработки,
реализацией управляет множество однотипных критериев. Каждое дерево
должно легко идентифицироваться. Освобождение ресурсов должно быть
быстрым и ясным. Полезно иметь возможность перемещать и архивировать
деревья. В добавок ко всему, структура файловой системы должна быть
близка структуре процесса разработки организации. Это позволит
перемещаться по хранилищу непрограммистам, таким, как менеджеры,
инженеры по качеству и составители технической документации. 

%%-------------------------------------------------------------------
%% Automating builds and testing
%%-------------------------------------------------------------------
\section{Автоматические сборки и тестирование}

Как правило важно иметь возможность максимально возможной
автоматизации процесса сборки. Это позволит производить сборку
справочных деревьев каталогов по ночам, сохраняя дневное время
разработчиков. Это также позволяет разработчикам запускать сборки на
собственных машинах без предварительной подготовки.

Для программного обеспечения, находящегося в разработке, часто
возникает множество заявок на сборку различных версий различных
продуктов. Для человека, выполняющего эти заявки, возможность
запланировать несколько сборок и <<пойти прогуляться>> часто является
критичной для поддержки и выполнения заявок.

Автоматизированное тестирование создаёт дополнительные трудности.
Для управления процессом тестирования большинства консольных
приложений могут быть использованы простые сценарии. Для тестирования
консольных приложений, требующих взаимодействия с пользователем, можно
\index{dejaGnu}
использовать утилиту GNU \utility{dejaGnu}. Разумеется, каркасы,
\index{JUnit}
подобные JUnit (\filename{\url{http://www.junit.org}}), также
предоставляют поддержку модульного тестирования приложений, не
требующего графической среды.

Тестирование приложений с графическим пользовательским интерфейсом
готовит дополнительные проблемы.  Для систем, использующих X11, я с
успехом применял тестирование по расписанию с использованием
\index{Xvbf}
виртуального оконного буфера (virtual frame buffer), Xvfb. Для Windows
я не смог найти удовлетворительного решения для автоматизированного
тестирования. Все подходы основаны на сохранении тестовой учётной
записи зарегистрированной системе, а экрана~--- не заблокированным.


%%%------------------------------------------------------------------
%%% Portable makefiles
%%%------------------------------------------------------------------
\chapter{Переносимые \Makefile{}'ы}
\label{chap:portable_makefiles}

Какой же \Makefile{} мы будем считать переносимым? В качестве
экстремального примера мы хотим иметь \Makefile{}, запускающийся без
изменений на любой платформе, позволяющей запустить GNU \GNUmake{}.
Однако это практически невозможно по причине огромного количества
различных операционных систем. Более разумным будет назвать
переносимым \Makefile{}, который легко изменить для запуска на другой
платформе. Однако перенос на другую платформу не должен препятствовать
поддержке всех предыдущих платформ, это будет дополнительным важным
ограничением.

Мы можем достичь этого уровня переносимости \Makefile{}'ов, используя
те же техники, что и в традиционном программировании: инкапсуляция и
абстракция. Используя переменные и функции, определяемые
пользователем, мы можем инкапсулировать приложения и алгоритмы.
Определяя переменные для аргументов командной строки и параметров, мы
можем абстрагироваться от элементов, варьирующихся от платформы к
платформе.

Затем вам потребуется определить, какие инструменты может предложить
каждая платформа для выполнения вашей работы, и какие из них нужно
использовать в случае каждой конкретной платформы. Наибольшую
переносимость приносит использование только тех инструментов, которые
присутствуют на всех интересующих платформах. Обычно это называют
подходом <<наименьшего общего знаменателя>>, который, очевидно, может
сделать базовый набор инструментов довольно скудным.

Другой версией подхода наименьшего общего знаменателя является
следующая парадигма: используйте мощный набор инструментов и
убедитесь, что можете взять его с собой на любую платформу. Это
гарантирует, что команды, которые вы вызываете в \Makefile{}'е,
работают совершенно одинаково в любой системе. Осуществить это, как
правило, нелегко, и с административной точки зрения, и в плане
убеждения вашей организации в необходимости кооперации её систем с
вашими  наработками. Однако такой подход может приносить результаты, и
позже я приведу приведу пример с пакетом Cygwin для Windows. Как вы
увидите, стандартизация инструментов не решает всех проблем, всегда
найдутся отличия операционных систем, которые нужно будет обрабатывать
особым образом.

Наконец, вы можете принять различия между системами как данность и
обходить их с помощью аккуратного выбора функций и макросов. Я приведу
пример такого подхода в этой главе.

Таким образом, рассудительно используя переменные и функции,
определяемые пользователем, минимизируя использование экзотических
возможностей и полагаясь на стандартные инструменты, мы можем
увеличить переносимость наших \Makefile{}'ов. Как уже было замечено,
идеальная переносимость недостижима, поэтому нашей задачей является
нахождение баланса между затратами и переносимостью. Однако прежде,
чем мы исследуем специфические техники, давайте произведём обзор
основных проблем переносимости \Makefile{}'ов.

%%--------------------------------------------------------------------
%% Portability issues
%%--------------------------------------------------------------------
\section{Проблемы переносимости}

Проблемы переносимости может быть нелегко охарактеризовать, поскольку
они могут варьироваться от тотальной смены парадигмы (например,
отличие классической Mac OS от System V \UNIX{}) до исправлений
тривиальных ошибок (таких, как исправление кода возврата программы).
Тем не менее, ниже перечислены основные проблемы переносимости, с
которыми рано или поздно сталкивается любой \Makefile{}:

\begin{description}
%---------------------------------------------------------------------
% Program names
%---------------------------------------------------------------------
\item[Имена программ] \hfill \\
Довольно часто в различных платформах для программ, реализующих схожую
(или даже одинаковую) функциональность, используются различные имена.
Наиболее ярким примером являются имена компиляторов языков \Clang{} и 
\Cplusplus{} (например, \utility{cc} и \utility{xlc}). Также общим
является добавления префикса \filename{g} для GNU-версий программ,
установленных на не GNU системах (например, \utility{gmake},
\utility{gawk}). 

%---------------------------------------------------------------------
% Paths
%---------------------------------------------------------------------
\item[Пути] \hfill \\
Расположение файлов и программ варьируется от платформы к платформе. 
Например, в операционной системе Solaris каталогом X-сервера
является \filename{/usr/X}, в то время как на многих других системах
этим каталогом является \filename{/usr/X11R6}. К тому же, различие
между \filename{/bin}, \filename{/usr/bin}, \filename{/sbin} и
\filename{/usr/sbin} часто неясно при переходе от одной системе к
другой.

%---------------------------------------------------------------------
% Options
%---------------------------------------------------------------------
\item[Аргументы командной строки] \hfill \\
Аргументы командной строки программы могут отличаться, в частности при
использовании альтернативной реализации. Более того, если на какой-то
платформе отсутствует нужная вам программа (или присутствующая версия
этой программы вам не подходит), вам, возможно, придётся заменить эту
программу другой, использующей другие аргументы командной строки.

%---------------------------------------------------------------------
% Shell features
%---------------------------------------------------------------------
\item[Возможности интерпретатора] \hfill \\
По умолчанию \GNUmake{} выполняет командные сценарии с помощью
\filename{/bin/sh}, однако возможности различных реализаций
интерпретатора \utility{sh} варьируются от платформы к платформе. В
частности, интерпретаторы, выпущенные до принятия стандарта \POSIX{},
не имеют множества возможностей и не принимают синтаксис современных
интерпретаторов.

У Open Group есть очень полезная статья, описывающая различия между
интерпретаторами System V и \POSIX{}. Её можно найти по адресу
\filename{\url{http://www.unix-systems.org/whitepapers/shdiffs.html}}.
Те, кому интересны детали, смогут найти спецификацию командного
интерпретатора \POSIX{} по адресу
\filename{\url{http://www.opengroup.org/onlinepubs/
007904975/utilities/xcu_chap02.html}}

%---------------------------------------------------------------------
% Program behavior
%---------------------------------------------------------------------
\item[Поведение программ] \hfill \\
Переносимым \Makefile{}'ам приходится бороться с программам, которые
ведут себя по-разному на различных платформах. Это встречается
повсеместно, поскольку различные поставщики исправляют (и совершают)
ошибки и добавляют новые возможности. Существуют также обновления
программ, которые поставщик может включить или не включить в релиз.
Например, в 1987 году программа \utility{awk} сменила старший номер
версии. Тем не менее, даже спустя двадцать лет некоторые системы всё
ещё не используют новую версию в качестве стандартной программы
\utility{awk}.

%---------------------------------------------------------------------
% Operating system
%---------------------------------------------------------------------
\item[Операционная система] \hfill \\
Наконец, существуют проблемы переносимости, связанные с совершенно
различными операционными системами, например, Windows и \UNIX{}, Linux
и VMS.
\end{description}

%%--------------------------------------------------------------------
%% Cygwin
%%--------------------------------------------------------------------
\section{Cygwin}

Несмотря на то, что есть порт \GNUmake{} под Win32, это лишь малая
часть проблемы переноса \Makefile{}'ов на Windows, поскольку командным
интерпретатором, используемым этим портом по умолчанию, является
\filename{cmd.exe} (или \filename{command.exe}). Это, наряду с
отсутствием большинства инструментов \UNIX{}, делает реализацию
кросс-платформенной переносимости очень сложной задачей. К счастью,
проект Cygwin (\filename{\url{http://www.cygwin.com}}) реализовал
для Windows библиотеку совместимости с Linux, с использованием которой
были перенесены многие программы. Я уверен, что Windows разработчики,
желающие достичь совместимости с Linux или получить доступ к
инструментам GNU, не найти смогут лучшего инструмента.

Я использовал Cygwin на протяжении десяти лет для различных проектов,
начиная с CAD-приложения, построенного на смеси \Cplusplus{} и Lisp, и
заканчивая приложением для управления рабочим процессом, написанным на
чистом \Java{}. Набор инструментов Cygwin включает компиляторы и
интерпретаторы многих языков программирования. Однако Cygwin можно
выгодно использовать даже в том случае, если приложение реализовано
без использования набора компиляторов и интерпретаторов Cygwin. Набор
инструментов Cygwin можно использовать как вспомогательное средство
для координации процессов разработки и сборки. Другими словами, совсем
не обязательно писать <<Cygwin>> приложение или использовать средства
разработки Cygwin, чтобы извлечь выгоду из Cygwin-окружения.

Тем не менее, Linux~--- это не Windows (слава богам!), поэтому при
использовании Cygwin инструментов применительно к <<родным>>
приложениям Windows возникает ряд проблем. Практически все эти
проблемы решаются на уровне символов окончаний строки в файлах и форм
путей к файлам, передающихся между Cygwin и Windows.

%---------------------------------------------------------------------
% Line termination
%---------------------------------------------------------------------
\subsection*{Окончания строк}

Файловая система Windows использует для индикации окончания строки
последовательность из двух символов: символа возврата каретки и
символа окончания строки (CRLF). \POSIX{} системы используют для этой
цели один символ~--- символ окончания строки (LF). Иногда это различие
может стать причиной удивления, если программа вдруг сообщит о
синтаксической ошибке или перейдёт к неверной позиции в файле.
Библиотека Cygwin делает всё возможное для избежания этих
неприятностей. Во время установки Cygwin (или при использовании
команды \utility{mount}) вы можете выбрать, следует ли Cygwin
выполнять преобразование файлов, содержащих последовательность CRLF в
качестве индикатора окончания строки. Если выбран формат файлов DOS,
Cygwin будет заменять последовательной CRLF на символ LF при открытии
файла и производить обратное преобразование при записи текста, таким
образом, \UNIX{}-программы могут правильно работать с текстовыми
файлами DOS.  Если вы планируете использовать родные инструменты
наподобие Visual C++ или Sun Java SDK, выбирайте формат файлов DOS.
Если же вы планируете использовать компиляторы Cygwin, используйте
формат \UNIX{} (вы сможете изменить своё решение в любое время).

В добавок ко всему, Cygwin поставляется с набором инструментов для
перевода форматов файлов. Программы \utility{dos2unix} и
\utility{unix2dos} помогут преобразовать файлы в нужный формат в
случае необходимости.

%---------------------------------------------------------------------
% Filesystem
%---------------------------------------------------------------------
\subsection*{Файловая система}

Cygwin предоставляет \POSIX{}-взгляд на файловую систему Windows.
Корневой каталог файловой системы \POSIX{}, \filename{/}, отображается
в каталог, в который установлен Cygwin. Диски Windows доступны из
псевдокаталога \filename{/cygdrive/\texttt{буква}}. Таким образом,
если Cygwin установлен в каталог
\filename{C:\textbackslash{}usr\textbackslash{}cygwin} (я предпочитаю
именно этот каталог), будет производиться отображение каталогов,
представленное в таблице~\ref{tab:cygwin_dir_mapping}.

\begin{table}
\footnotesize
\center
\begin{tabular}{|l|l|l|}
\hline
Путь Windows & Путь Cygwin & Альтернативный путь Cygwin \\
\hline
\hline
\filename{c:\textbackslash{}usr\textbackslash{}cygwin} &%
\filename{/} &%
\filename{/cygdrive/c/usr/cygwin} \\
\hline
\filename{c:\textbackslash{}Program Files} &%
\filename{/cygdrive/c/Program Files} & \\
\hline
\filename{c:\textbackslash{}usr\textbackslash{}cygwin\textbackslash{}bin} &%
\filename{/bin} &%
\filename{/cygdrive/c/usr/cygwin/bin} \\
\hline
\end{tabular}
\caption{Стандартное отображение каталогов Cygwin}
\label{tab:cygwin_dir_mapping}
\end{table}

Поначалу такое преобразование может быть немного непривычным, однако
на работу программ оно никак не влияет. Cygwin также предоставляет
команду \utility{mount}, позволяющую пользователям получать доступ к
файлам и каталогам более удобным способом. Oпция \utility{mount}
\index{Опции!Cygwin!change-cygdrive-prefix@\command{-{}-change-cygdrive-prefix}}
\command{-{}-change\hyp{}cygdrive\hyp{}prefix} позволяет вам изменить
префикс. Мне кажется, что изменение префикса на \filename{/} может
быть полезно, поскольку в этом случае доступ к дискам становится более
естественным:

\begin{alltt}
\footnotesize
\$ \textbf{mount --change-cygdrive-prefix} /
\$ \textbf{ls /c}
AUTOEXEC.BAT            IO.SYS                     WINDOWS
BOOT.INI                MSDOS.SYS                  WUTemp
CD                      NTDETECT.COM               hp
CONFIG.SYS              PERSIST                    ntldr
C\_DILLA                 Program Files              pagefile.sys
Documents and Settings  RECYCLER                   tmp
Home                    System Volume Information  usr
I386                    Temp                       work
\end{alltt}

Как только вы произведёте это действие, предыдущее отображение
каталогов поменяется на отображение, представленное в
таблице~\ref{tab:modified_dir_mapping}.

\begin{table}
\footnotesize
\center
\begin{tabular}{|l|l|l|}
\hline
Путь Windows & Путь Cygwin & Альтернативный путь Cygwin \\
\hline
\hline
\filename{c:\textbackslash{}usr\textbackslash{}cygwin} &%
\filename{/} &%
\filename{/c/usr/cygwin} \\
\hline
\filename{c:\textbackslash{}Program Files} &%
\filename{/c/Program Files} & \\
\hline
\filename{c:\textbackslash{}usr\textbackslash{}cygwin\textbackslash{}bin} &%
\filename{/bin} &%
\filename{/c/usr/cygwin/bin} \\
\hline
\end{tabular}
\caption{Модифицированное отображение каталогов Cygwin}
\label{tab:modified_dir_mapping}
\end{table}

Если вам нужно передать имя файла Windows-программе (например,
компилятору Visual \Cplusplus{}), вы можете просто передать
относительный путь к файлу, использовав \POSIX{} стиль, предполагающий
использование прямых слэшей. Win32 API не различает прямых и обратных
слэшей. К сожалению, некоторые программы, осуществляющие разбор
аргументов командной строки, интерпретируют все прямые слэши как
опции. Одной из таких программ является команда DOS \utility{print},
ещё одним примером является команда \utility{net}.

Если же используется абсолютный путь, синтаксис, основанный на именах
дисков, всегда вызывает проблемы. Несмотря на то, что программы
Windows обычно легко воспринимают прямые слэши в именах файлов, они
совершенно не способны воспринять синтаксис \filename{/c}. Имя диска
всегда должно преобразовываться в формат \filename{c:}. Для
осуществления прямых и обратных преобразований путей \POSIX{} в пути
Windows Cygwin предоставляет команду \utility{cygpath}:

\begin{alltt}
\footnotesize
\$ \textbf{cygpath --windows /c/work/src/lib/foo.c}
c:\textbackslash{}work\textbackslash{}src\textbackslash{}lib\textbackslash{}foo.c
\$ \textbf{cygpath --mixed /c/work/src/lib/foo.c}
c:/work/src/lib/foo.c
\$ \textbf{cygpath --mixed --path "/c/work/src:/c/work/include"}
c:/work/src;c:/work/include
\end{alltt}

\index{Опции!Cygwin!windows@\command{-{}-windows}}
Опция \command{-{}-windows} преобразует заданный путь \POSIX{} в путь
Windows (или, при указании соответствующего аргумента, наоборот). Я
\index{Опции!Cygwin!mixed@\command{-{}-mixed}}
предпочитаю использовать опцию \command{--mixed}, возвращающую путь
Windows, в котором все обратные слэши заменены на прямые (таком
образом, этот путь может использоваться для работы с программами
Windows). Такие пути гораздо удобнее использовать в командном
интерпретаторе Cygwin, воспринимающем обратный слэш как символ
экранирования. Программа \utility{cygpath} имеет множество опций,
предоставляющих переносимый доступ к важным каталогам Windows:

\begin{alltt}
\footnotesize
\$ \textbf{cygpath --desktop}
/c/Documents and Settings/Owner/Desktop
\$ \textbf{cygpath --homeroot}
/c/Documents and Settings
\$ \textbf{cygpath --smprograms}
/c/Documents and Settings/Owner/Start Menu/Programs
\$ \textbf{cygpath --sysdir}
/c/WINDOWS/SYSTEM32
\$ \textbf{cygpath --windir}
/c/WINDOWS
\end{alltt}

Если вы используете \utility{cygpath} в смешанной Windows/\UNIX{}
среде, вы можете захотеть обернуть его вызовы в переносимые функции:

{\footnotesize
\begin{verbatim}
ifdef COMSPEC
  cygpath-mixed         = $(shell cygpath -m "$1")
  cygpath-unix          = $(shell cygpath -u "$1")
  drive-letter-to-slash = /$(subst :,,$1)
else
  cygpath-mixed         = $1
  cygpath-unix          = $1
  drive-letter-to-slash = $1
endif
\end{verbatim}
}

Если вам нужно только преобразовать букву диска в \POSIX{} форму,
функция \function{drive-letter-to-slash} будет работать быстрее, чем
запуск программы \utility{cygpath}.

Наконец, Cygwin не может спрятать все причуды Windows. Имена файлов,
недопустимые в Windows, также недопустимы в Cygwin. Например, такие
имена, как \filename{aux.h}, \filename{com1} и \filename{prn} не могут
использоваться в \POSIX{} путях, даже при наличии расширения.

%---------------------------------------------------------------------
% Program conflicts
%---------------------------------------------------------------------
\subsection*{Конфликты программ}

Несколько программ Windows имеют точно такие же имена, что и
\UNIX{}-программы. Разумеется, программы Windows не принимают тех же
самых аргументов командной строки и не ведут себя совместимым с
\UNIX{}-программами образом. Если вы случайно вызвали Windows версию
программы, обычным результатом является серьёзное недоумение. Наиболее
проблемными программами в этом плане являются \utility{find},
\utility{sort}, \utility{ftp} и \utility{telnet}. Для достижения
максимальной переносимости убедитесь в том, что вы используете
абсолютные пути к этим программам при работе с \UNIX{}, Windows
и Cygwin.

Если вы тесно используете Cygwin и для сборки вам не нужны базовые
инструменты Windows, вы можете спокойно поместить каталог
\filename{/bin} в начало переменной окружения \variable{PATH}. Это
будет гарантией того, что в первую очередь будут использоваться
инструменты Cygwin, а не их Windows аналоги.

Если ваш \Makefile{} использует инструменты \Java{}, учтите, что
Cygwin включает программу GNU \utility{jar}, не совместимую по формату
со стандартными Sun \filename{jar} файлами. Поэтому каталог \Java{}
jdk \filename{bin} следует поместить в вашей переменной
\variable{PATH} раньше каталога Cygwin \filename{/bin}. Это поможет
вам избежать использования программы Cygwin \filename{jar}.

%%--------------------------------------------------------------------
%% Managing programs and files
%%--------------------------------------------------------------------
\section{Управление программами и файлами}

Наиболее общий способ управления программами заключается в
использовании переменных для имён программ или путей, которые могут
измениться. Переменные могут быть определены в простом блоке, как мы
уже видели прежде:
{\footnotesize
\begin{verbatim}
MV ?= mv -f
RM ?= rm -f
\end{verbatim}
}
{\flushleft или же в условном блоке:}
{\footnotesize
\begin{verbatim}
ifdef COMSPEC
  MV ?= move
  RM ?= del
else
  MV ?= mv -f
  RM ?= rm -f
endif
\end{verbatim}
}

Если используется простой блок, значения переменных могут измениться
при использовании аргументов командной строки, при редактировании
\Makefile{}'а или (именно для этого случая мы использовали оператор
условного присваивания \texttt{?=}) при наличии соответствующей
переменной окружения. Как уже было ранее замечено, одним из способов
определения текущей платформы является проверка существования
переменной \variable{COMSPEC}, используемой всеми версиями
операционной системы Windows. Иногда в коррекции нуждаются только
пути:

{\footnotesize
\begin{verbatim}
ifdef COMSPEC
  OUTPUT_ROOT := d:
  GCC_HOME    := c:/gnu/usr/bin
else
  OUTPUT_ROOT := $(HOME)
  GCC_HOME    := /usr/bin
endif

OUTPUT_DIR := $(OUTPUT_ROOT)/work/binaries
CC := $(GCC_HOME)/gcc
\end{verbatim}
}

Этот стиль приводит к \Makefile{}'ам, в которых б\'{о}льшая часть
программ вызывается при помощи переменных \GNUmake{}. Пока вы не
привыкните к этому, читать такие \Makefile{}'ы будет немного сложнее.
Однако использовать переменные в любом случае разумнее, поскольку их
имена могут быть значительно короче, чем имена программ, в частности,
если используются абсолютные пути.

Та же техника может быть использована для управления опциями командной
строки. Например, встроенные правила содержат переменную
\variable{TARGET\_ARCH}, которая может быть использована для указания
опций, зависящих от платформы:

{\footnotesize
\begin{verbatim}
ifeq "$(MACHINE)" "hpux-hppa"
  TARGET_ARCH := -mdisable-fpregs
endif
\end{verbatim}
}

При определении собственных программных переменных можно использовать
подобный подход:

{\footnotesize
\begin{verbatim}
MV := mv $(MV_FLAGS)

ifeq "$(MACHINE)" "solaris-sparc"
  MV_FLAGS := -f
endif
\end{verbatim}
}

Если вы переносите продукт на несколько платформ, цепочки секций
условной обработки могут стать неуклюжими и трудными в поддержке.
Вместо использования директивы \directive{ifdef} поместите каждый
набор переменных, зависящих от платформы, в собственный файл, имя
которого содержит название платформы. Например, если вы определяете
платформу по параметрам команды \utility{uname}, можете выбрать
соответствующий файл для включения следующим образом:

{\footnotesize
\begin{verbatim}
MACHINE := $(shell uname -smo | sed 's/ /-/g')
include $(MACHINE)-defines.mk
\end{verbatim}
}

Имена файлов, содержащие пробелы, являются особенно раздражающей
проблемой при использовании \GNUmake{}. Предположение о том, что
пробелы используются для разделения символов при синтаксическом
разборе, является для \GNUmake{} фундаментальным. Множество встроенных
функций, таких как \function{word}, \function{filter} и
\function{wildcard}, предполагают, что их аргументами является список
слов, разделённых пробелами. Тем не менее, есть несколько приёмов,
которые могут немного помочь в этом вопросе.  Первый приём, описанный
в разделе~\nameref{sec:supporting_multiple_binary_trees}
главы~\ref{chap:c_and_cpp}, заключается в замене пробелов другими
символами при помощи функции \function{subst}:

{\footnotesize
\begin{verbatim}
space = $(empty) $(empty)
# $(call space-to-question,file-name)
space-to-question = $(subst $(space),?,$1)
\end{verbatim}
}

Функция \function{space-to-question} заменяет все пробелы символом
вопросительного знака, используемым командным интерпретатором при
определении шаблонов. Теперь мы можем реализовать функции
\function{wildcard} и \function{file-exists}, умеющие правильно
работать с пробелами:

{\footnotesize
\begin{verbatim}
# $(call wildcard-spaces,file-name)
wildcard-spaces = $(wildcard $(call space-to-question,$1))

# $(call file-exists,file-name)
file-exists = $(strip                                \
                $(if $1,,$(warning $1 has no value)) \
                $(call wildcard-spaces,$1))
\end{verbatim}
}

Функция \function{wildcard-spaces} использует
\function{space-to-question} для осуществления операции поиска по
шаблону, содержащему пробелы. Можно использовать функцию
\function{wildcard-spaces} для реализации функции
\function{file-exists}. Разумеется, использование символа знака
вопроса может привести к тому, что функция \function{wildcard-spaces}
будет возвращать файлы, не соответствующие первоначальному шаблону
поиска (например, <<my documents.doc>> и <<my-documents.doc>>), однако
вряд ли можно найти что-то получше.

Функцию \function{space-to-question} можно использовать для
преобразования имён файлов, содержащих пробелы, в спецификациях целей
и реквизитов, поскольку они допускают использование шаблонов:

{\footnotesize
\begin{verbatim}
space := $(empty) $(empty)

# $(call space-to-question,file-name)
space-to-question = $(subst $(space),?,$1)

# $(call question-to-space,file-name)
question-to-space = $(subst ?,$(space),$1)
$(call space-to-question,foo bar): $(call space-to-question,bar baz)
       touch "$(call question-to-space,$@)"
\end{verbatim}
}

Если файл <<\filename{bar baz}>> существует, при первом выполнении
\Makefile{}'а реквизит будет найден, поскольку существующий файл
соответствует шаблону. Однако поиск файла по шаблону, соответствующего
цели, закончится неудачей, поскольку файл цели не существует. В
результате переменная \variable{\$@} примет значение \texttt{foo?bar}.
После этого командный сценарий вызовет функцию
\function{question-to-space}, чтобы преобразовать значение переменной
\variable{\$@} обратно в имя файла, содержащее пробел. При следующем
запуске файл цели, содержащий в имени пробел, будет найден по шаблону.
Этот приём выглядит немного неуклюже, однако я нашёл ему применение в
реальных \Makefile{}'ах.

%---------------------------------------------------------------------
% Source tree layout
%---------------------------------------------------------------------
\subsection*{Структура каталогов исходного кода}

Другим аспектом переносимости является возможность предоставления
разработчикам свободы в управлении средой разработки по собственному
усмотрению. Если система сборки будет требовать от них, к примеру,
помещать исходный код, бинарные файлы, библиотеки и инструменты
разработки в один и тот же каталог или диск Windows, рано или поздно
возникнут проблемы. В конце концов, разработчики, ограниченные в
дисковом пространстве, будут вынуждены разделить эти файлы.

Вместо этого имеет смысл реализовать \Makefile{} с использованием
переменных для хранения коллекций файлов и инициализировать эти
переменные разумными значениями по умолчанию. Для доступа к каждой
используемой библиотеке или инструменту может быть использована
соответствующая переменная, это позволит разработчикам настраивать
местоположение файлов по собственному усмотрению. Используйте оператор
условного присваивания при определении таких переменных, это даст
разработчикам простой способ переопределения их значений через
переменные окружения.

К тому же, возможность простой поддержки нескольких копий дерева
каталогов с исходными и бинарными файлами является благом для
разработчиков. Даже если им не приходится поддерживать несколько
платформ или использовать различные опции компиляции, разработчикам
часто приходится работать с несколькими рабочими копиями исходного
кода в целях отладки или при параллельной работе в нескольких
проектах. Мы уже рассмотрели два возможных пути реализации этой
возможности: использование высокоуровневых переменных окружения для
идентификации корневого каталога дерева исходных и бинарных файлов,
либо использование каталога, в котором находится \Makefile{}, в
совокупности с фиксированным относительным путём для определения
корневого каталога дерева бинарных файлов. Любой из этих подходов
предоставляет разработчикам механизм для поддержки нескольких деревьев
каталогов.

%%--------------------------------------------------------------------
%% Working with nonportable tools
%%--------------------------------------------------------------------
\section{Работа с непереносимыми инструментами}

Как уже было замечено, одной из альтернатив написания \Makefile{}'ов
по принципу наименьшего общего знаменателя является адаптация
стандартного набора инструментов. Разумеется, цель этого подхода~---
убедиться в том, что стандартный набор инструментов по меньшей мере
так же переносим, как и ваше приложение. Очевидным выбором переносимых
инструментов является набор программ проекта GNU, однако существует
довольно много проектов переносимых инструментов. Два других
инструмента, приходящие на ум~--- Perl и Python.

При отсутствии переносимых инструментов хорошей альтернативой является
инкапсуляция непереносимых инструментов в функции \GNUmake{}.
Например, для поддержки различных компиляторов Enterprise JavaBeans
(каждый из которых имеет собственный синтаксис вызова), мы можем
написать функцию для компиляции архива EJB и параметризовать её для
возможности подключения другого компилятора.

{\footnotesize
\begin{verbatim}
EJB_TMP_JAR = $(TMPDIR)/temp.jar
# $(call compile-generic-bean, bean-type, jar-name,
#        bean-files-wildcard, manifest-name-opt )
define compile-generic-bean
  $(RM) $(dir $(META_INF))
  $(MKDIR) $(META_INF)
  $(if $(filter %.xml %.xmi, $3),             \
    cp $(filter %.xml %.xmi, $3) $(META_INF))
  $(call compile-$1-bean-hook,$2)
  cd $(OUTPUT_DIR) &&                         \
  $(JAR) -cf0 $(EJB_TMP_JAR)                  \
         $(call jar-file-arg,$(META_INF))     \
         $(call bean-classes,$3)
  $(call $1-compile-command,$2)
  $(call create-manifest,$(if $4,$4,$2),,)
endef
\end{verbatim}
}

Первым аргументом этой общей функции компиляции EJB~--- это тип
компилятора компонентов, такого как Weblogic, Websphere и т.д.
Остальными аргументами являются имя архива, список файлов архива
(включая конфигурационные файлы) и необязательный файл манифеста.
Сначала шаблонная функция создаёт пустой временный каталог, удаляя и
создавая заново предыдущий временный каталог. Затем функция производит
копирование \filename{xml} и \filename{xmi} файлов, указанных в
качестве реквизитов каталога \variable{\$(META\_INF)}. На данном этапе
нам может понадобиться осуществление вспомогательных действий, будь то
очистка каталога \filename{META-INF} или подготовка \filename{.class}
файлов. Для поддержки этих операций мы включили функцию-триггер,
\function{compile-\$1-bean-hook}, которую пользователь может
реализовать по собственному усмотрению. Например, если компилятор
Websphere требует дополнительный контрольный файл, например,
\filename{xsl} файл, мы можем реализовать триггер следующим образом:

{\footnotesize
\begin{verbatim}
# $(call compile-websphere-bean-hook, file-list)
define compile-websphere-bean-hook
  cp $(filter %.xsl, $1) $(META_INF)
endef
\end{verbatim}
}

Просто определив эту функцию, мы убеждаемся в том, что вызов
\function{call} в функции \function{compile-generic-bean} будет
осуществлён успешно. Если не будем писать триггер, соответствующий
вызов в \function{compile-generic-bean} вычислится в пустую строку.

Затем наша функция создаёт jar архив. Вспомогательная функция
\function{jar\hyp{}file\hyp{}arg} производит преобразование обычного
пути к файлу в конкатенацию опции \texttt{-C} и относительного пути:

{\footnotesize
\begin{verbatim}
# $(call jar-file-arg, file-name)
define jar-file-arg
  -C "$(patsubst %/,%,$(dir $1))" $(notdir $1)
endef
\end{verbatim}
}

Вспомогательная функция \function{bean\hyp{}classes} извлекает
подходящий class файл из списка исходных файлов (в jar архив нужно
включать только интерфейсы и home классы):

{\footnotesize
\begin{verbatim}
# $(call bean-classes, bean-files-list)
define bean-classes
  $(subst $(SOURCE_DIR)/,,                 \
    $(filter %Interface.class %Home.class, \
      $(subst .java,.class,$1)))
endef
\end{verbatim}
}

Затем общая функция вызывает соответствующую команду компиляции
\texttt{\$(call \$1\hyp{}compile\hyp{}command,\$2)}:

{\footnotesize
\begin{verbatim}
define weblogic-compile-command
  cd $(TMPDIR) && \
  $(JVM) weblogic.ejbc -compiler $(EJB_JAVAC) $(EJB_TMP_JAR) $1
endef
\end{verbatim}
}

Наконец, общая функция добавляет файл манифеста.

После определения функции \function{compile-generic-bean} мы можем
обернуть её вызов в специальную функцию для каждого компилятора,
который мы хотим поддерживать.

{\footnotesize
\begin{verbatim}
# $(call compile-weblogic-bean, jar-name,
#        bean-files-wildcard, manifest-name-opt )
define compile-weblogic-bean
  $(call compile-generic-bean,weblogic,$1,$2,$3)
endef
\end{verbatim}
}

%---------------------------------------------------------------------
% A standard shell
%---------------------------------------------------------------------
\subsection*{Стандартный интерпретатор}

Следует ещё раз подчеркнуть, что одним из самых досадных источников
непереносимости при переходе на другую систему являются возможности
интерпретатора \filename{/bin/sh}, используемого \GNUmake{} по
умолчанию. Если вам приходится настраивать командные сценарии вашего
\Makefile{}'а, рассмотрите возможность стандартизации вашего
интерпретатора. Разумеется, это не очень подходит для типичных
проектов с открытым исходным кодом, \Makefile{}'ы которых выполняются
в неконтролируемой среде. Однако в случае, если вы управляете средой и
контролируете число машин, которые нужно настроить, такой подход
вполне разумен.

Многие интерпретаторы предоставляют возможности, которые могут
исключить использование большого числа небольших программ. Например,
\utility{bash} включает расширенные возможности работы с переменными,
такие как \texttt{\%\%} и \texttt{\#\#}, которые могут помочь избежать
использования инструментов, таких как \utility{sed} и \utility{expr}.

%%--------------------------------------------------------------------
%% Automake
%%--------------------------------------------------------------------
\section{Automake}

В этой главе мы сосредоточились на использовании GNU \GNUmake{} и
эффективной поддержке инструментов для достижения переносимости систем
сборки. Однако иногда даже эти скромные цели недостижимы. Если вы не
можете использовать мощные возможности GNU \GNUmake{} и вынуждены
полагаться на ограниченный набор возможностей, продиктованный подходом
наименьшего общего знаменателя, вам следует рассмотреть
\index{automake}
возможность использования программы \utility{automake},
\filename{\url{http://www.gnu.org/software/automake/automake.html}}.


Программа \utility{automake} принимает на вход стилизованный
\Makefile{} и производит переносимый \Makefile{}. Работа
\utility{automake} основывается на применении макроязыка \utility{m4},
допускающего довольно сжатый синтаксис входных файлов (обычно их имя
\filename{makefile.am}). Как правило, \utility{automake} используется
в совокупности с программой \utility{autoconf}, пакетом поддержки
переносимости программ, написанных на \Clang{}/\Cplusplus{}, однако
использование \utility{autoconf} не обязательно.

В то время как \utility{automake} является хорошим решением для систем
сборки, требующих максимальной переносимости, \Makefile{}'ы, которые
производит эта программа, не имеют доступа к богатым возможностям GNU
\GNUmake{} (за исключением оператора \texttt{+=}, для поддержки
которого используются особые средства). Более того, синтаксис входных
файлов \utility{automake} имеет мало общего с синтаксисом обычных
\Makefile{}'ов. Поэтому использование \utility{automake}
(без \utility{autoconf}) не очень сильно отличается от подхода
наименьшего общего знаменателя.


%%%-------------------------------------------------------------------
%%% C and C++
%%%-------------------------------------------------------------------
\chapter{\Clang{} и \Cplusplus{}}
\label{chap:c_and_cpp}

Проблемы и техники, показанные в главе~\ref{chap:managing_large_proj},
раскрываются и применяются в этой главе к проектам, написанным на
\Clang{} и \Cplusplus{}. Мы продолжим рассматривать наш пример mp3
плеера, сборка которого осуществляется нерекурсивным \Makefile{}'ом.

%%--------------------------------------------------------------------
%% Separating source and binary
%%--------------------------------------------------------------------
\section{Разделение исходных и бинарных файлов}
\label{sec:separating_source_and_binary}

Итак, что же нам делать, если мы хотим поддерживать единственное
дерево каталогов с исходным кодом, но множество платформ и множество
сборок для каждой платформы, разделяя при необходимости деревья
каталогов исходных и бинарных файлов? Изначально программа \GNUmake{}
была написана для эффективной работы с файлами, находящимися в одном
каталоге. И хотя со времени своего создания она очень изменилась,
истоки забыты не были. Лучше всего \GNUmake{} работает с несколькими
каталогами, если модифицируемые им файлы находятся в текущем каталоге
или в его подкаталогах.

%---------------------------------------------------------------------
% The easy way
%---------------------------------------------------------------------
\subsection*{Простой способ}

Наиболее простой способ заставить \GNUmake{} помещать бинарные файлы в
отдельный каталог~--- это запустить \GNUmake{} из этого каталога.
Доступ к выходным файлам осуществляется через относительные пути, как
показано в предыдущей главе, в то время как исходные файлы должны быть
указаны явно или при помощи \directive{vpath}. В любом случае, нам
придётся ссылаться на каталог с исходными файлами из нескольких мест,
поэтому нам нужно завести переменную для хранения пути к нему:

{\footnotesize
\begin{verbatim}
SOURCE_DIR := ../mp3_player
\end{verbatim}
}

Возьмём за основу наш предыдущий \Makefile{}. Функция
\function{source-to-object} остаётся неизменной, а вот функцию
\function{subdirectory} нужно изменить так, чтобы она учитывала
относительные пути к исходным файлам.

{\footnotesize
\begin{verbatim}
# $(call source-to-object, source-file-list)
source-to-object = $(subst .c,.o,$(filter %.c,$1)) \
                   $(subst .y,.o,$(filter %.y,$1)) \
                   $(subst .l,.o,$(filter %.l,$1))
# $(subdirectory)
subdirectory = $(patsubst $(SOURCE_DIR)/%/module.mk,%, \
                 $(word                                \
                   $(words $(MAKEFILE_LIST)),$(MAKEFILE_LIST)))

\end{verbatim}
}

Файлы, перечисленные в переменной \variable{MAKEFILE\_LIST}, будут
включать относительные пути к исходным файлам. Таким образом, чтобы
получить относительный путь к каталогу модуля, нам нужно помимо
суффикса \filename{module.mk} отсечь и префикс.

Далее, чтобы помочь \GNUmake{} найти исходные файлы, мы используем
директиву \directive{vpath}:

{\footnotesize
\begin{verbatim}
vpath %.y $(SOURCE_DIR)
vpath %.l $(SOURCE_DIR)
vpath %.c $(SOURCE_DIR)
\end{verbatim}
}

Это позволит нам использовать для исходных файлов такие же простые
относительные пути, как и для выходных файлов. Когда \GNUmake{} будет
искать исходный файл и не сможет найти его в текущем каталоге, он
будет сканировать каталог \variable{SOURCE\_DIR}. Далее, нам нужно
обновить значение переменной \variable{include\_dirs}:

{\footnotesize
\begin{verbatim}
include_dirs := lib $(SOURCE_DIR)/lib $(SOURCE_DIR)/include
\end{verbatim}
}

В дополнение к каталогам исходных файлов это переменная включает
подкаталог \filename{lib} дерева бинарных файлов, ведь именно туда
будут попадать заголовочные файлы, составленные \utility{lex} и
\utility{yacc}.

Также нужно обновить директиву \directive{include} для получения
доступа к файлам \filename{module.mk}, поскольку \GNUmake{} не
использует директиву \directive{vpath} для поиска включаемых файлов:

{\footnotesize
\begin{verbatim}
include $(patsubst %,$(SOURCE_DIR)/%/module.mk,$(modules))
\end{verbatim}
}

Наконец, мы создаём каталоги для размещения выходных файлов:

{\footnotesize
\begin{verbatim}
create-output-directories :=                    \
    $(shell for f in $(modules);                \
            do                                  \
              $(TEST) -d $$f || $(MKDIR) $$f;   \
            done)
\end{verbatim}
}

Это присваивание создаёт пустую переменную, значение которой никогда
не используется. Это даёт гарантию, что необходимые каталоги будут
созданы до начала основной работы \GNUmake{}. Нам приходится самим
создавать каталоги, так как \utility{lex}, \utility{yacc} и
инструменты автоматической генерации зависимостей не сделают этого за
нас.

Ещё один способ убедиться в том, что необходимые каталоги созданы~---
добавить эти каталоги в качестве реквизитов файлов зависимостей (эти
файлы имеют расширение \filename{.d}). Однако это плохая идея, так как
каталог на самом деле не является реквизитом. Файлы \utility{yacc},
\utility{lex} или файлы зависимостей не зависят от \emph{содержимого}
каталога, и не должны создаваться заново только от того, что
временн\'{а}я метка каталога изменилась. На самом деле это будет
источником неэффективности в случае добавления или удаления файлов из
каталога, содержащего выходные файлы.

Изменения, которые нужно сделать в файле \filename{module.mk}, ещё
проще:

{\footnotesize
\begin{verbatim}
local_src := $(addprefix $(subdirectory)/,playlist.y scanner.l)

$(eval $(call make-library, $(subdirectory)/libdb.a, $(local_src)))

.SECONDARY: $(call generated-source, $(local_src))

$(subdirectory)/scanner.d: $(subdirectory)/playlist.d
\end{verbatim}
}

Эта версия не использует функцию \function{wildcard} для определения
исходных файлов. Пусть восстановление этого функционала станет
упражнением для читателя. В первоначальном варианте был небольшой
сбой. Когда я запустил этот \Makefile{} я обнаружил, что файл
зависимостей \filename{scanner.d} был создан до файла
\filename{playlist.h}, от которого он зависит. Эта зависимость не была
отражена в исходном \Makefile{}'е, однако по счастливой случайности
всё работало правильно. Правильное определение \emph{всех}
зависимостей является трудной задачей даже для небольших проектов.

Если предположить, что исходные файлы находятся в подкаталоге
\filename{mp3\_player}, то сборка проекта будет осуществляться
следующим образом:

{\footnotesize
\begin{alltt}
\$ \textbf{mkdir mp3\_player\_out}
\$ \textbf{cd mp3\_player\_out}
\$ \textbf{make --file=../mp3\_player/makefile}
\end{alltt}
}

Наш \Makefile{} правилен и хорошо работает, однако немного раздражает
то, что приходится изменять каталог и добавлять опцию
\command{-{}-file} (\command{-f}). Эту проблему можно решить с помощью
простого сценария:

{\footnotesize
\begin{verbatim}
#! /bin/bash
if [[ ! -d $OUTPUT_DIR ]]
then
  if ! mkdir -p $OUTPUT_DIR
  then
    echo "Cannot create output directory" > /dev/stderr
  exit 1
  fi
fi

cd $OUTPUT_DIR
make --file=$SOURCE_DIR/makefile "$@"
\end{verbatim}
}

Этот сценарий подразумевает, что каталог с исходными файлами и каталог
выходных файлов хранятся в переменных окружения \variable{SOURCE\_DIR}
и \variable{OUTPUT\_DIR} соответственно. Это является обычной
практикой, позволяющей разработчикам легко переключать деревья
каталогов, не печатая при этом пути слишком часто.

Наш \Makefile{} не содержит никаких средств, позволяющих запретить
разработчикам выполнять \Makefile{} из каталога с исходными файлами.
Это частая ошибка, и некоторые командные сценарии могут вести себя
неправильно. Например, цель \target{clean}:

{\footnotesize
\begin{verbatim}
.PHONY: clean
clean:
    $(RM) -r *
\end{verbatim}
}

{\flushleft удалит всё дерево каталогов с исходным кодом пользователя!
Благоразумным решением является добавление соответствующей проверки
в самом начале выполнения \Makefile{}'а. Ниже приведён возможный
вариант такой проверки:}

{\footnotesize
\begin{verbatim}
$(if $(filter $(notdir $(SOURCE_DIR)),$(notdir $(CURDIR))),\
  $(error Пожалуйста, запустите makefile из бинарного дерева каталогов.))
\end{verbatim}
}

Этот код проверяет совпадение имени текущего рабочего каталога
\command{(\$(notdir \$(CURDIR)))} с именем каталога, содержащего
исходный код: \command{(\$(notdir \$(SOURCE\_DIR)))}. Если эти два
выражения совпадают, выводится сообщение об ошибке, и выполнение
\GNUmake{} прекращается. Поскольку результатом вычисления функций
\function{if} и \function{error} является пустая строка, мы можем
поместить эти две строчки кода сразу после определения переменной
\variable{SOURCE\_DIR}.

%---------------------------------------------------------------------
% The hard way
%---------------------------------------------------------------------
\subsection*{Сложный способ}

Некоторые разработчики находят необходимость перемещения в дерево
каталогов, содержащих бинарные файлы, настолько раздражающей, что
готовы проделать огромную работу для того, чтобы этого избежать. Или,
возможно, разработчик \Makefile{}'а использует среду, в которой
использование сценариев\hyp{}обёрток или псевдонимов программ
недопустимо.  В любом случае, можно изменить \Makefile{} таким
образом, чтобы была возможность производить запуск \GNUmake{} из
дерева каталогов, содержащих исходный код, и помещать бинарные файлы в
другое дерево каталогов с помощью добавления к путям выходных файлов
соответствующих префиксов. Обычно в этом случае я использую абсолютные
пути, поскольку такой подход предоставляет больше гибкости, хотя и
обостряет проблему конечности длины командной строки. Для исходных
файлов используются простые относительные пути (они вычисляются
относительно каталога, в котором располагается \Makefile{}).

Следующий пример содержит модифицированный \Makefile{}, позволяющий
запускать \GNUmake{} из каталога с исходным кодом и записывающий
бинарные файлы в отдельное дерево каталогов:

{\footnotesize
\begin{verbatim}
SOURCE_DIR := /test/book/examples/ch07-separate-binaries-1
BINARY_DIR := /test/book/out/mp3_player_out

# $(call source-dir-to-binary-dir, directory-list)
source-dir-to-binary-dir = $(addprefix $(BINARY_DIR)/, $1)

# $(call source-to-object, source-file-list)
source-to-object = $(call source-dir-to-binary-dir,  \
                     $(subst .c,.o,$(filter %.c,$1)) \
                     $(subst .y,.o,$(filter %.y,$1)) \
                     $(subst .l,.o,$(filter %.l,$1)))
# $(subdirectory)
subdirectory = $(patsubst %/module.mk,%,  \
                 $(word                   \
                   $(words $(MAKEFILE_LIST)),$(MAKEFILE_LIST)))

# $(call make-library, library-name, source-file-list)
define make-library
  libraries += $(BINARY_DIR)/$1
  sources   += $2
  $(BINARY_DIR)/$1: $(call source-dir-to-binary-dir,  \
                      $(subst .c,.o,$(filter %.c,$2)) \
                      $(subst .y,.o,$(filter %.y,$2)) \
                      $(subst .l,.o,$(filter %.l,$2)))
      $(AR) $(ARFLAGS) $$@ $$^
endef

# $(call generated-source, source-file-list)
generated-source = $(call source-dir-to-binary-dir,   \
                     $(subst .y,.c,$(filter %.y,$1))  \
                     $(subst .y,.h,$(filter %.y,$1))  \
                     $(subst .l,.c,$(filter %.l,$1))) \
                   $(filter %.c,$1)

# $(compile-rules)
define compile-rules
  $(foreach f, $(local_src),\
  $(call one-compile-rule,$(call source-to-object,$f),$f))
endef

# $(call one-compile-rule, binary-file, source-files)
define one-compile-rule
  $1: $(call generated-source,$2)
      $(COMPILE.c) -o $$@ $$<

  $(subst .o,.d,$1): $(call generated-source,$2)
      $(CC) $(CFLAGS) $(CPPFLAGS) $(TARGET_ARCH) -M $$< | \
      $(SED) 's,\($$(notdir $$*)\.o\) *:,$$(dir $$@)\1 $$@: ,' > $$@.tmp
      $(MV) $$@.tmp $$@
endef

modules      := lib/codec lib/db lib/ui app/player
programs     :=
libraries    :=
sources      :=

objects      = $(call source-to-object,$(sources))
dependencies = $(subst .o,.d,$(objects))

include_dirs := $(BINARY_DIR)/lib lib include
CPPFLAGS     += $(addprefix -I ,$(include_dirs))
vpath %.h $(include_dirs)

MKDIR := mkdir -p
MV    := mv -f
RM    := rm -f
SED   := sed
TEST  := test

create-output-directories :=                                      \
    $(shell for f in $(call source-dir-to-binary-dir,$(modules)); \
            do                                                    \
              $(TEST) -d $$f || $(MKDIR) $$f;                     \
            done)

all:

include $(addsuffix /module.mk,$(modules))

.PHONY: all
all: $(programs)

.PHONY: libraries
libraries: $(libraries)

.PHONY: clean
clean:
    $(RM) -r $(BINARY_DIR)

ifneq "$(MAKECMDGOALS)" "clean"
include $(dependencies)
endif
\end{verbatim}
}

В этой версии функция \function{source-to-object} модифицирована так,
чтобы осуществлять исправление путей к бинарным файлам. Поскольку эта
операция осуществляется несколько раз, реализуем её в виде функции:

{\footnotesize
\begin{verbatim}
SOURCE_DIR := /test/book/examples/ch07-separate-binaries-1
BINARY_DIR := /test/book/out/mp3_player_out

# $(call source-dir-to-binary-dir, directory-list)
source-dir-to-binary-dir = $(addprefix $(BINARY_DIR)/, $1)

# $(call source-to-object, source-file-list)
source-to-object = $(call source-dir-to-binary-dir,  \
                     $(subst .c,.o,$(filter %.c,$1)) \
                     $(subst .y,.o,$(filter %.y,$1)) \
                     $(subst .l,.o,$(filter %.l,$1)))
\end{verbatim}
}

Функция \function{make-library} изменена подобным образом, теперь она
добавляет к выходным файлам префикс \variable{BINARY\_DIR}. Теперь мы
используем предыдущую функции \function{subdirectory}, поскольку путь
к каталогам с включаемыми файлами снова стал простым относительным
путём. Есть одна небольшая загвоздка: ошибка в \GNUmake{} 3.80
препятствует вызову функции \function{source-to-object} из новой
версии функции \function{make-library}. Эта ошибка была исправлена в
версии 3.81. Мы можем обойти эту ошибку, подставив вместо вызова
функции \function{source-to-object} её тело.

Рассмотрим теперь по-настоящему неуклюжую часть. Неявные правила не
работают, если доступ к выходному файлу нельзя осуществить с помощью
относительного пути. Например, основное правило компиляции
\command{\%.o: \%.c} отлично работает, когда оба файла находятся в
одном каталоге, или даже если исходный файл располагается в
каком-нибудь подкаталоге, например, \filename{lib/codec/codec.c}.
Если исходный файл располагается в удалённом каталоге, мы можем
указать \GNUmake{} на необходимость его поиска при помощи директивы
\directive{vpath}. Однако если объектный файл располагается в
удалённом каталоге, \GNUmake{} не сможет определить местоположения
этого файла, и цепочка правил будет нарушена. Единственный способ
информировать \GNUmake{} о расположении выходного файла~--- это
предоставить явное правило, соединяющее исходный и объектный файлы:

{\footnotesize
\begin{verbatim}
$(BINARY_DIR)/lib/codec/codec.o: lib/codec/codec.c
\end{verbatim}
}

Такая операция должна быть осуществлена для каждого объектного файла.

Хуже того, эта пара не соответствует неявному правилу \command{\%.o:
\%.c}. Это значит, что нам самим нужно предоставить командный
сценарий, повторяющий правило встроенной базы правил, и, возможно,
повторить этот сценарий много раз. Проблема также касается правила
автоматического составления файлов зависимостей, которое мы
используем. Добавление двух явных правил для каждого объектного файла,
упоминающегося в \Makefile{}'е,~--- это кошмар для человека, который
будет поддерживать программный продукт. Однако мы можем минимизировать
дублирование кода, написав функцию для автоматического составления
этих правил:

{\footnotesize
\begin{verbatim}
# $(call one-compile-rule, binary-file, source-files)
define one-compile-rule
  $1: $(call generated-source,$2)
      $(COMPILE.c) $$@ $$<

  $(subst .o,.d,$1): $(call generated-source,$2)
      $(CC) $(CFLAGS) $(CPPFLAGS) $(TARGET_ARCH) -M $$< | \
      $(SED) 's,\($$(notdir $$*)\.o\) *:,$$(dir $$@)\1 $$@: ,' > $$@.tmp
      $(MV) $$@.tmp $$@
endef
\end{verbatim}
}

Первые две строки функции~--- это явные правила для описания
зависимости объектного файла от исходного. Реквизит для этого правила
должен быть вычислен при помощи функции \function{generated-source},
описанной в главе~\ref{chap:managing_large_proj}. Если этого не
сделать, исходные файлы \utility{lex} и \utility{yacc} вызовут ошибку
компиляции при появлении в командном сценарии (после подстановки
переменной \variable{\$\^}). Автоматические переменные экранированы,
поэтому они будут вычислены позднее, при выполнении командного
сценария (а не при выполнении функции \function{eval}). Функция
\function{generated-source} была модифицирована так, чтобы возвращать
пути к исходным файлам, написанным на языке \Clang{}, неизменными:

{\footnotesize
\begin{verbatim}
# $(call generated-source, source-file-list)
generated-source = $(call source-dir-to-binary-dir,   \
                     $(subst .y,.c,$(filter %.y,$1))  \
                     $(subst .y,.h,$(filter %.y,$1))  \
                     $(subst .l,.c,$(filter %.l,$1))) \
                   $(filter %.c,$1)
\end{verbatim}
}

С учётом этого изменения функция будет возвращать результаты,
представленные в следующей таблице:

\begin{center}
\begin{tabular}{ll}
Аргумент & Результат \\
\filename{lib/db/playlist.y} &
\filename{/c/mp3\_player\_out/lib/db/playlist.c} \\
 & \filename{/c/mp3\_player\_out/lib/db/playlist.h} \\
\filename{lib/db/scanner.l} &
\filename{/c/mp3\_player\_out/lib/db/scanner.c} \\
\filename{app/player/play\_mp3.c} &
\filename{app/player/play\_mp3.c} \\
\end{tabular}
\end{center}

Явное правило для создания файлов зависимостей устроено точно так же.
Ещё раз обратите внимание на экранирование (двойные символы доллара),
которое требуется для правильной работы сценария.

Теперь для каждого исходного файла в модуле нужно вычислить нашу новую
функцию:

{\footnotesize
\begin{verbatim}
# $(compile-rules)
define compile-rules
  $(foreach f, $(local_src),\
    $(call one-compile-rule,$(call source-to-object,$f),$f))
endef
\end{verbatim}
}

Эта функция использует глобальную переменную \variable{local\_src},
переопределяемую в файлах \filename{module.mk}. Более общий подход
заключается в передаче списка файлов в качестве аргумента, однако в
нашем проекте это делать не обязательно. Просто добавим эти функции в
файлы \filename{module.mk}:

{\footnotesize
\begin{verbatim}
local_src := $(subdirectory)/codec.c

$(eval $(call make-library,$(subdirectory)/libcodec.a,$(local_src)))

$(eval $(compile-rules))
\end{verbatim}
}

Мы должны использовать функцию \function{eval}, так как результат
вычисления функции \function{compile-rules} содержит более одной
строки кода \GNUmake{}.

Наконец, последняя сложность. Так как стандартные шаблонные правила
компиляции исходных файлов \Clang{} не могут определить путь к
выходным файлам, неявное правило для \utility{lex} и наше шаблонное
правило для \utility{yacc} также не смогут этого сделать. Мы можем
легко изменить эти правила самостоятельно. Поскольку они больше не
применимы к остальным \utility{lex} и \utility{yacc} файлам, мы можем
вынести их в файл \filename{lib/db/module.mk}:

{\footnotesize
\begin{verbatim}
local_dir := $(BINARY_DIR)/$(subdirectory)
local_src := $(addprefix $(subdirectory)/,playlist.y scanner.l)

$(eval $(call make-library,$(subdirectory)/libdb.a,$(local_src)))

$(eval $(compile-rules))

.SECONDARY: $(call generated-source, $(local_src))

$(local_dir)/scanner.d: $(local_dir)/playlist.d

$(local_dir)/%.c $(local_dir)/%.h: $(subdirectory)/%.y
    $(YACC.y) --defines $<
    $(MV) y.tab.c $(dir $@)$*.c
    $(MV) y.tab.h $(dir $@)$*.h

$(local_dir)/scanner.c: $(subdirectory)/scanner.l
    @$(RM) $@
    $(LEX.l) $< > $@
\end{verbatim}
}

Правила для \utility{lex} файлов реализованы как обычные явные
правила, а правила для \utility{yacc} файлов~--- как шаблонные.
Почему? Потому что правила для \utility{yacc} файлов используются для
сборки двух целей: исходного и заголовочного файла на языке \Clang{}.
Если мы используем обычное явное правило, \GNUmake{} выполнит
командный сценарий дважды: один раз для исходного файла, а второй~---
для заголовочного. Однако \GNUmake{} считает, что шаблонное правило,
определяющее несколько целей, при выполнении обновляет обе цели.  Если
возможно, вместо \Makefile{}'ов, приведённых в этом разделе, я буду
использовать более простой подход, основанный на компиляции из дерева
каталогов бинарных файлов. Как вы могли заметить, при попытке
компиляции из дерева исходных файлов немедленно возникают трудности, и
с ростом размеров проекта эти трудности только растут.

%%--------------------------------------------------------------------
%% Read only source
%%--------------------------------------------------------------------
\section{Объявляем права <<только для чтения>>}

Как только деревья каталогов с исходными и бинарными файлами
разделены, возможность объявления прав доступа <<только для чтения>>
для справочного дерева каталогов с исходным кодом получается
практически бесплатно, если только все объектные файлы, создаваемые
сборкой, помещаются в дерево каталогов бинарных файлов. Однако если
при сборке создаются исходные файлы, мы должны позаботиться о том,
чтобы они также были помещены в дерево каталогов бинарных файлов.

В более простом подходе, основанном на компиляции в бинарном дереве,
все создаваемые файлы помещались в бинарное дерево автоматически, так
как именно из него происходил вызов программ \utility{lex} и
\utility{yacc}. При подходе, основанном на компиляции из дерева
каталогов с исходными файлами, мы были вынуждены указывать явные пути
для исходных и целевых файлов, поэтому спецификация пути к файлу в
бинарном дереве каталогов не потребует дополнительной работы, нужно
просто не забыть сделать это.

Источником остальных препятствий для объявления дерева каталогов с
исходными файлами доступным только для чтения обычно является среда.
Часто система сборки, доставшаяся вам по наследству, содержит
действия, создающие файлы в дереве каталогов с исходными файлами, так
как первоначальный её автор не осознал преимущества исходного дерева,
доступного только для чтения. В качестве примеров можно привести
автоматически составленную документацию, файлы журналов и временные
файлы. Перемещение этих файлов в дерево каталогов с бинарными файлами
может потребовать значительных усилий, однако, если нужна поддержка
нескольких деревьев каталогов бинарных файлов, эта жертва является
необходимой. Альтернативой является поддержка и синхронизация
нескольких идентичных деревьев каталогов с исходными файлами.

%%--------------------------------------------------------------------
%% Dependency generation
%%--------------------------------------------------------------------
\section{Генерация зависимостей}

Небольшое введение в технику автоматической генерации зависимостей
можно найти в разделе <<\nameref{sec:auto_dep_gen}>>
главы~\ref{chap:rules}, однако это введение умалчивает о ряде важных
проблем. Этот раздел описывает несколько альтернатив
простого решения, рассмотренного нами ранее\footnote{%
Б\'{о}льшая часть материала, рассмотренного в этом разделе, была
разработана Томом Троми (Tom Tromey, tromey@cygnus.com) для проекта
GNU \utility{automake} и взята из прекрасной сводной статьи Пола Смита
(Paul Smith, программист, осуществляющий поддержку GNU \GNUmake{}),
прочитать которую можно на его веб-сайте
\filename{\url{http://make.paulandlesley.org}}. (прим. автора)}.
В частности, описанный ранее подход, предлагаемый руководством
пользователя GNU \GNUmake{}, обладает следующими недостатками:

\begin{itemize}
%---------------------------------------------------------------------
\item Он неэффективен. Когда \GNUmake{} обнаруживает, что файл
зависимостей не существует или устарел, он обновляет этот файл и
стартует заново. Повторное чтение \Makefile{}'а может быть
неэффективным, если во время чтения осуществляется много задач или
требуется анализ графа зависимостей.
%---------------------------------------------------------------------
\item Каждый раз при запуске сборки после добавления новых исходных
файлов \GNUmake{} выдаёт предупреждение. На момент запуска файл
зависимостей, ассоциированный с новым исходным файлом, ещё не
существует, поэтому при попытке прочитать этот файл зависимостей
\GNUmake{} выдаст предупреждение до того, как осуществит генерацию
файла. Это не критично, но порой очень раздражает.
%---------------------------------------------------------------------
\item Если вы удалите исходный файл, при попытке осуществления сборки
\GNUmake{} будет завершать своё выполнение с ошибкой. Причиной ошибки
является существование файл зависимостей, содержащего удалённый
исходный файл в качестве реквизита. Поскольку \GNUmake{} не может
найти удалённый файл и не имеет правила для его сборки, выдаётся
следующее сообщение об ошибке:

{\footnotesize
\begin{verbatim}
make: *** No rule to make target foo.h, needed by foo.d. Stop.
\end{verbatim}
}

Более того, из-за этой ошибки \GNUmake{} не сможет обновить файл
зависимостей. Единственный возможный выход~--- удалить файл
зависимостей вручную, однако обычно найти эти файлы нелегко, и
пользователи, как правило, удаляют все файлы зависимостей и
осуществляют чистую сборку. Та же проблема возникает при
переименовании файлов.

Обратите внимание на то, что эта проблема чаще всего появляется при
удалении или переименовании заголовочных (\filename{.h}) файлов.
Причина этого заключается в том, что \filename{.c} файлы будут удалены
из списка зависимостей автоматически и не вызовут проблем при сборке.
%---------------------------------------------------------------------
\end{itemize}

%---------------------------------------------------------------------
% Tromey's way
%---------------------------------------------------------------------
\subsection{Решение Троми}

Давайте разбирать проблемы по очереди.

Как нам избежать повторного запуска \GNUmake{}?

После небольшого размышления можно понять, что повторный запуск
\GNUmake{} не требуется. Если файл зависимостей обновлён, это значит,
что хотя бы один его реквизит изменился, что, в свою очередь, значит,
что нам нужно собрать целевой файл заново. На данном этапе \GNUmake{}
не нуждается в дополнительной информации о зависимостях, поскольку она
не изменит его поведения. Однако нам нужно, чтобы файл зависимостей
был обновлён, чтобы при следующем запуске \GNUmake{} обладал полной
информацией о зависимостях.

Поскольку в текущей сборке нам не нужен файл зависимостей, мы можем
обновить его при сборке целевого файла. Мы можем добиться этого путём
соответствующего изменения правила компиляции:

{\footnotesize
\begin{verbatim}
# $(call make-depend,source-file,object-file,depend-file)
define make-depend
  $(CC) $(CFLAGS) $(CPPFLAGS) $(TARGET_ARCH) -M $1 | \
  $(SED) 's,\($$(notdir $2)\) *:,$$(dir $2) $3: ,' > $3.tmp
  $(MV) $3.tmp $3
endef

%.o: %.c
    $(call make-depend,$<,$@,$(subst .o,.d,$@))
    $(COMPILE.c) -o $@ $<
\end{verbatim}
}

Мы реализовали возможность создания файлов зависимостей в форме
функции \function{make-depend}, принимающей в качестве аргументов
имена исходного и объектного файлов, а также имя файла
зависимостей. Это предоставляет нам максимальную гибкость на случай,
если мы решим повторно использовать эту функцию в другом контексте.
После подобного изменения правила компиляции следует удалить шаблонное
правило \command{\%.d: \%.c}, это позволит избежать повторного
составления файла зависимостей.

Теперь объектный файл и файл зависимостей логически связаны: если
существует один из них, должен существовать и второй. Таким образом,
нам не нужно больше беспокоиться об отсутствии файла зависимостей.
Если он не существует, объектный файл тоже не существует, оба этих
файла будут созданы при следующей сборке. Теперь мы можем игнорировать
любые предупреждения, возникающие из-за отсутствия файлов
зависимостей.

В разделе <<\nameref{sec:cond_inc_processing}>> главы~\ref{chap:vars}
была описана альтернативная форма директивы \directive{include},
\index{Директивы!sinclude@\directive{-include}}
\index{Директивы!sinclude@\directive{sinclude}}
\directive{-include} или (\directive{sinclude}), игнорирующая ошибки и
не выдающая предупреждений:

{\footnotesize
\begin{verbatim}
ifneq "$(MAKECMDGOALS)" "clean"
  -include $(dependencies)
endif
\end{verbatim}
}

Это решает вторую проблему~--- раздражающие сообщения, возникающие при
отсутствии файлов зависимостей.

Наконец, мы можем избежать предупреждений об отсутствующих реквизитах
с помощью небольшого трюка. Трюк заключается в спецификации для
отсутствующего файла цели без реквизитов и команд. Предположим для
примера, что генератор зависимостей создал следующую зависимость:

{\footnotesize
\begin{verbatim}
target.o target.d: header.h
\end{verbatim}
}

Допустим теперь, что в результате рефакторинга кода файл
\filename{header.h} был удалён. При следующем запуске \Makefile{}'а мы
получим следующую ошибку:

{\footnotesize
\begin{verbatim}
make: *** No rule to make target header.h, needed by target.d. Stop.
\end{verbatim}
}

Однако если мы добавим цель \target{header.h}, не имеющую
ассоциированного командного сценария, ошибки не будет:

{\footnotesize
\begin{verbatim}
target.o target.d: header.h
header.h:
\end{verbatim}
}

Это происходит потому, что если файл \filename{header.h} не
существует, он просто помечается устаревшим и все цели, имеющие этот
файл в качестве реквизита, собираются заново. Таким образом, файл
зависимостей будет создан заново и уже не будет содержать
\filename{header.h}. Если же файл \filename{header.h} существует,
\GNUmake{} просто продолжит выполнение.  Теперь всё, что нужно
сделать~--- это убедиться в том, что каждый реквизит имеет
соответствующее пустое правило. Этот вид привил впервые встретился нам
в разделе <<\nameref{sec:phony_targets}>> главы~\ref{chap:rules}. Ниже
приведена версия функции \function{make-depend}, добавляющая новую
цель:

{\footnotesize
\begin{verbatim}
# $(call make-depend,source-file,object-file,depend-file)
define make-depend
  $(CC) $(CFLAGS) $(CPPFLAGS) $(TARGET_ARCH) -M $1 |        \
  $(SED) 's,\($$(notdir $2)\) *:,$$(dir $2) $3: ,' > $3.tmp
  $(SED) -e 's/#.*//'                                       \
         -e 's/^[^:]*: *//'                                 \
         -e 's/ *\\$$$$//'                                  \
         -e '/^$$$$/ d'                                     \
         -e 's/$$$$/ :/' $3.tmp >> $3.tmp
  $(MV) $3.tmp $3
endef
\end{verbatim}
}

Чтобы создать дополнительные правила, мы применяем новую команду
\utility{sed} к файлу зависимостей. Этот кусок кода \utility{sed}
осуществляет пять преобразований: \begin{enumerate} \item Удаляет
комментарии.  \item Удаляет целевые файлы и соответствующие пробелы.
\item Удаляет оконечные пробелы.  \item Удаляет пустые строки.  \item
Добавляет в конец каждой строки двоеточие.  \end{enumerate} (GNU
\utility{sed} может читать файл и добавлять к нему текст в одной
команде, предохраняя нас от необходимости использования второго
временного файла. Этот код может не работать в других системах.)
Новая версия команды \utility{sed} принимает на вход текст следующего
вида:

{\footnotesize
\begin{verbatim}
# любые комментарии
target.o target.d: prereq1 prereq2 prereq3 \
    prereq4
\end{verbatim}
}

{\flushleft и преобразует его к виду:}

{\footnotesize
\begin{verbatim}
prereq1 prereq2 prereq3:
prereq4:
\end{verbatim}
}

Таким образом, функция \function{make-depend} добавляет новые цели к
исходному файлу зависимостей. Это решает проблему <<No rule to make
target>>.

%---------------------------------------------------------------------
% makedepend programs
%---------------------------------------------------------------------
\subsection{Программы \utility{makedepend}}

Всё это время мы могли использовать опцию \command{-M}, которой
обладает б\'{о}льшая часть компиляторов, но что бы мы делали, если бы
этой опции не существовало? Кроме того, есть ли более удачные решения,
чем использование опции \command{-M}?

На данный момент практически все компиляторы языка \Clang{} имеют
поддержку генерации зависимостей исходных файлов, однако не так давно
всё было иначе. На заре проекта X Window System его разработчики
создали программу \utility{makedepend}, определяющую зависимости для
заданного набора исходных файлов \Clang{} и \Cplusplus{}. Доступ к
этой программе можно получить бесплатно через сеть Internet.
Использовать эту программу немного неудобно, поскольку она написана
так, чтобы добавлять свой вывод в \Makefile{}, чего нам не хотелось
бы. Программа \utility{makedepend} подразумевает, что объектные файлы
располагаются в том же каталоге, что и исходные. Это, в свою очередь,
означает, что нам нужно изменить сценарий \utility{sed}:

{\footnotesize
\begin{verbatim}
# $(call make-depend,source-file,object-file,depend-file)
define make-depend
  $(MAKEDEPEND) -f- $(CFLAGS) $(CPPFLAGS) $(TARGET_ARCH) $1 | \
  $(SED) 's,^.*/\([^/]*\.o\) *:,$(dir $2)\1 $3: ,' > $3.tmp
  $(SED) -e 's/#.*//'                                         \
         -e 's/^[^:]*: *//'                                   \
         -e 's/ *\\$$$$//'                                    \
         -e '/^$$$$/ d'                                       \
         -e 's/$$$$/ :/' $3.tmp >> $3.tmp
  $(MV) $3.tmp $3
endef
\end{verbatim}
}

Опция \command{-f-} программы \utility{makedepend} означает, что
информация о зависимостях должна выводиться на стандартный поток
вывода.

Альтернативой использованию \utility{makedepend} или вашего
собственного компилятора является использование компилятора
\utility{gcc}. Этот компилятор имеет огромное количество опций для
составления информации о зависимостях. Наиболее подходящими для нашего
случая выглядят опции, используемые в следующем примере:

{\footnotesize
\begin{verbatim}
ifneq "$(MAKECMDGOALS)" "clean"
  -include $(dependencies)
endif

# $(call make-depend,source-file,object-file,depend-file)
define make-depend
  $(GCC) -MM            \
         -MF $3         \
         -MP            \
         -MT $2         \
         $(CFLAGS)      \
         $(CPPFLAGS)    \
         $(TARGET_ARCH) \
         $1
endef

%.o: %.c
    $(call make-depend,$<,$@,$(subst .o,.d,$@))
    $(COMPILE.c) $(OUTPUT_OPTION) $<
\end{verbatim}
}

Опция \command{-MM} призывает \utility{gcc} убрать все системные
заголовочные файлы из списка реквизитов. Это удобно, так как эти файлы
меняются редко (если вообще меняются), и помогает сократить
беспорядок, возникающий с ростом и усложнением системы сборки.
Изначально эта опция могла быть введена из соображений
производительности. Однако при использовании современных процессоров
разница в производительности едва ли может быть измерена.

Опция \command{-MF} специфицирует имя файла зависимостей. В качестве
имени будет использоваться имя объектного файла, расширение которого
заменяется на \filename{.d}. \utility{gcc} имеет ещё одну опцию,
\command{-MD} или \command{-MMD}, которая автоматически определяет имя
файла зависимостей, используя правило, подобное описанному выше. В
общем случае мы предпочли бы использовать эту опцию, однако встроенное
правило компилятора не сможет добавить соответствующий относительный
путь к каталогу с объектными файлами, вместо этого оно просто поместит
файлы зависимостей в текущий каталог. Поэтому мы вынуждены делать эту
работу самостоятельно и использовать опцию \command{-MF}.

Опция \command{-MP} призывает \utility{gcc} включать для каждого
реквизита абстрактную цель. Это делает ненужным наше неуклюжее
пятизвенное выражение для \utility{sed}, использовавшееся в функции
\function{make-depend}. Похоже, эту опцию добавили в \utility{gcc} по
просьбе разработчиков \utility{automake}, которые изобрели технику
абстрактных целей. 

Наконец, опция \command{-MT} специфицирует строку, которая будет
использоваться для целей в файле зависимостей. Повторим, без этой
опции \utility{gcc} не сможет включить относительный путь к каталогу
объектных файлов.

Используя \utility{gcc}, мы можем заменить четыре команды, требующиеся
для генерации зависимостей, одной. Даже если вы используете
коммерческий компилятор, вы можете использовать \utility{gcc} для
управления зависимостями.

%%--------------------------------------------------------------------
%% Supporting multiple binary trees
%%--------------------------------------------------------------------
\section{Поддержка нескольких каталогов бинарных файлов}
\label{sec:supporting_multiple_binary_trees}

После реализации \Makefile{}'а, осуществляющего запись бинарных файлов
в отдельное дерево каталогов, реализовать поддержку множества таких
деревьев довольно просто. Для интерактивных сборок, инициируемых
разработчиками при помощи клавиатуры, требуется совсем мало
подготовки. Разработчик создаёт каталог для бинарных файлов, переходит
в него и вызывает \GNUmake{}, указав нужный \Makefile{}.

{\footnotesize
\begin{alltt}
\$ \textbf{mkdir -p ~/work/mp3\_player\_out}
\$ \textbf{cd ~/work/mp3\_player\_out}
\$ \textbf{make -f ~/work/mp3\_player/makefile}
\end{alltt}
}

Если процесс запуска сборки требует от разработчика больше участия, то
сценарий\hyp{}обёртка будет наилучшим решением. Этот сценарий может
также анализировать текущий каталог и выставлять соответствующим
образом переменные окружения, используемые в \Makefile{}'е (например,
\variable{BINARY\_DIR}).

{\footnotesize
\begin{verbatim}
#! /bin/bash
# Работаем в каталоге с исходными файлами.
curr=$PWD
export SOURCE_DIR=$curr
while [[ $SOURCE_DIR ]]
do
  if [[ -e $SOURCE_DIR/[Mm]akefile ]]
  then
    break;
  fi
  SOURCE_DIR=${SOURCE_DIR%/*}
done

# Если makefile не найден, выводим сообщение об ошибке.
if [[ ! $SOURCE_DIR ]]
then
  printf "run-make: Cannot find a makefile" > /dev/stderr
  exit 1
fi

# Если каталог для выходных файлов не задан, используем значение
# по умолчанию.
if [[ ! $BINARY_DIR ]]
then
  BINARY_DIR=${SOURCE_DIR}_out
fi

# Создаём каталог для бинарных файлов.
mkdir --parents $BINARY_DIR

# Запускаем make.
make --directory="$BINARY_DIR" "$@"
\end{verbatim}
}

Этот сценарий не очень сложен. Он производит поиск \Makefile{}'а в
текущем каталоге, и в случае неудачи поднимается вверх по дереву
каталогов, пока не найдёт \Makefile{}. Затем происходит проверка
наличия переменной окружения, содержащей каталог для бинарных файлов.
Если переменная не определена, ей присваивается значение по умолчанию,
получаемое добавлением суффикса <<\_out>> к имени каталога с исходными
файлами. Затем сценарий создаёт каталог для бинарных файлов и
осуществляет запуск \GNUmake{}.

Если осуществляются сборки для различных платформ, требуются методы
для определения нужной платформы. Наиболее простой подход требует от
разработчика определения переменной окружения для каждого типа
платформы и добавления условных директив, использующих эту переменную,
в \Makefile{} и в исходный код. Лучшим подходом является
автоматическое определение платформы на основании вывода программы
\utility{uname}.

{\footnotesize
\begin{verbatim}
space := $(empty) $(empty)
export MACHINE := $(subst $(space),-,$(shell uname -smo))
\end{verbatim}
}

Я считаю, что при автоматическом запуске сборки программой
\utility{cron} использование вспомогательного сценария командного
интерпретатора является более предпочтительным подходом, нежели прямой
вызов \GNUmake{}. Сценарий-обёртка предоставляет больше возможностей
для подготовки, обработки ошибок и завершения автоматизированной
сборки. Сценарий также является подходящим местом для определения
переменных и опций командной строки.

Наконец, если проект поддерживает фиксированное число деревьев
каталогов и платформ, вы можете использовать имена каталогов для
автоматического определения параметров текущей сборки. Например:

{\footnotesize
\begin{verbatim}
ALL_TREES := /builds/hp-386-windows-optimized \
             /builds/hp-386-windows-debug     \
             /builds/sgi-irix-optimzed        \
             /builds/sgi-irix-debug           \
             /builds/sun-solaris8-profiled    \
             /builds/sun-solaris8-debug

BINARY_DIR := $(foreach t,$(ALL_TREES),\
                $(filter $(ALL_TREES)/%,$(CURDIR)))

BUILD_TYPE := $(notdir $(subst -,/,$(BINARY_DIR)))

MACHINE_TYPE := $(strip              \
                  $(subst /,-,       \
                    $(patsubst %/,%, \
                      $(dir          \
                        $(subst -,/, \
                          $(notdir $(BINARY_DIR)))))))
\end{verbatim}
}

Переменная \variable{ALL\_TREES} содержит список всех возможных
каталогов бинарных файлов. Цикл \command{foreach} осуществляет
проверку соответствия текущего каталога одному из возможных каталогов
бинарных файлов, причём соответствовать может только один каталог. Как
только каталог определён, мы можем извлечь из имени каталога параметры
сборки (например, оптимизированная, отладочная или профилировочная).
Мы получаем последний компонент имени каталога, преобразуя набор слов,
разделённых запятой, в набор слов, разделённых слэшем, и извлекая
последнее слово этого набора при помощи функции \function{notdir}.
Извлечение названия целевой платформы осуществляется таким же
способом.

%%--------------------------------------------------------------------
%% Partial source trees
%%--------------------------------------------------------------------
\section{Частичные рабочие копии}

В по-настоящему больших проектах простое создание рабочей копии и
поддержка исходного кода может быть тяжким бременем для разработчиков.
Если система состоит из большого числа модулей, и каждый разработчик
работает над небольшой её частью, создание полной рабочей копии и
компиляция всего проекта может быть непозволительной тратой времени.
Вместо этого можно использовать централизованные справочные
ночные сборки, служащие базой для заполнения недостающих файлов в
деревьях каталогов исходных и бинарных файлов разработчиков.

Реализация этого функционала потребует осуществления двух типов
поиска. Во-первых, если компилятору недостаёт заголовочного файла,
нужно дать ему инструкцию искать этот файл в справочном дереве
каталогов исходных файлов. Во-вторых, если \Makefile{}'у требуется
какая-то библиотека, нужно дать ему инструкцию искать её в справочном
дереве каталогов бинарных файлов. Для того, чтобы помочь компилятору
найти недостающий исходный код, мы можем просто указать дополнительную
опцию \command{-I} после аналогичной опции, специфицирующей локальные
каталоги заголовочных файлов. Чтобы помочь \GNUmake{} найти
библиотеки, мы можем указать дополнительные каталоги в директиве
\directive{vpath}.

{\footnotesize
\begin{verbatim}
SOURCE_DIR     := ../mp3_player
REF_SOURCE_DIR := /reftree/src/mp3_player
REF_BINARY_DIR := /binaries/mp3_player
...
include_dirs := lib $(SOURCE_DIR)/lib $(SOURCE_DIR)/include
CPPFLAGS     += $(addprefix -I ,$(include_dirs))                  \
                $(addprefix -I $(REF_SOURCE_DIR)/,$(include_dirs))
vpath %.h       $(include_dirs)                                   \
                $(addprefix $(REF_SOURCE_DIR)/,$(include_dirs))

vpath %.a       $(addprefix $(REF_BINARY_DIR)/lib/, codec db ui)
\end{verbatim}
}

Использование этого подхода предполагает, что наименьшей единицей,
которую можно извлечь из репозитория CVS, является библиотека или
программный модуль. В этом случае \GNUmake{} сможет пропустить
недостающие библиотеки и каталоги, если разработчик решил не делать их
рабочих копий. Когда будет нужно использовать эти библиотеки,
спецификация пути поиска поможет автоматически заполнить недостающие
файлы.

Переменная \variable{modules} нашего \Makefile{}'а содержит список
подкаталогов, в которых следует осуществлять поиск файлов
\filename{module.mk}. Если эти подкаталоги не содержатся в рабочей
копии, нужно удалить эти подкаталоги из списка. Кроме того, можно
присваивать значение переменной \variable{modules} при помощи функции 
\function{wildcard}:

{\footnotesize
\begin{verbatim}
modules := $(dir $(wildcard lib/*/module.mk))
\end{verbatim}
}

Это выражение вернёт список всех каталогов, содержащих файл
\filename{module.mk}. Заметьте, что благодаря использованию функции
\function{dir} имя каждого каталога будет оканчиваться слэшем.

\GNUmake{} также может осуществлять поддержку создания частичных
рабочих копий на уровне отдельных файлов, при сборке библиотеки
соединяя объектные файлы из локальной копии разработчика и, в случае
необходимости, из справочного дерева каталогов. Однако этот подход
имеет множество недостатков и, судя по моему опыту, разработчикам он
приносит больше вреда, чем пользы.

%%--------------------------------------------------------------------
%% Reference builds, libraries, and installers
%%--------------------------------------------------------------------
\section{Справочные сборки, библиотеки и инсталляторы}

Мы уже рассмотрели все средства, необходимые для реализации справочных
сборок. Настройка головного \Makefile{}'а для добавления этого
функционала не займёт много усилий. Просто заменим простые
присваивания значений переменных \variable{SOURCE\_DIR} и
\variable{BINARY\_DIR} условными (\command{?=}). Сценарий,
выполняемый программой \utility{cron}, может быть построен с
использованием следующего подхода:

\begin{enumerate}
%---------------------------------------------------------------------
\item Перенаправить поток вывода и определить имена файлов журналов.
%---------------------------------------------------------------------
\item Очистить каталоги старых сборок и удалить лишние файлы из
справочного дерева каталогов исходных файлов.
%---------------------------------------------------------------------
\item Сделать рабочую копию свежей версии исходного кода.
%---------------------------------------------------------------------
\item Определить переменные, отвечающие за расположение исходных и
бинарных файлов.
%---------------------------------------------------------------------
\item Осуществить вызов \GNUmake{}.
%---------------------------------------------------------------------
\item Проверить, содержат ли файлы журналов ошибки.
%---------------------------------------------------------------------
\item Составить файл символов (TAGS-файл), и, при необходимости,
\index{locate database} \index{База данных!файлов}
обновить базу данных файлов (locate database)\footnote{
База данных файлов~--- это совокупность всех имён файлов, существующих
в файловой системе. Использование такой базы позволяет быстро находить
файлы по имени. Трудно переоценить полезность такой базы при
управлении большими проектами. Я предпочитаю реализовывать
автоматическое обновление этой базы после ночной сборки.}.
%---------------------------------------------------------------------
\item Послать письмо с отчётом об успешном (или неудачном) завершении
сборки.
%---------------------------------------------------------------------
\end{enumerate}

В модели разработки с применением справочных сборок удобно
поддерживать несколько старых сборок на случай, если чьё-то
злонамеренное вмешательство повредит дерево каталогов. Я обычно храню
7 или 14 ночных сборок. Разумеется, сценарий, осуществляющий ночные
сборки, описывает свои действия в файле журнала и удаляет устаревшие
сборки и файлы журналов. Поиск ошибок в файле журнала обычно
осуществляется при помощи сценария \utility{awk}. Наконец, я использую
сценарий, обновляющий файл \filename{latest}, являющийся символической
ссылкой на последнюю сборку. Для определения успешности сборки я
включаю в каждый \Makefile{} цель \target{validate}. Сценарий,
ассоциированный с этой целью, осуществляет проверку наличия всех
необходимых целевых файлов:

{\footnotesize
\begin{verbatim}
.PHONY: validate_build
validate_build:
    test $(foreach f,$(RELEASE_FILES),-s $f -a) -e .
\end{verbatim}
}

Этот сценарий проверяет, что файлы, которые должны появиться в
результате сборки, существуют и не пусты. Разумеется, такой подход
никогда не заменит тестирования, однако он является удобной базовой
проверкой целостности сборки. Если этот тест не пройден, \GNUmake{}
завершается с ошибкой, и сценарий, осуществляющий ночную сборку,
сохраняет ссылку \filename{latest} указывающей на предыдущую сборку.

\index{Библиотека!сторонних разработчиков}
Сторонние библиотеки всегда немного мешают управлению проектом. Я
согласен с распространённым убеждением, что хранение больших бинарных
файлов в CVS является не лучшей идеей. Причина кроется в том, что CVS
не может хранить отличия бинарных файлов с помощью
дельта\hyp{}кодирования, в результате чего файлы системы RCS, лежащей
в основе CVS, могут вырасти до огромных размеров. Хранение файлов
больших размеров в репозитории CVS существенно замедляет многие
базовые операции CVS, что сказывается на всём цикле разработки.

Если библиотеки сторонних разработчиков не хранятся в CVS, нужно
управлять ими каким-то другим способом. Моим предложением является
создание в справочном дереве каталога библиотек и включение версии
каждой библиотеки в имя соответствующего каталога, как показано на
рисунке~\ref{fig:third_party_libs}.

\begin{figure}[b]
\begin{verbatim}
reftree
`--third-party
   |--oracle-8.0.7sp2
   `--oracle-9.0.1.1
\end{verbatim}
\caption{Структура каталогов, используемая для управления библиотеками
сторонних разработчиков.}
\label{fig:third_party_libs}
\end{figure}

Имена этих каталогов могут использоваться в \Makefile{}'е:

{\footnotesize
\begin{verbatim}
ORACLE_9011_DIR ?= /reftree/third_party/oracle-9.0.1.1/Ora90
ORACLE_9011_JAR ?= $(ORACLE_9011_DIR)/jdbc/lib/classes12.jar
\end{verbatim}
}

\index{Инсталлятор}
Когда поставщик изменит версию своей библиотеки, создайте новый
каталог в справочном дереве и объявите новую переменную в
\Makefile{}'е. При использовании этого подхода \Makefile{}, должным
образом поддерживаемый при помощи меток и ветвей системы контроля
версий, всегда будет явно отражать версию используемой библиотеки.

Создание и поддержка инсталляторов также является сложной проблемой. Я
уверен, что отделение базового процесса сборки от процесса создания
инсталлятора является правильным решением. Инструменты для создания
инсталляторов, существующие на момент написания этой книги, сложны и
неустойчивы. Соединение этих инструментов с системой сборки (часто
также являющейся сложной и неустойчивой) породит чрезвычайно сложную в
поддержке систему. Вместо этого сценарий базовой сборки может помещать
бинарные файлы в каталог продукта, содержащий все (или почти все)
данные, необходимые инструменту создания инсталляторов. Управления
этим инструментом может осуществляться при помощи собственного
\Makefile{}'а, в конечном счёте производящего исполняемый файл
установки.


%%%-------------------------------------------------------------------
%%% Java
%%%-------------------------------------------------------------------
\chapter{\Java{}}
\label{chap:java}

\index{Интегрированные среды разработки}
Многие Java\hyp{}разработчики предпочитают использовать
интегрированные среды разработки (Integrated Development Environments,
IDE), например, Eclipse. У вас может возникнуть вопрос, зачем вам
нужно использовать \GNUmake{} в \Java{} проектах, если есть такие
известные альтернативы, как Ant и среды разработки \Java{}? Эта глава
содержит исследование значения \GNUmake{} в среде \Java{}, в
частности, в ней приводится универсальный \Makefile{}, который может
быть помещён с минимальными модификациями практически в любой
\Java{}\hyp{}проект для осуществления всех стандартных задач сборки.

Использование \GNUmake{} в совокупности с \Java{} поднимает несколько
проблем и предоставляет некоторые дополнительные возможности. Причиной
этого является сочетание трёх основных факторов: во-первых, компилятор
\Java{} работает очень быстро; во-вторых, стандартный компилятор
\Java{} поддерживает синтаксис \command{@fi\-le\-na\-me} для чтения
параметров командной строки из файла; в третьих, если в коде
\Java{}\hyp{}класса указан пакет, путь к \filename{.class}\hyp{}файлу
определяется однозначно.

Стандартный компилятор \Java{} работает очень быстро. Главной причиной
этого является принцип работы директивы \directive{import}. Подобно
директиве \directive{\#include} препроцессора языка \Clang{}, эта
директива используется для обеспечения доступа к внешним символам.
Однако вместо повторного чтения исходного кода, который затем
потребует повторного разбора и анализа, компилятор \Java{} считывает
файлы классов напрямую. Поскольку символы, определяемые в файле
класса, не могут измениться в процессе компиляции, компилятор
производит кэширование классов. Даже в случае проектов среднего
размера это означает, что компилятор \Java{} избавлен от необходимости
повторно считывать, разбирать и анализировать буквально миллионы строк
кода, с которыми пришлось бы работать компилятору языка \Clang{}.
Менее существенный прирост производительности достигается за счёт
свед\'{е}ния к минимуму оптимизаций, выполняемых большинством
компиляторов \Java{}. Вместо статической оптимизации предпочтение
отдаётся сложным оптимизациям времени выполнения (just-in-time, JIT),
осуществляемым виртуальной машиной \Java{} (\Java{} virtual machine,
JVM).

\index{Java!пакет}
Практически все крупные \Java{}\hyp{}проекты интенсивно используют
\newword{пакеты} (\newword{pack\-ages}). Каждый класс инкапсулируется в
пакет, определяющий область видимости символов, определённых в файле.
Имена пакетов имеют иерархическую структуру и неявно определяют
структуру файловой системы, предназначенную для их хранения. Например,
пакет \command{a.b.c} неявно определяет структуру каталогов
\filename{a/b/c}. Код, объявленный соответствующей директивой как
принадлежащий пакету \command{a.b.c}, будет скомпилирован в файлы
классов и помещён в каталог \filename{a/b/c}. Это означает, что
обычный алгоритм \GNUmake{}, отвечающий за ассоциацию бинарных файлов
с соответствующими исходными файлами, не будет работать правильно.
Однако это также означает, что нам больше не нужно указывать опцию
\command{-o} для спецификации каталога, предназначенного для
размещения объектного файла. Достаточно указать корень дерева
каталогов бинарных файлов, одинаковый для всех исходных файлов. Это, в
свою очередь, означает, что исходный код из различных каталогов может
быть скомпилирован одной и той же командой.

Все стандартные компиляторы \Java{} поддерживают синтаксис
\command{@fi\-le\-na\-me}, позволяющий считывать параметры командной
строки из файла. Это имеет большое значение в сочетании с функционалом
пакетов, поскольку позволяет производить компиляцию всего исходного
кода единственным вызовом компилятора. Такой подход даёт значительный
выигрыш в производительности, так как время, требуемое для загрузки и
работы компилятора, является значительной частью времени выполнения
сборки.

Итак, после составления соответствующей командной строки, компиляция
400\,000 строк \Java{}\hyp{}кода занимает около трёх минут при
использовании процессора Pentium 4 (2,5ГГц). Компиляция эквивалентного
по размеру приложения, написанного на \Cplusplus{}, потребует
нескольких часов.

%%--------------------------------------------------------------------
%% Alternatives to make
%%--------------------------------------------------------------------
\section{Альтернативы \GNUmake{}}

Как уже было замечено, сообщество \Java{}\hyp{}разработчиков с
энтузиазмом принимает новые технологии. Рассмотрим две из них, имеющие
отношение к \GNUmake{}~--- \utility{Ant} и интегрированные среды
разработки.

%---------------------------------------------------------------------
% Ant
%---------------------------------------------------------------------
\subsection{Ant}
\index{Ant@\utility{Ant}}
Сообщество \Java{}\hyp{}разработчиков очень активно и производит новые
инструменты с впечатляющей скоростью. Одним из таких инструментов
является \utility{Ant}~--- система сборки, призванная занять место
\GNUmake{} в процессе разработки \Java{}\hyp{}приложений. Как и
\GNUmake{}, \utility{Ant} использует файл спецификации для определения
целей и реквизитов проекта. В отличие от \GNUmake{}, \utility{Ant}
написан на языке \Java{} и принимает файлы спецификации в формате XML.

Чтобы дать вам представление о файле спецификации в формате XML,
приведу небольшую выдержку из файла сборки для \utility{Ant}:

{\footnotesize
\begin{verbatim}
<target name="build"
        depends="prepare, check_for_optional_packages"
        description="--> compiles the source code">
  <mkdir dir="${build.dir}"/>
  <mkdir dir="${build.classes}"/>
  <mkdir dir="${build.lib}"/>

  <javac srcdir="${java.dir}"
         destdir="${build.classes}"
         debug="${debug}"
         deprecation="${deprecation}"
         target="${javac.target}"
         optimize="${optimize}" >
    <classpath refid="classpath"/>
  </javac>
  
  ...

  <copy todir="${build.classes}">
    <fileset dir="${java.dir}">
      <include name="**/*.properties"/>
      <include name="**/*.dtd"/>
    </fileset>
  </copy>
</target>
\end{verbatim}
}

Как вы могли заметить, цель объявляется при помощи XML тега
\command{<target>}. Каждая цель имеет имя и список зависимостей,
указанных в атрибутах \command{name} и \command{depends}
\index{Ant!задачи}
соответственно. Действия, выполняемые \utility{Ant}, называются
\newword{задачами} (\newword{tasks}). Задачи реализованы на языке
\Java{} и привязаны к XML тегу. Например, задача создания каталога
специфицируется при помощи тега \command{<mkdir>} и вызывает
выполнение метода \command{Mkdir.execute}, который в конечном итоге
вызывает метод \command{File.mkdir}. Насколько это возможно, все
задачи реализуются средствами \Java{} API.

Эквивалентный файл сборки \GNUmake{} содержит следующий код:

{\footnotesize
\begin{verbatim}
# производит компиляцию исходного кода
build: $(all_javas) prepare check_for_optional_packages
    $(MKDIR) -p $(build.dir) $(build.classes) $(build.lib)
    $(JAVAC) -sourcepath $(java.dir) \
             -d $(build.classes)     \
             $(debug)                \
             $(deprecation)          \
             -target $(javac.target) \
             $(optimize)             \
             -classpath $(classpath) \
             @$<
    ...
    $(FIND) . \( -name '*.properties' -o -name '*.dtd' \) | \
    $(TAR) -c -f - -T - | $(TAR) -C $(build.classes) -x -f -

\end{verbatim}
}

Отрывок кода, приведённый выше, использует техники, которые мы ещё не
обсуждали. Пока удовлетворимся тем, что реквизит \target{all.javac}
содержит список всех \filename{java} файлов, которые нужно
скомпилировать. Задачи \utility{Ant} \command{<mkdir>},
\command{<javac>} и \command{<copy>} также осуществляют проверку
зависимостей. К примеру, если каталог уже существует, задача
\command{mkdir} не выполнит никаких действий. Более того, если файлы
\Java{}\hyp{}классов имеют более позднюю дату модификации, чем
соответствующие исходные файлы, компиляция не будет осуществляться.
Тем не менее, командный сценарий \GNUmake{} осуществляет по существу
такие же функции. \utility{Ant} включает общую задачу, именуемую
\command{<exec>}, используемую для запуска локальных программ.

\utility{Ant} использует искусный и оригинальный подход, однако, при
его использовании возникает несколько проблем, которые стоит
рассмотреть:

\begin{itemize}
%---------------------------------------------------------------------
\item Несмотря на то, что \utility{Ant} получил широкое
распространение в \Java{}\hyp{}сообществе, вне сообщества
\utility{Ant} практически не распространён. К тому же, сомнительно,
что его популярность когда-нибудь выйдет за пределы
\Java{}\hyp{}проектов (по причинам, перечисленным далее). \GNUmake{},
в свою очередь, успешно применяется во многих областях, включая
разработку программного обеспечения, обработку документов и
типографское дело, поддержку веб\hyp{}сайтов. Понимание \GNUmake{}
очень важно для любого, кому требуется работать в различных
программных системах.
%---------------------------------------------------------------------
\item Выбор XML как языка спецификаций вполне разумен для
\Java{}\hyp{}приложения. Однако читать и писать спецификации на языке
XML большинству людей не очень удобно. Хороший XML\hyp{}редактор может
быть нелегко найти или интегрировать с существующими инструментами
(если моя интегрированная среда разработки не содержит хорошего
XML\hyp{}редактора, мне придётся либо менять среду разработки, либо
искать такой редактор и использовать его отдельно). Как вы могли
видеть из предыдущего примера, \utility{Ant}\hyp{}диалект XML довольно
избыточен по сравнению с синтаксисом \GNUmake{}, и полон специфических
для XML особенностей.
%---------------------------------------------------------------------
\item В процессе работы с файлами \utility{Ant} вам нужно преодолевать
некоторую косвенность ваших спецификаций. Задача \utility{Ant}
\command{<mkdir>} те вызывает соответствующую программу
\utility{mkdir} вашей системы. Вместо этого вызывается метод
\command{mkdir()} класса \command{java.io.File}. Результатом вызова
может быть совсем не то, что вы ожидаете. По существу, любое
предположение программиста о поведении основных инструментов
\utility{Ant} должно быть проверено с привлечением документации по
\utility{Ant} или \Java{}, либо исходного кода \utility{Ant}. В
добавок, для вызова, к примеру, компилятора \Java{}, вам может
понадобиться разобраться в использовании десятка или более незнакомых
XML атрибутов, например, \command{srcdir}, \command{debug} и т.д., не
вошедших в руководство пользователя компилятора. В противоположность
этому \GNUmake{} совершенно прозрачен; как правило, вы можете просто
набирать команды прямо в интерпретаторе и следить за их поведением.
%---------------------------------------------------------------------
\item И всё же, несомненно, \utility{Ant} переносим, как и \GNUmake{}.
Как показано в главе~\ref{chap:portable_makefiles}, написание
переносимых \Makefile{}'ов, как и написание переносимых спецификаций
\utility{Ant}, требуют опыта и особых знаний. Программисты писали
переносимые \Makefile{}'ы два десятилетия. Более того, в документации
\utility{Ant} отмечается, что \utility{Ant} имеет проблемы
переносимости, связанные с символическими ссылками \UNIX{} и длинными
именами файлов в Windows, а MacOS X является единственной операционной
системой Apple, поддерживаемой \utility{Ant}, поддержка же других
платформ не гарантируется. К тому же, базовые операции наподобие
выставления флага исполняемости файлов не могут осуществляться при
помощи \Java{} API, для этого требуется вызов внешней программы.
Переносимость никогда не может быть простой или полной.
%---------------------------------------------------------------------
\item Программа \utility{Ant} не предоставляет подробного отчёта о
своих действиях. Поскольку задачи \utility{Ant} реализованы не в виде
командных сценариев, отображение действий, совершаемых этими задачами,
вызывает определённые трудности. Как правило, вывод состоит из
выражений на естественном языке, выдаваемых выражениями
\command{print}, добавленными автором задачи. Эти выражения не могут
быть выполнены пользователем в командной строке. В противоположность
этому, строки текста, отображаемые \GNUmake{} являются выражениями
интерпретатора и могут быть копированы из вывода и вставлены в
командный интерпретатор для повторного выполнения. Это означает, что
\utility{Ant} менее полезен для разработчиков, пытающихся понять
процесс сборки и способ работы инструментов, используемых в этом
процессе. Кроме того, это не даёт разработчику возможности повторно
использовать элементы этих задач экспромтом, при помощи клавиатуры.
%---------------------------------------------------------------------
\item Последняя и наиболее важная проблема заключается в том, что
\utility{Ant} сдвигает парадигмы осуществления сборок, призывая
использовать компилируемый язык программирования взамен
интерпретируемого. Задачи \utility{Ant} написаны на языке \Java{}.
Если какая-то задача не реализована или делает не то, что вы хотите,
вам нужно либо реализовать собственную задачу на \Java{}, либо
использовать задачу \command{<exec>} (разумеется, если вам приходится
часто использовать задачу \command{<exec>}, то гораздо проще
использовать \GNUmake{} с его макросами, функциями и более компактным
синтаксисом).

С другой стороны, интерпретируемые языки программирования были
изобретены для решения именно таких проблем. \GNUmake{} существует
около тридцати лет и может быть использован в большинстве сложных
ситуаций без расширения своей реализации. Разумеется, за эти тридцать
лет была реализована поддержка множества новых возможностей. Многие из
них задуманы и реализованы в GNU \GNUmake{}.
%---------------------------------------------------------------------
\end{itemize}

\utility{Ant} является замечательной программой, широко
распространённой в \Java{}\hyp{}сообществе. Тем не менее, прежде, чем
приступить к новому проекту, тщательно убедитесь, что \utility{Ant}
является подходящим инструментом для вашей среды разработки. Надеюсь,
эта глава докажет вам, что \GNUmake{} может быть успешно использован
для осуществления сборки вашего \Java{}\hyp{}проекта.

%---------------------------------------------------------------------
% Ant
%---------------------------------------------------------------------
\subsection{Интегрированные среды разработки}

Многие \Java{}\hyp{}разработчики используют интегрированные среды
разработки, совмещающие в единой (как правило, графической) среде
редактор, компилятор, отладчик и инструмент для навигации по исходному
коду. В качестве примеров можно привести такие проекты с открытым
исходным кодом, как Eclipse (\filename{\url{http://www.eclipse.org}})
и Emacs JDEE (\filename{\url{http://jdee.sunsite.dk}}), а также, если
рассматривать коммерческие разработки, Sun Java Studio
(\filename{\url{http://www.sun.com/software/sundev/jde}}) и JBuilder
(\filename{\url{http://www.borland.com/jbuilder}}). Эти среды, как
правило, имеют понятие процесса сборки проекта, заключающегося в
компиляции необходимых файлов и запуска приложения на выполнение.

Если интегрированная среда разработки поддерживает все эти операции,
зачем тогда нам рассматривать использование \GNUmake{}? Наиболее
очевидной причиной является переносимость. Если возникнет
необходимость осуществить сборку проекта на другой платформе, сборка
может закончится неудачей. Несмотря на то, что код \Java{} сам по себе
является переносимым, инструменты для работы с ним, как правило,
таковыми не являются. Например, конфигурационные файлы вашего проекта
могут включать списки путей в стиле \UNIX{} или Windows, это может
стать причиной ошибки при попытке запуска сборки под управлением
другой операционной системы. Второй причиной является тот факт, что
\GNUmake{} поддерживает автоматические сборки. Некоторые
интегрированные среды разработки поддерживают пакетные сборки, а
некоторые нет. Качество этой поддержки также варьируется. Наконец,
встроенная поддержка сборок часто бывает довольно ограниченной. Если
вы хотите реализовать собственную структуру каталогов, соответствующую
структуре релизов вашего проекта, интегрировать файлы помощи внешних
приложений, поддерживать автоматическое тестирование, ветвление и
параллельные треки разработки, скорее всего, вы обнаружите, что
встроенная поддержка сборок не подходит для ваших нужд.

По собственному опыту я могу судить, что интегрированные среды
разработки вполне подходят для небольших немасштабируемых приложений,
однако промышленные системы сборки требуют большей поддержки,
и \GNUmake{} может её обеспечить. Обычно я использую интегрированную
среду разработки для написания и отладки кода и составляю \Makefile{}
для промышленных сборок и релизов. Во время разработки я использую
интегрированную среду для компиляции проекта в состояние, пригодное
для отладки. Однако если я изменяю много файлов или модифицирую файлы,
являющиеся входными файлами для генератора кода, я запускаю
\Makefile{}. Интегрированная среда разработки, которую я использовал,
не имела соответствующей поддержки внешних программ, осуществляющих
генерацию кода. Обычно сборки, полученные с помощью интегрированной
среды, не подходят для поставок внутренним или внешним потребителям.
Для таких задач я использую \GNUmake{}.

%%--------------------------------------------------------------------
%% A generic java makefile
%%--------------------------------------------------------------------
\section{Универсальный \Makefile{} для \Java{}}

Следующий пример демонстрирует универсальный \Makefile{} для сборки
\Java{}\hyp{}проектов. Я объясню каждую из его частей далее в этой
главе.

{\footnotesize
\begin{verbatim}

# Общий makefile для Java-проекта.
VERSION_NUMBER := 1.0

# Определения базовых каталогов
SOURCE_DIR     := src
OUTPUT_DIR     := classes

# Инструменты Unix
AWK            := awk
FIND           := /bin/find
MKDIR          := mkdir -p
RM             := rm -rf
SHELL          := /bin/bash

# Пути для поддержки работы программ
JAVA_HOME      := /opt/j2sdk1.4.2_03
AXIS_HOME      := /opt/axis-1_1
TOMCAT_HOME    := /opt/jakarta-tomcat-5.0.18
XERCES_HOME    := /opt/xerces-1_4_4
JUNIT_HOME     := /opt/junit3.8.1

# Инструменты Java
JAVA           := $(JAVA_HOME)/bin/java
JAVAC          := $(JAVA_HOME)/bin/javac

JFLAGS         := -sourcepath $(SOURCE_DIR)  \
                  -d $(OUTPUT_DIR)           \
                  -source 1.4

JVMFLAGS       := -ea                        \
                  -esa                       \
                  -Xfuture

JVM            := $(JAVA) $(JVMFLAGS)

JAR            := $(JAVA_HOME)/bin/jar
JARFLAGS       := cf

JAVADOC        := $(JAVA_HOME)/bin/javadoc
JDFLAGS        := -sourcepath $(SOURCE_DIR) \
                  -d $(OUTPUT_DIR)          \
                  -link http://java.sun.com/products/jdk/1.4/docs/api

# Jar архивы
COMMONS_LOGGING_JAR := $(AXIS_HOME)/lib/commons-logging.jar

LOG4J_JAR           := $(AXIS_HOME)/lib/log4j-1.2.8.jar
XERCES_JAR          := $(XERCES_HOME)/xerces.jar
JUNIT_JAR           := $(JUNIT_HOME)/junit.jar

# Определяем путь к классам Java
class_path := OUTPUT_DIR          \
              XERCES_JAR          \
              COMMONS_LOGGING_JAR \
              LOG4J_JAR           \
              JUNIT_JAR

# Пробел
space := $(empty) $(empty)

# $(call build-classpath, variable-list)
define build-classpath
  $(strip                                          \
    $(patsubst :%,%,                               \
      $(subst : ,:,                                \
        $(strip                                    \
          $(foreach j,$1,$(call get-file,$j):)))))
endef

# $(call get-file, variable-name)
define get-file
  $(strip                                         \
    $($1)                                         \
      $(if $(call file-exists-eval,$1),,          \
        $(warning Файл, указанный в переменной    \
                  '$1' ($($1)), не найден)))
endef

# $(call file-exists-eval, variable-name)
define file-exists-eval
  $(strip                                      \
    $(if $($1),,$(warning '$1' has no value))  \
    $(wildcard $($1)))
endef

# $(call brief-help, makefile)
define brief-help
  $(AWK) '$$1 ~ /^[^.][-A-Za-z0-9]*:/                   \
          { print substr($$1, 1, length($$1)-1) }' $1 | \
  sort |                                                \
  pr -T -w 80 -4
endef

# $(call file-exists, wildcard-pattern)
file-exists = $(wildcard $1)

# $(call check-file, file-list)
define check-file
  $(foreach f, $1,                       \
    $(if $(call file-exists, $($f)),,    \
      $(warning $f ($($f)) is missing)))
endef

# $(call make-temp-dir, root-opt)
define make-temp-dir
  mktemp -t $(if $1,$1,make).XXXXXXXXXX
endef

# MANIFEST_TEMPLATE - шаблон файла манифеста, предназначенный
#                     для обработки макропроцессором m4
MANIFEST_TEMPLATE := src/manifest/manifest.mf
TMP_JAR_DIR       := $(call make-temp-dir)
TMP_MANIFEST      := $(TMP_JAR_DIR)/manifest.mf

# $(call add-manifest, jar, jar-name, manifest-file-opt)
define add-manifest
  $(RM) $(dir $(TMP_MANIFEST))
  $(MKDIR) $(dir $(TMP_MANIFEST))
  m4 --define=NAME="$(notdir $2)"            \
     --define=IMPL_VERSION=$(VERSION_NUMBER) \
     --define=SPEC_VERSION=$(VERSION_NUMBER) \
     $(if $3,$3,$(MANIFEST_TEMPLATE))        \
     > $(TMP_MANIFEST)
  $(JAR) -ufm $1 $(TMP_MANIFEST)
  $(RM) $(dir $(TMP_MANIFEST))
endef

# Определяем переменную CLASSPATH
export CLASSPATH := $(call build-classpath, $(class_path))

# make-directories - убеждаемся, что выходной каталог существует
make-directories := $(shell $(MKDIR) $(OUTPUT_DIR))

# help - цель по умолчанию
.PHONY: help
help:
    @$(call brief-help, $(CURDIR)/Makefile)

# all - осуществляет полную сборки системы
.PHONY: all
all: compile jars javadoc

# all_javas - временный файл для хранения списка исходных файлов
all_javas := $(OUTPUT_DIR)/all.javas

# compile - компилирует исходный код
.PHONY: compile
compile: $(all_javas)
    $(JAVAC) $(JFLAGS) @$<

# all_javas - составляет список исходных файлов
.INTERMEDIATE: $(all_javas)
$(all_javas):
    $(FIND) $(SOURCE_DIR) -name '*.java' > $@

# jar_list - список всех jar-архивов
jar_list := server_jar ui_jar

# jars - создаёт все jar-архивы
.PHONY: jars
jars: $(jar_list)

# server_jar - создаёт архив $(server_jar)
server_jar_name     := $(OUTPUT_DIR)/lib/a.jar
server_jar_manifest := src/com/company/manifest/foo.mf
server_jar_packages := com/company/m com/company/n

# ui_jar - создаёт архив $(ui_jar)
ui_jar_name     := $(OUTPUT_DIR)/lib/b.jar
ui_jar_manifest := src/com/company/manifest/bar.mf
ui_jar_packages := com/company/o com/company/p

# Создаёт явные правила для каждого архива
# $(foreach j, $(jar_list), $(eval $(call make-jar,$j)))
$(eval $(call make-jar,server_jar))
$(eval $(call make-jar,ui_jar))

# javadoc - создаёт документацию Java doc
.PHONY: javadoc
javadoc: $(all_javas)
    $(JAVADOC) $(JDFLAGS) @$<

.PHONY: clean
clean:
    $(RM) $(OUTPUT_DIR)

.PHONY: classpath
classpath:
    @echo CLASSPATH='$(CLASSPATH)'

.PHONY: check-config
check-config:
    @echo Проверяем конфигурацию...
    $(call check-file, $(class_path) JAVA_HOME)

.PHONY: print
print:
    $(foreach v, $(V), \
      $(warning $v = $($v)))
\end{verbatim}
}

%%--------------------------------------------------------------------
%% Compiling Java
%%--------------------------------------------------------------------
\section{Компиляция \Java{} кода}

Есть два способа компиляции кода \Java{} с помощью \GNUmake{}:
традиционный подход, вызывающий \utility{javac} для компиляции каждого
файла, и более быстрый подход, изложенный ранее и использующий
синтаксис \command{@filename}.

%---------------------------------------------------------------------
% The fast approach: all-in-one compile
%---------------------------------------------------------------------
\subsection*{Быстрый подход: компиляция всех исходных файлов за один
раз} \label{sec:all_in_one_compile}

Давайте более детально рассмотрим быстрый подход. Обратите внимание на
следующий фрагмент универсального \Makefile{}'а:

{\footnotesize
\begin{verbatim}
# all_javas - временный файл для хранения списка исходных файлов
all_javas := $(OUTPUT_DIR)/all.javas

# compile - компилирует исходный код
.PHONY: compile
compile: $(all_javas)
    $(JAVAC) $(JFLAGS) @$<

# all_javas - составляет список исходных файлов
.INTERMEDIATE: $(all_javas)
$(all_javas):
    $(FIND) $(SOURCE_DIR) -name '*.java' > $@
\end{verbatim}
}

Абстрактная цель \target{compile} вызывает \utility{javac} для
компиляции всего исходного кода проекта.

Реквизит \target{\$(all\_java)}~--- это файл, \filename{all.javas},
содержащий список исходных файлов \Java{}, по одному файлу на каждой
строке. Вовсе необязательно размещать каждый файл на отдельной строке,
однако так гораздо легче производить фильтрацию этого списка командой
\command{grep -v}, если в этом возникнет необходимость. Правило создания
файла \filename{all.javas} помечено как \target{.INTERMEDIATE},
поэтому \GNUmake{} будет удалять этот файл после каждого запуска и
создавать его заново перед каждой компиляцией. Командный сценарий для
создания файла очень прост. Для обеспечения максимальной переносимости
мы используем команду \utility{find} для извлечения всех исходных
файлов \Java{} из дерева каталогов с исходными файлами. Эта команда
работает не очень быстро, однако мы можем быть уверены в её корректной
работе. Более того, при изменении структуры дерева каталогов с
исходным кодом нам практически не придётся вносить изменений в этот
командный сценарий.

Если список каталогов, содержащих исходный код, определён и может быть
указан в вашем \Makefile{}'е, вы можете использовать более
производительный способ составления файла \filename{all.javas}.
Если список каталогов с исходным кодом не очень велик и помещается в
командной строке, не нарушая ограничений, накладываемых операционной
системой, можно использовать следующий сценарий:

{\footnotesize
\begin{verbatim}
$(all_javas):
    shopt -s nullglob; \
    printf "%s\n" $(addsuffix /*.java,$(PACKAGE_DIRS)) > $@
\end{verbatim}
}

Этот сценарий использует шаблоны командного интерпретатора для
определения списка \Java{}\hyp{}файлов в каждом каталоге. Однако если
каталог не содержит \Java{}\hyp{}файлов, нам хотелось бы, чтобы
раскрытие шаблон порождало пустую строку, а не текст исходного шаблона
(именно таково поведение по умолчанию многих командных
интерпретаторов). Для достижения этого эффекта используется опция
командного интерпретатора \utility{bash} \command{shopt -s nullglob}.
Большинство других интерпретаторов имеет подобную опцию. Наконец, мы
используем шаблоны и команду \command{printf} вместо \command{ln -l},
поскольку эти инструменты интегрированы в \utility{bash}, поэтому
потребуется выполнение всего одной программы независимо от числа
каталогов.

Мы можем избежать использования шаблонов интерпретатора при помощи
вызова функции \function{wildcard}:

{\footnotesize
\begin{verbatim}
$(all_javas):
    print "%s\n" $(wildcard \
                   $(addsuffix /*.java,$(PACKAGE_DIRS))) > $@
\end{verbatim}
}

Если ваш проект содержит много каталогов с исходным кодом (или пути к
ним имеют очень большую длину), предыдущий сценарий может превысить
предел длины командной строки вашей системы. В этом случае более
предпочтительным является следующий вариант:

{\footnotesize
\begin{verbatim}
.INTERMEDIATE: $(all_javas)
$(all_javas):
    shopt -s nullglob;           \
    for f in $(PACKAGE_DIRS);    \
    do                           \
      printf "%s\n" $$f/*.java;  \
    done > $@

\end{verbatim}
}

Заметим, что цель \target{compile} и вспомогательное правило следуют
подходу, основанному на нерекурсивном вызове \GNUmake{}. Не важно,
сколько подкаталогов в нашем проекте, мы используем единственный
\Makefile{} и производим единственный вызов компилятора. Если вам
нужно произвести компиляцию всего исходного кода, этот подход является
наиболее быстрым.

К тому же, мы совершенно не используем информацию о зависимостях.
Используя эти правила, \GNUmake{} не знает о связях между файлами и не
заботится о датах их модификации. Он просто осуществляет компиляцию
всего исходного кода при каждом вызове. В качестве бонуса мы получаем
возможность вызывать \GNUmake{} из каталога с исходными, а не
бинарными файлами. В контексте возможностей управления зависимостями
\GNUmake{} это может выглядеть как неразумный способ организации
\Makefile{}'а, однако давайте примем во внимание следующие доводы:

\begin{itemize}
%---------------------------------------------------------------------
\item Альтернатива (краткий обзор которой мы произведём позже)
использует стандартный подход, основанный на зависимостях. При этом
для каждого файла создаётся новый процесс \utility{javac}, что
увеличивает накладные расходы. Однако если наш проект не очень велик,
компиляция всех исходных файлов займёт не намного больше времени, чем
компиляция нескольких файлов, поскольку компилятор \utility{javac}
работает очень быстро, а создание новых процессов происходит
относительно медленно. Любая сборка, занимающая менее 15 секунд,
практически эквивалентна другой такой же, независимо от количества
работы, которую необходимо выполнить. Например, компиляция
приблизительно пятисот исходных файлов  (дистрибутива \utility{Ant})
занимает 14 секунд при выполнении на моём Pentium 4 1.8 ГГц, имеющем
512 Мб оперативной памяти. Компиляция одного файла занимает пять
секунд.
%---------------------------------------------------------------------
\item Б\'{о}льшая часть разработчиков будет использовать некий аналог
рабочей среды, предоставляющей быструю компиляцию отдельных файлов.
\Makefile{} же в основном будет использоваться в том случае, если
изменения охватывают большой участок кода, или требуется чистая
сборка, или же сборка осуществляется без вмешательства человека.
%---------------------------------------------------------------------
\item Как мы увидим, усилия, требуемые для реализации и поддержки
подхода, основанного на зависимостях, сравнимы усилиями, необходимыми
для реализации разделения деревьев каталогов исходных и бинарных
файлов для проектов, написанных на \Clang{}/\Cplusplus (эта тема
обсуждается в главе~\ref{chap:c_and_cpp}). Эту задачу не стоит
недооценивать.
%---------------------------------------------------------------------
\end{itemize}

Как мы увидим в следующих примерах, переменная
\variable{PACKAGE\_DIRS} используется не только для составления файла
\filename{all.javas}. Поддержка корректного значения этой переменной
может быть трудоёмким и потенциально сложным шагом. В случае небольших
проектов список каталогов может указываться явно прямо в
\Makefile{}'е, однако при росте числа каталогов до нескольких сотен
ручное редактирование этого списка становится довольно неприятным
занятием может привести к ошибкам. Более благоразумным способом может
быть использование программы \utility{find} для поиска соответствующих
каталогов:

{\footnotesize
\begin{verbatim}
# $(call find-compilation-dirs, root-directory)
  find-compilation-dirs =                       \
    $(patsubst %/,%,                            \
      $(sort                                    \
        $(dir                                   \
          $(shell $(FIND) $1 -name '*.java'))))
  PACKAGE_DIRS := $(call find-compilation-dirs, $(SOURCE_DIR))

\end{verbatim}
}

Команда \command{find} возвращает список файлов, функция
\function{dir} отсекает лишнюю часть имени, оставляя только имя
каталога, функция \function{sort} удаляет из списка дубликаты, а
функция \function{patsubst} удаляет слэш на конце каждого имени.
Обратите внимание на то, что функция \function{find-compilation-dirs}
находит все файлы, подлежащие компиляции, только для того, чтобы
отсечь имена файлов, в то время как правило, ассоциированное с
\filename{all.javas} использует шаблоны для восстановления этих имён.
Это может показаться напрасным расточительством ресурсов, однако я
часто замечаю, что наличие списка пакетов, содержащих исходный код,
чрезвычайно удобно в других аспектах сборки, например, при
сканировании конфигурационных файлов EJB. Если в вашем случае список
пакетов не требуется, просто используйте один из более простых
методов, упомянутых при обсуждении составления файла
\filename{all.javas}.

%---------------------------------------------------------------------
% Compiling with dependencies
%---------------------------------------------------------------------
\subsection*{Компиляция с учётом зависимостей}

Для реализации сборки с полным учётом зависимостей нам потребуется
инструмент для извлечения информации о зависимостях из исходных файлов
\Java{}, подобный команде \command{cc -M}. Программа Jikes
(\filename{\url{http://www.ibm.com/developerworks/opensource/jikes}})~---
это компилятор \Java{} с открытым исходным кодом, поддерживающий
эту возможность при использовании опций \command{-makefile} или
\command{+M}. Jikes~--- не идеальный инструмент для разделения
исходного кода и бинарных файлов, потому что он всегда записывает файл
зависимостей в тот же каталог, в котором находится исходный файл,
однако он бесплатен и эффективен. Есть и положительная сторона: файлы
зависимостей создаются во время компиляции, поэтому дополнительный
вызов компилятора не требуется.

Ниже приведён пример функции для работы с зависимостями и правила,
использующего эту функцию:

{\footnotesize
\begin{verbatim}
%.class: %.java
    $(JAVAC) $(JFLAGS) +M $<
    $(call java-process-depend,$<,$@)

# $(call java-process-depend, source-file, object-file)
define java-process-depend
  $(SED) -e 's/^.*\.class *:/$2 $(subst .class,.d,$2):/' \
         $(subst .java,.u,$1) > $(subst .class,.tmp,$2)
  $(SED) -e 's/#.*//'                                    \
         -e 's/^[^:]*: *//'                              \
         -e 's/ *\\$$$$//'                               \
         -e '/^$$$$/ d'                                  \
         -e 's/$$$$/ :/' $(subst .class,.tmp,$2)         \
         >> $(subst .class,.tmp,$2)
  $(MV) $(subst .class,.tmp,$2).tmp $(subst .class,.d,$2)
endef
\end{verbatim}
}

Этот сценарий требует, чтобы запуск \GNUmake{} осуществлялся из
каталога с бинарными файлами, и чтобы директива \directive{vpath}
указывала расположение исходных файлов. Если вы хотите использовать
компилятор Jikes только для генерации зависимостей, обращаясь к
другому компилятору для непосредственной генерации кода, вы можете
использовать опцию \command{+B}, в этом случае Jikes не будет
генерировать байт\-код.

В небольшом тесте производительности, в рамках которого происходила
компиляция 223 \Java{}\hyp{}файлов, однострочная команда компиляции,
описанная ранее, выполнялась на моей машине 9.9 секунд. Компиляция тех
же 223 файлов с индивидуальным вызовом компилятора для каждого файла
потребовала 411.6 секунд, т.е. в 41.5 раз больше времени. Более того,
при использовании раздельной компиляции любая сборка, требующая
компиляции более четырёх исходных файлов, будет занимать больше
времени, чем компиляция всего проекта одной командой. Если генерация
зависимостей и компиляция будут осуществляться разными программами,
разница только увеличится.

Разумеется, среды разработки варьируются, однако всегда важно
внимательно обдумать ваши цели. Минимизация числа файлов, подлежащих
компиляции, не всегда будет означать минимизацию времени, требующегося
для сборки системы. В случае языка \Java{} полная проверка
зависимостей и минимизация числа компилируемых файлов не являются
необходимыми атрибутами хорошей среды программирования.

%---------------------------------------------------------------------
% Setting CLASSPATH
%---------------------------------------------------------------------
\subsection*{Определение переменной CLASSPATH}

Одной из самых важных проблем при разработке программного обеспечения
на языке \Java{} является корректное определение переменной
\variable{CLASSPATH}. Эта переменная определяет, откуда будет
загружаться код при разрешении ссылки на класс. Для корректной
компиляции \Java{}\hyp{}приложения \Makefile{} должен включать
правильное определение переменной \variable{CLASSPATH}. Эта задача
быстро становится сложной при добавлении \Java{}\hyp{}пакетов,
программных интерфейсов приложений (API) и вспомогательных
инструментов. Если правильное определение \variable{CLASSPATH} может
быть сложной задачей, имеет смысл делать эту задачу в каком-то одном
месте.

Техника, которую я нашёл полезной, заключается в определении
переменной \variable{CLASSPATH} \Makefile{}'е для нужд не только
\GNUmake{}, но и других программ. Например, цель \target{classpath}
может возвращать команду экспорта переменной \variable{CLASSPATH} в
среду командного интерпретатора, вызвавшего \GNUmake{}:

{\footnotesize
\begin{verbatim}
.PHONY: classpath
classpath:
    @echo "export CLASSPATH='$(CLASSPATH)'"
\end{verbatim}
}

Разработчики могут определять \variable{CLASSPATH} следующим образом
(если они используют \utility{bash}):

{\footnotesize
\begin{alltt}
\$ \textbf{eval \$(make classpath)}
\end{alltt}
}

Определить переменную \variable{CLASSPATH} в среде Windows можно
следующим образом:

{\footnotesize
\begin{verbatim}
.PHONY: windows_classpath
windows_classpath:
    regtool set /user/Environment/CLASSPATH \
	        "$(subst /,\\,$(CLASSPATH))"
    control sysdm.cpl,@1,3 &
    @echo "Теперь нажмите кнопку <<Переменные окружения>>, " \
	      "затем OK, затем снова OK."
\end{verbatim}
}

Программа \utility{regtool} является частью среды разработки Cygwin и
предназначена для работы с реестром Windows. Однако простое обновление
реестра не вызовет считывания нового значения. Одним из способов
осуществления этой задачи является посещение диалогового окна
<<Переменные окружения>> (Environment Variables) и закрытие этого
окна при помощи кнопки OK.

Вторая строка сценария сообщает Windows, что нужно отобразить
диалоговое окно <<Свойства системы>> (System Properties) и сделать
активной вкладку <<Дополнительно>> (Advanced). К сожалению, командный
сценарий не может отобразить диалоговое окно <<Переменные окружения>>
или активировать кнопку OK, поэтому последняя строка сценария
предлагает пользователю завершить работу самостоятельно.

Экспорт переменной \variable{CLASSPATH} в другие программы, такие как
проектные файлы Emacs JDEE или JBuilder, осуществляется очень просто.

Непосредственное определение переменной \variable{CLASSPATH} может
также управляться при помощи \GNUmake{}. Определение этой переменной
очевидным способом определённо является разумной идеей:

{\footnotesize
\begin{verbatim}
CLASSPATH = /third_party/toplink-2.5/TopLink.jar:/third_party/...
\end{verbatim}
}

Из соображений переносимости более предпочтительным способом является
использование переменных:

{\footnotesize
\begin{verbatim}
# Определение Java classpath
class_path := OUTPUT_DIR          \
              XERCES_JAR          \
              COMMONS_LOGGING_JAR \
              LOG4J_JAR           \
              JUNIT_JAR
...
# Определение CLASSPATH
export CLASSPATH := $(call build-classpath, $(class_path))
\end{verbatim}
}
(Определение \variable{CLASSPATH}, приведённое в коде универсального
\Makefile{}'а, более показательно и полезно). Должным образом
реализованная функция \function{build\hyp{}classpath} решает несколько
раздражающих проблем:

\begin{itemize}
%---------------------------------------------------------------------
\item Очень просто собрать значение \variable{CLASSPATH} из частей.
Например, если используется несколько серверов приложений, может
потребоваться изменение \variable{CLASSPATH}. Различные версии
\variable{CLASSPATH} могут заключаться в секции \directive{ifdef} и
выбираться на основании значения какой-либо переменной \GNUmake{}.
%---------------------------------------------------------------------
\item Люди, занимающиеся поддержкой \Makefile{}'а, не должны
волноваться о внутренних пробелах, символах новой строки или переносах
строк, функция \function{build\hyp{}classpath} осуществляет
необходимые операции самостоятельно.
%---------------------------------------------------------------------
\item Функция \function{build\hyp{}classpath} может выбирать
разделитель путей автоматически, делая тем самым значение переменной
корректным для Windows и \UNIX{}.
%---------------------------------------------------------------------
\item Функция \function{build\hyp{}classpath} может осуществлять
проверку правильности элементов списка путей. В частности, одной из
раздражающих проблем \GNUmake{} является то, что вычисление
переменных, значение которых не определено, просто возвращает пустую
строку.  В большинстве случаев такое поведение полезно, однако иногда
оно может встать на вашем пути. В этом случае значение переменной
\variable{CLASSPATH} будет иметь фиктивное значение%
\footnote{
Для обнаружения этой ситуации можно попробовать использовать опцию
\command{-{}-warn\hyp{}undefined\hyp{}variables}, однако это приведёт
к предупреждениям, относящимся к тем переменным, неопределённое
значение которых нас устраивает.}.
Мы можем решить эту проблему, добавив проверку определённости
переменных, входящих в список путей, в функцию
\function{build\hyp{}classpath}. Функция также может проверять
существование каждого файла или каталога, входящего в список, и, в
случае невыполнения ограничений, выводить соответствующее
предупреждение.
%---------------------------------------------------------------------
\item Наконец, для реализации наиболее изощрённого функционала
(например, обработки пробелов в именах файлов или путях поиска) может
быть удобно использовать триггер для обработки переменной
\variable{CLASSPATH}.
%---------------------------------------------------------------------
\end{itemize}

Ниже приведена реализация функции \function{build\hyp{}classpath},
учитывающая первые три пункта нашего списка:

{\footnotesize
\begin{verbatim}
# $(call build-classpath, variable-list)
define build-classpath
  $(strip                                         \
    $(patsubst %:,%,                              \
      $(subst : ,:,                               \
        $(strip                                   \
          $(foreach c,$1,$(call get-file,$c):)))))
endef

# $(call get-file, variable-name)
define get-file
  $(strip                                        \
    $($1)                                        \
    $(if $(call file-exists-eval,$1),,           \
      $(warning Файл, указанный в переменной     \
                '$1' ($($1)), не найден)))
endef

# $(call file-exists-eval, variable-name)
define file-exists-eval
  $(strip                                                \
    $(if $($1),,$(warning Переменная'$1' не определена)) \
    $(wildcard $($1)))
endef
\end{verbatim}
}

Функция \function{build\hyp{}classpath} проходит по всем словам своего
аргумента, производя проверку каждого элемента и соединяя эти
элементы, используя разделитель путей (в нашем случае это
\command{:}).  Реализовать автоматический выбор разделителя путей
теперь очень просто. Затем функция удаляет пробелы, добавленные
функцией \function{get\hyp{}file} и циклом \function{for\-each}. Затем
функция удаляет последний разделитель, добавленный циклом
\function{for\-each}. Наконец, весь список подаётся на вход функции
\function{strip}, благодаря чему удаляются лишние пробелы, добавленные
продолжением строк.

Функция \function{get\hyp{}file} принимает на вход имя переменной и
осуществляет проверку существования файла, имя которого является
значением переменной. Если файл не существует, генерируется
предупреждение. Функция возвращает значение переменной независимо от
того, существует ли соответствующий файл, так как это значение может
быть полезным для пользователя. В некоторых случаях функция
\function{get\hyp{}file} может быть применена к файлу, который будет
сгенерирован позже и пока не существует.

Последняя функция, \function{file\hyp{}exists\hyp{}eval}, принимает
в качестве аргумента имя переменной, содержащей имя файла. Если
переменная содержит пустую строку, генерируется предупреждение, в
противном случае для проверки существования файла (или списка файлов)
используется вызов функции \function{wild\-card}

Если функция \function{build\hyp{}classpath} применяется к фиктивным
значениям, при запуске мы увидим следующие ошибки:

{\footnotesize
\begin{verbatim}
Makefile:37: Файл, указанный в переменной 'TOPLINKX_25_JAR'
             (/usr/java/toplink-2.5/TopLinkX.jar), не найден
...
Makefile:37: Переменная 'XERCES_142_JAR' не определена
Makefile:37: Файл, указанный в переменной 'XERCES_142_JAR'
             ( ), не найден
\end{verbatim}
}

Этот пример демонстрирует существенный прогресс по сравнению с
молчанием, которое мы получаем при использовании простого подхода.

Существование функции \function{get\hyp{}file} подразумевает, что мы
можем обобщить задачу поиска входных файлов.

{\footnotesize
\begin{verbatim}
# $(call get-jar, variable-name)
define get-jar
  $(strip                                                     \
    $(if $($1),,$(warning Переменная '$1' пуста))             \
    $(if $(JAR_PATH),,$(warning Переменная JAR_PATH пуста))   \
    $(foreach d, $(dir $($1)) $(JAR_PATH),                    \
      $(if $(wildcard $d/$(notdir $($1))),                    \
        $(if $(get-jar-return),,                              \
          $(eval get-jar-return := $d/$(notdir $($1))))))     \
    $(if $(get-jar-return),                                   \
      $(get-jar-return)                                       \
      $(eval get-jar-return :=),                              \
      $($1)                                                   \
      $(warning get-jar: Файл '$1' не найден в $(JAR_PATH))))
endef
\end{verbatim}
}

Здесь мы определяем переменную \variable{JAR\_PATH}, содержащую пути
для поиска файлов. Возвращается первый найденный файл. Параметром
функции является имя переменной, содержащей путь к архиву jar. Мы
производим поиск jar\hyp{}файла, используя сначала путь, содержащийся
в переданной переменной, затем набор путей, содержащихся в переменной
\variable{JAR\_PATH}. Чтобы реализовать такое поведение, список
каталогов в цикле \function{for\-each} составляется из значения
переменной, за которым сделует значение переменной
\variable{JAR\_PATH}. Два других обращения к параметру обрамляются
вызовом функции \function{notdir}, благодаря чему имя архива можно
соединить с соответствующим элементом списка. Обратите внимание на то,
что мы не можем выйти из цикла \function{for\-each}. Вместо этого,
однако, мы используем функцию \function{eval} для определения
переменной \variable{get\hyp{}jar\hyp{}return}, используемую для
хранения первого найденного файла. После выхода из цикла мы возвращаем
значение временной переменной или генерируем предупреждение, если файл
не был найден. Важно не забыть сбросить значение временной переменной
перед завершением работы макроса.

Эта функция по существу является реализацией директивы
\directive{vpath} в контексте определения переменной
\variable{CLASSPATH}. Чтобы понять это, вспомним, что директива
\directive{vpath} используется \GNUmake{} для нахождения реквизитов,
которые не были найдены по их относительному пути. В такой
ситуации \GNUmake{} производит поиск реквизитов в каталогах, указанных
директивой \directive{vpath}, и подставляет дополненный путь в
автоматические переменные \variable{\$\^}, \variable{\$?} и
\variable{\$+}. Мы хотим, чтобы для определения переменной
\variable{CLASSPATH} \GNUmake{} производил поиск пути к каждому
jar\hyp{}файлу и производил конкатенацию переменной
\variable{CLASSPATH} с этим дополненным путём. Поскольку \GNUmake{} не
имеет встроенной поддержки этого функционала, мы добавляем его
самостоятельно.  Разумеется, вы можете просто всегда прописывать
полные пути к jar\hyp{}файлам, предоставив задачу поиска виртуальной
машине \Java{}, однако переменная \variable{CLASSPATH} и без того
быстро становится длинной. На некоторых операционных системах длина
переменных окружения ограничена и существует опасность усечения
длинных значений \variable{CLASSPATH}.  Например, на операционной
системе Windows XP длина значения переменной окружения ограничена 1023
символами. В добавок, даже если переменая \variable{CLASSPATH} не
будет усечена, виртуальная машина \Java{} должна производить поиск в
\variable{CLASSPATH} при загрузке классов, что замедляет работу
приложения.

%%--------------------------------------------------------------------
%% Managing jars
%%--------------------------------------------------------------------
\section{Управление архивами \Java{}}

Сборка и управление \Java{}\hyp{}архивами поднимают проблемы, отличные
от тех, с которыми мы сталкивались при сборке библиотек
\Clang{}/\Cplusplus{}. На это есть три причины. Во\hyp{}первых,
элементы \Java{}\hyp{}архива адресуются относительным путём, поэтому
точные имена файлов, передаваемых программе \utility{jar}, нужно
тщательно контролировать. Во\hyp{}вторых, в \Java{}\hyp{}сообществе
есть тенденция соединять архивы, чтобы всё приложение могло
размещаться в единственном архиве. Наконец, \Java{}\hyp{}архивы могут
содержать файлы, отличные от файлов классов, например, файл манифеста,
файлы свойств и XML\hyp{}файлы.

Базовая команда для создания \Java{}\hyp{}архива при помощи GNU
\GNUmake{} выглядит следующим образом:

{\footnotesize
\begin{verbatim}
JAR      := jar
JARFLAGS := -cf

$(FOO_JAR): реквизиты...
    $(JAR) $(JARFLAGS) $@ $^
\end{verbatim}
}

Программа \utility{jar} может принимать вместо имён файлов имена
каталогов, в этом случае в архив будет помещено всё содержимое
указанных каталогов. Это может быть очень удобно, особенно при
использовании совместно с опцией \command{-C}, временно изменяющей
текущий каталог:

{\footnotesize
\begin{verbatim}
JAR      := jar
JARFLAGS := -cf

.PHONY: $(FOO_JAR)
$(FOO_JAR):
    $(JAR) $(JARFLAGS) $@ -C $(OUTPUT_DIR) com
\end{verbatim}
}

Здесь файл архива объявлен абстрактной целью. Однако при повторном
запуске \Makefile{}'а архив не будет создаваться заново, поскольку эта
цель не имеет реквизитов. Как и в случае команды \utility{ar},
описанной в одной из предыдущих глав, смысла в использовании флага
обновления архива, \command{-u}, практически нет, поскольку эта
операция занимает практически такое же (или даже большее) время, что и
операция создания нового архива.

\Java{}\hyp{}архив часто включает файл манифеста, в котором указан
поставщик, API и номер версии. Простой файл манифеста может выглядеть
следующим образом:

{\footnotesize
\begin{verbatim}
Name: JAR_NAME
Specification-Title: SPEC_NAME
Implementation-Version: IMPL_VERSION
Specification-Vendor: Generic Innovative Company, Inc.
\end{verbatim}
}

Этот файл содержит три переменных, \variable{JAR\_NAME},
\variable{SPEC\_NAME} и \variable{IMPL\_VERSION}, которые могут быть
заменены реальными значениями при создании архива с помощью
\utility{sed}, \utility{m4} или вашего любимого редактора потоков.
Ниже приведена функция для обработки файла манифеста:

{\footnotesize
\begin{verbatim}
MANIFEST_TEMPLATE := src/manifests/default.mf
TMP_JAR_DIR       := $(call make-temp-dir)
TMP_MANIFEST      := $(TMP_JAR_DIR)/manifest.mf

# $(call add-manifest, jar, jar-name, manifest-file-opt)
define add-manifest
  $(RM) $(dir $(TMP_MANIFEST))
  $(MKDIR) $(dir $(TMP_MANIFEST))
  m4 --define=NAME="$(notdir $2)"            \
     --define=IMPL_VERSION=$(VERSION_NUMBER) \
     --define=SPEC_VERSION=$(VERSION_NUMBER) \
     $(if $3,$3,$(MANIFEST_TEMPLATE))        \
     > $(TMP_MANIFEST)
  $(JAR) -ufm $1 $(TMP_MANIFEST)
  $(RM) $(dir $(TMP_MANIFEST))
endef
\end{verbatim}
}

Функция \function{add\hyp{}manifest} оперирует файлом манифеста
методом, подобным описанному выше. Сначала функция создаёт временный
каталог, затем производит подстановку переменных в шаблоне файла
манифеста. Затем функция обновляет архив и удаляет временный каталог.
Обратите внимание на то, что последний аргумент функции является
необязательным. Если путь к файлу манифеста не указан, функция
использует значение переменной \variable{MANIFEST\_TEMPLATE}.

В универсальном \Makefile{}'е эти операции привязаны к общей функции,
осуществляющей составление явного правила для создания
\Java{}\hyp{}архива:

{\footnotesize
\begin{verbatim}
# $(call make-jar,jar-variable-prefix)
define make-jar
  .PHONY: $1 $$($1_name)
  $1: $($1_name)
  $$($1_name):
      cd $(OUTPUT_DIR); \
      $(JAR) $(JARFLAGS) $$(notdir $$@) $$($1_packages)
      $$(call add-manifest, $$@, $$($1_name), $$($1_manifest))
endef
\end{verbatim}
}

Эта функция принимает один аргумент, префикс переменной \GNUmake{},
который идентифицирует набор переменных, описывающих четыре параметра
архива: имя цели, имя архива, пакеты архива и файл манифеста.
Например, для создания архива \filename{ui.jar} мы напишем следующее:

{\footnotesize
\begin{verbatim}
ui_jar_name     := $(OUTPUT_DIR)/lib/ui.jar
ui_jar_manifest := src/com/company/ui/manifest.mf
ui_jar_packages := src/com/company/ui \
                   src/com/company/lib

$(eval $(call make-jar,ui_jar))
\end{verbatim}
}

Используя композицию имён переменных, мы можем сократить
последовательность действий, выполняемых функцией, достигнув в тоже
время гибкой её реализации.

Если нам нужно создать много архивов, мы можем автоматизировать этот
процесс, поместив список имён архивов в переменную:

{\footnotesize
\begin{verbatim}
jar_list := server_jar ui_jar

.PHONY: jars $(jar_list)
jars: $(jar_list)

$(foreach j, $(jar_list),\
  $(eval $(call make-jar,$j)))
\end{verbatim}
}

В некоторых случаях нам может понадобиться распаковать содержимое
архива во временный каталог. Ниже представлен пример простой функции,
реализующей это требование:

{\footnotesize
\begin{verbatim}
# $(call burst-jar, jar-file, target-directory)
define burst-jar
  $(call make-dir,$2)
  cd $2; $(JAR) -xf $1
endef
\end{verbatim}
}

%%--------------------------------------------------------------------
%% Reference trees and third-party jars
%%--------------------------------------------------------------------
\section{Справочные деревья и архивы сторонних %
разработчиков}

Для того, чтобы использовать единое разделяемое справочное дерево с
поддержкой создания разработчиками частичных рабочих копий, просто
настройте механизм ночных сборок, создающий \Java{}\hyp{}архивы
проекта, и включите эти архивы в \variable{CLASSPATH} компилятора.
После этого шага разработчики смогут сделать нужную им частичную
рабочую копию и инициировать процесс компиляции (в предположении, что
список исходных файлов создаётся динамически программой, подобной
\utility{find}). Когда компилятору \Java{} нужно будет найти символ,
определённый в отсутствующем исходном файле, компилятор произведёт
поиск, основываясь на значении \variable{CLASSPATH}, и обнаружит
соответствующие файлы классов в архиве.

Получение \Java{}\hyp{}архивов сторонних разработчиков из справочного
дерева реализуется также просто. Просто поместите пути к этим архивам
в переменную \variable{CLASSPATH}. Как уже было замечено, \Makefile{}
может быть очень полезным инструментом управления этим процессом.
Разумеется, функция \function{get\hyp{}file} может быть использована
для автоматического выбора стабильной или бета версии, локальных или
удалённых \Java{}\hyp{}архивов при помощи соответствующего определения
переменной \variable{JAR\_PATH}.

%%--------------------------------------------------------------------
%% Enterprise JavaBeans
%%--------------------------------------------------------------------
\section{Enterprise JavaBeans}

\Java{}\hyp{}компоненты уровня предприятия (Enterprise
Java\-Beans\trademark{}, EJB)~--- это мощная техника инкапсуляции и
повторного использования бизнес\hyp{}логики, каркасом которой является
механизм удалённых вызовов методов (Remote Method Invocation, RMI).
EJB определяет \Java{}\hyp{}классы, используемые для реализации API
сервера, используемого, в конечном счёте, удалёнными клиентами. Эти
объекты и службы настраиваются при помощи специальных файлов в формате
XML.  После написания \Java{}\hyp{}класса и соответствующего ему
конфигурационного XML\hyp{}файла эти файлы нужно упаковать вместе в
\Java{}\hyp{}архив. Затем вызывается специальный EJB\hyp{}компилятор,
создающий код заглушек и связок, реализующих поддержку RPC.

Следующий код может быть добавлен в код универсального \Makefile{}'а
для предоставления поддержки EJB:

{\footnotesize
\begin{verbatim}
EJB_TMP_JAR = $(EJB_TMP_DIR)/temp.jar
META_INF    = $(EJB_TMP_DIR)/META-INF

# $(call compile-bean, jar-name,
#        bean-files-wildcard, manifest-name-opt)
define compile-bean
  $(eval EJB_TMP_DIR := $(shell mktemp -d \
                          $(TMPDIR)/compile-bean.XXXXXXXX))
  $(MKDIR) $(META_INF)
  $(if $(filter %.xml, $2),cp $(filter %.xml, $2) $(META_INF))
  cd $(OUTPUT_DIR) &&                     \
  $(JAR) -cf0 $(EJB_TMP_JAR)              \
         $(call jar-file-arg,$(META_INF)) \
         $(filter-out %.xml, $2)
  $(JVM) weblogic.ejbc $(EJB_TMP_JAR) $1
  $(call add-manifest,$(if $3,$3,$1),,)
  $(RM) $(EJB_TMP_DIR)
endef

# $(call jar-file-arg, jar-file)
jar-file-arg = -C "$(patsubst %/,%,$(dir $1))" $(notdir $1)
\end{verbatim}
}

Функция \function{compile\hyp{}bean} принимает три параметра: имя
\Java{}\hyp{}архива, который требуется создать, список файлов,
входящих в архив, и необязательный файл манифеста. Сначала при помощи
программы \utility{mktemp} создаётся пустой временный каталог, имя
каталога сохраняется в переменной \variable{EJB\_TMP\_DIR}. Поместив
присваивание этой переменной в функцию \function{eval}, мы получаем
гарантию того, что значение \variable{EJB\_TMP\_DIR} будет указывать
на новый временный каталог при каждом вычислении функции
\function{compile\hyp{}bean}. Поскольку функция
\function{compile\hyp{}bean} используется в командном сценарии,
она будет вычисляться только при выполнении сценария. Затем функция
осуществляет копирование всех XML файлов из списка
\variable{bean\hyp{}files\hyp{}wild\-card} в каталог
\filename{META\hyp{}INF}. Именно в этом каталоге хранятся
конфигурационные файлы EJB. После этого функция создаёт временный
\Java{}\hyp{}архив, используемый в качестве входа для
EJB\hyp{}компилятора. Функция \function{jar\hyp{}file\hyp{}arg}
преобразует имена вида \filename{dir1/dir2/dir3} к виду
\filename{-C dir1/dir2 dir3}, поэтому относительные имена файлов
архива корректны. Этот наиболее подходящий формат для передачи команде
\utility{jar} пути к каталогу \filename{META\hyp{}INF}. Поскольку
XML\hyp{}файлы, содержавшиеся в списке, уже скопированы в каталог
\filename{META\hyp{}INF}, мы отсеиваем их из списка аргументов команды
\utility{jar} при помощи функции \function{filter\hyp{}out}. После
сборки временного архива вызывается EJB\hyp{}компилятор Web\-Lo\-gic,
создающий результирующий архив. Затем к составленному архиву
добавляется файл манифеста. Последним действием является удаление
временного архива.

Способ использования новой функции очевиден:

{\footnotesize
\begin{verbatim}
bean_files = com/company/bean/FooInterface.class      \
             com/company/bean/FooHome.class           \
             src/com/company/bean/ejb-jar.xml         \
             src/com/company/bean/weblogic-ejb-jar.xml

.PHONY: ejb_jar $(EJB_JAR)
ejb_jar: $(EJB_JAR)
$(EJB_JAR):
    $(call compile-bean, $@, $(bean_files), weblogic.mf)
\end{verbatim}
}

Список \variable{bean\_files} немного необычен. Пути к файлам классов,
входящих в этот список, указаны относительно каталога
\filename{classes}, в то время как пути к XML\hyp{}файлам будут
вычисляться относительно каталога, в котором располагается
\Makefile{}.

Это всё замечательно, но что если ваш архив содержит много файлов?
Существует ли способ составить список файлов автоматически?
Разумеется:

{\footnotesize
\begin{verbatim}
src_dirs := $(SOURCE_DIR)/com/company/...

bean_files =                                          \
  $(patsubst $(SOURCE_DIR)/%,%,                       \
    $(addsuffix /*.class,                             \
      $(sort                                          \
        $(dir                                         \
          $(wildcard                                  \
            $(addsuffix /*Home.java,$(src_dirs)))))))

.PHONY: ejb_jar $(EJB_JAR)
ejb_jar: $(EJB_JAR)
$(EJB_JAR):
    $(call compile-bean, $@, $(bean_files), weblogic.mf)
\end{verbatim}
}

Этот код подразумевает, что список каталогов с исходными файлами
хранится в переменной \variable{src\_dirs} (в списке могут находится и
каталоги, не содержащие кода EJB\hyp{}компонентов), и что все файлы,
оканчивающиеся строкой \emph{Home.java}, идентифицируют пакеты,
содержащие код EJB\hyp{}компонентов. Выражение для определения
переменной \variable{bean\_files} сначала добавляет суффикс шаблона к
имени каждого каталога в списке, а затем вызывает функцию
\function{wild\-card} для нахождения всех файлов, имя которых
оканчивается строкой \emph{Home.java}. Имена файлов отбрасываются,
полученный список каталогов сортируется, дублирующиеся элементы
удаляются из списка. К каждому каталогу добавляется суффикс
\command{/*.class}, в результате командный интерпретатор заменит
шаблон списком соответствующих файлов классов. Наконец, от каждого
элемента списка отсекается префикс, содержащий имя каталога с
исходными файлами (поскольку такого подкаталога каталога
\filename{classes} не существует). Причиной использования шаблонов
командного интерпретатора вместо функции \function{wild\-card}
является тот факт, что \GNUmake{} не сможет гарантированно выполнить
поиск файлов, соответствующих шаблону, \emph{после} компиляции и
генерации файлов классов. Если \GNUmake{} вычислит функцию
\function{wild\-card} слишком рано, файлы не будут обнаружены, а кэш
содержимого каталогов помешает найти эти файлы позже. Применение же
функции \function{wild\-card} в каталоге с исходными файлами
совершенно безопасно, поскольку мы подразумеваем, что исходные файлы
не будут добавляться во время работы \GNUmake{}.

Предыдущий код будет работать в том случае, если у нас имеется
небольшое число архивов компонентов. Другой стиль разработки
подразумевает помещение каждого EJB\hyp{}компонента в собственный
\Java{}\hyp{}архив. Большие проекты могут содержать десятки архивов.
Для того, чтобы осуществлять автоматическую обработку этой ситуации,
нам нужно составить явное правило для каждого EJB\hyp{}архива. В нашем
примере исходный код EJB\hyp{}компонентов самодостаточен: каждый
компонент располагается в отдельном каталоге вместе с ассоциированным
XML\hyp{}файлом. Определить каталоги, содержащие EJB\hyp{}компоненты,
можно по наличию файлов, оканчивающихся строкой \emph{Session.java}.

Основной подход заключается в поиске EJB\hyp{}компонентов в каталогах
с исходным кодом, построении явного правила для каждого компонента и
записи этих правил в файл. Затем файл с правилами для
EJB\hyp{}компонентов включается в наш \Makefile{}. Создание файла с
правилами для компонентов вызывается через механизм управления
включаемыми файлами \GNUmake{}.

{\footnotesize
\begin{verbatim}
# session_jars - архивы EJB, адресованные относительным путём.
session_jars =
  $(subst .java,.jar,                       \
    $(wildcard                              \
      $(addsuffix /*Session.java, $(COMPILATION_DIRS))))

# EJBS - список всех EJB-архивов.
EJBS = $(addprefix $(TMP_DIR)/,$(notdir $(session_jars)))

# ejbs - Create all EJB jar files.
.PHONY: ejbs
ejbs: $(EJBS)
$(EJBS):
    $(call compile-bean,$@,$^,)
\end{verbatim}
}

С помощью вызова функции \function{wild\-card} со списком всех
каталогов с исходным кодом в качестве аргумента мы находим все файлы,
имя которых оканчивается на \emph{Session.java}. В нашем примере имя
архива образуется из имени найденного исходного файла с добавлением
расширения \filename{.jar}. Архивы будут помещаться во временный
каталог. Переменная \variable{EJBS} содержит список архивов,
адресованных относительным путём от корня дерева бинарных файлов.
Эти архивы являются целью, которую мы хотим обновить. Командным
сценарием является вызов функции \function{compile\hyp{}bean},
реализованной нами ранее. Фокус заключается в том, что список файлов
указан в качестве реквизита каждого архива. Давайте посмотрим, как
они будут создаваться.

{\footnotesize
\begin{verbatim}
-include $(OUTPUT_DIR)/ejb.d

# $(call ejb-rule, ejb-name)
ejb-rule = $(TMP_DIR)/$(notdir $1):            \
             $(addprefix $(OUTPUT_DIR)/,       \
               $(subst .java,.class,           \
                 $(wildcard $(dir $1)*.java))) \
             $(wildcard $(dir $1)*.xml)

# ejb.d - файл зависимостей EJB
$(OUTPUT_DIR)/ejb.d: Makefile
    @echo Вычисляю зависимости ejb...
    @for f in $(session_jars);       \
    do                               \
      echo "\$$(call ejb-rule,$$f)"; \
    done > $@
\end{verbatim}
}

Зависимости для каждого EJB\hyp{}архива записываются в файл
\filename{ejb.d}, включаемый в \Makefile{}. Когда \GNUmake{} первый
раз производит поиск этого файла, файл ещё не существует. Поэтому
\GNUmake{} вызывает правило для обновления включаемого файла. Это
правило записывает по одной строке, подобной следующей, для каждого
EJB\hyp{}архива:

{\footnotesize
\begin{verbatim}
$(call ejb-rule,src/com/company/foo/FooSession.jar)
\end{verbatim}
}

Результатом вычисления функции \function{ejb\hyp{}rule} является
имя целевого архива и списка реквизитов, как показано ниже:

{\footnotesize
\begin{verbatim}
classes/lib/FooSession.jar:                  \
    classes/com/company/foo/FooHome.jar      \
    classes/com/company/foo/FooInterface.jar \
    classes/com/company/foo/FooSession.jar   \
    src/com/company/foo/ejb-jar.xml          \
    src/com/company/foo/ejb-weblogic-jar.xml
\end{verbatim}
}

Таким образом, \GNUmake{} предоставляет возможность управлять довольно
большим количеством архивов без необходимости ручной поддержки набора
явных правил.


%%%-------------------------------------------------------------------
%%% Improving the performance of make
%%%-------------------------------------------------------------------
\chapter{Повышаем производительность \GNUmake{}}
\label{chap:improving_the_performance}

\GNUmake{} имеет чрезвычайно важную роль в процессе разработки
программного обеспечения. Компонуя составляющие проекта в приложение,
\GNUmake{} позволяет разработчикам избежать трудноуловимых ошибок,
связанных со случайным пропуском какого-то шага сборки. Однако, если
разработчики избегают использования \GNUmake{} из-за низкой скорости
выполнения сборки, все преимущества использования \GNUmake{} теряются.
Таким образом, чрезвычайно важно убедиться в том, что \Makefile{} был
составлен с расчётом на максимальную производительность.

Проблемы производительности всегда довольно запутаны, однако, если
взять в рассмотрение восприятие пользователей и различные пути
выполнения кода, всё становится ещё сложнее. Не каждая цель в
\makefile{е} нуждается в оптимизации. В некоторых условиях даже
радикальные оптимизации могут не оправдать затраченных на них усилий. 
Например, сокращение времени сборки с 90 до 45 минут может быть
несущественным, поскольку даже с учётом оптимизации сборка становится
операцией, начав которую, можно <<идти на обед>>. С другой стороны,
сокращение времени сборки с двух минут до одной может сопровождаться
аплодисментами разработчиков, если во время сборки они вынуждены
сидеть сложа руки.

При написании эффективных \makefile{ов} важно знать стоимость
различных операций, а также ясно понимать, какие именно из этих
операций выполняются. В последующих разделах мы проведём несколько
простых тестов производительности, позволяющих дополнить эти общие
комментарии количественными данными и описать техники, помогающие
найти узкие места (bottlenecks).

Другим подходом к повышению производительности является использование
параллелизма и топологий локальных сетей. Одновременное исполнение
более одного командного сценария (даже на одном процессоре) может
существенно сократить время сборки.

%%--------------------------------------------------------------------
%% Benchmarking
%%--------------------------------------------------------------------
\section{Измеряем производительность}

В этом разделе мы измерим производительность базовых операций
\GNUmake{}. Таблица~\ref{tab:cost_of_operations} содержит результаты
этих измерений. Далее будет рассмотрен каждый из тестов, а также
представлены соображения по поводу влияния этих результатов на
написанные вами \Makefile{}'ы.

\begin{table}[!b]
{\footnotesize
\begin{tabular}{llllll}
\hline
\vspace{0.3em}
Операция &
Повторений &
\parbox[b]{2cm}{\flushleft Секунд на \break выполнение (Windows)} &
\parbox[b]{2cm}{\flushleft Число \break выполнений в секунду (Windows)} &
\parbox[b]{2cm}{\flushleft Секунд на \break выполнение (Linux)} &
\parbox[b]{2cm}{\flushleft Число \break выполнений в секунду (Linux)} \\
\hline
\vspace{0.5em}
make (bash) & $\hphantom{0.}1000$ & 0,0436 & $\hphantom{00}22$ & 0,0162 & $\hphantom{000.}61$ \\
\vspace{0.5em}
make (ash) & $\hphantom{0.}1000$ & 0,0413 & $\hphantom{00}24$ & 0,0151 & $\hphantom{000.}66$ \\
\vspace{0.5em}
make (bash) & $\hphantom{0.}1000$ & 0,0452 & $\hphantom{00}22$ & 0,0159 & $\hphantom{000.}62$ \\
\vspace{0.5em}
присваивание & 10.000 & 0,0001 & 8130 & 0,0001 & 10.989 \\
\vspace{0.5em}
subst (short) & 10.000 & 0,0003 & 3891 & 0,0003 & $\hphantom{0.}3846$ \\
\vspace{0.5em}
subst (long) & 10.000 & 0,0018 & $\hphantom{0}547$ & 0,0014 & $\hphantom{00.}704$ \\
\vspace{0.5em}
sed (bash) & $\hphantom{0.}1000$ & 0,0910 & $\hphantom{00}10$ & 0,0342 & $\hphantom{000.}29$ \\
\vspace{0.5em}
sed (ash) & $\hphantom{0.}1000$ & 0,0699 & $\hphantom{00}14$& 0,0069 & $\hphantom{00.}144$ \\
\vspace{0.5em}
sed (sh) & $\hphantom{0.}1000$ & 0,0911 & $\hphantom{00}10$ & 0,0139 & $\hphantom{000.}71$ \\
\vspace{0.5em}
shell (bash) & $\hphantom{0.}1000$ & 0,0398 & $\hphantom{00}25$ & 0,0261 & $\hphantom{000.}38$ \\
\vspace{0.5em}
shell (ash) & $\hphantom{0.}1000$ & 0,0253 & $\hphantom{00}39$ & 0,0018 & $\hphantom{00.}555$ \\
\vspace{0.3em}
shell (sh) & $\hphantom{0.}1000$ & 0,0399 & $\hphantom{00}25$ & 0,0050 & $\hphantom{00.}198$ \\
\hline
\end{tabular}
}
\caption{Стоимость базовых операций} \label{tab:cost_of_operations}
\end{table}

Тесты для Windows запускались на Pentium 4 с тактовой частотой 1,9 ГГц
(приблизительно 3578 BogoMips\footnote{
Объяснение величины BogoMips можно найти на сайте
\url{http://www.clifton.nl/bogomips.html } (прим. автора).})
и оперативной памятью 512 Мб под управлением операционной системы
Windows XP. Использовался Cygwin \GNUmake{} версии 3.80, запускаемый
из окна \utility{rxvt}. Тесты для Linux запускались на Pentium 2 с
тактовой частотой 450 ГГц (891 BogoMips) и оперативной памятью 256 Mб
под управлением операционной системы Linux RedHat 9.

Командный интерпретатор, используемый \GNUmake{}, может существенно
повлиять на производительность выполнения \makefile{а}. Командный
интерпретатор \utility{bash} сложен и обладает обширной
функциональностью, поэтому он достаточно тяжеловесен. Интерпретатор
\utility{ash} гораздо легче, он обладает меньшими возможностями,
впрочем, вполне подходящими для большинства задач. Чтобы усложнить
задачу, добавлю, что при запуске \utility{bash} командой
\filename{/bin/sh} его поведение существенно изменяется с целью
максимального соответствия возможностям стандартного интерпретатора.
На большинстве Linux\hyp{}систем файл \filename{/bin/sh} является
символической ссылкой на \utility{bash}, в то время как в Cygwin эта
ссылка указывает на \utility{ash}. Для учёта этих различий некоторые
тесты запускались трижды, по одному разу для каждого командного
интерпретатора. Командный интерпретатор, использованный в тесте,
указан в скобках. К примеру, <<(sh)>> означает, что использовался
интерпретатор \utility{bash}, запущенный при помощи символической
ссылки \filename{/bin/sh}.

Первые три теста, обозначенные как <<make>>, отображают стоимость
запуска \GNUmake{}, не совершающего полезной работы. \makefile{}
содержит следующие строки:

{\footnotesize
\begin{verbatim}
SHELL := /bin/bash
.PHONY: x
x:
    $(MAKE) --no-print-directory --silent --question make-bash.mk; \
    ...Эта команда повторяется ещё 99 раз...
\end{verbatim}
}

По необходимости слово <<bash>> заменялось соответствующим названием
командного интерпретатора.

Мы использовали опции \command{-{}-no\hyp{}print\hyp{}directory} и
\command{-{}-{}silent} для исключения ненужных вычислений, которые
могли повлиять на время выполнения и затмить измеренные временные
величины грудой бесполезного текста. Опция \command{-{}\hyp{}question}
сообщает \GNUmake{}, что выполнять команды не требуется, нужна только
проверка зависимостей. В этом случае, если файл не требует обновления,
\GNUmake{} завершит работу с нулевым кодом возврата. Это позволяет
\GNUmake{} делать настолько мало работы, насколько это возможно. Этот
\makefile{} не будет выполнять команд, зависимости в нём существуют
только для одной абстрактной цели. Файл \filename{make-bash.mk}
выполняется родительским процессом \GNUmake{} 10 раз. Содержимое этого
файла представлено ниже:

{\footnotesize
\begin{verbatim}
define ten-times
  TESTS += $1
  .PHONY: $1
  $1:
      @echo $(MAKE) --no-print-directory --silent $2; \
      time $(MAKE) --no-print-directory --silent $2; \
      time $(MAKE) --no-print-directory --silent $2; \
      time $(MAKE) --no-print-directory --silent $2; \
      time $(MAKE) --no-print-directory --silent $2; \
      time $(MAKE) --no-print-directory --silent $2; \
      time $(MAKE) --no-print-directory --silent $2; \
      time $(MAKE) --no-print-directory --silent $2; \
      time $(MAKE) --no-print-directory --silent $2; \
      time $(MAKE) --no-print-directory --silent $2; \
      time $(MAKE) --no-print-directory --silent $2
endef

.PHONY: all
all:

$(eval $(call ten-times, make-bash, -f make-bash.mk))

all: $(TESTS)
\end{verbatim}
}

После этого время, требуемое для тысячи запусков, усредняется.

Как вы можете видеть из таблицы, Cygwin \GNUmake{} выполняется
примерно 22 раза в секунду, или 0,044 секунд, в то время как под
управлением операционной системы Linux (не смотря на гораздо более
медленный процессор) выполнение осуществляется примерно 61 раз в
секунду (т.е. одно выполнение занимает 0,016 секунд). Для проверки
этих результатов был протестирован порт \GNUmake{} под Windwos, не
показавший, впрочем, существенного выигрыша в производительности.
Заключение: хоть создание процесса Cygwin \GNUmake{} и
занимает немного больше времени, чем та же операция в Windows
\GNUmake{}, оба этих варианта значительно уступают в
производительности аналогичной операции в Linux. Отсюда следует, что
рекурсивное выполнение \GNUmake{} под Windows может занимать
значительно больше времени, чем рекурсивная сборка, запущенная под
управлением Linux.

Как вы могли ожидать, используемый командный интерпретатор практически
не влияет на скорость выполнения. Поскольку командный сценарий не
содержит специальных символов, командный интерпретатор даже не
вызывался. \GNUmake{} выполнял команды самостоятельно. Это можно
проверить, присвоив переменной \variable{SHELL} произвольное значение
и убедившись в том, что тест выполняется корректно. Разница в
производительности при использовании различных интерпретаторов можно
списать на нормальную вариацию времени выполнения процесса в системе.

Следующий тест измеряет время, требуемое для присваивания переменной
значения~--- наиболее элементарной операции \GNUmake{}. \makefile{},
называющийся \filename{assign.mk}, содержит следующие строки:

{\footnotesize
\begin{verbatim}
# 10000 assignments
z := 10
...предыдущая строка повторяется 10000 раз...
.PHONY: x
x: ;
\end{verbatim}
}

Этот \makefile{} выполняется в родительском \makefile{е} с
использованием нашей функции \function{ten\hyp{}times}.

Очевидно, присваивание выполняется очень быстро. Cygwin \GNUmake{}
выполняет 8130 присваиваний в секунду, в то время как в системе Linux
этот показатель доходит до 10.989. Я уверен, что производительность
выполнения этой операции в системе Windows на самом деле выше, чем
показывают наши измерения, поскольку точное время создания десяти
процессов \GNUmake{} невозможно отделить от времени выполнения
присваивания. Заключение: поскольку вероятность того, что в среднем
\makefile{е} будет осуществляться 10.000 присваиваний, довольно мала,
стоимость выполнения присваиваний в среднем \makefile{е} можно не
учитывать.

Следующие два теста измеряют время выполнения функции \function{subst}.
Первый тест осуществляет подстановку трёх символов в коротких строках,
состоящих из десяти символов:

{\footnotesize
\begin{verbatim}
# 10000 подстановок в строке из 10 символов
dir := ab/cd/ef/g
x := $(subst /, ,$(dir))
...предыдущая строка повторяется 10000 раз...
.PHONY: x
x: ;
\end{verbatim}
}

Операция занимает примерно в два раза больше времени чем простое
присваивание, выполняясь в Windows 3891 раз в секунду. Повторюсь,
показатели производительности в системе Linux значительно превосходят
аналогичные показатели в Windows (как вы помните, производительность
процессора компьютера, на котором установлена система Linux, примерно
в четыре раза меньше производительности процессора компьютера с
системой Windows).

Второй тест осуществляет примерно 100 подстановок в строке длиной в
1000 символов:

{\footnotesize
\begin{verbatim}
# Имя файла из 10 символов
dir := ab/cd/ef/g
# список путей из 1000 символов
p100 := $(dir);$(dir);$(dir);$(dir);$(dir);...
p1000 := $(p100)$(p100)$(p100)$(p100)$(p100)...

# 10000 подстановок в строке длиной в 1000 символов
x := $(subst ;, ,$(p1000))
...предыдущая строка повторяется 10000 раз...
.PHONY: x
x: ;
\end{verbatim}
}

Следующие три теста измеряют скорость той же подстановки при
использовании \utility{sed}. Содержимое тестового файла представлено
ниже:

{\footnotesize
\begin{verbatim}
# 100 sed using bash
SHELL := /bin/bash

.PHONY: sed-bash
sed-bash:
echo '$(p1000)' | sed 's/;/ /g' > /dev/null
...предыдущая строка повторяется 100 раз...
\end{verbatim}
}

Как и раньше, \makefile{} выполняется с помощью функции
\function{ten-times}. В системе Windows \utility{sed} выполняется
примерно в 50 раз медленнее, чем функция \function{subst}. В системе
Linux \utility{sed} работает в 24 раза медленнее.

Если учесть время, затраченное на запуск командного интерпретатора,
становится очевидным, что использование командного интерпретатора
\utility{ash} в Windows даёт небольшую прибавку в скорости. При
использовании \utility{ash} \utility{sed} всего лишь в 39 раз
медленнее \function{subst}! В Linux влияние используемого командного
интерпретатора на скорость выполнения прослеживается более чётко. При
использовании \utility{ash} \utility{sed} всего в пять раз медленнее
\function{subst}. Здесь же можно проследить эффект замены
\utility{bash} на \utility{sh}. В среде Cygwin разница между
\utility{bash}, вызванного через \filename{/bin/bash}, и
\utility{bash}, вызванного через \filename{/bin/sh}, не
прослеживается. В Linux же \utility{/bin/sh} выполняется значительно
быстрее.

Последний тест измеряет затраты на выполнение команды в дочернем
командном интерпретаторе, вызывая команду \command{make shell}.
\makefile{} содержит следующие строки:

{\footnotesize
\begin{verbatim}
# 100 $(shell ) using bash
SHELL := /bin/bash
x := $(shell :)
...предыдущая строка повторяется 100 раз...
.PHONY: x
x: ;
\end{verbatim}
}

Впрочем, результаты были вполне предсказуемы. Система Windows работает
медленнее, чем Linux, командный интерпретатор \utility{ash}
работает быстрее, чем \utility{bash}. Выигрыш от использования
\utility{ash} выражен в этом тесте более ярко и составляет примерно
50\%. В системе Linux наибольшая производительность достигается при
использовании \utility{ash}, наименьшая~--- при использовании
\utility{bash} (вызванного из файла \filename{/bin/bash}).

Тесты производительности являются неиссякаемым источником задач,
тем не менее, сделанные нами измерения могут помочь нам извлечь
некоторую полезную информацию. Создавайте столько переменных, сколько
считаете нужным, если, конечно, они помогают упростить структуру
\makefile{а}, поскольку их использование практически ничего не стоит.
Встроенные функции более предпочтительны, чем запуск внешних программ,
даже если структура вашего кода обязывает вас последовательно
выполнять вызовы функций \GNUmake{}. Избегайте использования
рекурсивного \GNUmake{} или избыточного порождения процессов в
Windows. Если вы работаете в Linux и вам нужно создавать множество
процессов, используйте \utility{ash}.

Наконец, запомните, что для большинства \makefile{ов} справедливо
следующее утверждение: время выполнения \makefile{а} практически
полностью определяется временем выполнения внешних программ, а вовсе
не нагрузкой \GNUmake{} и не структурой \makefile{а}. Как правило,
сокращение числа запусков внешних программ сократит и время выполнения
\makefile{а}.

%%--------------------------------------------------------------------
%% Identifying and handling bottlenecks
%%--------------------------------------------------------------------
\section{Определяем и устраняем узкие места}
Излишние задержки выполнения \makefile{а} могут появляться по одной
из трёх причин: неудачный выбор структуры \makefile{а}, неверный
анализ зависимостей, и неправильное использование функций и переменных
\GNUmake{}. Эти проблемы могут маскироваться функциями \GNUmake{},
подобными \function{shell}, которые вызывают команды, но не печатают
их в терминал, что существенно затрудняет поиск источника задержек.

Анализ зависимостей~--- это палка о двух концах. С одной стороны,
выполнение полного анализа зависимостей может вызвать существенные
задержки. Без специальной поддержки компилятора, предоставляемой, к
примеру, \utility{gcc} и \utility{jikes}, создание файла зависимостей
требует запуска внешней программы, что практически удваивает время
компиляции\footnote{
На практике время компиляции линейно зависит от размера входного
текста и практически всегда определяется скоростью операций
ввода/вывода. Точно так же время вычисления зависимостей с помощью
опции \command{-M} линейно зависит от размера файла и ограничено
скоростью операций ввода/вывода.}. Преимуществом полного анализа
зависимостей является возможность \GNUmake{} осуществлять меньшее
количество компиляций. К сожалению, разработчики часто не верят, что
эта возможность себя оправдает, и пишут \makefile{ы} с неполной
информацией о зависимостях. Этот компромисс почти всегда превращается
в проблему, заставляющую других разработчиков платить за эту скупость
вдвойне, компилируя больше кода, чем потребовалось бы, будь у
\GNUmake{} полная информация о зависимостях.

Чтобы сформулировать стратегию анализа зависимостей, начните с
понимания зависимостей, присущих вашему проекту. Когда все
зависимости осознаны, можно приступать к представлению
этих зависимостей в \makefile{е} (вычисленных или перечисленных
вручную) и выбору сокращённых путей осуществления сборки. Хоть и не
все представленные шаги являются легко осуществимыми, этот метод сам
по себе является наиболее простым.

Когда вы определили структуру \makefile{а} и необходимые зависимости,
эффективность \makefile{а} достигается за счёт обхода некоторых
известных ловушек.

%---------------------------------------------------------------------
% Simple variables versus recursive
%---------------------------------------------------------------------
\subsection{Выбор переменных: простые или рекурсивные}
Одной из наиболее общих проблем, относящихся к производительности,
является использование рекурсивных переменных. Например, поскольку
код, приведённый ниже, использует оператор \command{=} вместо
оператора \command{:=}, при каждом обращении к переменной
\variable{DATE} её значение будет вычисляться заново:

{\footnotesize
\begin{verbatim}
DATE = $(shell date +%F)
\end{verbatim}
}

Опция \command{+\%F} сообщает программе \utility{date}, что дату
требуется возвращать в формате <<гггг-мм-дд>>, таким образом,
большинство пользователей не заметят эффекта от многократного вызова
\utility{date}. Разумеется, разработчики, засидевшиеся в офисе до
полуночи, могут быть приятно удивлены.

Поскольку \GNUmake{} не выводит команды, выполняемые при помощи
функции \function{shell}, идентифицировать, что же именно выполняется,
может быть довольно трудно. Определив переменную \variable{SHELL} как
\command{/bin/sh -x}, вы можете выявить все команды, выполняемые
\GNUmake{}.

Следующий \makefile{} создаёт каталог перед осуществлением прочих
действий. Имя каталога составляется из слова <<out>> и текущей даты:

{\footnotesize
\begin{verbatim}
DATE = $(shell date +%F)
OUTPUT_DIR = out-$(DATE)
make-directories := \
    $(shell [ -d $(OUTPUT_DIR) ] || mkdir -p $(OUTPUT_DIR))
all: ;
\end{verbatim}
}

После запуска \makefile{а} с отладочной опцией интерпретатора мы
увидим следующий вывод:

{\footnotesize
\begin{alltt}
\$ \textbf{make SHELL='/bin/sh -x'}
+ date +\%F
+ date +\%F
+ '[' -d out-2004-03-30 ']'
+ mkdir -p out-2004-03-30
make: all is up to date.
\end{alltt}
}

Теперь отчётливо видно, что команда \utility{date} выполняется дважды.
Если вам часто требуется осуществлять подобного рода отладку, вы
можете упростить её, используя конструкцию следующего вида:

{\footnotesize
\begin{verbatim}
ifdef DEBUG_SHELL
  SHELL = /bin/sh -x
endif
\end{verbatim}
}

%---------------------------------------------------------------------
% Disabling @
%---------------------------------------------------------------------
\subsection{Отключаем @}
Ещё одним способом сокрытия команд является модификатор \command{@}.
Иногда бывает полезным отключить эту возможность. Это легко
осуществить с помощью определения вспомогательной переменной
\variable{QUIET}, содержащей символ \command{@}, и использования этой
переменной в командах:

{\footnotesize
\begin{verbatim}
ifndef VERBOSE
  QUIET := @
endif
...
target:
    $(QUIET) echo Собираю цель target...
\end{verbatim}
}

Когда нужно будет увидеть команды, скрытые при помощи модификатора,
просто определите переменную \variable{VERBOSE} через интерфейс
командной строки:

{\footnotesize
\begin{alltt}
\$ \textbf{make VERBOSE=1}
echo Собираю цель target...
Собираю цель target...
\end{alltt}
}

%---------------------------------------------------------------------
% Lazy initialization
%---------------------------------------------------------------------
\subsection{Ленивая инициализация}
При использовании простых переменных в сочетании с функцией
\function{shell}, \GNUmake{} осуществляет вызовы функции
\function{shell} во время чтения \makefile{а}. Если таких вызовов
много, или если они осуществляют сложные вычисления, выполнение
\GNUmake{} может существенно замедлиться. Время отклика \GNUmake{}
можно измерить, вызвав \GNUmake{} со спецификацией несуществующей
цели:

{\footnotesize
\begin{alltt}
\$ \textbf{time make no-such-target}
make: *** No rule to make target no-such-target. Stop.
real    0m0.058s
user    0m0.062s
sys     0m0.015s
\end{alltt}
}

Приведённый выше код измеряет время, добавляемое \GNUmake{} к каждой
выполняемой команде, даже если эта команда тривиальна или ошибочна.

Поскольку рекурсивные переменные вычисляются заново при каждом
обращении к ним, существует тенденция оформлять результаты сложных
вычислений в виде простых переменных. С другой стороны, такой подход
увеличивает время отклика \GNUmake{} при сборке любой цели. Похоже,
существует необходимость в дополнительном виде переменных, правая
часть которых вычисляется в точности один раз при первом обращении к
переменной.

Пример, иллюстрирующий необходимость подобного рода инициализации, был
приведён в функции \function{find\hyp{}compilation\hyp{}dir} в разделе
<<\nameref{sec:all_in_one_compile}>> главы \ref{chap:java}:

{\footnotesize
\begin{verbatim}
# $(call find-compilation-dirs, root-directory)
find-compilation-dirs =                      \
  $(patsubst %/,%,                           \
    $(sort                                   \
      $(dir                                  \
        $(shell $(FIND) $1 -name '*.java'))))
PACKAGE_DIRS := $(call find-compilation-dirs, $(SOURCE_DIR))
\end{verbatim}
}

В идеале нам хотелось бы осуществлять операцию \command{find} только
один раз при первом обращении к переменной \variable{PACKAGE\_DIR}.
\index{Ленивая инициализация}
Это можно назвать \newword{ленивой инициализацией} (\newword{lazy
initialization}). Мы можем создать подобного рода переменную с помощью
функции \function{eval}:

{\footnotesize
\begin{verbatim}
PACKAGE_DIRS = $(redefine-package-dirs) $(PACKAGE_DIRS)
redefine-package-dirs =                                \
  $(eval PACKAGE_DIRS := $(call find-compilation-dirs, \
                           $(SOURCE_DIR)))
\end{verbatim}
}

Этот подход заключается в определении \variable{PACKAGE\_DIR} как
изначально рекурсивной переменной. При первом обращении к переменной
вычисляется ресурсоёмкая функция, в данном случае
\function{find\hyp{}compilation\hyp{}dir}, и переменная
переопределяется как простая. Наконец, значение переменной (теперь уже
простой) возвращается как результат обращения к первоначально
рекурсивной переменной.

Давайте рассмотрим этот пример более детально:
\begin{enumerate}
%---------------------------------------------------------------------
\item Когда \GNUmake{} считывает эти переменные, он просто сохраняет
правые части их определений, так как обе переменные являются
рекурсивными.
%---------------------------------------------------------------------
\item При первом обращении к переменной \variable{PACKAGE\_DIRS}
\GNUmake{} извлекает соответствующую правую часть и производит
вычисление переменной \variable{redefine\hyp{}package\hyp{}dirs}.
%---------------------------------------------------------------------
\item Значением переменной \variable{redefine\hyp{}package\hyp{}dirs}
является единственный вызов функции \function{eval}.
%---------------------------------------------------------------------
\item Тело функции \function{eval} переопределяет переменную
\variable{PACKAGE\_DIRS} как простую переменную, присваивая ей
результат вычисления функции
\function{find\hyp{}compilation\hyp{}dirs}. Теперь
\variable{PACKAGE\_DIRS} инициализирована списком каталогов.
%---------------------------------------------------------------------
\item Результатом вычисления функции
\function{redefine\hyp{}package\hyp{}dir} является пустая строка
(поскольку результатом вычисления функции \function{eval} также
является пустая строка).
%---------------------------------------------------------------------
\item \GNUmake{} продолжает вычислять изначальное значение переменной
\variable{PACKAGE\_DIRS}. Остаётся только подставить значение
переменной \variable{PACKAGE\_DIRS}. \GNUmake{} производит поиск
переменной, находит простую переменную и возвращает её значение.
%---------------------------------------------------------------------
\end{enumerate}

Единственным по-настоящему тонким моментом в этом коде является
предположение, согласно которому \GNUmake{} вычисляет правую часть
определения переменной слева направо. Если, к примеру, \GNUmake{}
решит вычислить выражение \command{\$(PACKAGE\_DIRS)} раньше выражения
\command{\$(redefine\hyp{}package\hyp{}dirs)}, этот код не будет
работать.

Процедура, описанная мной выше, может быть преобразована в функцию
\function{lazy-init}:

{\footnotesize
\begin{verbatim}
# $(call lazy-init,variable-name,value)
define lazy-init
  $1 = $$(redefine-$1) $$($1)
  redefine-$1 = $$(eval $1 := $2)
endef

# PACKAGE_DIRS - ленивое вычисление списка каталогов
$(eval                           \
  $(call lazy-init,PACKAGE_DIRS, \
    $$(call find-compilation-dirs,$(SOURCE_DIRS))))
\end{verbatim}
}

%%--------------------------------------------------------------------
%% Parallel make
%%--------------------------------------------------------------------
\section{Параллельное выполнение \GNUmake{}}

Ещё одним способом увеличения производительности сборок является
использование параллелизма, присущего проблеме обработки
\makefile{а}. Большинство \makefile{ов} предназначены для
осуществления задач, многие из которых можно обрабатывать параллельно,
например, компиляцию исходных файлов \Clang{} в объектные или создание
библиотек из объектных файлов. Более того, сама структура хорошо
написанных \makefile{ов} предоставляет всю необходимую информацию для
автоматического управления конкурирующими процессами.

Следующий пример демонстрирует сборку нашей программы mp3 плеера с
опцией управления задачами, \command{-{}-jobs=2} (или \command{-j
  2}). На рисунке~\ref{fig:parallel_make} изображён тот же самый
запуск \GNUmake{}, представленный с помощью диаграммы
UML. Использование опции \command{-{}-jobs} сообщает \GNUmake{}, что
при возможности следует обновлять параллельно две цели. Когда
\GNUmake{} обновляет цели параллельно, он выводит команды в том
порядке, в каком они выполняются, чередуя в выводе команды сборки
разных целей. Это может затруднить чтение вывода \GNUmake{},
осуществляющего параллельную сборку. Давайте внимательно рассмотрим
вывод.

{\footnotesize
\begin{alltt}
\$ \textbf{make -f ../ch07-separate-binaries/makefile --jobs=2}
\end{alltt}
\begin{verbatim}
1  bison -y --defines ../ch07-separate-binaries/lib/db/playlist.y
2  flex -t ../ch07-separate-binaries/lib/db/scanner.l >
   lib/db/scanner.c
3  gcc -I lib -I ../ch07-separate-binaries/lib
   -I ../ch07-separate-binaries/include
   -M ../ch07-separate-binaries/app/player/play_mp3.c | \
   sed 's,\(play_mp3\.o\) *:,app/player/\1 app/player/play_mp3.d: ,'
   > app/player/play_mp3.d.tmp
4  mv -f y.tab.c lib/db/playlist.c
5  mv -f y.tab.h lib/db/playlist.h
6  gcc -I lib -I ../ch07-separate-binaries/lib
   -I ../ch07-separate-binaries/include
   -M ../ch07-separate-binaries/lib/codec/codec.c | \
   sed 's,\(codec\.o\) *:,lib/codec/\1 lib/codec/codec.d: ,' >
   lib/codec/codec.d.tmp
7  mv -f app/player/play_mp3.d.tmp app/player/play_mp3.d
8  gcc -I lib -I ../ch07-separate-binaries/lib
   -I ../ch07-separate-binaries/include -M lib/db/playlist.c | \
   sed 's,\(playlist\.o\) *:,lib/db/\1 lib/db/playlist.d: ,' >
   lib/db/playlist.d.tmp
9  mv -f lib/codec/codec.d.tmp lib/codec/codec.d
10 gcc -I lib -I ../ch07-separate-binaries/lib
   -I ../ch07-separate-binaries/include
   -M ../ch07-separate-binaries/lib/ui/ui.c | \
   sed 's,\(ui\.o\) *:,lib/ui/\1 lib/ui/ui.d: ,' > lib/ui/ui.d.tmp
11 mv -f lib/db/playlist.d.tmp lib/db/playlist.d
12 gcc -I lib -I ../ch07-separate-binaries/lib
   -I ../ch07-separate-binaries/include
   -M lib/db/scanner.c | \
   sed 's,\(scanner\.o\) *:,lib/db/\1 lib/db/scanner.d: ,' >
   lib/db/scanner.d.tmp
13 mv -f lib/ui/ui.d.tmp lib/ui/ui.d
14 mv -f lib/db/scanner.d.tmp lib/db/scanner.d
15 gcc -I lib -I ../ch07-separate-binaries/lib
   -I ../ch07-separate-binaries/include -c
   -o app/player/play_mp3.o
   ../ch07-separate-binaries/app/player/play_mp3.c
16 gcc -I lib -I ../ch07-separate-binaries/lib
   -I ../ch07-separate-binaries/include -c
   -o lib/codec/codec.o
   ../ch07-separate-binaries/lib/codec/codec.c
17 gcc -I lib -I ../ch07-separate-binaries/lib
   -I ../ch07-separate-binaries/include -c
   -o lib/db/playlist.o lib/db/playlist.c
18 gcc -I lib -I ../ch07-separate-binaries/lib
   -I ../ch07-separate-binaries/include -c
   -o lib/db/scanner.o lib/db/scanner.c
   ../ch07-separate-binaries/lib/db/scanner.l: In function yylex:
   ../ch07-separate-binaries/lib/db/scanner.l:9: warning:
   return makes integer from pointer without a cast
19 gcc -I lib -I ../ch07-separate-binaries/lib
   -I ../ch07-separate-binaries/include -c
   -o lib/ui/ui.o ../ch07-separate-binaries/lib/ui/ui.c
20 ar rv lib/codec/libcodec.a lib/codec/codec.o
   ar: creating lib/codec/libcodec.a
   a - lib/codec/codec.o
21 ar rv lib/db/libdb.a lib/db/playlist.o lib/db/scanner.o
   ar: creating lib/db/libdb.a
   a - lib/db/playlist.o
   a - lib/db/scanner.o
22 ar rv lib/ui/libui.a lib/ui/ui.o
   ar: creating lib/ui/libui.a
   a - lib/ui/ui.o
23 gcc app/player/play_mp3.o lib/codec/libcodec.a lib/db/libdb.a
   lib/ui/libui.a app/player/play_mp3
\end{verbatim}
}

\begin{figure}
\begin{center}
\includegraphics{./src/latex/figures/parallel_make.eps}
\end{center}
\caption{Диаграмма выполнения \GNUmake{} при \command{-{}-jobs=2}}
\label{fig:parallel_make}
\end{figure}

Сначала \GNUmake{} должен сгенерировать исходные файлы и файлы
зависимостей. Два сгенерированных исходных файла являются выводом
команд \utility{yacc} и \utility{lex} (команды 1 и 2
соответственно). Третья команда генерирует файл зависимостей для
\filename{play\_mp3.c}, её выполнение начинается до завершения
генерации файлов зависимостей для \filename{playlist.c} или
\filename{scanner.c} (командами 4, 5, 8, 9, 12 и 14). Таким образом,
\GNUmake{} выполняет одновременно три задачи, не смотря на то, что
опция командной строки требует одновременного выполнения двух задач.

Команды \command{mv} (4 и 5) завершают генерацию исходного файла
\filename{playlist.c}, начавшуюся первой командой. Команда 6 начинает
генерацию очередного файла зависимостей. Каждый командный сценарий
выполняется одним процессом \GNUmake{}, однако каждая цель и каждый
реквизит формируют отдельный поток. Таким образом, команда 7,
являющаяся второй командой сценария генерации файла зависимостей,
исполняется тем же процессом \GNUmake{}, что и команда 3. Команда 6
выполняется, скорее всего, процессом \GNUmake{}, порождённым сразу
после выполнения команд 1-4-5 (обработки грамматики \utility{yacc}),
но до генерации файла зависимостей, осуществляющейся командой 8.

Определение зависимостей продолжается в том же духе вплоть до команды
14. Все файлы зависимостей должны быть созданы до того, как \GNUmake{}
сможет приступить к следующей фазе обработки~--- повторному считыванию
\makefile{а}. Эта фаза образует естественную точку синхронизации.

Как только завершается повторное чтение \makefile{а}, \GNUmake{} может
снова продолжить параллельное выполнение сборки. В этот раз \GNUmake{}
решает произвести компиляцию всех объектных файлов до того, как начать
упаковывать их в библиотечные архивы. Этот порядок является
недетерминированным. В следующий раз \GNUmake{} может собрать
библиотеку \filename{libcodec.a} до того, как будет скомпилирован файл
\filename{playlist.c}, поскольку библиотека не требует наличия никаких
объектных файлов, кроме \filename{codec.o}. Таким образом, этот пример
демонстрирует один порядок выполнения из множества возможных.

Наконец, происходит компоновка программы. В нашем случае фаза
компоновки также является естественной точкой синхронизации и всегда
будет происходить в последнюю очередь. Если, однако, целью была не
единственная программа, а множество программ или библиотек, последняя
исполняемая инструкция может также быть другой.

Запуск множества задач на многопроцессорном компьютере может иметь
смысл, однако запуск более одной задачи на процессор может быть также
очень полезным. Причина кроется в латентности дискового ввода/вывода и
наличии кэширования на большинстве систем. Например, если процесс, к
примеру, \utility{gcc}, ожидает поступления данных с диска, данные для
других задач (таких как \utility{mv} или \utility{yacc}) могут
располагаться в памяти компьютера. В этом случае лучшим выходом будет
разрешить задаче обработку данных. В общем случае запуск \GNUmake{} с
несколькими потоками выполнения на однопроцессорной системе
практически всегда быстрее, чем запуск однопоточной сборки, и не так уж
редко запуск трёх или даже четырёх потоков приводит к лучшему
результату, чем при запуске двух потоков.

Опция \command{-{}-jobs} может использоваться без аргумента. В этом
случае \GNUmake{} порождает по одному потоку на каждую обновляемую
цель. Как правило, это плохая идея, поскольку на переключение
контекста может уходить настолько много времени, что итоговая
производительность будет гораздо ниже, чем в случае однопоточного
выполнения.

Ещё одним способом управления множеством задач является использование
средней загрузки системы как отправной точки. Средняя загрузка
системы~--- это число запущенных задач, усреднённое за некоторый
промежуток времени (как правило,1, 5 или 15 минут). Средняя загрузка
выражается как число с плавающей точкой. Опция
\command{-{}-load-average} (или \command{-l}) задаёт верхнюю границу
числа порождаемых задач. Например, следующая команда:

{\footnotesize
\begin{alltt}
\$ \textbf{make --load-average=3.5}
\end{alltt}
}

сообщает \GNUmake{}, что потоки задач должны порождаться таким
образом, чтобы средняя средняя загрузка системы была не более
3.5. Если средняя загрузка выше, \GNUmake{} будет ждать, пока она не
уменьшится до допустимых пределов, или пока все потоки не закончат
свою работу.

Когда вы пишите \makefile{} для параллельного выполнения, задача
правильного указания реквизитов становятся ещё более важной. Как уже
было замечено, когда опция \command{-{}-jobs} имеет значение 1, список
реквизитов обычно вычисляется слева направо. Когда \command{-{}-jobs}
больше 1, реквизиты могут вычисляться параллельно. Поэтому любые
отношения зависимости, неявно присутствующие в порядке вычисления
реквизитов, при параллельном запуске должны быть указаны явно.

Ещё одной неприятностью при использовании параллельных сборок является
проблема разделяемых временных файлов. Например, если каталог содержит
файлы \filename{foo.y} и \filename{bar.y}, параллельный запуск
\utility{yacc} может привести к тому, что один из экземпляров файла
\filename{y.tab.c} или \filename{y.tab.h} перепишет другой. С подобной
проблемой вы также сталкиваетесь при использовании в своих сценариях
временных файлов с фиксированными именами.

Ещё одной идиомой, препятствующей параллельному выполнению, является
рекурсивный вызов \GNUmake{} из цикла \command{for}:

{\footnotesize
\begin{verbatim}
dir:
    for d in $(SUBDIRS);         \
    do                           \
        $(MAKE) --directory=$$d; \
    done
\end{verbatim}
}

Как уже упоминалось в разделе \nameref{sec:recursive_make} главы
\ref{chap:managing_large_proj}, \GNUmake{} не может выполнять
рекурсивные вызовы параллельно. Чтобы достичь параллельного
выполнения, объявите каталоги абстрактными целями:

{\footnotesize
\begin{verbatim}
.PHONY: $(SUBDIRS)
$(SUBDIRS):
    $(MAKE) --directory=$@
\end{verbatim}
}

%%--------------------------------------------------------------------
%% Distributed make
%%--------------------------------------------------------------------
\section{Распределённое выполнение \GNUmake{}}

GNU \GNUmake{} поддерживает малоизвестную (и практически не
тестированную) опцию для управления сборками, распределёнными среди
нескольких рабочих станций, соединённых сетью. Этот функционал основан
на библиотеке Customs, распространяемой с дистрибутивом
\utility{Pmake}. \utility{Pmake}~--- это альтернативная версия
\GNUmake{}, реализованная Адамом де Буром (Adam de Boor) в 1989 году
для операционной системы Sprite и всё ещё поддерживаемая Андреасом
Столке (Andreas Stolcke). Библиотека Customs помогает распределить
выполнение \GNUmake{} между множеством компьютеров. GNU \GNUmake{}
включает поддержку этой библиотеки начиная с версии 3.77.

Чтобы включить поддержку библиотеки Customs, вам нужно собрать
\GNUmake{} из исходного кода. Инструкцию по осуществлению этого
процесса можно найти в файле \filename{README.customs} в дистрибутиве
\GNUmake{}. Сначала вам нужно загрузить дистрибутив \utility{pmake}
(URL указан в инструкции), затем собрать \GNUmake{} с опцией
\command{-{}\hyp{}with\hyp{}customs}.

Сердцем библиотеки Customs является демон (daemon), запускаемый на
каждом узле распределённой вычислительной сети \GNUmake{}. Все узлы
должны иметь доступ к разделяемой файловой системе, предоставляемый,
например, NFS. Один экземпляр демона назначается управляющим.
Управляющий процесс назначает задачи участникам вычислительной сети.
Когда \GNUmake{} запускается с опцией \command{-{}\hyp{}jobs} больше
1, \GNUmake{} контактирует с управляющим процессом, вместе они
порождают задачи, распределяя их среди доступных узлов сети.

Библиотека Customs поддерживает множество возможностей. Узлы могут
группироваться по архитектуре и ранжироваться по производительности.
Узлам могут назначаться произвольные атрибуты, и задачи могут
назначаться на основании значений атрибутов и булевых операторов. В
добавок к этому, такие характеристики работы узлов, как время
простоя, свободное дисковое пространство, свободное пространство в
разделе подкачки, текущая средняя загрузка могут быть посчитаны во
время выполнения задач.

Если ваш проект реализован на \Clang{}, \Cplusplus{} или Objective-C
вам следует рассмотреть возможность применения программы
\utility{distcc} (\filename{\url{http://distcc.samba.org}}),
предназначенной для распределённой компиляции. \utility{distcc}
написана Мартином Пулом (Martin Pool) и другими программистами для
ускорения сборок проекта Samba. Это законченное робастное решение для
проектов, написанных на \Clang{}, \Cplusplus{} или Objective-C.
Для использования этого инструмента достаточно заменить компилятор
\Clang{} программой \utility{distcc}:

{\footnotesize
\begin{alltt}
\$ \textbf{make --jobs=8 CC=distcc}
\end{alltt}
}

Для каждой компиляции \utility{distcc} использует препроцессор
для обработки исходного кода, затем отправляет результат другим узлам
сети для компиляции. Наконец, удалённые узлы возвращают полученные
объектные файлы управляющему процессу. Этот подход устраняет
необходимость в разделяемой файловой системе, что существенно упрощает
установку и конфигурацию.

Множество рабочих узлов или \newword{добровольцев} можно указать
несколькими способами. Наиболее простым является перечисление
узлов\hyp{}добровольцев в переменной окружения перед запуском
\utility{distcc}:

{\footnotesize
\begin{alltt}
\$ \textbf{export DISTCC\_HOSTS='localhost wasatch oops'}
\end{alltt}
}

\utility{distcc} имеет много опций для управления списком удалённых
узлов, интеграцией с компилятором, управления компрессией, путями
поиска, а также обнаружения и исправления ошибок.

Ещё одним инструментом увеличения скорости компиляции является
программа \utility{ccache}, написанная руководителем проекта Samba
Эндрю Тридгеллом (Andrew Tridgell). Идея очень проста:
\utility{ccache} кэширует результаты предыдущих сборок. Перед
осуществлением компиляции осуществляется проверка, содержит ли кэш
нужные объектные файлы. Это не требует участие нескольких узлов сети,
не требуется даже существование сети. Автор сообщает о 5-10 кратном
ускорении основного процесса компиляции. Наиболее простым способом
использования этого инструмента является переопределение команды
компиляции через интерфейс командной строки:

{\footnotesize
\begin{alltt}
\$ \textbf{make CC='ccache gcc'}
\end{alltt}
}

\utility{ccache} можно использовать совместно с \utility{distcc} для
ещё большего ускорения процесса сборки. В добавок ко всему, оба этих
инструмента доступны в наборе инструментов Cygwin.


%%%-------------------------------------------------------------------
%%% Example Makefiles
%%%-------------------------------------------------------------------
\chapter{Примеры \makefile{ов}}
\label{chap:example_makefiles}

\makefile{ы}, приведённые ранее в этой книге, достаточно хороши для
промышленного использования и вполне могут буть адаптированы под более
сложные требования. Тем не менее, стоит всё же рассмотреть
\makefile{ы} некоторых реальных проектов, чтобы оценить, что люди
могут сделать с помощью \GNUmake{} под давлением требований конкретных
продуктов. В этой главе мы детально рассмотрим несколько
\makefile{ов}. Первый \Makefile{} использовался для сборки этой
книги\TranslatorFootnote{Подразумевается оригинал книги, перевод был
  подготовлен при помощи других технологий}. Второй~--- для сборки
ядра Linux версии 2.6.7.

%%--------------------------------------------------------------------
%% The Book Makefile
%%--------------------------------------------------------------------
\section{\Makefile{} этой книги}

Написание книги о программировании само по себе является интересным
упражнением в построении систем сборки. Текст книги состоит из
множества файлов, каждый из которых требует различной
обработки. Примеры являются реальными программами, каждая из которых
должна быть запущена, а их вывод нуждается в сборе, обработке и
включении в основной текст (благодаря этому вывод не нужно копировать
и вставлять в текст, что избавляет от риска внесения ошибок). В
процессе написания книги полезно иметь возможность просмотреть текст в
различных форматах. Наконец, доставка материала требует
архивирования. Разумеется, все шаги должны быть воспроизводимыми и
относительно простыми в поддержке.

Выглядит как работа для \GNUmake{}! Возможность применения для
удивительно разнородных потребностей ~--- одна из самых замечательных
особенностей \GNUmake{}. Эта книга была написана в формате DocBook
(т.е. XML). \GNUmake{} ~--- стандартный инструмент при работе с
\TeX{}, \LaTeX{} и \command{troff}.

Следующий пример содержит полный \Makefile{} книги. В нём примерно 440
строк. \Makefile{} разделяется на следующие базовые задачи:

\begin{itemize}
  \item{} Управление примерами.
  \item{} Обработка XML.
  \item{} Генерация документов различных форматов.
  \item{} Проверка исходного кода.
  \item{} Базовые задачи поддержки.
\end{itemize}

\begin{verbatim}
# Сборка книги.
#
# Основные цели этого файла:
#
# show_pdf  Генерация pdf и запуск программы просмотра
# pdf       Генерация pdf
# print     Печать pdf
# show_html Генерация html и запуск программы просмотра
# html      Генерация html
# xml       Генерация xml
# release   Создание архива релиза
# clean     Удаление файлов, созданных в процессе сборки
#

BOOK_DIR     := /test/book
SOURCE_DIR   := text
OUTPUT_DIR   := out
EXAMPLES_DIR := examples

QUIET = @

SHELL       =  shell
AWK         := awk
CP          := cp
EGREP       := egrep
HTML_VIEWER := cygstart
KILL        := /bin/kill
M4          := m4
MV          := mv
PDF_VIEWER  := cygstart
RM          := rm -f
MKDIR       := mkdir -p
LNDIR       := lndir
SED         := sed
SORT        := sort
TOUCH       := touch
XMLTO       := xmlto
XMLTO_FLAGS =  -o $(OUTPUT_DIR) $(XML_VERBOSE)
process-pgm := bin/process-include
make-depend := bin/make-depend

m4-macros := text/macros.m4

# $(call process-includes, input-file, output-file)
# Осуществляет замену символов табуляции прбелами,
# подстановку макросов и обработку директив включения.
define process-includes
  expand $1 |                                             \
  $(M4) --prefix-builtins --include=text $(m4-macros) - | \
  $(process-pgm) > $2
endef

# $(call file-exists, file-name)
# Возвращает ненулевое значение в случае существования
# файла с заданным именем
file-exists = $(wildcard $1)

# $(call maybe-mkdir, directory-name-opt)
# Создаёт каталог, если он ещё не существует.
# Если параметр directory-name-opt опущен, в качестве имени
# каталога используется значение $@.
maybe-mkdir = $(if $(call file-exists,        \
                     $(if $1,$1,$(dir $@))),, \
                $(MKDIR) $(if $1,$1,$(dir $@)))

# $(kill-acroread)
# Завершает процесс Acrobat Reader
define kill-acroread
  $(QUIET) ps -W |                                 \
  $(AWK) 'BEGIN { FIELDWIDTHS = "9 47 100" }       \
          /AcroRd32/ {                             \
                       print "Killing " $$3;       \
                       system( "$(KILL) -f " $$1 ) \
                     }'
endef

# $(call source-to-output, file-name)
# Преобразует имя исходного файла в имя выходного файла.
define source-to-output
$(subst $(SOURCE_DIR),$(OUTPUT_DIR),$1)
endef

# $(call run-script-example, script-name, output-file)
# Запускает makefile примера.
define run-script-example
  ( cd $(dir $1);                                   \
    $(notdir $1) 2>&1 |                             \
    if $(EGREP) --silent '\$$\(MAKE\)' [mM]akefile; \
    then                                            \
      $(SED) -e 's/^++*/$$/';                       \
    else                                            \
      $(SED) -e 's/^++*/$$/'                        \
             -e '/ing directory /d'                 \
             -e 's/\[[0-9]\]//';                    \
    fi )                                            \
  > $(TMP)/out.$$$$ &                               \
  $(MV) $(TMP)/out.$$$$ $2
endef

# $(call generic-program-example,example-directory)
# Создаёт общие правила сборки примера.
define generic-program-example
  $(eval $1_dir      := $(OUTPUT_DIR)/$1)
  $(eval $1_make_out := $($1_dir)/make.out)
  $(eval $1_run_out  := $($1_dir)/run.out)
  $(eval $1_clean    := $($1_dir)/clean)
  $(eval $1_run_make := $($1_dir)/run-make)
  $(eval $1_run_run  := $($1_dir)/run-run)
  $(eval $1_sources  := $(filter-out %/CVS, \
                          $(wildcard $(EXAMPLES_DIR)/$1/*)))
  $($1_run_out): $($1_make_out) $($1_run_run)
      $$(call run-script-example, $($1_run_run), $$@)

  $($1_make_out): $($1_clean) $($1_run_make)
      $$(call run-script-example, $($1_run_make), $$@)

  $($1_clean): $($1_sources) Makefile
      $(RM) -r $($1_dir)
      $(MKDIR) $($1_dir)
      $(LNDIR) -silent ../../$(EXAMPLES_DIR)/$1 $($1_dir)
      $(TOUCH) $$@

  $($1_run_make):
      printf "#! /bin/bash -x\nmake\n" > $$@
endef

# Конечные форматы книги
BOOK_XML_OUT     := $(OUTPUT_DIR)/book.xml
BOOK_HTML_OUT    := $(subst xml,html,$(BOOK_XML_OUT))
BOOK_FO_OUT      := $(subst xml,fo,$(BOOK_XML_OUT))
BOOK_PDF_OUT     := $(subst xml,pdf,$(BOOK_XML_OUT))
ALL_XML_SRC      := $(wildcard $(SOURCE_DIR)/*.xml)
ALL_XML_OUT      := $(call source-to-output,$(ALL_XML_SRC))
DEPENDENCY_FILES := $(call source-to-output,\
                           $(subst .xml,.d,$(ALL_XML_SRC)))
# xml/html/pdf - Производит желаемые конечные форматы книги.
.PHONY: xml html pdf
xml:  $(OUTPUT_DIR)/validate
html: $(BOOK_HTML_OUT)
pdf:  $(BOOK_PDF_OUT)

# show_pdf - Создаёт pdf документ и отображает его.
.PHONY: show_pdf show_html print
show_pdf: $(BOOK_PDF_OUT)
    $(kill-acroread)
    $(PDF_VIEWER) $(BOOK_PDF_OUT)

# show_html - Создаёт html файл и отображает его.
show_html: $(BOOK_HTML_OUT)
    $(HTML_VIEWER) $(BOOK_HTML_OUT)

# print - Печатает заданные страницы книги.
print: $(BOOK_FO_OUT)
    $(kill-acroread)
    java -Dstart=15 -Dend=15 $(FOP) $< -print > /dev/null

# $(BOOK_PDF_OUT) - Создаёт pdf файл.
$(BOOK_PDF_OUT): $(BOOK_FO_OUT) Makefile

# $(BOOK_HTML_OUT) - Создаёт html файл.
$(BOOK_HTML_OUT): $(ALL_XML_OUT) $(OUTPUT_DIR)/validate Makefile

# $(BOOK_FO_OUT) - Создаёт временный файл в формате fo.
.INTERMEDIATE: $(BOOK_FO_OUT)
    $(BOOK_FO_OUT): $(ALL_XML_OUT) $(OUTPUT_DIR)/validate Makefile

# $(BOOK_XML_OUT) - Обрабатывает все входные xml файлы.
$(BOOK_XML_OUT): Makefile

#################################################################
# Поддержка FOP
#
FOP := org.apache.fop.apps.Fop

# DEBUG_FOP - Определите этот макрос для просмотра вывода fop.
ifndef DEBUG_FOP
  FOP_FLAGS := -q
  FOP_OUTPUT := | $(SED) -e '/not implemented/d'       \
                         -e '/relative-align/d'        \
                         -e '/xsl-footnote-separator/d'
endif

# CLASSPATH - Compute the appropriate CLASSPATH for fop.
export CLASSPATH
CLASSPATH = $(patsubst %;,%,                                \
              $(subst ; ,;,                                 \
                $(addprefix c:/usr/xslt-process-2.2/java/,  \
                  $(addsuffix .jar;,                        \
                    xalan                                   \
                    xercesImpl                              \
                    batik                                   \
                    fop                                     \
                    jimi-1.0                                \
                    avalon-framework-cvs-20020315))))

# %.pdf - Шаблонное правило создания pdf из fo.
%.pdf: %.fo
    $(kill-acroread)
    java -Xmx128M $(FOP) $(FOP_FLAGS) $< $@ $(FOP_OUTPUT)

# %.fo - Шаблонное правило для создания fo из xml.
PAPER_SIZE := letter
%.fo: %.xml
    XSLT_FLAGS="--stringparam paper.type $(PAPER_SIZE)" \
    $(XMLTO) $(XMLTO_FLAGS) fo $<

# %.html - Шаблонное правило для создания html из xml.
%.html: %.xml
    $(XMLTO) $(XMLTO_FLAGS) html-nochunks $<

# fop_help - Отображение спавки по использованию fop.
.PHONY: fop_help
fop_help:
    -java org.apache.fop.apps.Fop -help
    -java org.apache.fop.apps.Fop -print help

#################################################################
# release - Создаёт релиз книги
#
RELEASE_TAR   := mpwm-$(shell date +%F).tar.gz
RELEASE_FILES := README Makefile *.pdf bin examples out text
.PHONY: release
release: $(BOOK_PDF_OUT)
    ln -sf $(BOOK_PDF_OUT) .
    tar --create                 \
        --gzip                   \
        --file=$(RELEASE_TAR)    \
        --exclude=CVS            \
        --exclude=semantic.cache \
        --exclude=*~             \
        $(RELEASE_FILES)
    ls -l $(RELEASE_TAR)

#################################################################
# Правила для примеров из первой главы.
#
# Все каталоги с примерами.
EXAMPLES :=
    ch01-bogus-tab
    ch01-cw1
    ch01-hello
    ch01-cw2
    ch01-cw2a
    ch02-cw3
    ch02-cw4
    ch02-cw4a
    ch02-cw5
    ch02-cw5a
    ch02-cw5b
    ch02-cw6
    ch02-make-clean
    ch03-assert-not-null
    ch03-debug-trace
    ch03-debug-trace-1
    ch03-debug-trace-2
    ch03-filter-failure
    ch03-find-program-1
    ch03-find-program-2
    ch03-findstring-1
    ch03-grep
    ch03-include
    ch03-invalid-variable
    ch03-kill-acroread
    ch03-kill-program
    ch03-letters
    ch03-program-variables-1
    ch03-program-variables-2
    ch03-program-variables-3
    ch03-program-variables-5
    ch03-scoping-issue
    ch03-shell
    ch03-trailing-space
    ch04-extent
    ch04-for-loop-1
    ch04-for-loop-2
    ch04-for-loop-3
    ch06-simple
    appb-defstruct
    appb-arithmetic

# Я бы с радостью использовал этот цикл foreach, но ошибка
# в версии 3.80 приводит к аварийному останову.
#$(foreach e,$(EXAMPLES),$(eval $(call generic-program-example,$e)))

# Вместо этого приходитя раскрывать цикл вручную:
$(eval $(call generic-program-example,ch01-bogus-tab))
$(eval $(call generic-program-example,ch01-cw1))
$(eval $(call generic-program-example,ch01-hello))
$(eval $(call generic-program-example,ch01-cw2))
$(eval $(call generic-program-example,ch01-cw2a))
$(eval $(call generic-program-example,ch02-cw3))
$(eval $(call generic-program-example,ch02-cw4))
$(eval $(call generic-program-example,ch02-cw4a))
$(eval $(call generic-program-example,ch02-cw5))
$(eval $(call generic-program-example,ch02-cw5a))
$(eval $(call generic-program-example,ch02-cw5b))
$(eval $(call generic-program-example,ch02-cw6))
$(eval $(call generic-program-example,ch02-make-clean))
$(eval $(call generic-program-example,ch03-assert-not-null))
$(eval $(call generic-program-example,ch03-debug-trace))
$(eval $(call generic-program-example,ch03-debug-trace-1))
$(eval $(call generic-program-example,ch03-debug-trace-2))
$(eval $(call generic-program-example,ch03-filter-failure))
$(eval $(call generic-program-example,ch03-find-program-1))
$(eval $(call generic-program-example,ch03-find-program-2))
$(eval $(call generic-program-example,ch03-findstring-1))
$(eval $(call generic-program-example,ch03-grep))
$(eval $(call generic-program-example,ch03-include))
$(eval $(call generic-program-example,ch03-invalid-variable))
$(eval $(call generic-program-example,ch03-kill-acroread))
$(eval $(call generic-program-example,ch03-kill-program))
$(eval $(call generic-program-example,ch03-letters))
$(eval $(call generic-program-example,ch03-program-variables-1))
$(eval $(call generic-program-example,ch03-program-variables-2))
$(eval $(call generic-program-example,ch03-program-variables-3))
$(eval $(call generic-program-example,ch03-program-variables-5))
$(eval $(call generic-program-example,ch03-scoping-issue))
$(eval $(call generic-program-example,ch03-shell))
$(eval $(call generic-program-example,ch03-trailing-space))
$(eval $(call generic-program-example,ch04-extent))
$(eval $(call generic-program-example,ch04-for-loop-1))
$(eval $(call generic-program-example,ch04-for-loop-2))
$(eval $(call generic-program-example,ch04-for-loop-3))
$(eval $(call generic-program-example,ch06-simple))
$(eval $(call generic-program-example,ch10-echo-bash))
$(eval $(call generic-program-example,appb-defstruct))
$(eval $(call generic-program-example,appb-arithmetic))

#################################################################
# validate
#
# Производит проверку
# a) неподставленных макросов m4;
# b) символов табуляций;
# c) комментариев FIXME;
# d) RM: мои ответы Andy;
# e) дубликатов макросов m4.
#
validation_checks := $(OUTPUT_DIR)/chk_macros_tabs      \
                     $(OUTPUT_DIR)/chk_fixme            \
                     $(OUTPUT_DIR)/chk_duplicate_macros \
                     $(OUTPUT_DIR)/chk_orphaned_examples

.PHONY: validate-only
validate-only: $(OUTPUT_DIR)/validate
$(OUTPUT_DIR)/validate: $(validation_checks)
    $(TOUCH) $@

$(OUTPUT_DIR)/chk_macros_tabs: $(ALL_XML_OUT)
    # Ищем макросы и символы табуляции...
    $(QUIET)! $(EGREP) --ignore-case          \
                       --line-number          \
                       --regexp='\b(m4_|mp_)' \
                       --regexp='\011'
                       $^
    $(TOUCH) $@

$(OUTPUT_DIR)/chk_fixme: $(ALL_XML_OUT)
    # Ищем комментарии RM: и FIXME...
    $(QUIET)$(AWK)                                                \
            '/FIXME/  { printf "%s:%s: %s\n", FILENAME, NR, $$0 } \
             /^ *RM:/ {                                           \
                        if ( $$0 !~ /RM: Done/ )                  \
                        printf "%s:%s: %s\n", FILENAME, NR, $$0   \
                      }' $(subst $(OUTPUT_DIR)/,$(SOURCE_DIR)/,$^) 
    $(TOUCH) $@ 

$(OUTPUT_DIR)/chk_duplicate_macros: $(SOURCE_DIR)/macros.m4
    # Ищем дубликаты макросов...
    $(QUIET)! $(EGREP) --only-matching              \
        "\`[^']+'," $< |                            \
    $(SORT) |                                       \
    uniq -c |                                       \
    $(AWK) '$$1 > 1 { printf "$<:0: %s\n", $$0 }' | \
    $(EGREP) "^"
    $(TOUCH) $@

ALL_EXAMPLES := $(TMP)/all_examples

$(OUTPUT_DIR)/chk_orphaned_examples: $(ALL_EXAMPLES) $(DEPENDENCY_FILES)
    $(QUIET)$(AWK) -F/ '/(EXAMPLES|OUTPUT)_DIR/ { print $$3 }' \
            $(filter %.d,$^) |                                 \
    $(SORT) -u |                                               \
    comm -13 - $(filter-out %.d,$^)
    $(TOUCH) $@

.INTERMEDIATE: $(ALL_EXAMPLES)
$(ALL_EXAMPLES):
    # Ищем неиспользованные примеры...
    $(QUIET) ls -p $(EXAMPLES_DIR) | \
    $(AWK) '/CVS/ { next }           \
            /\// { print substr($$0, 1, length - 1) }' > $@
#################################################################
# clean
#
clean:
    $(kill-acroread)
    $(RM) -r $(OUTPUT_DIR)
    $(RM) $(SOURCE_DIR)/*~ $(SOURCE_DIR)/*.log semantic.cache
    $(RM) book.pdf

#################################################################
# Управление зависимостями
#
# Если выполняется цель clean, не стоит производить чтение или
# запуск включаемых файлов.
#
ifneq "$(MAKECMDGOALS)" "clean"
  -include $(DEPENDENCY_FILES)
endif

vpath %.xml $(SOURCE_DIR)
vpath %.tif $(SOURCE_DIR)
vpath %.eps $(SOURCE_DIR)

$(OUTPUT_DIR)/%.xml: %.xml $(process-pgm) $(m4-macros)
    $(call process-includes, $<, $@)

$(OUTPUT_DIR)/%.tif: %.tif
    $(CP) $< $@

$(OUTPUT_DIR)/%.eps: %.eps
    $(CP) $< $@

$(OUTPUT_DIR)/%.d: %.xml $(make-depend)
    $(make-depend) $< > $@

#################################################################
# Создание каталогов для вывода
#
# Создаём каталоги для вывода по мере необходимости.
#
DOCBOOK_IMAGES := $(OUTPUT_DIR)/release/images
DRAFT_PNG      := /usr/share/docbook-xsl/images/draft.png

ifneq "$(MAKECMDGOALS)" "clean"
  _CREATE_OUTPUT_DIR :=                                                 \
    $(shell                                                             \
      $(MKDIR) $(DOCBOOK_IMAGES) &                                      \
      $(CP) $(DRAFT_PNG) $(DOCBOOK_IMAGES);                             \
      if ! [[ $(foreach d,                                              \
                $(notdir                                                \
                  $(wildcard $(EXAMPLES_DIR)/ch*)),                     \
                -e $(OUTPUT_DIR)/$d &) -e . ]];                         \
      then                                                              \
        echo Компоновка примеров... > /dev/stderr;                \
        $(LNDIR) $(BOOK_DIR)/$(EXAMPLES_DIR) $(BOOK_DIR)/$(OUTPUT_DIR); \
      fi)
endif
\end{verbatim}

Этот \Makefile{} написан для запуска в Cygwin без серьёзных претензий
на переносимость в \UNIX{}. Тем не менее, я уверен, что в нём очень
мало (если вообще есть) несовместимостей с \UNIX{}, которые нельзя
было бы решить переопределением значений переменных или, возможно,
введением новых переменных.

Раздел глобальных переменных определяет расположение корневого
каталога и относительные пути к каталогам с текстом и примерами, а
также к каталогу для вывода. Имя каждой нетривиальной программы,
используемой в \makefile{е}, оформляется в виде переменной.

%---------------------------------------------------------------------
% Managing Examples
%---------------------------------------------------------------------
\subsection{Управление примерами}

Первая задача, управление примерами, является самой сложной. Каждый
пример располагается в собственном подкаталоге каталога
\filename{book/examples/chn-<title>}. Каждый пример содержит
собственный \Makefile{}. Для обработки примера мы сначала создаём
каталог, содержащий символические ссылки на деревья каталогов выходных
файлов и работаем в нём, поэтому артифакты, созданные в процессе
работы \GNUmake{}, не попадают в дерево каталогов с исходным
кодом. Более того, б\'{о}льшая часть примеров для производства
ожидаемого результата требует работы в том же каталоге, в котором
находится \Makefile{}. После создания символических ссылок на каталоги
с исходным кодом мы выполняем сценарий командного интерпретатора,
\command{run-make}, вызывающий \Makefile{} с правильными
аргументами. Если в каталоге с исходным кодом нет такого сценария, мы
выполняем стандартную версию сценария. Вывод сценария
\command{run-make} сохраняется в файле \filename{make.out}. Некоторые
примеры порождают исполняемые файлы, которые также нужно
выполнить. Эта работа выполняется сценарием \command{run-run}, его
вывод сохраняется в файле \filename{run.out}.

Создание каталога с символическими ссылками осуществляется следующим
кодом, находящимся в конце \makefile{а}:

\begin{verbatim}
ifneq "$(MAKECMDGOALS)" "clean"
  _CREATE_OUTPUT_DIR :=                                                 \
    $(shell                                                             \
      $(MKDIR) $(DOCBOOK_IMAGES) &                                      \
      $(CP) $(DRAFT_PNG) $(DOCBOOK_IMAGES);                             \
      if ! [[ $(foreach d,                                              \
                $(notdir                                                \
                  $(wildcard $(EXAMPLES_DIR)/ch*)),                     \
                -e $(OUTPUT_DIR)/$d &) -e . ]];                         \
      then                                                              \
        echo Компоновка примеров... > /dev/stderr;                \
        $(LNDIR) $(BOOK_DIR)/$(EXAMPLES_DIR) $(BOOK_DIR)/$(OUTPUT_DIR); \
      fi)
endif
\end{verbatim}

Этот код осуществляет одно присваивание простой переменной, обёрнутое
в директиву условной обработки \directive{ifneq}. Условная директива
нужна для того, чтобы \GNUmake{} не создавал структуру каталогов в
случае запуска команды \command{make clean}. На самом деле переменная
является фиктивной: её значение никогда не используется. Тем не менее,
функция \function{shell} справа от оператора присваивания выполняется
в процессе чтения \makefile{а}. Эта функция проверяет существование
каталога каждого примера в дереве выходных файлов. В случае отсутствия
какого-либо каталога вызывается команда \utility{lndir}, обновляющая
каталог с символическими ссылками.

Тест, выполняемый командой \command{if}, заслуживает более тщательного
рассмотрения. Он состоит из одного условия \command{-e} (существует ли
каталог?) для каталога каждого примера. Реальный код выглядит примерно
следующим образом: для нахождения всех примеров используется функция
\function{wildcard}, затем имена каталогов примеров отсекаются
функцией \function{notdir}, после чего для каждого каталога создаётся
текст \command{-e \$(OUTPUT\_DIR)/\textit{каталог} \&\&}. Все эти
элементы объединяются и подставляются в условие \command{bash
  [[\ldots{}]]}. Наконец, результат проверки условия
инвертируется. Одно дополнительное условие, \command{-e .}, включается
для того, чтобы позволить циклу \function{foreach} просто добавить
\command{\&\&} к каждому условию.

Этого достаточно для того, чтобы убедится в том, что новые каталоги
всегда будут включены в процесс сборки при обнаружении.

Следующим шагом является создание правил, обновляющие два выходных
файла, \filename{make.out} и \filename{run.out}. Это выполняется для
\filename{.out} файлов всех примеров следущей функцией:

\begin{verbatim}
# $(call generic-program-example,example-directory)
# Создаёт общие правила сборки примера.
define generic-program-example
  $(eval $1_dir      := $(OUTPUT_DIR)/$1)
  $(eval $1_make_out := $($1_dir)/make.out)
  $(eval $1_run_out  := $($1_dir)/run.out)
  $(eval $1_clean    := $($1_dir)/clean)
  $(eval $1_run_make := $($1_dir)/run-make)
  $(eval $1_run_run  := $($1_dir)/run-run)
  $(eval $1_sources  := $(filter-out %/CVS, \
                          $(wildcard $(EXAMPLES_DIR)/$1/*)))
  $($1_run_out): $($1_make_out) $($1_run_run)
      $$(call run-script-example, $($1_run_run), $$@)

  $($1_make_out): $($1_clean) $($1_run_make)
      $$(call run-script-example, $($1_run_make), $$@)

  $($1_clean): $($1_sources) Makefile
      $(RM) -r $($1_dir)
      $(MKDIR) $($1_dir)
      $(LNDIR) -silent ../../$(EXAMPLES_DIR)/$1 $($1_dir)
      $(TOUCH) $$@

  $($1_run_make):
      printf "#! /bin/bash -x\nmake\n" > $$@
endef
\end{verbatim}

Эта функция должна быть вызвана единожды с именем каталога каждого
примера в качестве аргумента:

\begin{verbatim}
$(eval $(call generic-program-example,ch01-bogus-tab))
$(eval $(call generic-program-example,ch01-cw1))
$(eval $(call generic-program-example,ch01-hello))
$(eval $(call generic-program-example,ch01-cw2))
\end{verbatim}

%---------------------------------------------------------------------
% XML Processing
%---------------------------------------------------------------------
\subsection{Обработка XML}

Рискуя выставить себя перед потомками в дурном свете, хочу сообщить,
что я не очень люблю формат XML. Я нахожу его неуклюжим и
многословным. Поэтому когда я узнал, что рукопись должна быть написана
в DocBook, я начал поиск более традиционных инструментов, которые
смогли бы упростить мою работу. Препроцессор \utility{m4} и
\utility{awk}~--- это два инструмента, которые мне очень помогли.

Есть две проблемы, связанные с DocBook и XML, с которыми \utility{m4}
отлично справляется: неудобство многословного синтаксиса XML и
необходимость управления идентификаторами, используемыми в
перекрёстных ссылках. К примеру, чтобы выделить слово в DocBook, вам
нужно написать:

\begin{verbatim}
<emphasis>not</emphasis>
\end{verbatim}

Используя \utility{m4}, я написал простой макрос, позволяющий записать
тоже самое следующим образом:

\begin{verbatim}
mp_em(not)
\end{verbatim}

Да, так уже лучше. В добавок я ввёл множество символических стилей
форматирования, таких как \command{mp\_variable} и
\command{mp\_target}. Это позволило мне выбрать тривиальный формат для
этих сущностей (к примеру, отсутсвие выделения) и изменять его позже
по желанию редактора без необходимости осуществлять поиск и замену по
всему документу.

Возможно, поклонники XML завалят меня письмами с описанием решения
этой задачи средствами XML (с помощью XML-сущностей или чего-нибудь в
этом роде). Однако не стоит забывать, что \UNIX{} нужен, чтобы решать
текущие задачи теми инструментами, которые у тебя есть. Как любит
говорить Ларри Уолл (Larry Wall), <<Есть более одного способа сделать
это>>(<<There is more then one way to do it>>). Кроме того, я
опасаюсь, что чрезмерное изучение XML заморочит мне голову.

Вторая задача для \utility{m4}~--- управление XML-идентификаторами,
используемыми в перекрёстных ссылках. Каждая глава, раздел, пример и
таблица имеют свой идентификатор:

\begin{verbatim}
<sect1 id="MPWM-CH-7-SECT-1">
\end{verbatim}

Ссылки на раздел должны использовать этот идентификатор. С точки
зрения программирования эта проблема довольно ясна. Идентификаторы
являются сложными константами, расбросанными по всему <<коду>>. Более
того, символы сами по себе не имеют значения. Я не имею понятия, о чём
может идти речь в первом разделе седьмой главы. Используя
\utility{m4}, я могу избежать дублирования сложных идентификаторов,
используя вместо них имеющие смысл имена:

\begin{verbatim}
<sect1 id="mp_se_makedepend">
\end{verbatim}

Что более важно, при смене нумерации разделов или глав (что в процессе
написания книги происходило много раз) идентификатор нужно будет
поменять только в одном файле. Это преимущество наиболее ощутимо при
смене нумерации разделов в главе. Подобная операция может потребовать
полдесятка операций поиска и замены по всем файлам, в которых я не
ввёл символические ссылки.

Вот несколько примеров макросов \utility{m4}\AuthorFootnote{Префикс
  \command{mp} является сокращением Managing Projects (название
  книги), слова macro processor (макро-процессор) или make pretty
  (буквально <<сделай красивым>>). Выберите наиболее понравившийся вам
  вариант}:
\begin{verbatim}
m4_define(`mp_tag',    `<$1>`$2'</$1>')
m4_define(`mp_lit',    `mp_tag(literal, `$1')')
m4_define(`mp_cmd',    `mp_tag(command,`$1')')
m4_define(`mp_target', `mp_lit($1)')
m4_define(`mp_all',    `mp_target(all)')
m4_define(`mp_bash',   `mp_cmd(bash)')
m4_define(`mp_ch_examples',     `MPWM-CH-11')
m4_define(`mp_se_book',         `MPWM-CH-11.1')
m4_define(`mp_ex_book_makefile',`MPWM-CH-11-EX-1')
\end{verbatim}

Ещё одной задачей предварительной обработки была реализация
возможности включения текста примеров. Этот текст требует замены
символов табуляций пробелами (посколько конвертер DocBook O'Reilly не
может обрабатывать символы табуляции, а в \makefile{ах} их полно),
обёртки содержимого в \verb|<![CDATA[...]]>| для экранирования
специальных символов, и, наконец, отсечения лишних символов переноса
строки в начале и конце текста примеров. Я смог решить эту задачу
благодаря следующей небольшой программе на \utility{awk}, которую я
назвал \utility{process-includes}:

\begin{verbatim}
#! /usr/bin/awk -f
function expand_cdata( dir )
{
  start_place = match( $1, "include-" )
  if ( start_place > 0 )
  {
    prefix = substr( $1, 1, start_place - 1 )
  }
  else
  {
    print "Bogus include '" $0 "'" > "/dev/stderr"
  }
  end_place = match( $2, "(</(programlisting|screen)>.*)$", tag )

  if ( end_place > 0 )
  {
    file = dir substr( $2, 1, end_place - 1 )
  }
  else
  {
    print "Bogus include '" $0 "'" > "/dev/stderr"
  }

  command = "expand " file

  printf "%s>&33;&91;CDATA[", prefix
  tail = 0
  previous_line = ""
  while ( (command | getline line) > 0 )
  {
    if ( tail )
      print previous_line;

    tail = 1
    previous_line = line
  }

  printf "%s&93;&93;&62;%s\n", previous_line, tag[1]
  close( command )
}

/include-program/ {
  expand_cdata( "examples/" )
  next;
}

/include-output/ {
  expand_cdata( "out/" )
  next;
}

/<(programlisting|screen)> *$/ {
  # Find the current indentation.
  offset = match( $0, "<(programlisting|screen)>" )

  # Strip newline from tag.
  printf $0

  # Read the program...
  tail = 0
  previous_line = ""
  while ( (getline line) > 0 )
  {
    if ( line ~ "</(programlisting|screen)>" )
    {
      gsub( /^ */, "", line )
      break
    }
    if ( tail )
      print previous_line

    tail = 1
    previous_line = substr( line, offset + 1 )
  }

  printf "%s%s\n", previous_line, line

  next
}

{
  print
}
\end{verbatim}

Сначала в \makefile{е} мы копируем XML-файлы из дерева каталогов с
исходными файлами в каталог с выходными файлами, заменяем символы
табуляции пробелами, производим подстановку макросов и производим
включение файлов примеров:

\begin{verbatim}
process-pgm := bin/process-include
m4-macros   := text/macros.m4

# $(call process-includes, input-file, output-file)
# Осуществляет замену символов табуляции прбелами,
# подстановку макросов и обработку директив включения.
define process-includes
  expand $1 |                                             \
  $(M4) --prefix-builtins --include=text $(m4-macros) - | \
  $(process-pgm) > $2
endef

vpath %.xml $(SOURCE_DIR)

$(OUTPUT_DIR)/%.xml: %.xml $(process-pgm) $(m4-macros)
    $(call process-includes, $<, $@)
\end{verbatim}

Шаблонное правило определяет способ составления выходного XML-файла из
исходного XML-файла. Оно также утвержает, что все выходные XML-файлы
должны быть составлены заново, если макросы или файл сценария
включения изменились.

%---------------------------------------------------------------------
% Generating Output
%---------------------------------------------------------------------
\subsection{Генерация документов}

%---------------------------------------------------------------------
% Validating the Source
%---------------------------------------------------------------------
\subsection{Проверка исходного кода}

  %%--------------------------------------------------------------------
%% The Linux Kernel Makefile
%%--------------------------------------------------------------------
\section{\Makefile{} ядра Linux}

\Makefile{} ядра Linux является отличным примером использования
\GNUmake{} для сборки в рамках сложной инфраструктуры. Поскольку целью
этой книги не является описание сруктуры и процесса сборки ядра Linux,
мы рассмотим лишь несколько интересных применений \GNUmake{} внутри
системы сборки ядра. Более подробное обсуждением системы сборки ядра
версий 2.5/2.6 и эволюции этой системы по сравнению с версией 2.4 вы
можете найти по адресу
\url{http://macarchive.linuxsymposium.org/ols2003/Proceedings/All-Reprints/Reprint-Germaschewski-OLS2003.pdf}.

Поскольку упомянутый \Makefile{} имеет так много аспектов, мы обсудим
лишь некоторые из тех, что могут быть использованы в различных
приложениях. Сначала мы рассмотрим, как однобуквенные переменные
\GNUmake{} используются для симуляции ключей командной строки. Мы
увидим, как разделить деревья каталогов с исходными и с бинарными
файлами разделяются, чтобы пользователи смогли вызывать \GNUmake{}
прямо из каталога с исходным кодом. Затем мы исследуем методику,
позволяющую \makefile{у} контролировать степень детализации
вывода. Далее, мы рассмотрим наиболее интересные функции, определяемые
пользователем, и увидим, как они препятствуют дублированию кода,
улучшают читаемость кода и инкапсулируют сложность. Наконец, мы
рассмотрим базовый функционал справки, реализованный с помощью
\GNUmake{}.

Ядро Linux следует известному шаблону \textit{конфигурация, сборка,
  установка} (configure, build, install), применяемому большинством
проектов свободного программного обеспечения. В то время как множество
открытых проектов используют отдельный сценарий
\filename{con\-fi\-gure} (обычно созданный при помощи
\utility{autoconf}), ядро Linux реализует стадию конфигурации с
помощью \GNUmake{}, вызывая внешние сценарии и вспомогательные
программы неявно.

Когда стадия конфигурации завершена, команда \command{make} или
\command{make all} собирает ядро, все модули и создаёт сжатый образ
ядра (цели \target{vmlinux}, \target{modules} и
\target{bzImage}, соответственно). Каждой сборке ядра присваивается
уникальный номер, хранящийся в файле \filename{version.o},
прилинкованном к ядру. Это число (и файл \filename{version.o})
обновляются средствами самого \makefile{а}.

%---------------------------------------------------------------------
% Command-Line Options
%---------------------------------------------------------------------
\subsection{Опции командной строки}
Первая часть \makefile{а} содержит код, устанавливающий общие опции
сборки, полученные из командной строки. Ниже приведена выдержка,
осуществляющая контроль флага детализации вывода:

\begin{verbatim}
# Чтобы предупреждения были более заметными, по-умолчанию
# отображается минимум сообщений.
# Для более детального вывода используйте 'make V=1'.
ifdef V
  ifeq ("$(origin V)", "command line")
    KBUILD_VERBOSE = $(V)
  endif
endif
ifndef KBUILD_VERBOSE
  KBUILD_VERBOSE = 0
endif
\end{verbatim}

Вложенная пара \directive{ifdef}/\directive{ifeq} проверяет, что
переменная \variable{KBUILD\_VERBOSE} выставляется только в том
случае, когда переменная \variable{V} была определена в командной
строке. Определение \variable{V} в окружении или внутри \makefile{а}
не возымеет эффекта. Следующая директива \directive{ifndef} выключает
флаг \variable{KBUILD\_VERBOSE}, если его значение ещё не было
определено. Чтобы включить детальный вывод из окружения или
\makefile{а}, вам нужно явно определить значение переменной
\variable{KBUILD\_VERBOSE}, а не \variable{V}.

Заметим, однако, что определение опции \variable{KBUILD\_VERBOSE} явно
в командной строке разрешено и работает именно так, как вы
ожидаете. Это может быть удобно для написания сценарев командного
интерпретатора (или псевдонимов команд) для вызова
\makefile{а}. Использование полного имени будут более
самодокументируем и похожим на длинные опции GNU.

Другие опции командной строки, запрос запуска анализатора
\utility{sparse} (\command{C}) и требование сборки внешних модулей
(\command{M}), используют аналогичную проверку, чтобы избежать
случайного их переопределения внутри \makefile{а}.

Следующая секция \makefile{а} производит оработку опции,
устанавливающей каталога вывода (\command{O}). Это довольно сложный
участок кода. Чтобы прояснить его структуру, мы заменим некоторые
части этого отрывка многоточиями:

\begin{verbatim}
# Система kbuild поддерживает сохранение выходных файлов в отдельном
# каталоге.
# Есть два пути воспользоваться этой возможностью. В обоих случаях
# рабочим каталогом должен быть каталог с исходным кодом ядра.
# 1) O=
# Используйте опцию O: "make O=dir/to/store/output/files/"
#
# 2) Определите  KBUILD_OUTPUT
# Определите переменную окружения KBUILD_OUTPUT так, чтобы она
# указывала на каталог, в который следует поместить выходные файлы.
# export KBUILD_OUTPUT=dir/to/store/output/files/
# make
#
# Опция O= имеет более высокий приоритет, чем переменная окружения
# KBUILD_OUTPUT.
# Переменная KBUILD_SRC выставляется в значение OBJ после старта make
# На данный момент не предполагается, что KBUILD_SRC будет
# использоваться пользователями без прав суперпользователя.
ifeq ($(KBUILD_SRC),)
  # Нас вызвали из каталога, в котором распологается исходный код
  # ядра. Требуется ли помещать объектные файлы в отдельный каталог?
  ifdef O
    ifeq ("$(origin O)", "command line")
      KBUILD_OUTPUT := $(O)
    endif
  endif
  ...
  ifneq ($(KBUILD_OUTPUT),)
    ...
    .PHONY: $(MAKECMDGOALS)
    $(filter-out _all,$(MAKECMDGOALS)) _all:
        $(if $(KBUILD_VERBOSE:1=),@)$(MAKE) -C $(KBUILD_OUTPUT)       \
        KBUILD_SRC=$(CURDIR)         KBUILD_VERBOSE=$(KBUILD_VERBOSE) \
        KBUILD_CHECK=$(KBUILD_CHECK) KBUILD_EXTMOD="$(KBUILD_EXTMOD)" \
        -f $(CURDIR)/Makefile $@
    # Поручаем сборку уже вызванному make
    skip-makefile := 1
  endif # ifneq ($(KBUILD_OUTPUT),)
endif # ifeq ($(KBUILD_SRC),)

# Оставшаяся часть makefile обрабатывается только в том случае, если
# текущий вызов make - последний.
ifeq ($(skip-makefile),)
  ...the rest of the makefile here...
endif
# skip-makefile
\end{verbatim}

Вкратце, этот участок кода проверяет, определена ли переменная
\variable{KBUILD\_OUTPUT}, и, если это так, вызывает \GNUmake{}
рекурсивно в каталоге, имя которого хранится в переменной
\variable{KBUILD\_OUTPUT}, определяя переменную \variable{KBUILD\_SRC}
так, чтобы она содержала путь к каталогу, в котором \GNUmake{} был
выполнен в первый раз. При этом для сборки используется исходный
\Makefile{}. Также выставляется флаг \variable{skip-makefile}, из-за
чего оставшаяся часть \makefile{а} не будет видна
\GNUmake{}. Рекурсивный \GNUmake{} прочтёт тот же самый \Makefile{}
ещё раз, только в этот раз переменная \variable{KBUILD\_SRC} будет
определена, поэтому флаг \variable{skip-makefile} не будет определён,
и остаток \makefile{а} будет прочитан и обработан.

На этом мы закончим рассмотрение опций командной строки. Большая часть
\makefile{а} находится в секции \command{ifeq(\$(skip-makefile),)}.

%---------------------------------------------------------------------
% Configuration Versus Building
%---------------------------------------------------------------------
\subsection{Конфигурация или сборка?}
\Makefile{} содержит цели для конфирурации и сборки. Конфигурационные
цели имеют форму \target{menuconfig}, \target{defconfig}, и
т.д. Вспомогательные цели, такие как \target{clean}, также трактуются
как конфигурационные цели. Другие цели, такие как \target{all},
\target{vmlinux} и \target{modules}, являются целями сборки. Главным
результатом вызова конфигурационных целей являются два файла:
\filename{.config} и \filename{.config.cmd}. Эти два файла включаются
\makefile{ом} для целей сборки и не включаются для конфигурационных
целей (поскольку именно конфигурационные цели создают эти
файлы). Возможно также смешивать конфигурационные цели и цели сборки в
одном вызове \GNUmake{}:

\begin{verbatim}
\$ make oldconfig all
\end{verbatim}

В этом случае \Makefile{} вызывает \GNUmake{} рекурсивно для
индивидуальной обработки каждой цели, таким образом обрабатывая
конфигурационные цели отдельно от целей сборки.

Ниже приводится начало участка кода, управляющего конфигурацией,
сборкой и смешением целей.

\begin{verbatim}
# Чтобы убедиться в том, что мы не включаем файл .config для
# конфигурационный целей, мы обрабатываем их заранее, передавая
# их scripts/kconfig/Makefile
# При вызове \GNUmake{} разрешается указывать несколько целей, а также
# смешивать конфигурационные цели и цели сборки.
# Пример: 'make oldconfig all'.
# Ситуацию со смешением целей нужно обрабатывать особым образом:
# производить повторный вызов make так, чтобы файл .config не
# включался для конфигурационных целей и в этом случае.

no-dot-config-targets := clean mrproper distclean \
                         cscope TAGS tags help %docs check%

config-targets := 0
mixed-targets  := 0
dot-config     := 1
\end{verbatim}

% TODO(rkashitsyn): Missing text

%---------------------------------------------------------------------
% Managing Command Echo
%---------------------------------------------------------------------
\subsection{Управление командой \command{echo}}
\label{sec:managing_command_echo}
\makefile{ы} ядра используют новаторскую технику управления уровнем
детализации вывода, производимого выполняемыми комадами. Каждая важная
задача представлена в двух вариантах: с тихим и с детальным режимом
вывода. Делальная версия содержит только команду, которую нужно
выполнить, в естественной форме и сохранена в переменной
\variable{cmd\_\ItalicMono{action}}. Тихая версия содержит короткое
сообщение, описывающее выполняемое действие, и хранится в переменной
\variable{quiet\_cmd\_\ItalicMono{action}}. Например, команада, создающая
файл символов для \utility{emacs}, выглядит следующим образом:

\begin{verbatim}
quiet_cmd_TAGS = MAKE $@
cmd_TAGS = $(all-sources) | etags -
\end{verbatim}

Команда может быть выполнена с помощью вызова функции \function{cmd}:

\begin{verbatim}
# Если переменная quiet выставлена, использовать короткую версию команды
cmd = @$(if $($(quiet)cmd_$(1)),\
         echo ' $($(quiet)cmd_$(1))' &&) $(cmd_$(1))
\end{verbatim}

Чтобы вызвать код, формирующий файл символов для \utility{emacs},
\Makefile{} должен содержать следующий код:

\begin{verbatim}
TAGS:
    $(call cmd,TAGS)
\end{verbatim}

Обратите внимание на то, что функция \function{cmd} начинается с символа
\Monospace{@}, поэтому единственный вывод, производимые ей, является
выводом команды \command{echo}. В нормальном режиме переменная
\variable{queit} не определена, и условие
\command{if, \$(\$(quiet)cmd\_\$(1))} возвращает \command{\$(cmd\_TAGS)}.
Поскольку эта переменная определена, результатом всего выражения будет
команда

\begin{verbatim}
echo ' $(all-sources) | etags -' && $(all-sources) | etags -
\end{verbatim}

Если тихий режим вывода более предпочтителен, переменная
\variable{queit} содержит текст \command{queit\_}, и результатом
вычисления функции будет выражение

\begin{verbatim}
echo ' MAKE $@' && $(all-sources) | etags -
\end{verbatim}

Значением переменной также может быть \command{silent\_}. Поскольку команда
\command{silent\_cmd\_TAGS} не определена, вызов функции \command{cmd} ничего
не выводит.

Вывод команд иногда более затруднителен, в частности, если команды содержат
апострофы. В этом случае \Makefile{} содержит следующий код:

\begin{verbatim}
$(if $($(quiet)cmd_$(1)),echo ' $(subst ','\'',$($(quiet)cmd_$(1)))';)
\end{verbatim}

Команда \command{echo} содержит подстановку, которая экранирует апострофы,
благодаря чему они выводятся правильным образом.

Небольшие команды, не требующие определения переменных \command{cmd\_}
и \command{quiet\_cmd\_}, имеют префикс \command{\$(0)}, который может
быть пуст или равен \command{@}:

\begin{verbatim}
ifeq ($(KBUILD_VERBOSE),1)
  quiet =
  Q =
else
  quiet=quiet_
  Q = @
endif

# Если пользователь выполняет команду make -s ("тихий" режим),
# подавить вывод команд

ifneq ($(findstring s,$(MAKEFLAGS)),)
  quiet=silent_
endif
\end{verbatim}

%---------------------------------------------------------------------
% User-Defined Functions
%---------------------------------------------------------------------
\subsection{Функции, опредённые пользователем}

\Makefile{} ядра определяет несколько функций. В этом разделе мы
рассмотрим наиболее интересные из них. Форматирование кода было
изменено для улучшения читабельности.

Функция \function{check\_gcc} используется для выбора опций
командной строки \utility{gcc}.

\begin{verbatim}
# $(call check_gcc,preferred-option,alternate-option)
check_gcc = \
  $(shell if $(CC) $(CFLAGS) $(1) -S -o /dev/null \
             -xc /dev/null > /dev/null 2>&1;      \
          then                                    \
            echo "$(1)";                          \
          else                                    \
            echo "$(2)";                          \
          fi ;)
\end{verbatim}

Функция вызывает \utility{gcc} с пустым списком исходных файлов
с предопределёнными опциями командной строки. Выходной файл,
а также содержимое стандартных потоков вывода и ошибки отбрасываются.
Если выполнение \utility{gcc} завершается успехом, это означает, что
предопределённые опции командной строки допустимы на данной архитектуре,
и функция возвращает эти опции в качестве результата. В противном
случае опции являются недопустимыми, и функция возвращает алтернативный
набор опций. Пример использования этой опции может быть найден в файле
\filename{arch/i386/Makefile}:

\begin{verbatim}
# Запрещаем gcc сохранять 16-байтовое выравнивание сегмента стека
CFLAGS += $(call check_gcc,-mpreferred-stack-boundary=2,)
\end{verbatim}

Функция \function{if\_changed\_dep} генерирует информацию о
зависимостях, используя довольно интересную технику.

\begin{verbatim}
# Выполняет команду и выполняет пост-обработку составленного
# .d файла зависимостей
if_changed_dep =                                         \
    $(if                                                 \
      $(strip $?                                         \
        $(filter-out FORCE $(wildcard $^),$^)            \
        $(filter-out $(cmd_$(1)),$(cmd_$@))              \
        $(filter-out $(cmd_$@),$(cmd_$(1)))),            \
      @set -e;                                           \
      $(if $($(quiet)cmd_$(1)),                          \
        echo ' $(subst ','\'',$($(quiet)cmd_$(1)))';)    \
      $(cmd_$(1));                                       \
      scripts/basic/fixdep                               \
          $(depfile)                                     \
          $@                                             \
          '$(subst $$,$$$$,$(subst ','\'',$(cmd_$(1))))' \
          > $(@D)/.$(@F).tmp;                            \
      rm -f $(depfile);                                  \
      mv -f $(@D)/.$(@F).tmp $(@D)/.$(@F).cmd)
\end{verbatim}

Функция состоит из одного условного выражения. Детали условия
достаточно запутанны, однако, достаточно чётко выделяется
намерение получить непустое значение, если файл с зависимостями
должен быть обновлён. Обычно информация о зависимостях
рассматривается в контексте даты последней модификации файлов.
Система сборки ядра добавляет к этой задаче дополнительные ньюансы.
Сборка ядра требует огромного количества опций компиляции для
контроля сборки и поведения компонентов. Чтобы убедиться, что
опции командной строки учитываются при сборке правильным образом,
\Makefile{} производит перекомпиляцию файла при изменении опций
для соответствующей цели. Перейдём к более детальному рассмотрению
механизма, с помощью которого это реализовано.

Команда, используемая для компиляции каждого файла ядра, сохраняется в
файле с расширением \filename{.cmd}. Когда выполняется повторная
сборка, \GNUmake{} читает \filename{.cmd}-файл и сравнивает текущую
команду компиляции с предыдущей. Если они отличаются, в
\filename{.cmd}-файл записывается новое значение команды, что вызывает
пересборку объектного файла. \filename{.cmd} файл обычно состоит из
двух частей: списка файлов-зависимостей целевого файла и одной
переменной, содержащей список опций компилятора. Например, файл
\filename{arch/i386/cpu/mtrr/if.c} порождает следующий (сокращённый)
\filename{.cmd}-файл:

\begin{verbatim}
cmd_arch/i386/kernel/cpu/mtrr/if.o := gcc -Wp,-MD ...; if.c

deps_arch/i386/kernel/cpu/mtrr/if.o := \
arch/i386/kernel/cpu/mtrr/if.c \
...

arch/i386/kernel/cpu/mtrr/if.o: $(deps_arch/i386/kernel/cpu/mtrr/if.o)
$(deps_arch/i386/kernel/cpu/mtrr/if.o):
\end{verbatim}

Вернёмся к функции \function{if\_changed\_dep}. Первый аргумент
\function{strip}~--- это (возможно, пустой) список реквизитов,
модифицированных позднее, чем цель. Второй аргумент~--- это все
реквизиты, не являющиеся файлами или специальной целью
\target{FORCE}. Предназначение последних двух вызовов
\function{filter-out} определённо требует пояснения:

\begin{verbatim}
$(filter-out $(cmd\_$(1)),$(cmd\_$@))
$(filter-out $(cmd\_$@),$(cmd\_$(1)))
\end{verbatim}

Результат вычисления этих вызовов будет непустой строкой, если
аргументы командной строки изменились. Результатом подстановки макроса
\command{\$(cmd\_\$(1))} является текущая команда, а макроса
\command{\$(cmd\_\$@)}~--- предыдущая команда, например, переменная
\variable{cmd\_arch/i386/kernel/cpu/mtrr/if.o} из предудущего
примера. Если новая команда содержит дополнительные опции, результатом
первого вызова \function{filter-out} будет пустая строка, а
результатом второго~--- новые опции. Если же новая команда содержит
меньше опций, первый результат будет содержать удалённые опции, а
второй будет пустым. Заметим, что поскольку \function{filter-out}
принимает список слов (каждое из которых интерпретируется как
независимый шаблон), случай изменения порядка опций будет обработан
корректно. Довольно изящное решение.

Первая инструкция в теле команды устанавливает опции интерпретатора,
вызывающие немедленное завершение выполнения в случае ошибки. Это
предотвращает повреждение файлов многострочными сценариями в случае
возникновения проблем. В случае простых сценариев альтернативой может
быть соединение инструкций оператором \command{\&\&}, а не точкой с
запятой.

Следующая инструкция~--- это команда \command{echo}, записанная с
использованием техники, описанной в разделе
\nameref{sec:managing_command_echo} текущей главы. Далее следует
непосредственно команда генерации зависимостей, создающая файл
\variable{\$(depfile)}, который затем трансформируется сценарием
\filename{scripts/basic/fixdep}. Вложенные в \command{fixdep} вызовы
\function{subst} экранируют вхождения последовательности \command{\$\$}
(командный интерпретатор сопоставляет ей идентификатор текущего
процесса).

В случае отсутсвия ошибок вспомогательный файл \variable{\$(depfile)}
удаляется, а сгенерированный файл зависимостей (с расширением
\filename{.cmd}) перемещается в соответствующий каталог.

Следующая функция, \function{if\_changed\_rule}, использует для
управления командами ту же технику сравнения, что и
\function{if\_changed\_dep}:

\begin{verbatim}
# Usage: $(call if_changed_rule,foo)
# will check if $(cmd_foo) changed, or any of the prequisites changed,
# and if so will execute $(rule_foo)
if_changed_rule =                                   \
    $(if $(strip $?                                 \
           $(filter-out $(cmd_$(1)),$(cmd_$(@F)))   \
           $(filter-out $(cmd_$(@F)),$(cmd_$(1)))), \
      @$(rule_$(1)))
\end{verbatim}

Эта функция используется внутри макросов в \makefile{е} верхнего
уровня, чтобы слинковаять ядро:

\begin{verbatim}
# Нетривиальный момент: Если мы хотим произвести повторную
# линковку vmlinux, желательно увеличить номер версии, что
# означает перекомпиляцию init/version.o и линковку init/init.o.
# Однако, мы не можем сделать это во время фазы рекурсивной
# сборки (descending-into-subdirs phase), поскольку на этом
# этапе мы не можем знать, потребуется ли повторная линковка
# vmlinux. Поэтому мы снова рекурсивно запускаем make в каталоге
# init/ в рамках правила для vmlinux.

...

quiet_cmd_vmlinux__ = LD $@
define cmd_vmlinux__
  $(LD) $(LDFLAGS) $(LDFLAGS_vmlinux) \
  ...
endef

# set -e заставляет правило завершиться немедлненно
# в случае ошибки

define rule_vmlinux__
  +set -e;                                             \
  $(if $(filter .tmp_kallsyms%,$^),,                   \
    echo ' GEN    .version';                           \
    . $(srctree)/scripts/mkversion > .tmp_version;     \
    mv -f .tmp_version .version;                       \
    $(MAKE) $(build)=init;)                            \
  $(if $($(quiet)cmd_vmlinux__),                       \
    echo ' $($(quiet)cmd_vmlinux__)' &&)               \
  $(cmd_vmlinux__);                                    \
  echo 'cmd_$@ := $(cmd_vmlinux__)' > $(@D)/.$(@F).cmd
endef

define rule_vmlinux
  $(rule_vmlinux__);            \
  $(NM) $@ |                    \
  grep -v '\(compiled\)\|...' | \
  sort > System.map
endef
\end{verbatim}

Функция \function{if\_changed\_rule} используется для вызова правила
\function{rule\_vmlinux}, которое выполняет линковку и собирает
финальный файл \filename{System.map}. Как указано в комментариях к
\makefile{у}, функция \function{rule\_vmlinux\_\_} отвечает за
генерацию версии ядра и повторную линковку \filename{init.o} перед
повторной линковкой \filename{vmlinux} (за это отвечает первый
оператор \function{if}). Второй опрератор \function{if} контролирует
вывод команды линковки,
\variable{\$(cmd\_vmlinux\_\_)}. Непосредственно выполняемая команда
записывается в \filename{.cmd}-файле для возможности сравнения при
следующей сборке.


%%%-------------------------------------------------------------------
%%% Debugging Makefiles
%%%-------------------------------------------------------------------
\chapter{Отладка \makefile{ов}}
\label{chap:debugging_makefiles}

%%--------------------------------------------------------------------
%% Debugging Features of Make
%%--------------------------------------------------------------------
\section{Отладочные возможности \GNUmake{}}

\index{Функции!встроенные!warning@\function{warning}}
Функция \function{warning} очень полезна для отладки
\makefile{ов}. Поскольку результатом вычисления этой функции является
пустая строка, её можно помещать в любом месте \makefile{а}: на самом
верхнем уровне, в списке реквизитов или в секции команд правила. Это
позволяет вам печатать значения переменных в том месте, где это
наиболее удобно. Например:

\begin{verbatim}
$(warning Предупреждение верхнего уровня)

FOO := $(warning Правая часть простой переменной)bar
BAZ = $(warning Правая часть рекурсивной переменной)boo

$(warning Цель)target: $(warning Список реквизитов)makefile $(BAZ)
    $(warning Командный сценарий)
    ls

$(BAZ):
\end{verbatim}

Производит следующий вывод:

\begin{verbatim}
$ make
makefile:1: Предупреждение верхнего уровня
makefile:2: Правая часть простой переменной
makefile:5: Цель
makefile:5: Список реквизитов
makefile:5: Правая часть рекурсивной переменной
makefile:8: Правая часть рекурсивной переменной
makefile:6: Командный сценарий
ls
makefile
\end{verbatim}

Обратите внимание на вычисление функции \function{warning} следует
обычной схеме аппликативных и отложенных вычислений
\GNUmake{}. Несмотря на то, что вычисление \variable{BAZ} содержит
вызов \function{warning}, соответствующее сообщение не будет
напечатано до вычисление \variable{BAZ} в списке реквизитов.

Возможность внедрять \function{warning} в любое место состовляет
по сути основной отладочный механизм.

\subsection{Опции командной строки}

Следующие три опции командной строки я нахожу наиболее удобными для
отладки: \command{-{}-just-print (-n)}, \command{-{}-print-data-base
  (-p)}, \command{-{}-warn-undefined-variables}.

\subsubsection{\function{-{}-just-print}}
\index{Опции!just-print@\command{-{}-just-print (-n)}}

Первым тестом, выполняемым мной для новой цели \makefile{а}, это вызов
\GNUmake{} с опцией \command{-{}-just-print (-n)}. Эта опция вынуждает
\GNUmake{} читать \Makefile{} и печатать все команды, которую он бы
выполнял при сборке цели, без их непосредственного выполнения. Для
удобства \GNUmake{} также печатает все команды \command{echo},
помеченные модификатором подавления вывода \command{(@)}.

Эта опция подавляет выполнение всех команд. Однако вам нужно быть
осторожным: хоть \GNUmake{} и не будет выполнять сценарии правил,
вызовы функции \function{shell} в аппликативном контексте всё же будут
выполнены. Например:

\begin{verbatim}
REQUIRED_DIRS = ...
_MKDIRS := $(shell for d in $(REQUIRED_DIRS); \
             do                               \
               [[ -d $$d ]] || mkdir -p $$d;  \
             done)

$(objects) : $(sources)
\end{verbatim}

Как мы уже видели раньше, назначение простой переменной
\variable{\_MKDIRS}~--- создание необходимых каталогов. При выполнении
с опцией \command{-{}-just-print} \GNUmake{} вызовет функцию
\function{shell} во время чтения \makefile{а}. Уже после этого
\GNUmake{} будет печатать (не выполняя) команды компиляции,
необходимые для сборки файлов из списка \command{\$(objects)}.

%%--------------------------------------------------------------------
%% Writing Code for Debugging
%%--------------------------------------------------------------------
\section{Отладочный код}

%%--------------------------------------------------------------------
%% Common Error Messages
%%--------------------------------------------------------------------
\section{Основные сообщения об ошибках}


