%%--------------------------------------------------------------------
%% Read only source
%%--------------------------------------------------------------------
\section{Объявляем права <<только для чтения>>}

Как только деревья каталогов с исходными и бинарными файлами
разделены, возможность объявления прав доступа <<только для чтения>>
для справочного дерева каталогов с исходным кодом получается
практически бесплатно, если только все объектные файлы, создаваемые
сборкой, помещаются в дерево каталогов бинарных файлов. Однако если
при сборке создаются исходные файлы, мы должны позаботиться о том,
чтобы они также были помещены в дерево каталогов бинарных файлов.

В более простом подходе, основанном на компиляции в бинарном дереве,
все создаваемые файлы помещались в бинарное дерево автоматически, так
как именно из него происходил вызов программ \utility{lex} и
\utility{yacc}. При подходе, основанном на компиляции из дерева
каталогов с исходными файлами, мы были вынуждены указывать явные пути
для исходных и целевых файлов, поэтому спецификация пути к файлу в
бинарном дереве каталогов не потребует дополнительной работы, нужно
просто не забыть сделать это.

Источником остальных препятствий для объявления дерева каталогов с
исходными файлами доступным только для чтения обычно является среда.
Часто система сборки, доставшаяся вам по наследству, содержит
действия, создающие файлы в дереве каталогов с исходными файлами, так
как первоначальный её автор не осознал преимущества исходного дерева,
доступного только для чтения. В качестве примеров можно привести
автоматически составленную документацию, файлы журналов и временные
файлы. Перемещение этих файлов в дерево каталогов с бинарными файлами
может потребовать значительных усилий, однако, если нужна поддержка
нескольких деревьев каталогов бинарных файлов, эта жертва является
необходимой. Альтернативой является поддержка и синхронизация
нескольких идентичных деревьев каталогов с исходными файлами.
