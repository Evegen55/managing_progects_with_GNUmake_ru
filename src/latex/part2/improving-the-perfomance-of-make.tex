%%%-------------------------------------------------------------------
%%% Improving the performance of make
%%%-------------------------------------------------------------------
\chapter{Повышаем производительность \GNUmake{}}
\label{chap:improving_the_performance}

\GNUmake{} имеет чрезвычайно важную роль в процессе разработки
программного обеспечения. Компонуя составляющие проекта в приложение,
\GNUmake{} позволяет разработчикам избежать трудноуловимых ошибок,
связанных со случайным пропуском какого-то шага сборки. Однако, если
разработчики избегают использования \GNUmake{} из-за низкой скорости
выполнения сборки, все преимущества использования \GNUmake{} теряются.
Таким образом, чрезвычайно важно убедиться в том, что \Makefile{} был
составлен с расчётом на максимальную производительность.

Проблемы производительности всегда довольно запутаны, однако, если
взять в рассмотрение восприятие пользователей и различные пути
выполнения кода, всё становится ещё сложнее. Не каждая цель в
\makefile{е} нуждается в оптимизации. В некоторых условиях даже
радикальные оптимизации могут не оправдать затраченных на них усилий. 
Например, сокращение времени сборки с 90 до 45 минут может быть
несущественным, поскольку даже с учётом оптимизации сборка становится
операцией, начав которую, можно <<идти на обед>>. С другой стороны,
сокращение времени сборки с двух минут до одной может сопровождаться
аплодисментами разработчиков, если во время сборки они вынуждены
сидеть сложа руки.

При написании эффективных \makefile{ов} важно знать стоимость
различных операций, а также ясно понимать, какие именно из этих
операций выполняются. В последующих разделах мы проведём несколько
простых тестов производительности, позволяющих дополнить эти общие
комментарии количественными данными и описать техники, помогающие
найти узкие места (bottlenecks).

Другим подходом к повышению производительности является использование
параллелизма и топологий локальных сетей. Одновременное исполнение
более одного командного сценария (даже на одном процессоре) может
существенно сократить время сборки.

%%--------------------------------------------------------------------
%% Benchmarking
%%--------------------------------------------------------------------
\section{Измеряем производительность}

В этом разделе мы измерим производительность базовых операций
\GNUmake{}. Таблица~\ref{tab:cost_of_operations} содержит результаты
этих измерений. Далее будет рассмотрен каждый из тестов, а также
представлены соображения по поводу влияния этих результатов на
написанные вами \Makefile{}'ы.

\begin{table}[!b]
{\footnotesize
\begin{tabular}{llllll}
\hline
\vspace{0.3em}
Операция &
Повторений &
\parbox[b]{2cm}{\flushleft Секунд на \break выполнение (Windows)} &
\parbox[b]{2cm}{\flushleft Число \break выполнений в секунду (Windows)} &
\parbox[b]{2cm}{\flushleft Секунд на \break выполнение (Linux)} &
\parbox[b]{2cm}{\flushleft Число \break выполнений в секунду (Linux)} \\
\hline
\vspace{0.5em}
make (bash) & $\hphantom{0.}1000$ & 0,0436 & $\hphantom{00}22$ & 0,0162 & $\hphantom{000.}61$ \\
\vspace{0.5em}
make (ash) & $\hphantom{0.}1000$ & 0,0413 & $\hphantom{00}24$ & 0,0151 & $\hphantom{000.}66$ \\
\vspace{0.5em}
make (bash) & $\hphantom{0.}1000$ & 0,0452 & $\hphantom{00}22$ & 0,0159 & $\hphantom{000.}62$ \\
\vspace{0.5em}
присваивание & 10.000 & 0,0001 & 8130 & 0,0001 & 10.989 \\
\vspace{0.5em}
subst (short) & 10.000 & 0,0003 & 3891 & 0,0003 & $\hphantom{0.}3846$ \\
\vspace{0.5em}
subst (long) & 10.000 & 0,0018 & $\hphantom{0}547$ & 0,0014 & $\hphantom{00.}704$ \\
\vspace{0.5em}
sed (bash) & $\hphantom{0.}1000$ & 0,0910 & $\hphantom{00}10$ & 0,0342 & $\hphantom{000.}29$ \\
\vspace{0.5em}
sed (ash) & $\hphantom{0.}1000$ & 0,0699 & $\hphantom{00}14$& 0,0069 & $\hphantom{00.}144$ \\
\vspace{0.5em}
sed (sh) & $\hphantom{0.}1000$ & 0,0911 & $\hphantom{00}10$ & 0,0139 & $\hphantom{000.}71$ \\
\vspace{0.5em}
shell (bash) & $\hphantom{0.}1000$ & 0,0398 & $\hphantom{00}25$ & 0,0261 & $\hphantom{000.}38$ \\
\vspace{0.5em}
shell (ash) & $\hphantom{0.}1000$ & 0,0253 & $\hphantom{00}39$ & 0,0018 & $\hphantom{00.}555$ \\
\vspace{0.3em}
shell (sh) & $\hphantom{0.}1000$ & 0,0399 & $\hphantom{00}25$ & 0,0050 & $\hphantom{00.}198$ \\
\hline
\end{tabular}
}
\caption{Стоимость базовых операций} \label{tab:cost_of_operations}
\end{table}

Тесты для Windows запускались на Pentium 4 с тактовой частотой 1,9 ГГц
(приблизительно 3578 BogoMips\footnote{
Объяснение величины BogoMips можно найти на сайте
\url{http://www.clifton.nl/bogomips.html } (прим. автора).})
и оперативной памятью 512 Мб под управлением операционной системы
Windows XP. Использовался Cygwin \GNUmake{} версии 3.80, запускаемый
из окна \utility{rxvt}. Тесты для Linux запускались на Pentium 2 с
тактовой частотой 450 ГГц (891 BogoMips) и оперативной памятью 256 Mб
под управлением операционной системы Linux RedHat 9.

Командный интерпретатор, используемый \GNUmake{}, может существенно
повлиять на производительность выполнения \makefile{а}. Командный
интерпретатор \utility{bash} сложен и обладает обширной
функциональностью, поэтому он достаточно тяжеловесен. Интерпретатор
\utility{ash} гораздо легче, он обладает меньшими возможностями,
впрочем, вполне подходящими для большинства задач. Чтобы усложнить
задачу, добавлю, что при запуске \utility{bash} командой
\filename{/bin/sh} его поведение существенно изменяется с целью
максимального соответствия возможностям стандартного интерпретатора.
На большинстве Linux\hyp{}систем файл \filename{/bin/sh} является
символической ссылкой на \utility{bash}, в то время как в Cygwin эта
ссылка указывает на \utility{ash}. Для учёта этих различий некоторые
тесты запускались трижды, по одному разу для каждого командного
интерпретатора. Командный интерпретатор, использованный в тесте,
указан в скобках. К примеру, <<(sh)>> означает, что использовался
интерпретатор \utility{bash}, запущенный при помощи символической
ссылки \filename{/bin/sh}.

Первые три теста, обозначенные как <<make>>, отображают стоимость
запуска \GNUmake{}, не совершающего полезной работы. \makefile{}
содержит следующие строки:

{\footnotesize
\begin{verbatim}
SHELL := /bin/bash
.PHONY: x
x:
    $(MAKE) --no-print-directory --silent --question make-bash.mk; \
    ...Эта команда повторяется ещё 99 раз...
\end{verbatim}
}

По необходимости слово <<bash>> заменялось соответствующим названием
командного интерпретатора.

Мы использовали опции \command{-{}-no\hyp{}print\hyp{}directory} и
\command{-{}-{}silent} для исключения ненужных вычислений, которые
могли повлиять на время выполнения и затмить измеренные временные
величины грудой бесполезного текста. Опция \command{-{}\hyp{}question}
сообщает \GNUmake{}, что выполнять команды не требуется, нужна только
проверка зависимостей. В этом случае, если файл не требует обновления,
\GNUmake{} завершит работу с нулевым кодом возврата. Это позволяет
\GNUmake{} делать настолько мало работы, насколько это возможно. Этот
\makefile{} не будет выполнять команд, зависимости в нём существуют
только для одной абстрактной цели. Файл \filename{make-bash.mk}
выполняется родительским процессом \GNUmake{} 10 раз. Содержимое этого
файла представлено ниже:

{\footnotesize
\begin{verbatim}
define ten-times
  TESTS += $1
  .PHONY: $1
  $1:
      @echo $(MAKE) --no-print-directory --silent $2; \
      time $(MAKE) --no-print-directory --silent $2; \
      time $(MAKE) --no-print-directory --silent $2; \
      time $(MAKE) --no-print-directory --silent $2; \
      time $(MAKE) --no-print-directory --silent $2; \
      time $(MAKE) --no-print-directory --silent $2; \
      time $(MAKE) --no-print-directory --silent $2; \
      time $(MAKE) --no-print-directory --silent $2; \
      time $(MAKE) --no-print-directory --silent $2; \
      time $(MAKE) --no-print-directory --silent $2; \
      time $(MAKE) --no-print-directory --silent $2
endef

.PHONY: all
all:

$(eval $(call ten-times, make-bash, -f make-bash.mk))

all: $(TESTS)
\end{verbatim}
}

После этого время, требуемое для тысячи запусков, усредняется.

Как вы можете видеть из таблицы, Cygwin \GNUmake{} выполняется
примерно 22 раза в секунду, или 0,044 секунд, в то время как под
управлением операционной системы Linux (не смотря на гораздо более
медленный процессор) выполнение осуществляется примерно 61 раз в
секунду (т.е. одно выполнение занимает 0,016 секунд). Для проверки
этих результатов был протестирован порт \GNUmake{} под Windwos, не
показавший, впрочем, существенного выигрыша в производительности.
Заключение: хоть создание процесса Cygwin \GNUmake{} и
занимает немного больше времени, чем та же операция в Windows
\GNUmake{}, оба этих варианта значительно уступают в
производительности аналогичной операции в Linux. Отсюда следует, что
рекурсивное выполнение \GNUmake{} под Windows может занимать
значительно больше времени, чем рекурсивная сборка, запущенная под
управлением Linux.

Как вы могли ожидать, используемый командный интерпретатор практически
не влияет на скорость выполнения. Поскольку командный сценарий не
содержит специальных символов, командный интерпретатор даже не
вызывался. \GNUmake{} выполнял команды самостоятельно. Это можно
проверить, присвоив переменной \variable{SHELL} произвольное значение
и убедившись в том, что тест выполняется корректно. Разница в
производительности при использовании различных интерпретаторов можно
списать на нормальную вариацию времени выполнения процесса в системе.

Следующий тест измеряет время, требуемое для присваивания переменной
значения~--- наиболее элементарной операции \GNUmake{}. \makefile{},
называющийся \filename{assign.mk}, содержит следующие строки:

{\footnotesize
\begin{verbatim}
# 10000 assignments
z := 10
...предыдущая строка повторяется 10000 раз...
.PHONY: x
x: ;
\end{verbatim}
}

Этот \makefile{} выполняется в родительском \makefile{е} с
использованием нашей функции \function{ten\hyp{}times}.

Очевидно, присваивание выполняется очень быстро. Cygwin \GNUmake{}
выполняет 8130 присваиваний в секунду, в то время как в системе Linux
этот показатель доходит до 10.989. Я уверен, что производительность
выполнения этой операции в системе Windows на самом деле выше, чем
показывают наши измерения, поскольку точное время создания десяти
процессов \GNUmake{} невозможно отделить от времени выполнения
присваивания. Заключение: поскольку вероятность того, что в среднем
\makefile{е} будет осуществляться 10.000 присваиваний, довольно мала,
стоимость выполнения присваиваний в среднем \makefile{е} можно не
учитывать.

Следующие два теста измеряют время выполнения функции \function{subst}.
Первый тест осуществляет подстановку трёх символов в коротких строках,
состоящих из десяти символов:

{\footnotesize
\begin{verbatim}
# 10000 подстановок в строке из 10 символов
dir := ab/cd/ef/g
x := $(subst /, ,$(dir))
...предыдущая строка повторяется 10000 раз...
.PHONY: x
x: ;
\end{verbatim}
}

Операция занимает примерно в два раза больше времени чем простое
присваивание, выполняясь в Windows 3891 раз в секунду. Повторюсь,
показатели производительности в системе Linux значительно превосходят
аналогичные показатели в Windows (как вы помните, производительность
процессора компьютера, на котором установлена система Linux, примерно
в четыре раза меньше производительности процессора компьютера с
системой Windows).

Второй тест осуществляет примерно 100 подстановок в строке длиной в
1000 символов:

{\footnotesize
\begin{verbatim}
# Имя файла из 10 символов
dir := ab/cd/ef/g
# список путей из 1000 символов
p100 := $(dir);$(dir);$(dir);$(dir);$(dir);...
p1000 := $(p100)$(p100)$(p100)$(p100)$(p100)...

# 10000 подстановок в строке длиной в 1000 символов
x := $(subst ;, ,$(p1000))
...предыдущая строка повторяется 10000 раз...
.PHONY: x
x: ;
\end{verbatim}
}

Следующие три теста измеряют скорость той же подстановки при
использовании \utility{sed}. Содержимое тестового файла представлено
ниже:

{\footnotesize
\begin{verbatim}
# 100 sed using bash
SHELL := /bin/bash

.PHONY: sed-bash
sed-bash:
echo '$(p1000)' | sed 's/;/ /g' > /dev/null
...предыдущая строка повторяется 100 раз...
\end{verbatim}
}

Как и раньше, \makefile{} выполняется с помощью функции
\function{ten-times}. В системе Windows \utility{sed} выполняется
примерно в 50 раз медленнее, чем функция \function{subst}. В системе
Linux \utility{sed} работает в 24 раза медленнее.

Если учесть время, затраченное на запуск командного интерпретатора,
становится очевидным, что использование командного интерпретатора
\utility{ash} в Windows даёт небольшую прибавку в скорости. При
использовании \utility{ash} \utility{sed} всего лишь в 39 раз
медленнее \function{subst}! В Linux влияние используемого командного
интерпретатора на скорость выполнения прослеживается более чётко. При
использовании \utility{ash} \utility{sed} всего в пять раз медленнее
\function{subst}. Здесь же можно проследить эффект замены
\utility{bash} на \utility{sh}. В среде Cygwin разница между
\utility{bash}, вызванного через \filename{/bin/bash}, и
\utility{bash}, вызванного через \filename{/bin/sh}, не
прослеживается. В Linux же \utility{/bin/sh} выполняется значительно
быстрее.

Последний тест измеряет затраты на выполнение команды в дочернем
командном интерпретаторе, вызывая команду \command{make shell}.
\makefile{} содержит следующие строки:

{\footnotesize
\begin{verbatim}
# 100 $(shell ) using bash
SHELL := /bin/bash
x := $(shell :)
...предыдущая строка повторяется 100 раз...
.PHONY: x
x: ;
\end{verbatim}
}

Впрочем, результаты были вполне предсказуемы. Система Windows работает
медленнее, чем Linux, командный интерпретатор \utility{ash}
работает быстрее, чем \utility{bash}. Выигрыш от использования
\utility{ash} выражен в этом тесте более ярко и составляет примерно
50\%. В системе Linux наибольшая производительность достигается при
использовании \utility{ash}, наименьшая~--- при использовании
\utility{bash} (вызванного из файла \filename{/bin/bash}).

Тесты производительности являются неиссякаемым источником задач,
тем не менее, сделанные нами измерения могут помочь нам извлечь
некоторую полезную информацию. Создавайте столько переменных, сколько
считаете нужным, если, конечно, они помогают упростить структуру
\makefile{а}, поскольку их использование практически ничего не стоит.
Встроенные функции более предпочтительны, чем запуск внешних программ,
даже если структура вашего кода обязывает вас последовательно
выполнять вызовы функций \GNUmake{}. Избегайте использования
рекурсивного \GNUmake{} или избыточного порождения процессов в
Windows. Если вы работаете в Linux и вам нужно создавать множество
процессов, используйте \utility{ash}.

Наконец, запомните, что для большинства \makefile{ов} справедливо
следующее утверждение: время выполнения \makefile{а} практически
полностью определяется временем выполнения внешних программ, а вовсе
не нагрузкой \GNUmake{} и не структурой \makefile{а}. Как правило,
сокращение числа запусков внешних программ сократит и время выполнения
\makefile{а}.

%%--------------------------------------------------------------------
%% Identifying and handling bottlenecks
%%--------------------------------------------------------------------
\section{Определяем и устраняем узкие места}
Излишние задержки выполнения \makefile{а} могут появляться по одной
из трёх причин: неудачный выбор структуры \makefile{а}, неверный
анализ зависимостей, и неправильное использование функций и переменных
\GNUmake{}. Эти проблемы могут маскироваться функциями \GNUmake{},
подобными \function{shell}, которые вызывают команды, но не печатают
их в терминал, что существенно затрудняет поиск источника задержек.

Анализ зависимостей~--- это палка о двух концах. С одной стороны,
выполнение полного анализа зависимостей может вызвать существенные
задержки. Без специальной поддержки компилятора, предоставляемой, к
примеру, \utility{gcc} и \utility{jikes}, создание файла зависимостей
требует запуска внешней программы, что практически удваивает время
компиляции\footnote{
На практике время компиляции линейно зависит от размера входного
текста и практически всегда определяется скоростью операций
ввода/вывода. Точно так же время вычисления зависимостей с помощью
опции \command{-M} линейно зависит от размера файла и ограничено
скоростью операций ввода/вывода.}. Преимуществом полного анализа
зависимостей является возможность \GNUmake{} осуществлять меньшее
количество компиляций. К сожалению, разработчики часто не верят, что
эта возможность себя оправдает, и пишут \makefile{ы} с неполной
информацией о зависимостях. Этот компромисс почти всегда превращается
в проблему, заставляющую других разработчиков платить за эту скупость
вдвойне, компилируя больше кода, чем потребовалось бы, будь у
\GNUmake{} полная информация о зависимостях.

Чтобы сформулировать стратегию анализа зависимостей, начните с
понимания зависимостей, присущих вашему проекту. Когда все
зависимости осознаны, можно приступать к представлению
этих зависимостей в \makefile{е} (вычисленных или перечисленных
вручную) и выбору сокращённых путей осуществления сборки. Хоть и не
все представленные шаги являются легко осуществимыми, этот метод сам
по себе является наиболее простым.

Когда вы определили структуру \makefile{а} и необходимые зависимости,
эффективность \makefile{а} достигается за счёт обхода некоторых
известных ловушек.

%---------------------------------------------------------------------
% Simple variables versus recursive
%---------------------------------------------------------------------
\subsection{Выбор переменных: простые или рекурсивные}
Одной из наиболее общих проблем, относящихся к производительности,
является использование рекурсивных переменных. Например, поскольку
код, приведённый ниже, использует оператор \command{=} вместо
оператора \command{:=}, при каждом обращении к переменной
\variable{DATE} её значение будет вычисляться заново:

{\footnotesize
\begin{verbatim}
DATE = $(shell date +%F)
\end{verbatim}
}

Опция \command{+\%F} сообщает программе \utility{date}, что дату
требуется возвращать в формате <<гггг-мм-дд>>, таким образом,
большинство пользователей не заметят эффекта от многократного вызова
\utility{date}. Разумеется, разработчики, засидевшиеся в офисе до
полуночи, могут быть приятно удивлены.

Поскольку \GNUmake{} не выводит команды, выполняемые при помощи
функции \function{shell}, идентифицировать, что же именно выполняется,
может быть довольно трудно. Определив переменную \variable{SHELL} как
\command{/bin/sh -x}, вы можете выявить все команды, выполняемые
\GNUmake{}.

Следующий \makefile{} создаёт каталог перед осуществлением прочих
действий. Имя каталога составляется из слова <<out>> и текущей даты:

{\footnotesize
\begin{verbatim}
DATE = $(shell date +%F)
OUTPUT_DIR = out-$(DATE)
make-directories := \
    $(shell [ -d $(OUTPUT_DIR) ] || mkdir -p $(OUTPUT_DIR))
all: ;
\end{verbatim}
}

После запуска \makefile{а} с отладочной опцией интерпретатора мы
увидим следующий вывод:

{\footnotesize
\begin{alltt}
\$ \textbf{make SHELL='/bin/sh -x'}
+ date +\%F
+ date +\%F
+ '[' -d out-2004-03-30 ']'
+ mkdir -p out-2004-03-30
make: all is up to date.
\end{alltt}
}

Теперь отчётливо видно, что команда \utility{date} выполняется дважды.
Если вам часто требуется осуществлять подобного рода отладку, вы
можете упростить её, используя конструкцию следующего вида:

{\footnotesize
\begin{verbatim}
ifdef DEBUG_SHELL
  SHELL = /bin/sh -x
endif
\end{verbatim}
}

%---------------------------------------------------------------------
% Disabling @
%---------------------------------------------------------------------
\subsection{Отключаем @}
Ещё одним способом сокрытия команд является модификатор \command{@}.
Иногда бывает полезным отключить эту возможность. Это легко
осуществить с помощью определения вспомогательной переменной
\variable{QUIET}, содержащей символ \command{@}, и использования этой
переменной в командах:

{\footnotesize
\begin{verbatim}
ifndef VERBOSE
  QUIET := @
endif
...
target:
    $(QUIET) echo Собираю цель target...
\end{verbatim}
}

Когда нужно будет увидеть команды, скрытые при помощи модификатора,
просто определите переменную \variable{VERBOSE} через интерфейс
командной строки:

{\footnotesize
\begin{alltt}
\$ \textbf{make VERBOSE=1}
echo Собираю цель target...
Собираю цель target...
\end{alltt}
}

%---------------------------------------------------------------------
% Lazy initialization
%---------------------------------------------------------------------
\subsection{Ленивая инициализация}
При использовании простых переменных в сочетании с функцией
\function{shell}, \GNUmake{} осуществляет вызовы функции
\function{shell} во время чтения \makefile{а}. Если таких вызовов
много, или если они осуществляют сложные вычисления, выполнение
\GNUmake{} может существенно замедлиться. Время отклика \GNUmake{}
можно измерить, вызвав \GNUmake{} со спецификацией несуществующей
цели:

{\footnotesize
\begin{alltt}
\$ \textbf{time make no-such-target}
make: *** No rule to make target no-such-target. Stop.
real    0m0.058s
user    0m0.062s
sys     0m0.015s
\end{alltt}
}

Приведённый выше код измеряет время, добавляемое \GNUmake{} к каждой
выполняемой команде, даже если эта команда тривиальна или ошибочна.

Поскольку рекурсивные переменные вычисляются заново при каждом
обращении к ним, существует тенденция оформлять результаты сложных
вычислений в виде простых переменных. С другой стороны, такой подход
увеличивает время отклика \GNUmake{} при сборке любой цели. Похоже,
существует необходимость в дополнительном виде переменных, правая
часть которых вычисляется в точности один раз при первом обращении к
переменной.

Пример, иллюстрирующий необходимость подобного рода инициализации, был
приведён в функции \function{find\hyp{}compilation\hyp{}dir} в разделе
<<\nameref{sec:all_in_one_compile}>> главы \ref{chap:java}:

{\footnotesize
\begin{verbatim}
# $(call find-compilation-dirs, root-directory)
find-compilation-dirs =                      \
  $(patsubst %/,%,                           \
    $(sort                                   \
      $(dir                                  \
        $(shell $(FIND) $1 -name '*.java'))))
PACKAGE_DIRS := $(call find-compilation-dirs, $(SOURCE_DIR))
\end{verbatim}
}

В идеале нам хотелось бы осуществлять операцию \command{find} только
один раз при первом обращении к переменной \variable{PACKAGE\_DIR}.
\index{Ленивая инициализация}
Это можно назвать \newword{ленивой инициализацией} (\newword{lazy
initialization}). Мы можем создать подобного рода переменную с помощью
функции \function{eval}:

{\footnotesize
\begin{verbatim}
PACKAGE_DIRS = $(redefine-package-dirs) $(PACKAGE_DIRS)
redefine-package-dirs =                                \
  $(eval PACKAGE_DIRS := $(call find-compilation-dirs, \
                           $(SOURCE_DIR)))
\end{verbatim}
}

Этот подход заключается в определении \variable{PACKAGE\_DIR} как
изначально рекурсивной переменной. При первом обращении к переменной
вычисляется ресурсоёмкая функция, в данном случае
\function{find\hyp{}compilation\hyp{}dir}, и переменная
переопределяется как простая. Наконец, значение переменной (теперь уже
простой) возвращается как результат обращения к первоначально
рекурсивной переменной.

Давайте рассмотрим этот пример более детально:
\begin{enumerate}
%---------------------------------------------------------------------
\item Когда \GNUmake{} считывает эти переменные, он просто сохраняет
правые части их определений, так как обе переменные являются
рекурсивными.
%---------------------------------------------------------------------
\item При первом обращении к переменной \variable{PACKAGE\_DIRS}
\GNUmake{} извлекает соответствующую правую часть и производит
вычисление переменной \variable{redefine\hyp{}package\hyp{}dirs}.
%---------------------------------------------------------------------
\item Значением переменной \variable{redefine\hyp{}package\hyp{}dirs}
является единственный вызов функции \function{eval}.
%---------------------------------------------------------------------
\item Тело функции \function{eval} переопределяет переменную
\variable{PACKAGE\_DIRS} как простую переменную, присваивая ей
результат вычисления функции
\function{find\hyp{}compilation\hyp{}dirs}. Теперь
\variable{PACKAGE\_DIRS} инициализирована списком каталогов.
%---------------------------------------------------------------------
\item Результатом вычисления функции
\function{redefine\hyp{}package\hyp{}dir} является пустая строка
(поскольку результатом вычисления функции \function{eval} также
является пустая строка).
%---------------------------------------------------------------------
\item \GNUmake{} продолжает вычислять изначальное значение переменной
\variable{PACKAGE\_DIRS}. Остаётся только подставить значение
переменной \variable{PACKAGE\_DIRS}. \GNUmake{} производит поиск
переменной, находит простую переменную и возвращает её значение.
%---------------------------------------------------------------------
\end{enumerate}

Единственным по-настоящему тонким моментом в этом коде является
предположение, согласно которому \GNUmake{} вычисляет правую часть
определения переменной слева направо. Если, к примеру, \GNUmake{}
решит вычислить выражение \command{\$(PACKAGE\_DIRS)} раньше выражения
\command{\$(redefine\hyp{}package\hyp{}dirs)}, этот код не будет
работать.

Процедура, описанная мной выше, может быть преобразована в функцию
\function{lazy-init}:

{\footnotesize
\begin{verbatim}
# $(call lazy-init,variable-name,value)
define lazy-init
  $1 = $$(redefine-$1) $$($1)
  redefine-$1 = $$(eval $1 := $2)
endef

# PACKAGE_DIRS - ленивое вычисление списка каталогов
$(eval                           \
  $(call lazy-init,PACKAGE_DIRS, \
    $$(call find-compilation-dirs,$(SOURCE_DIRS))))
\end{verbatim}
}

%%--------------------------------------------------------------------
%% Parallel make
%%--------------------------------------------------------------------
\section{Параллельное выполнение \GNUmake{}}

Ещё одним способом увеличения производительности сборок является
использование параллелизма, присущего проблеме обработки
\makefile{а}. Большинство \makefile{ов} предназначены для
осуществления задач, многие из которых можно обрабатывать параллельно,
например, компиляцию исходных файлов \Clang{} в объектные или создание
библиотек из объектных файлов. Более того, сама структура хорошо
написанных \makefile{ов} предоставляет всю необходимую информацию для
автоматического управления конкурирующими процессами.

Следующий пример демонстрирует сборку нашей программы mp3 плеера с
опцией управления задачами, \command{-{}-jobs=2} (или \command{-j
  2}). На рисунке~\ref{fig:parallel_make} изображён тот же самый
запуск \GNUmake{}, представленный с помощью диаграммы
UML. Использование опции \command{-{}-jobs} сообщает \GNUmake{}, что
при возможности следует обновлять параллельно две цели. Когда
\GNUmake{} обновляет цели параллельно, он выводит команды в том
порядке, в каком они выполняются, чередуя в выводе команды сборки
разных целей. Это может затруднить чтение вывода \GNUmake{},
осуществляющего параллельную сборку. Давайте внимательно рассмотрим
вывод.

{\footnotesize
\begin{alltt}
\$ \textbf{make -f ../ch07-separate-binaries/makefile --jobs=2}
\end{alltt}
\begin{verbatim}
1  bison -y --defines ../ch07-separate-binaries/lib/db/playlist.y
2  flex -t ../ch07-separate-binaries/lib/db/scanner.l >
   lib/db/scanner.c
3  gcc -I lib -I ../ch07-separate-binaries/lib
   -I ../ch07-separate-binaries/include
   -M ../ch07-separate-binaries/app/player/play_mp3.c | \
   sed 's,\(play_mp3\.o\) *:,app/player/\1 app/player/play_mp3.d: ,'
   > app/player/play_mp3.d.tmp
4  mv -f y.tab.c lib/db/playlist.c
5  mv -f y.tab.h lib/db/playlist.h
6  gcc -I lib -I ../ch07-separate-binaries/lib
   -I ../ch07-separate-binaries/include
   -M ../ch07-separate-binaries/lib/codec/codec.c | \
   sed 's,\(codec\.o\) *:,lib/codec/\1 lib/codec/codec.d: ,' >
   lib/codec/codec.d.tmp
7  mv -f app/player/play_mp3.d.tmp app/player/play_mp3.d
8  gcc -I lib -I ../ch07-separate-binaries/lib
   -I ../ch07-separate-binaries/include -M lib/db/playlist.c | \
   sed 's,\(playlist\.o\) *:,lib/db/\1 lib/db/playlist.d: ,' >
   lib/db/playlist.d.tmp
9  mv -f lib/codec/codec.d.tmp lib/codec/codec.d
10 gcc -I lib -I ../ch07-separate-binaries/lib
   -I ../ch07-separate-binaries/include
   -M ../ch07-separate-binaries/lib/ui/ui.c | \
   sed 's,\(ui\.o\) *:,lib/ui/\1 lib/ui/ui.d: ,' > lib/ui/ui.d.tmp
11 mv -f lib/db/playlist.d.tmp lib/db/playlist.d
12 gcc -I lib -I ../ch07-separate-binaries/lib
   -I ../ch07-separate-binaries/include
   -M lib/db/scanner.c | \
   sed 's,\(scanner\.o\) *:,lib/db/\1 lib/db/scanner.d: ,' >
   lib/db/scanner.d.tmp
13 mv -f lib/ui/ui.d.tmp lib/ui/ui.d
14 mv -f lib/db/scanner.d.tmp lib/db/scanner.d
15 gcc -I lib -I ../ch07-separate-binaries/lib
   -I ../ch07-separate-binaries/include -c
   -o app/player/play_mp3.o
   ../ch07-separate-binaries/app/player/play_mp3.c
16 gcc -I lib -I ../ch07-separate-binaries/lib
   -I ../ch07-separate-binaries/include -c
   -o lib/codec/codec.o
   ../ch07-separate-binaries/lib/codec/codec.c
17 gcc -I lib -I ../ch07-separate-binaries/lib
   -I ../ch07-separate-binaries/include -c
   -o lib/db/playlist.o lib/db/playlist.c
18 gcc -I lib -I ../ch07-separate-binaries/lib
   -I ../ch07-separate-binaries/include -c
   -o lib/db/scanner.o lib/db/scanner.c
   ../ch07-separate-binaries/lib/db/scanner.l: In function yylex:
   ../ch07-separate-binaries/lib/db/scanner.l:9: warning:
   return makes integer from pointer without a cast
19 gcc -I lib -I ../ch07-separate-binaries/lib
   -I ../ch07-separate-binaries/include -c
   -o lib/ui/ui.o ../ch07-separate-binaries/lib/ui/ui.c
20 ar rv lib/codec/libcodec.a lib/codec/codec.o
   ar: creating lib/codec/libcodec.a
   a - lib/codec/codec.o
21 ar rv lib/db/libdb.a lib/db/playlist.o lib/db/scanner.o
   ar: creating lib/db/libdb.a
   a - lib/db/playlist.o
   a - lib/db/scanner.o
22 ar rv lib/ui/libui.a lib/ui/ui.o
   ar: creating lib/ui/libui.a
   a - lib/ui/ui.o
23 gcc app/player/play_mp3.o lib/codec/libcodec.a lib/db/libdb.a
   lib/ui/libui.a app/player/play_mp3
\end{verbatim}
}

\begin{figure}
\begin{center}
\includegraphics{./src/latex/figures/parallel_make.eps}
\end{center}
\caption{Диаграмма выполнения \GNUmake{} при \command{-{}-jobs=2}}
\label{fig:parallel_make}
\end{figure}

Сначала \GNUmake{} должен сгенерировать исходные файлы и файлы
зависимостей. Два сгенерированных исходных файла являются выводом
команд \utility{yacc} и \utility{lex} (команды 1 и 2
соответственно). Третья команда генерирует файл зависимостей для
\filename{play\_mp3.c}, её выполнение начинается до завершения
генерации файлов зависимостей для \filename{playlist.c} или
\filename{scanner.c} (командами 4, 5, 8, 9, 12 и 14). Таким образом,
\GNUmake{} выполняет одновременно три задачи, не смотря на то, что
опция командной строки требует одновременного выполнения двух задач.

Команды \command{mv} (4 и 5) завершают генерацию исходного файла
\filename{playlist.c}, начавшуюся первой командой. Команда 6 начинает
генерацию очередного файла зависимостей. Каждый командный сценарий
выполняется одним процессом \GNUmake{}, однако каждая цель и каждый
реквизит формируют отдельный поток. Таким образом, команда 7,
являющаяся второй командой сценария генерации файла зависимостей,
исполняется тем же процессом \GNUmake{}, что и команда 3. Команда 6
выполняется, скорее всего, процессом \GNUmake{}, порождённым сразу
после выполнения команд 1-4-5 (обработки грамматики \utility{yacc}),
но до генерации файла зависимостей, осуществляющейся командой 8.

Определение зависимостей продолжается в том же духе вплоть до команды
14. Все файлы зависимостей должны быть созданы до того, как \GNUmake{}
сможет приступить к следующей фазе обработки~--- повторному считыванию
\makefile{а}. Эта фаза образует естественную точку синхронизации.

Как только завершается повторное чтение \makefile{а}, \GNUmake{} может
снова продолжить параллельное выполнение сборки. В этот раз \GNUmake{}
решает произвести компиляцию всех объектных файлов до того, как начать
упаковывать их в библиотечные архивы. Этот порядок является
недетерминированным. В следующий раз \GNUmake{} может собрать
библиотеку \filename{libcodec.a} до того, как будет скомпилирован файл
\filename{playlist.c}, поскольку библиотека не требует наличия никаких
объектных файлов, кроме \filename{codec.o}. Таким образом, этот пример
демонстрирует один порядок выполнения из множества возможных.

Наконец, происходит компоновка программы. В нашем случае фаза
компоновки также является естественной точкой синхронизации и всегда
будет происходить в последнюю очередь. Если, однако, целью была не
единственная программа, а множество программ или библиотек, последняя
исполняемая инструкция может также быть другой.

Запуск множества задач на многопроцессорном компьютере может иметь
смысл, однако запуск более одной задачи на процессор может быть также
очень полезным. Причина кроется в латентности дискового ввода/вывода и
наличии кэширования на большинстве систем. Например, если процесс, к
примеру, \utility{gcc}, ожидает поступления данных с диска, данные для
других задач (таких как \utility{mv} или \utility{yacc}) могут
располагаться в памяти компьютера. В этом случае лучшим выходом будет
разрешить задаче обработку данных. В общем случае запуск \GNUmake{} с
несколькими потоками выполнения на однопроцессорной системе
практически всегда быстрее, чем запуск однопоточной сборки, и не так уж
редко запуск трёх или даже четырёх потоков приводит к лучшему
результату, чем при запуске двух потоков.

Опция \command{-{}-jobs} может использоваться без аргумента. В этом
случае \GNUmake{} порождает по одному потоку на каждую обновляемую
цель. Как правило, это плохая идея, поскольку на переключение
контекста может уходить настолько много времени, что итоговая
производительность будет гораздо ниже, чем в случае однопоточного
выполнения.

Ещё одним способом управления множеством задач является использование
средней загрузки системы как отправной точки. Средняя загрузка
системы~--- это число запущенных задач, усреднённое за некоторый
промежуток времени (как правило,1, 5 или 15 минут). Средняя загрузка
выражается как число с плавающей точкой. Опция
\command{-{}-load-average} (или \command{-l}) задаёт верхнюю границу
числа порождаемых задач. Например, следующая команда:

{\footnotesize
\begin{alltt}
\$ \textbf{make --load-average=3.5}
\end{alltt}
}

сообщает \GNUmake{}, что потоки задач должны порождаться таким
образом, чтобы средняя средняя загрузка системы была не более
3.5. Если средняя загрузка выше, \GNUmake{} будет ждать, пока она не
уменьшится до допустимых пределов, или пока все потоки не закончат
свою работу.

Когда вы пишите \makefile{} для параллельного выполнения, задача
правильного указания реквизитов становятся ещё более важной. Как уже
было замечено, когда опция \command{-{}-jobs} имеет значение 1, список
реквизитов обычно вычисляется слева направо. Когда \command{-{}-jobs}
больше 1, реквизиты могут вычисляться параллельно. Поэтому любые
отношения зависимости, неявно присутствующие в порядке вычисления
реквизитов, при параллельном запуске должны быть указаны явно.

Ещё одной неприятностью при использовании параллельных сборок является
проблема разделяемых временных файлов. Например, если каталог содержит
файлы \filename{foo.y} и \filename{bar.y}, параллельный запуск
\utility{yacc} может привести к тому, что один из экземпляров файла
\filename{y.tab.c} или \filename{y.tab.h} перепишет другой. С подобной
проблемой вы также сталкиваетесь при использовании в своих сценариях
временных файлов с фиксированными именами.

Ещё одной идиомой, препятствующей параллельному выполнению, является
рекурсивный вызов \GNUmake{} из цикла \command{for}:

{\footnotesize
\begin{verbatim}
dir:
    for d in $(SUBDIRS);         \
    do                           \
        $(MAKE) --directory=$$d; \
    done
\end{verbatim}
}

Как уже упоминалось в разделе \nameref{sec:recursive_make} главы
\ref{chap:managing_large_proj}, \GNUmake{} не может выполнять
рекурсивные вызовы параллельно. Чтобы достичь параллельного
выполнения, объявите каталоги абстрактными целями:

{\footnotesize
\begin{verbatim}
.PHONY: $(SUBDIRS)
$(SUBDIRS):
    $(MAKE) --directory=$@
\end{verbatim}
}

%%--------------------------------------------------------------------
%% Distributed make
%%--------------------------------------------------------------------
\section{Распределённое выполнение \GNUmake{}}

GNU \GNUmake{} поддерживает малоизвестную (и практически не
тестированную) опцию для управления сборками, распределёнными среди
нескольких рабочих станций, соединённых сетью. Этот функционал основан
на библиотеке Customs, распространяемой с дистрибутивом
\utility{Pmake}. \utility{Pmake}~--- это альтернативная версия
\GNUmake{}, реализованная Адамом де Буром (Adam de Boor) в 1989 году
для операционной системы Sprite и всё ещё поддерживаемая Андреасом
Столке (Andreas Stolcke). Библиотека Customs помогает распределить
выполнение \GNUmake{} между множеством компьютеров. GNU \GNUmake{}
включает поддержку этой библиотеки начиная с версии 3.77.

Чтобы включить поддержку библиотеки Customs, вам нужно собрать
\GNUmake{} из исходного кода. Инструкцию по осуществлению этого
процесса можно найти в файле \filename{README.customs} в дистрибутиве
\GNUmake{}. Сначала вам нужно загрузить дистрибутив \utility{pmake}
(URL указан в инструкции), затем собрать \GNUmake{} с опцией
\command{-{}\hyp{}with\hyp{}customs}.

Сердцем библиотеки Customs является демон (daemon), запускаемый на
каждом узле распределённой вычислительной сети \GNUmake{}. Все узлы
должны иметь доступ к разделяемой файловой системе, предоставляемый,
например, NFS. Один экземпляр демона назначается управляющим.
Управляющий процесс назначает задачи участникам вычислительной сети.
Когда \GNUmake{} запускается с опцией \command{-{}\hyp{}jobs} больше
1, \GNUmake{} контактирует с управляющим процессом, вместе они
порождают задачи, распределяя их среди доступных узлов сети.

Библиотека Customs поддерживает множество возможностей. Узлы могут
группироваться по архитектуре и ранжироваться по производительности.
Узлам могут назначаться произвольные атрибуты, и задачи могут
назначаться на основании значений атрибутов и булевых операторов. В
добавок к этому, такие характеристики работы узлов, как время
простоя, свободное дисковое пространство, свободное пространство в
разделе подкачки, текущая средняя загрузка могут быть посчитаны во
время выполнения задач.

Если ваш проект реализован на \Clang{}, \Cplusplus{} или Objective-C
вам следует рассмотреть возможность применения программы
\utility{distcc} (\filename{\url{http://distcc.samba.org}}),
предназначенной для распределённой компиляции. \utility{distcc}
написана Мартином Пулом (Martin Pool) и другими программистами для
ускорения сборок проекта Samba. Это законченное робастное решение для
проектов, написанных на \Clang{}, \Cplusplus{} или Objective-C.
Для использования этого инструмента достаточно заменить компилятор
\Clang{} программой \utility{distcc}:

{\footnotesize
\begin{alltt}
\$ \textbf{make --jobs=8 CC=distcc}
\end{alltt}
}

Для каждой компиляции \utility{distcc} использует препроцессор
для обработки исходного кода, затем отправляет результат другим узлам
сети для компиляции. Наконец, удалённые узлы возвращают полученные
объектные файлы управляющему процессу. Этот подход устраняет
необходимость в разделяемой файловой системе, что существенно упрощает
установку и конфигурацию.

Множество рабочих узлов или \newword{добровольцев} можно указать
несколькими способами. Наиболее простым является перечисление
узлов\hyp{}добровольцев в переменной окружения перед запуском
\utility{distcc}:

{\footnotesize
\begin{alltt}
\$ \textbf{export DISTCC\_HOSTS='localhost wasatch oops'}
\end{alltt}
}

\utility{distcc} имеет много опций для управления списком удалённых
узлов, интеграцией с компилятором, управления компрессией, путями
поиска, а также обнаружения и исправления ошибок.

Ещё одним инструментом увеличения скорости компиляции является
программа \utility{ccache}, написанная руководителем проекта Samba
Эндрю Тридгеллом (Andrew Tridgell). Идея очень проста:
\utility{ccache} кэширует результаты предыдущих сборок. Перед
осуществлением компиляции осуществляется проверка, содержит ли кэш
нужные объектные файлы. Это не требует участие нескольких узлов сети,
не требуется даже существование сети. Автор сообщает о 5-10 кратном
ускорении основного процесса компиляции. Наиболее простым способом
использования этого инструмента является переопределение команды
компиляции через интерфейс командной строки:

{\footnotesize
\begin{alltt}
\$ \textbf{make CC='ccache gcc'}
\end{alltt}
}

\utility{ccache} можно использовать совместно с \utility{distcc} для
ещё большего ускорения процесса сборки. В добавок ко всему, оба этих
инструмента доступны в наборе инструментов Cygwin.

