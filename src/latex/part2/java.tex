%%%-------------------------------------------------------------------
%%% Java
%%%-------------------------------------------------------------------
\chapter{\Java{}}
\label{chap:java}

\index{Интегрированные среды разработки}
Многие Java\hyp{}разработчики предпочитают использовать
интегрированные среды разработки (Integrated Development Environments,
IDE), например, Eclipse. У вас может возникнуть вопрос, зачем вам
нужно использовать \GNUmake{} в \Java{} проектах, если есть такие
известные альтернативы, как Ant и среды разработки \Java{}? Эта глава
содержит исследование значения \GNUmake{} в среде \Java{}, в
частности, в ней приводится универсальный \Makefile{}, который может
быть помещён с минимальными модификациями практически в любой
\Java{}\hyp{}проект для осуществления всех стандартных задач сборки.

Использование \GNUmake{} в совокупности с \Java{} поднимает несколько
проблем и предоставляет некоторые дополнительные возможности. Причиной
этого является сочетание трёх основных факторов: во-первых, компилятор
\Java{} работает очень быстро; во-вторых, стандартный компилятор
\Java{} поддерживает синтаксис \command{@fi\-le\-na\-me} для чтения
параметров командной строки из файла; в третьих, если в коде
\Java{}\hyp{}класса указан пакет, путь к \filename{.class}\hyp{}файлу
определяется однозначно.

Стандартный компилятор \Java{} работает очень быстро. Главной причиной
этого является принцип работы директивы \directive{import}. Подобно
директиве \directive{\#include} препроцессора языка \Clang{}, эта
директива используется для обеспечения доступа к внешним символам.
Однако вместо повторного чтения исходного кода, который затем
потребует повторного разбора и анализа, компилятор \Java{} считывает
файлы классов напрямую. Поскольку символы, определяемые в файле
класса, не могут измениться в процессе компиляции, компилятор
производит кэширование классов. Даже в случае проектов среднего
размера это означает, что компилятор \Java{} избавлен от необходимости
повторно считывать, разбирать и анализировать буквально миллионы строк
кода, с которыми пришлось бы работать компилятору языка \Clang{}.
Менее существенный прирост производительности достигается за счёт
свед\'{е}ния к минимуму оптимизаций, выполняемых большинством
компиляторов \Java{}. Вместо статической оптимизации предпочтение
отдаётся сложным оптимизациям времени выполнения (just-in-time, JIT),
осуществляемым виртуальной машиной \Java{} (\Java{} virtual machine,
JVM).

\index{Java!пакет}
Практически все крупные \Java{}\hyp{}проекты интенсивно используют
\newword{пакеты} (\newword{pack\-ages}). Каждый класс инкапсулируется в
пакет, определяющий область видимости символов, определённых в файле.
Имена пакетов имеют иерархическую структуру и неявно определяют
структуру файловой системы, предназначенную для их хранения. Например,
пакет \command{a.b.c} неявно определяет структуру каталогов
\filename{a/b/c}. Код, объявленный соответствующей директивой как
принадлежащий пакету \command{a.b.c}, будет скомпилирован в файлы
классов и помещён в каталог \filename{a/b/c}. Это означает, что
обычный алгоритм \GNUmake{}, отвечающий за ассоциацию бинарных файлов
с соответствующими исходными файлами, не будет работать правильно.
Однако это также означает, что нам больше не нужно указывать опцию
\command{-o} для спецификации каталога, предназначенного для
размещения объектного файла. Достаточно указать корень дерева
каталогов бинарных файлов, одинаковый для всех исходных файлов. Это, в
свою очередь, означает, что исходный код из различных каталогов может
быть скомпилирован одной и той же командой.

Все стандартные компиляторы \Java{} поддерживают синтаксис
\command{@fi\-le\-na\-me}, позволяющий считывать параметры командной
строки из файла. Это имеет большое значение в сочетании с функционалом
пакетов, поскольку позволяет производить компиляцию всего исходного
кода единственным вызовом компилятора. Такой подход даёт значительный
выигрыш в производительности, так как время, требуемое для загрузки и
работы компилятора, является значительной частью времени выполнения
сборки.

Итак, после составления соответствующей командной строки, компиляция
400\,000 строк \Java{}\hyp{}кода занимает около трёх минут при
использовании процессора Pentium 4 (2,5ГГц). Компиляция эквивалентного
по размеру приложения, написанного на \Cplusplus{}, потребует
нескольких часов.

%%--------------------------------------------------------------------
%% Alternatives to make
%%--------------------------------------------------------------------
\section{Альтернативы \GNUmake{}}

Как уже было замечено, сообщество \Java{}\hyp{}разработчиков с
энтузиазмом принимает новые технологии. Рассмотрим две из них, имеющие
отношение к \GNUmake{}~--- \utility{Ant} и интегрированные среды
разработки.

%---------------------------------------------------------------------
% Ant
%---------------------------------------------------------------------
\subsection{Ant}
\index{Ant@\utility{Ant}}
Сообщество \Java{}\hyp{}разработчиков очень активно и производит новые
инструменты с впечатляющей скоростью. Одним из таких инструментов
является \utility{Ant}~--- система сборки, призванная занять место
\GNUmake{} в процессе разработки \Java{}\hyp{}приложений. Как и
\GNUmake{}, \utility{Ant} использует файл спецификации для определения
целей и реквизитов проекта. В отличие от \GNUmake{}, \utility{Ant}
написан на языке \Java{} и принимает файлы спецификации в формате XML.

Чтобы дать вам представление о файле спецификации в формате XML,
приведу небольшую выдержку из файла сборки для \utility{Ant}:

{\footnotesize
\begin{verbatim}
<target name="build"
        depends="prepare, check_for_optional_packages"
        description="--> compiles the source code">
  <mkdir dir="${build.dir}"/>
  <mkdir dir="${build.classes}"/>
  <mkdir dir="${build.lib}"/>

  <javac srcdir="${java.dir}"
         destdir="${build.classes}"
         debug="${debug}"
         deprecation="${deprecation}"
         target="${javac.target}"
         optimize="${optimize}" >
    <classpath refid="classpath"/>
  </javac>
  
  ...

  <copy todir="${build.classes}">
    <fileset dir="${java.dir}">
      <include name="**/*.properties"/>
      <include name="**/*.dtd"/>
    </fileset>
  </copy>
</target>
\end{verbatim}
}

Как вы могли заметить, цель объявляется при помощи XML тега
\command{<target>}. Каждая цель имеет имя и список зависимостей,
указанных в атрибутах \command{name} и \command{depends}
\index{Ant!задачи}
соответственно. Действия, выполняемые \utility{Ant}, называются
\newword{задачами} (\newword{tasks}). Задачи реализованы на языке
\Java{} и привязаны к XML тегу. Например, задача создания каталога
специфицируется при помощи тега \command{<mkdir>} и вызывает
выполнение метода \command{Mkdir.execute}, который в конечном итоге
вызывает метод \command{File.mkdir}. Насколько это возможно, все
задачи реализуются средствами \Java{} API.

Эквивалентный файл сборки \GNUmake{} содержит следующий код:

{\footnotesize
\begin{verbatim}
# производит компиляцию исходного кода
build: $(all_javas) prepare check_for_optional_packages
    $(MKDIR) -p $(build.dir) $(build.classes) $(build.lib)
    $(JAVAC) -sourcepath $(java.dir) \
             -d $(build.classes)     \
             $(debug)                \
             $(deprecation)          \
             -target $(javac.target) \
             $(optimize)             \
             -classpath $(classpath) \
             @$<
    ...
    $(FIND) . \( -name '*.properties' -o -name '*.dtd' \) | \
    $(TAR) -c -f - -T - | $(TAR) -C $(build.classes) -x -f -

\end{verbatim}
}

Отрывок кода, приведённый выше, использует техники, которые мы ещё не
обсуждали. Пока удовлетворимся тем, что реквизит \target{all.javac}
содержит список всех \filename{java} файлов, которые нужно
скомпилировать. Задачи \utility{Ant} \command{<mkdir>},
\command{<javac>} и \command{<copy>} также осуществляют проверку
зависимостей. К примеру, если каталог уже существует, задача
\command{mkdir} не выполнит никаких действий. Более того, если файлы
\Java{}\hyp{}классов имеют более позднюю дату модификации, чем
соответствующие исходные файлы, компиляция не будет осуществляться.
Тем не менее, командный сценарий \GNUmake{} осуществляет по существу
такие же функции. \utility{Ant} включает общую задачу, именуемую
\command{<exec>}, используемую для запуска локальных программ.

\utility{Ant} использует искусный и оригинальный подход, однако, при
его использовании возникает несколько проблем, которые стоит
рассмотреть:

\begin{itemize}
%---------------------------------------------------------------------
\item Несмотря на то, что \utility{Ant} получил широкое
распространение в \Java{}\hyp{}сообществе, вне сообщества
\utility{Ant} практически не распространён. К тому же, сомнительно,
что его популярность когда-нибудь выйдет за пределы
\Java{}\hyp{}проектов (по причинам, перечисленным далее). \GNUmake{},
в свою очередь, успешно применяется во многих областях, включая
разработку программного обеспечения, обработку документов и
типографское дело, поддержку веб\hyp{}сайтов. Понимание \GNUmake{}
очень важно для любого, кому требуется работать в различных
программных системах.
%---------------------------------------------------------------------
\item Выбор XML как языка спецификаций вполне разумен для
\Java{}\hyp{}приложения. Однако читать и писать спецификации на языке
XML большинству людей не очень удобно. Хороший XML\hyp{}редактор может
быть нелегко найти или интегрировать с существующими инструментами
(если моя интегрированная среда разработки не содержит хорошего
XML\hyp{}редактора, мне придётся либо менять среду разработки, либо
искать такой редактор и использовать его отдельно). Как вы могли
видеть из предыдущего примера, \utility{Ant}\hyp{}диалект XML довольно
избыточен по сравнению с синтаксисом \GNUmake{}, и полон специфических
для XML особенностей.
%---------------------------------------------------------------------
\item В процессе работы с файлами \utility{Ant} вам нужно преодолевать
некоторую косвенность ваших спецификаций. Задача \utility{Ant}
\command{<mkdir>} те вызывает соответствующую программу
\utility{mkdir} вашей системы. Вместо этого вызывается метод
\command{mkdir()} класса \command{java.io.File}. Результатом вызова
может быть совсем не то, что вы ожидаете. По существу, любое
предположение программиста о поведении основных инструментов
\utility{Ant} должно быть проверено с привлечением документации по
\utility{Ant} или \Java{}, либо исходного кода \utility{Ant}. В
добавок, для вызова, к примеру, компилятора \Java{}, вам может
понадобиться разобраться в использовании десятка или более незнакомых
XML атрибутов, например, \command{srcdir}, \command{debug} и т.д., не
вошедших в руководство пользователя компилятора. В противоположность
этому \GNUmake{} совершенно прозрачен; как правило, вы можете просто
набирать команды прямо в интерпретаторе и следить за их поведением.
%---------------------------------------------------------------------
\item И всё же, несомненно, \utility{Ant} переносим, как и \GNUmake{}.
Как показано в главе~\ref{chap:portable_makefiles}, написание
переносимых \Makefile{}'ов, как и написание переносимых спецификаций
\utility{Ant}, требуют опыта и особых знаний. Программисты писали
переносимые \Makefile{}'ы два десятилетия. Более того, в документации
\utility{Ant} отмечается, что \utility{Ant} имеет проблемы
переносимости, связанные с символическими ссылками \UNIX{} и длинными
именами файлов в Windows, а MacOS X является единственной операционной
системой Apple, поддерживаемой \utility{Ant}, поддержка же других
платформ не гарантируется. К тому же, базовые операции наподобие
выставления флага исполняемости файлов не могут осуществляться при
помощи \Java{} API, для этого требуется вызов внешней программы.
Переносимость никогда не может быть простой или полной.
%---------------------------------------------------------------------
\item Программа \utility{Ant} не предоставляет подробного отчёта о
своих действиях. Поскольку задачи \utility{Ant} реализованы не в виде
командных сценариев, отображение действий, совершаемых этими задачами,
вызывает определённые трудности. Как правило, вывод состоит из
выражений на естественном языке, выдаваемых выражениями
\command{print}, добавленными автором задачи. Эти выражения не могут
быть выполнены пользователем в командной строке. В противоположность
этому, строки текста, отображаемые \GNUmake{} являются выражениями
интерпретатора и могут быть копированы из вывода и вставлены в
командный интерпретатор для повторного выполнения. Это означает, что
\utility{Ant} менее полезен для разработчиков, пытающихся понять
процесс сборки и способ работы инструментов, используемых в этом
процессе. Кроме того, это не даёт разработчику возможности повторно
использовать элементы этих задач экспромтом, при помощи клавиатуры.
%---------------------------------------------------------------------
\item Последняя и наиболее важная проблема заключается в том, что
\utility{Ant} сдвигает парадигмы осуществления сборок, призывая
использовать компилируемый язык программирования взамен
интерпретируемого. Задачи \utility{Ant} написаны на языке \Java{}.
Если какая-то задача не реализована или делает не то, что вы хотите,
вам нужно либо реализовать собственную задачу на \Java{}, либо
использовать задачу \command{<exec>} (разумеется, если вам приходится
часто использовать задачу \command{<exec>}, то гораздо проще
использовать \GNUmake{} с его макросами, функциями и более компактным
синтаксисом).

С другой стороны, интерпретируемые языки программирования были
изобретены для решения именно таких проблем. \GNUmake{} существует
около тридцати лет и может быть использован в большинстве сложных
ситуаций без расширения своей реализации. Разумеется, за эти тридцать
лет была реализована поддержка множества новых возможностей. Многие из
них задуманы и реализованы в GNU \GNUmake{}.
%---------------------------------------------------------------------
\end{itemize}

\utility{Ant} является замечательной программой, широко
распространённой в \Java{}\hyp{}сообществе. Тем не менее, прежде, чем
приступить к новому проекту, тщательно убедитесь, что \utility{Ant}
является подходящим инструментом для вашей среды разработки. Надеюсь,
эта глава докажет вам, что \GNUmake{} может быть успешно использован
для осуществления сборки вашего \Java{}\hyp{}проекта.

%---------------------------------------------------------------------
% Ant
%---------------------------------------------------------------------
\subsection{Интегрированные среды разработки}

Многие \Java{}\hyp{}разработчики используют интегрированные среды
разработки, совмещающие в единой (как правило, графической) среде
редактор, компилятор, отладчик и инструмент для навигации по исходному
коду. В качестве примеров можно привести такие проекты с открытым
исходным кодом, как Eclipse (\filename{\url{http://www.eclipse.org}})
и Emacs JDEE (\filename{\url{http://jdee.sunsite.dk}}), а также, если
рассматривать коммерческие разработки, Sun Java Studio
(\filename{\url{http://www.sun.com/software/sundev/jde}}) и JBuilder
(\filename{\url{http://www.borland.com/jbuilder}}). Эти среды, как
правило, имеют понятие процесса сборки проекта, заключающегося в
компиляции необходимых файлов и запуска приложения на выполнение.

Если интегрированная среда разработки поддерживает все эти операции,
зачем тогда нам рассматривать использование \GNUmake{}? Наиболее
очевидной причиной является переносимость. Если возникнет
необходимость осуществить сборку проекта на другой платформе, сборка
может закончится неудачей. Несмотря на то, что код \Java{} сам по себе
является переносимым, инструменты для работы с ним, как правило,
таковыми не являются. Например, конфигурационные файлы вашего проекта
могут включать списки путей в стиле \UNIX{} или Windows, это может
стать причиной ошибки при попытке запуска сборки под управлением
другой операционной системы. Второй причиной является тот факт, что
\GNUmake{} поддерживает автоматические сборки. Некоторые
интегрированные среды разработки поддерживают пакетные сборки, а
некоторые нет. Качество этой поддержки также варьируется. Наконец,
встроенная поддержка сборок часто бывает довольно ограниченной. Если
вы хотите реализовать собственную структуру каталогов, соответствующую
структуре релизов вашего проекта, интегрировать файлы помощи внешних
приложений, поддерживать автоматическое тестирование, ветвление и
параллельные треки разработки, скорее всего, вы обнаружите, что
встроенная поддержка сборок не подходит для ваших нужд.

По собственному опыту я могу судить, что интегрированные среды
разработки вполне подходят для небольших немасштабируемых приложений,
однако промышленные системы сборки требуют большей поддержки,
и \GNUmake{} может её обеспечить. Обычно я использую интегрированную
среду разработки для написания и отладки кода и составляю \Makefile{}
для промышленных сборок и релизов. Во время разработки я использую
интегрированную среду для компиляции проекта в состояние, пригодное
для отладки. Однако если я изменяю много файлов или модифицирую файлы,
являющиеся входными файлами для генератора кода, я запускаю
\Makefile{}. Интегрированная среда разработки, которую я использовал,
не имела соответствующей поддержки внешних программ, осуществляющих
генерацию кода. Обычно сборки, полученные с помощью интегрированной
среды, не подходят для поставок внутренним или внешним потребителям.
Для таких задач я использую \GNUmake{}.

%%--------------------------------------------------------------------
%% A generic java makefile
%%--------------------------------------------------------------------
\section{Универсальный \Makefile{} для \Java{}}

Следующий пример демонстрирует универсальный \Makefile{} для сборки
\Java{}\hyp{}проектов. Я объясню каждую из его частей далее в этой
главе.

{\footnotesize
\begin{verbatim}

# Общий makefile для Java-проекта.
VERSION_NUMBER := 1.0

# Определения базовых каталогов
SOURCE_DIR     := src
OUTPUT_DIR     := classes

# Инструменты Unix
AWK            := awk
FIND           := /bin/find
MKDIR          := mkdir -p
RM             := rm -rf
SHELL          := /bin/bash

# Пути для поддержки работы программ
JAVA_HOME      := /opt/j2sdk1.4.2_03
AXIS_HOME      := /opt/axis-1_1
TOMCAT_HOME    := /opt/jakarta-tomcat-5.0.18
XERCES_HOME    := /opt/xerces-1_4_4
JUNIT_HOME     := /opt/junit3.8.1

# Инструменты Java
JAVA           := $(JAVA_HOME)/bin/java
JAVAC          := $(JAVA_HOME)/bin/javac

JFLAGS         := -sourcepath $(SOURCE_DIR)  \
                  -d $(OUTPUT_DIR)           \
                  -source 1.4

JVMFLAGS       := -ea                        \
                  -esa                       \
                  -Xfuture

JVM            := $(JAVA) $(JVMFLAGS)

JAR            := $(JAVA_HOME)/bin/jar
JARFLAGS       := cf

JAVADOC        := $(JAVA_HOME)/bin/javadoc
JDFLAGS        := -sourcepath $(SOURCE_DIR) \
                  -d $(OUTPUT_DIR)          \
                  -link http://java.sun.com/products/jdk/1.4/docs/api

# Jar архивы
COMMONS_LOGGING_JAR := $(AXIS_HOME)/lib/commons-logging.jar

LOG4J_JAR           := $(AXIS_HOME)/lib/log4j-1.2.8.jar
XERCES_JAR          := $(XERCES_HOME)/xerces.jar
JUNIT_JAR           := $(JUNIT_HOME)/junit.jar

# Определяем путь к классам Java
class_path := OUTPUT_DIR          \
              XERCES_JAR          \
              COMMONS_LOGGING_JAR \
              LOG4J_JAR           \
              JUNIT_JAR

# Пробел
space := $(empty) $(empty)

# $(call build-classpath, variable-list)
define build-classpath
  $(strip                                          \
    $(patsubst :%,%,                               \
      $(subst : ,:,                                \
        $(strip                                    \
          $(foreach j,$1,$(call get-file,$j):)))))
endef

# $(call get-file, variable-name)
define get-file
  $(strip                                         \
    $($1)                                         \
      $(if $(call file-exists-eval,$1),,          \
        $(warning Файл, указанный в переменной    \
                  '$1' ($($1)), не найден)))
endef

# $(call file-exists-eval, variable-name)
define file-exists-eval
  $(strip                                      \
    $(if $($1),,$(warning '$1' has no value))  \
    $(wildcard $($1)))
endef

# $(call brief-help, makefile)
define brief-help
  $(AWK) '$$1 ~ /^[^.][-A-Za-z0-9]*:/                   \
          { print substr($$1, 1, length($$1)-1) }' $1 | \
  sort |                                                \
  pr -T -w 80 -4
endef

# $(call file-exists, wildcard-pattern)
file-exists = $(wildcard $1)

# $(call check-file, file-list)
define check-file
  $(foreach f, $1,                       \
    $(if $(call file-exists, $($f)),,    \
      $(warning $f ($($f)) is missing)))
endef

# $(call make-temp-dir, root-opt)
define make-temp-dir
  mktemp -t $(if $1,$1,make).XXXXXXXXXX
endef

# MANIFEST_TEMPLATE - шаблон файла манифеста, предназначенный
#                     для обработки макропроцессором m4
MANIFEST_TEMPLATE := src/manifest/manifest.mf
TMP_JAR_DIR       := $(call make-temp-dir)
TMP_MANIFEST      := $(TMP_JAR_DIR)/manifest.mf

# $(call add-manifest, jar, jar-name, manifest-file-opt)
define add-manifest
  $(RM) $(dir $(TMP_MANIFEST))
  $(MKDIR) $(dir $(TMP_MANIFEST))
  m4 --define=NAME="$(notdir $2)"            \
     --define=IMPL_VERSION=$(VERSION_NUMBER) \
     --define=SPEC_VERSION=$(VERSION_NUMBER) \
     $(if $3,$3,$(MANIFEST_TEMPLATE))        \
     > $(TMP_MANIFEST)
  $(JAR) -ufm $1 $(TMP_MANIFEST)
  $(RM) $(dir $(TMP_MANIFEST))
endef

# Определяем переменную CLASSPATH
export CLASSPATH := $(call build-classpath, $(class_path))

# make-directories - убеждаемся, что выходной каталог существует
make-directories := $(shell $(MKDIR) $(OUTPUT_DIR))

# help - цель по умолчанию
.PHONY: help
help:
    @$(call brief-help, $(CURDIR)/Makefile)

# all - осуществляет полную сборки системы
.PHONY: all
all: compile jars javadoc

# all_javas - временный файл для хранения списка исходных файлов
all_javas := $(OUTPUT_DIR)/all.javas

# compile - компилирует исходный код
.PHONY: compile
compile: $(all_javas)
    $(JAVAC) $(JFLAGS) @$<

# all_javas - составляет список исходных файлов
.INTERMEDIATE: $(all_javas)
$(all_javas):
    $(FIND) $(SOURCE_DIR) -name '*.java' > $@

# jar_list - список всех jar-архивов
jar_list := server_jar ui_jar

# jars - создаёт все jar-архивы
.PHONY: jars
jars: $(jar_list)

# server_jar - создаёт архив $(server_jar)
server_jar_name     := $(OUTPUT_DIR)/lib/a.jar
server_jar_manifest := src/com/company/manifest/foo.mf
server_jar_packages := com/company/m com/company/n

# ui_jar - создаёт архив $(ui_jar)
ui_jar_name     := $(OUTPUT_DIR)/lib/b.jar
ui_jar_manifest := src/com/company/manifest/bar.mf
ui_jar_packages := com/company/o com/company/p

# Создаёт явные правила для каждого архива
# $(foreach j, $(jar_list), $(eval $(call make-jar,$j)))
$(eval $(call make-jar,server_jar))
$(eval $(call make-jar,ui_jar))

# javadoc - создаёт документацию Java doc
.PHONY: javadoc
javadoc: $(all_javas)
    $(JAVADOC) $(JDFLAGS) @$<

.PHONY: clean
clean:
    $(RM) $(OUTPUT_DIR)

.PHONY: classpath
classpath:
    @echo CLASSPATH='$(CLASSPATH)'

.PHONY: check-config
check-config:
    @echo Проверяем конфигурацию...
    $(call check-file, $(class_path) JAVA_HOME)

.PHONY: print
print:
    $(foreach v, $(V), \
      $(warning $v = $($v)))
\end{verbatim}
}

%%--------------------------------------------------------------------
%% Compiling Java
%%--------------------------------------------------------------------
\section{Компиляция \Java{} кода}

Есть два способа компиляции кода \Java{} с помощью \GNUmake{}:
традиционный подход, вызывающий \utility{javac} для компиляции каждого
файла, и более быстрый подход, изложенный ранее и использующий
синтаксис \command{@filename}.

%---------------------------------------------------------------------
% The fast approach: all-in-one compile
%---------------------------------------------------------------------
\subsection*{Быстрый подход: компиляция всех исходных файлов за один
раз} \label{sec:all_in_one_compile}

Давайте более детально рассмотрим быстрый подход. Обратите внимание на
следующий фрагмент универсального \Makefile{}'а:

{\footnotesize
\begin{verbatim}
# all_javas - временный файл для хранения списка исходных файлов
all_javas := $(OUTPUT_DIR)/all.javas

# compile - компилирует исходный код
.PHONY: compile
compile: $(all_javas)
    $(JAVAC) $(JFLAGS) @$<

# all_javas - составляет список исходных файлов
.INTERMEDIATE: $(all_javas)
$(all_javas):
    $(FIND) $(SOURCE_DIR) -name '*.java' > $@
\end{verbatim}
}

Абстрактная цель \target{compile} вызывает \utility{javac} для
компиляции всего исходного кода проекта.

Реквизит \target{\$(all\_java)}~--- это файл, \filename{all.javas},
содержащий список исходных файлов \Java{}, по одному файлу на каждой
строке. Вовсе необязательно размещать каждый файл на отдельной строке,
однако так гораздо легче производить фильтрацию этого списка командой
\command{grep -v}, если в этом возникнет необходимость. Правило создания
файла \filename{all.javas} помечено как \target{.INTERMEDIATE},
поэтому \GNUmake{} будет удалять этот файл после каждого запуска и
создавать его заново перед каждой компиляцией. Командный сценарий для
создания файла очень прост. Для обеспечения максимальной переносимости
мы используем команду \utility{find} для извлечения всех исходных
файлов \Java{} из дерева каталогов с исходными файлами. Эта команда
работает не очень быстро, однако мы можем быть уверены в её корректной
работе. Более того, при изменении структуры дерева каталогов с
исходным кодом нам практически не придётся вносить изменений в этот
командный сценарий.

Если список каталогов, содержащих исходный код, определён и может быть
указан в вашем \Makefile{}'е, вы можете использовать более
производительный способ составления файла \filename{all.javas}.
Если список каталогов с исходным кодом не очень велик и помещается в
командной строке, не нарушая ограничений, накладываемых операционной
системой, можно использовать следующий сценарий:

{\footnotesize
\begin{verbatim}
$(all_javas):
    shopt -s nullglob; \
    printf "%s\n" $(addsuffix /*.java,$(PACKAGE_DIRS)) > $@
\end{verbatim}
}

Этот сценарий использует шаблоны командного интерпретатора для
определения списка \Java{}\hyp{}файлов в каждом каталоге. Однако если
каталог не содержит \Java{}\hyp{}файлов, нам хотелось бы, чтобы
раскрытие шаблон порождало пустую строку, а не текст исходного шаблона
(именно таково поведение по умолчанию многих командных
интерпретаторов). Для достижения этого эффекта используется опция
командного интерпретатора \utility{bash} \command{shopt -s nullglob}.
Большинство других интерпретаторов имеет подобную опцию. Наконец, мы
используем шаблоны и команду \command{printf} вместо \command{ln -l},
поскольку эти инструменты интегрированы в \utility{bash}, поэтому
потребуется выполнение всего одной программы независимо от числа
каталогов.

Мы можем избежать использования шаблонов интерпретатора при помощи
вызова функции \function{wildcard}:

{\footnotesize
\begin{verbatim}
$(all_javas):
    print "%s\n" $(wildcard \
                   $(addsuffix /*.java,$(PACKAGE_DIRS))) > $@
\end{verbatim}
}

Если ваш проект содержит много каталогов с исходным кодом (или пути к
ним имеют очень большую длину), предыдущий сценарий может превысить
предел длины командной строки вашей системы. В этом случае более
предпочтительным является следующий вариант:

{\footnotesize
\begin{verbatim}
.INTERMEDIATE: $(all_javas)
$(all_javas):
    shopt -s nullglob;           \
    for f in $(PACKAGE_DIRS);    \
    do                           \
      printf "%s\n" $$f/*.java;  \
    done > $@

\end{verbatim}
}

Заметим, что цель \target{compile} и вспомогательное правило следуют
подходу, основанному на нерекурсивном вызове \GNUmake{}. Не важно,
сколько подкаталогов в нашем проекте, мы используем единственный
\Makefile{} и производим единственный вызов компилятора. Если вам
нужно произвести компиляцию всего исходного кода, этот подход является
наиболее быстрым.

К тому же, мы совершенно не используем информацию о зависимостях.
Используя эти правила, \GNUmake{} не знает о связях между файлами и не
заботится о датах их модификации. Он просто осуществляет компиляцию
всего исходного кода при каждом вызове. В качестве бонуса мы получаем
возможность вызывать \GNUmake{} из каталога с исходными, а не
бинарными файлами. В контексте возможностей управления зависимостями
\GNUmake{} это может выглядеть как неразумный способ организации
\Makefile{}'а, однако давайте примем во внимание следующие доводы:

\begin{itemize}
%---------------------------------------------------------------------
\item Альтернатива (краткий обзор которой мы произведём позже)
использует стандартный подход, основанный на зависимостях. При этом
для каждого файла создаётся новый процесс \utility{javac}, что
увеличивает накладные расходы. Однако если наш проект не очень велик,
компиляция всех исходных файлов займёт не намного больше времени, чем
компиляция нескольких файлов, поскольку компилятор \utility{javac}
работает очень быстро, а создание новых процессов происходит
относительно медленно. Любая сборка, занимающая менее 15 секунд,
практически эквивалентна другой такой же, независимо от количества
работы, которую необходимо выполнить. Например, компиляция
приблизительно пятисот исходных файлов  (дистрибутива \utility{Ant})
занимает 14 секунд при выполнении на моём Pentium 4 1.8 ГГц, имеющем
512 Мб оперативной памяти. Компиляция одного файла занимает пять
секунд.
%---------------------------------------------------------------------
\item Б\'{о}льшая часть разработчиков будет использовать некий аналог
рабочей среды, предоставляющей быструю компиляцию отдельных файлов.
\Makefile{} же в основном будет использоваться в том случае, если
изменения охватывают большой участок кода, или требуется чистая
сборка, или же сборка осуществляется без вмешательства человека.
%---------------------------------------------------------------------
\item Как мы увидим, усилия, требуемые для реализации и поддержки
подхода, основанного на зависимостях, сравнимы усилиями, необходимыми
для реализации разделения деревьев каталогов исходных и бинарных
файлов для проектов, написанных на \Clang{}/\Cplusplus (эта тема
обсуждается в главе~\ref{chap:c_and_cpp}). Эту задачу не стоит
недооценивать.
%---------------------------------------------------------------------
\end{itemize}

Как мы увидим в следующих примерах, переменная
\variable{PACKAGE\_DIRS} используется не только для составления файла
\filename{all.javas}. Поддержка корректного значения этой переменной
может быть трудоёмким и потенциально сложным шагом. В случае небольших
проектов список каталогов может указываться явно прямо в
\Makefile{}'е, однако при росте числа каталогов до нескольких сотен
ручное редактирование этого списка становится довольно неприятным
занятием может привести к ошибкам. Более благоразумным способом может
быть использование программы \utility{find} для поиска соответствующих
каталогов:

{\footnotesize
\begin{verbatim}
# $(call find-compilation-dirs, root-directory)
  find-compilation-dirs =                       \
    $(patsubst %/,%,                            \
      $(sort                                    \
        $(dir                                   \
          $(shell $(FIND) $1 -name '*.java'))))
  PACKAGE_DIRS := $(call find-compilation-dirs, $(SOURCE_DIR))

\end{verbatim}
}

Команда \command{find} возвращает список файлов, функция
\function{dir} отсекает лишнюю часть имени, оставляя только имя
каталога, функция \function{sort} удаляет из списка дубликаты, а
функция \function{patsubst} удаляет слэш на конце каждого имени.
Обратите внимание на то, что функция \function{find-compilation-dirs}
находит все файлы, подлежащие компиляции, только для того, чтобы
отсечь имена файлов, в то время как правило, ассоциированное с
\filename{all.javas} использует шаблоны для восстановления этих имён.
Это может показаться напрасным расточительством ресурсов, однако я
часто замечаю, что наличие списка пакетов, содержащих исходный код,
чрезвычайно удобно в других аспектах сборки, например, при
сканировании конфигурационных файлов EJB. Если в вашем случае список
пакетов не требуется, просто используйте один из более простых
методов, упомянутых при обсуждении составления файла
\filename{all.javas}.

%---------------------------------------------------------------------
% Compiling with dependencies
%---------------------------------------------------------------------
\subsection*{Компиляция с учётом зависимостей}

Для реализации сборки с полным учётом зависимостей нам потребуется
инструмент для извлечения информации о зависимостях из исходных файлов
\Java{}, подобный команде \command{cc -M}. Программа Jikes
(\filename{\url{http://www.ibm.com/developerworks/opensource/jikes}})~---
это компилятор \Java{} с открытым исходным кодом, поддерживающий
эту возможность при использовании опций \command{-makefile} или
\command{+M}. Jikes~--- не идеальный инструмент для разделения
исходного кода и бинарных файлов, потому что он всегда записывает файл
зависимостей в тот же каталог, в котором находится исходный файл,
однако он бесплатен и эффективен. Есть и положительная сторона: файлы
зависимостей создаются во время компиляции, поэтому дополнительный
вызов компилятора не требуется.

Ниже приведён пример функции для работы с зависимостями и правила,
использующего эту функцию:

{\footnotesize
\begin{verbatim}
%.class: %.java
    $(JAVAC) $(JFLAGS) +M $<
    $(call java-process-depend,$<,$@)

# $(call java-process-depend, source-file, object-file)
define java-process-depend
  $(SED) -e 's/^.*\.class *:/$2 $(subst .class,.d,$2):/' \
         $(subst .java,.u,$1) > $(subst .class,.tmp,$2)
  $(SED) -e 's/#.*//'                                    \
         -e 's/^[^:]*: *//'                              \
         -e 's/ *\\$$$$//'                               \
         -e '/^$$$$/ d'                                  \
         -e 's/$$$$/ :/' $(subst .class,.tmp,$2)         \
         >> $(subst .class,.tmp,$2)
  $(MV) $(subst .class,.tmp,$2).tmp $(subst .class,.d,$2)
endef
\end{verbatim}
}

Этот сценарий требует, чтобы запуск \GNUmake{} осуществлялся из
каталога с бинарными файлами, и чтобы директива \directive{vpath}
указывала расположение исходных файлов. Если вы хотите использовать
компилятор Jikes только для генерации зависимостей, обращаясь к
другому компилятору для непосредственной генерации кода, вы можете
использовать опцию \command{+B}, в этом случае Jikes не будет
генерировать байт\-код.

В небольшом тесте производительности, в рамках которого происходила
компиляция 223 \Java{}\hyp{}файлов, однострочная команда компиляции,
описанная ранее, выполнялась на моей машине 9.9 секунд. Компиляция тех
же 223 файлов с индивидуальным вызовом компилятора для каждого файла
потребовала 411.6 секунд, т.е. в 41.5 раз больше времени. Более того,
при использовании раздельной компиляции любая сборка, требующая
компиляции более четырёх исходных файлов, будет занимать больше
времени, чем компиляция всего проекта одной командой. Если генерация
зависимостей и компиляция будут осуществляться разными программами,
разница только увеличится.

Разумеется, среды разработки варьируются, однако всегда важно
внимательно обдумать ваши цели. Минимизация числа файлов, подлежащих
компиляции, не всегда будет означать минимизацию времени, требующегося
для сборки системы. В случае языка \Java{} полная проверка
зависимостей и минимизация числа компилируемых файлов не являются
необходимыми атрибутами хорошей среды программирования.

%---------------------------------------------------------------------
% Setting CLASSPATH
%---------------------------------------------------------------------
\subsection*{Определение переменной CLASSPATH}

Одной из самых важных проблем при разработке программного обеспечения
на языке \Java{} является корректное определение переменной
\variable{CLASSPATH}. Эта переменная определяет, откуда будет
загружаться код при разрешении ссылки на класс. Для корректной
компиляции \Java{}\hyp{}приложения \Makefile{} должен включать
правильное определение переменной \variable{CLASSPATH}. Эта задача
быстро становится сложной при добавлении \Java{}\hyp{}пакетов,
программных интерфейсов приложений (API) и вспомогательных
инструментов. Если правильное определение \variable{CLASSPATH} может
быть сложной задачей, имеет смысл делать эту задачу в каком-то одном
месте.

Техника, которую я нашёл полезной, заключается в определении
переменной \variable{CLASSPATH} \Makefile{}'е для нужд не только
\GNUmake{}, но и других программ. Например, цель \target{classpath}
может возвращать команду экспорта переменной \variable{CLASSPATH} в
среду командного интерпретатора, вызвавшего \GNUmake{}:

{\footnotesize
\begin{verbatim}
.PHONY: classpath
classpath:
    @echo "export CLASSPATH='$(CLASSPATH)'"
\end{verbatim}
}

Разработчики могут определять \variable{CLASSPATH} следующим образом
(если они используют \utility{bash}):

{\footnotesize
\begin{alltt}
\$ \textbf{eval \$(make classpath)}
\end{alltt}
}

Определить переменную \variable{CLASSPATH} в среде Windows можно
следующим образом:

{\footnotesize
\begin{verbatim}
.PHONY: windows_classpath
windows_classpath:
    regtool set /user/Environment/CLASSPATH \
	        "$(subst /,\\,$(CLASSPATH))"
    control sysdm.cpl,@1,3 &
    @echo "Теперь нажмите кнопку <<Переменные окружения>>, " \
	      "затем OK, затем снова OK."
\end{verbatim}
}

Программа \utility{regtool} является частью среды разработки Cygwin и
предназначена для работы с реестром Windows. Однако простое обновление
реестра не вызовет считывания нового значения. Одним из способов
осуществления этой задачи является посещение диалогового окна
<<Переменные окружения>> (Environment Variables) и закрытие этого
окна при помощи кнопки OK.

Вторая строка сценария сообщает Windows, что нужно отобразить
диалоговое окно <<Свойства системы>> (System Properties) и сделать
активной вкладку <<Дополнительно>> (Advanced). К сожалению, командный
сценарий не может отобразить диалоговое окно <<Переменные окружения>>
или активировать кнопку OK, поэтому последняя строка сценария
предлагает пользователю завершить работу самостоятельно.

Экспорт переменной \variable{CLASSPATH} в другие программы, такие как
проектные файлы Emacs JDEE или JBuilder, осуществляется очень просто.

Непосредственное определение переменной \variable{CLASSPATH} может
также управляться при помощи \GNUmake{}. Определение этой переменной
очевидным способом определённо является разумной идеей:

{\footnotesize
\begin{verbatim}
CLASSPATH = /third_party/toplink-2.5/TopLink.jar:/third_party/...
\end{verbatim}
}

Из соображений переносимости более предпочтительным способом является
использование переменных:

{\footnotesize
\begin{verbatim}
# Определение Java classpath
class_path := OUTPUT_DIR          \
              XERCES_JAR          \
              COMMONS_LOGGING_JAR \
              LOG4J_JAR           \
              JUNIT_JAR
...
# Определение CLASSPATH
export CLASSPATH := $(call build-classpath, $(class_path))
\end{verbatim}
}
(Определение \variable{CLASSPATH}, приведённое в коде универсального
\Makefile{}'а, более показательно и полезно). Должным образом
реализованная функция \function{build\hyp{}classpath} решает несколько
раздражающих проблем:

\begin{itemize}
%---------------------------------------------------------------------
\item Очень просто собрать значение \variable{CLASSPATH} из частей.
Например, если используется несколько серверов приложений, может
потребоваться изменение \variable{CLASSPATH}. Различные версии
\variable{CLASSPATH} могут заключаться в секции \directive{ifdef} и
выбираться на основании значения какой-либо переменной \GNUmake{}.
%---------------------------------------------------------------------
\item Люди, занимающиеся поддержкой \Makefile{}'а, не должны
волноваться о внутренних пробелах, символах новой строки или переносах
строк, функция \function{build\hyp{}classpath} осуществляет
необходимые операции самостоятельно.
%---------------------------------------------------------------------
\item Функция \function{build\hyp{}classpath} может выбирать
разделитель путей автоматически, делая тем самым значение переменной
корректным для Windows и \UNIX{}.
%---------------------------------------------------------------------
\item Функция \function{build\hyp{}classpath} может осуществлять
проверку правильности элементов списка путей. В частности, одной из
раздражающих проблем \GNUmake{} является то, что вычисление
переменных, значение которых не определено, просто возвращает пустую
строку.  В большинстве случаев такое поведение полезно, однако иногда
оно может встать на вашем пути. В этом случае значение переменной
\variable{CLASSPATH} будет иметь фиктивное значение%
\footnote{
Для обнаружения этой ситуации можно попробовать использовать опцию
\command{-{}-warn\hyp{}undefined\hyp{}variables}, однако это приведёт
к предупреждениям, относящимся к тем переменным, неопределённое
значение которых нас устраивает.}.
Мы можем решить эту проблему, добавив проверку определённости
переменных, входящих в список путей, в функцию
\function{build\hyp{}classpath}. Функция также может проверять
существование каждого файла или каталога, входящего в список, и, в
случае невыполнения ограничений, выводить соответствующее
предупреждение.
%---------------------------------------------------------------------
\item Наконец, для реализации наиболее изощрённого функционала
(например, обработки пробелов в именах файлов или путях поиска) может
быть удобно использовать триггер для обработки переменной
\variable{CLASSPATH}.
%---------------------------------------------------------------------
\end{itemize}

Ниже приведена реализация функции \function{build\hyp{}classpath},
учитывающая первые три пункта нашего списка:

{\footnotesize
\begin{verbatim}
# $(call build-classpath, variable-list)
define build-classpath
  $(strip                                         \
    $(patsubst %:,%,                              \
      $(subst : ,:,                               \
        $(strip                                   \
          $(foreach c,$1,$(call get-file,$c):)))))
endef

# $(call get-file, variable-name)
define get-file
  $(strip                                        \
    $($1)                                        \
    $(if $(call file-exists-eval,$1),,           \
      $(warning Файл, указанный в переменной     \
                '$1' ($($1)), не найден)))
endef

# $(call file-exists-eval, variable-name)
define file-exists-eval
  $(strip                                                \
    $(if $($1),,$(warning Переменная'$1' не определена)) \
    $(wildcard $($1)))
endef
\end{verbatim}
}

Функция \function{build\hyp{}classpath} проходит по всем словам своего
аргумента, производя проверку каждого элемента и соединяя эти
элементы, используя разделитель путей (в нашем случае это
\command{:}).  Реализовать автоматический выбор разделителя путей
теперь очень просто. Затем функция удаляет пробелы, добавленные
функцией \function{get\hyp{}file} и циклом \function{for\-each}. Затем
функция удаляет последний разделитель, добавленный циклом
\function{for\-each}. Наконец, весь список подаётся на вход функции
\function{strip}, благодаря чему удаляются лишние пробелы, добавленные
продолжением строк.

Функция \function{get\hyp{}file} принимает на вход имя переменной и
осуществляет проверку существования файла, имя которого является
значением переменной. Если файл не существует, генерируется
предупреждение. Функция возвращает значение переменной независимо от
того, существует ли соответствующий файл, так как это значение может
быть полезным для пользователя. В некоторых случаях функция
\function{get\hyp{}file} может быть применена к файлу, который будет
сгенерирован позже и пока не существует.

Последняя функция, \function{file\hyp{}exists\hyp{}eval}, принимает
в качестве аргумента имя переменной, содержащей имя файла. Если
переменная содержит пустую строку, генерируется предупреждение, в
противном случае для проверки существования файла (или списка файлов)
используется вызов функции \function{wild\-card}

Если функция \function{build\hyp{}classpath} применяется к фиктивным
значениям, при запуске мы увидим следующие ошибки:

{\footnotesize
\begin{verbatim}
Makefile:37: Файл, указанный в переменной 'TOPLINKX_25_JAR'
             (/usr/java/toplink-2.5/TopLinkX.jar), не найден
...
Makefile:37: Переменная 'XERCES_142_JAR' не определена
Makefile:37: Файл, указанный в переменной 'XERCES_142_JAR'
             ( ), не найден
\end{verbatim}
}

Этот пример демонстрирует существенный прогресс по сравнению с
молчанием, которое мы получаем при использовании простого подхода.

Существование функции \function{get\hyp{}file} подразумевает, что мы
можем обобщить задачу поиска входных файлов.

{\footnotesize
\begin{verbatim}
# $(call get-jar, variable-name)
define get-jar
  $(strip                                                     \
    $(if $($1),,$(warning Переменная '$1' пуста))             \
    $(if $(JAR_PATH),,$(warning Переменная JAR_PATH пуста))   \
    $(foreach d, $(dir $($1)) $(JAR_PATH),                    \
      $(if $(wildcard $d/$(notdir $($1))),                    \
        $(if $(get-jar-return),,                              \
          $(eval get-jar-return := $d/$(notdir $($1))))))     \
    $(if $(get-jar-return),                                   \
      $(get-jar-return)                                       \
      $(eval get-jar-return :=),                              \
      $($1)                                                   \
      $(warning get-jar: Файл '$1' не найден в $(JAR_PATH))))
endef
\end{verbatim}
}

Здесь мы определяем переменную \variable{JAR\_PATH}, содержащую пути
для поиска файлов. Возвращается первый найденный файл. Параметром
функции является имя переменной, содержащей путь к архиву jar. Мы
производим поиск jar\hyp{}файла, используя сначала путь, содержащийся
в переданной переменной, затем набор путей, содержащихся в переменной
\variable{JAR\_PATH}. Чтобы реализовать такое поведение, список
каталогов в цикле \function{for\-each} составляется из значения
переменной, за которым сделует значение переменной
\variable{JAR\_PATH}. Два других обращения к параметру обрамляются
вызовом функции \function{notdir}, благодаря чему имя архива можно
соединить с соответствующим элементом списка. Обратите внимание на то,
что мы не можем выйти из цикла \function{for\-each}. Вместо этого,
однако, мы используем функцию \function{eval} для определения
переменной \variable{get\hyp{}jar\hyp{}return}, используемую для
хранения первого найденного файла. После выхода из цикла мы возвращаем
значение временной переменной или генерируем предупреждение, если файл
не был найден. Важно не забыть сбросить значение временной переменной
перед завершением работы макроса.

Эта функция по существу является реализацией директивы
\directive{vpath} в контексте определения переменной
\variable{CLASSPATH}. Чтобы понять это, вспомним, что директива
\directive{vpath} используется \GNUmake{} для нахождения реквизитов,
которые не были найдены по их относительному пути. В такой
ситуации \GNUmake{} производит поиск реквизитов в каталогах, указанных
директивой \directive{vpath}, и подставляет дополненный путь в
автоматические переменные \variable{\$\^}, \variable{\$?} и
\variable{\$+}. Мы хотим, чтобы для определения переменной
\variable{CLASSPATH} \GNUmake{} производил поиск пути к каждому
jar\hyp{}файлу и производил конкатенацию переменной
\variable{CLASSPATH} с этим дополненным путём. Поскольку \GNUmake{} не
имеет встроенной поддержки этого функционала, мы добавляем его
самостоятельно.  Разумеется, вы можете просто всегда прописывать
полные пути к jar\hyp{}файлам, предоставив задачу поиска виртуальной
машине \Java{}, однако переменная \variable{CLASSPATH} и без того
быстро становится длинной. На некоторых операционных системах длина
переменных окружения ограничена и существует опасность усечения
длинных значений \variable{CLASSPATH}.  Например, на операционной
системе Windows XP длина значения переменной окружения ограничена 1023
символами. В добавок, даже если переменая \variable{CLASSPATH} не
будет усечена, виртуальная машина \Java{} должна производить поиск в
\variable{CLASSPATH} при загрузке классов, что замедляет работу
приложения.

%%--------------------------------------------------------------------
%% Managing jars
%%--------------------------------------------------------------------
\section{Управление архивами \Java{}}

Сборка и управление \Java{}\hyp{}архивами поднимают проблемы, отличные
от тех, с которыми мы сталкивались при сборке библиотек
\Clang{}/\Cplusplus{}. На это есть три причины. Во\hyp{}первых,
элементы \Java{}\hyp{}архива адресуются относительным путём, поэтому
точные имена файлов, передаваемых программе \utility{jar}, нужно
тщательно контролировать. Во\hyp{}вторых, в \Java{}\hyp{}сообществе
есть тенденция соединять архивы, чтобы всё приложение могло
размещаться в единственном архиве. Наконец, \Java{}\hyp{}архивы могут
содержать файлы, отличные от файлов классов, например, файл манифеста,
файлы свойств и XML\hyp{}файлы.

Базовая команда для создания \Java{}\hyp{}архива при помощи GNU
\GNUmake{} выглядит следующим образом:

{\footnotesize
\begin{verbatim}
JAR      := jar
JARFLAGS := -cf

$(FOO_JAR): реквизиты...
    $(JAR) $(JARFLAGS) $@ $^
\end{verbatim}
}

Программа \utility{jar} может принимать вместо имён файлов имена
каталогов, в этом случае в архив будет помещено всё содержимое
указанных каталогов. Это может быть очень удобно, особенно при
использовании совместно с опцией \command{-C}, временно изменяющей
текущий каталог:

{\footnotesize
\begin{verbatim}
JAR      := jar
JARFLAGS := -cf

.PHONY: $(FOO_JAR)
$(FOO_JAR):
    $(JAR) $(JARFLAGS) $@ -C $(OUTPUT_DIR) com
\end{verbatim}
}

Здесь файл архива объявлен абстрактной целью. Однако при повторном
запуске \Makefile{}'а архив не будет создаваться заново, поскольку эта
цель не имеет реквизитов. Как и в случае команды \utility{ar},
описанной в одной из предыдущих глав, смысла в использовании флага
обновления архива, \command{-u}, практически нет, поскольку эта
операция занимает практически такое же (или даже большее) время, что и
операция создания нового архива.

\Java{}\hyp{}архив часто включает файл манифеста, в котором указан
поставщик, API и номер версии. Простой файл манифеста может выглядеть
следующим образом:

{\footnotesize
\begin{verbatim}
Name: JAR_NAME
Specification-Title: SPEC_NAME
Implementation-Version: IMPL_VERSION
Specification-Vendor: Generic Innovative Company, Inc.
\end{verbatim}
}

Этот файл содержит три переменных, \variable{JAR\_NAME},
\variable{SPEC\_NAME} и \variable{IMPL\_VERSION}, которые могут быть
заменены реальными значениями при создании архива с помощью
\utility{sed}, \utility{m4} или вашего любимого редактора потоков.
Ниже приведена функция для обработки файла манифеста:

{\footnotesize
\begin{verbatim}
MANIFEST_TEMPLATE := src/manifests/default.mf
TMP_JAR_DIR       := $(call make-temp-dir)
TMP_MANIFEST      := $(TMP_JAR_DIR)/manifest.mf

# $(call add-manifest, jar, jar-name, manifest-file-opt)
define add-manifest
  $(RM) $(dir $(TMP_MANIFEST))
  $(MKDIR) $(dir $(TMP_MANIFEST))
  m4 --define=NAME="$(notdir $2)"            \
     --define=IMPL_VERSION=$(VERSION_NUMBER) \
     --define=SPEC_VERSION=$(VERSION_NUMBER) \
     $(if $3,$3,$(MANIFEST_TEMPLATE))        \
     > $(TMP_MANIFEST)
  $(JAR) -ufm $1 $(TMP_MANIFEST)
  $(RM) $(dir $(TMP_MANIFEST))
endef
\end{verbatim}
}

Функция \function{add\hyp{}manifest} оперирует файлом манифеста
методом, подобным описанному выше. Сначала функция создаёт временный
каталог, затем производит подстановку переменных в шаблоне файла
манифеста. Затем функция обновляет архив и удаляет временный каталог.
Обратите внимание на то, что последний аргумент функции является
необязательным. Если путь к файлу манифеста не указан, функция
использует значение переменной \variable{MANIFEST\_TEMPLATE}.

В универсальном \Makefile{}'е эти операции привязаны к общей функции,
осуществляющей составление явного правила для создания
\Java{}\hyp{}архива:

{\footnotesize
\begin{verbatim}
# $(call make-jar,jar-variable-prefix)
define make-jar
  .PHONY: $1 $$($1_name)
  $1: $($1_name)
  $$($1_name):
      cd $(OUTPUT_DIR); \
      $(JAR) $(JARFLAGS) $$(notdir $$@) $$($1_packages)
      $$(call add-manifest, $$@, $$($1_name), $$($1_manifest))
endef
\end{verbatim}
}

Эта функция принимает один аргумент, префикс переменной \GNUmake{},
который идентифицирует набор переменных, описывающих четыре параметра
архива: имя цели, имя архива, пакеты архива и файл манифеста.
Например, для создания архива \filename{ui.jar} мы напишем следующее:

{\footnotesize
\begin{verbatim}
ui_jar_name     := $(OUTPUT_DIR)/lib/ui.jar
ui_jar_manifest := src/com/company/ui/manifest.mf
ui_jar_packages := src/com/company/ui \
                   src/com/company/lib

$(eval $(call make-jar,ui_jar))
\end{verbatim}
}

Используя композицию имён переменных, мы можем сократить
последовательность действий, выполняемых функцией, достигнув в тоже
время гибкой её реализации.

Если нам нужно создать много архивов, мы можем автоматизировать этот
процесс, поместив список имён архивов в переменную:

{\footnotesize
\begin{verbatim}
jar_list := server_jar ui_jar

.PHONY: jars $(jar_list)
jars: $(jar_list)

$(foreach j, $(jar_list),\
  $(eval $(call make-jar,$j)))
\end{verbatim}
}

В некоторых случаях нам может понадобиться распаковать содержимое
архива во временный каталог. Ниже представлен пример простой функции,
реализующей это требование:

{\footnotesize
\begin{verbatim}
# $(call burst-jar, jar-file, target-directory)
define burst-jar
  $(call make-dir,$2)
  cd $2; $(JAR) -xf $1
endef
\end{verbatim}
}

%%--------------------------------------------------------------------
%% Reference trees and third-party jars
%%--------------------------------------------------------------------
\section{Справочные деревья и архивы сторонних %
разработчиков}

Для того, чтобы использовать единое разделяемое справочное дерево с
поддержкой создания разработчиками частичных рабочих копий, просто
настройте механизм ночных сборок, создающий \Java{}\hyp{}архивы
проекта, и включите эти архивы в \variable{CLASSPATH} компилятора.
После этого шага разработчики смогут сделать нужную им частичную
рабочую копию и инициировать процесс компиляции (в предположении, что
список исходных файлов создаётся динамически программой, подобной
\utility{find}). Когда компилятору \Java{} нужно будет найти символ,
определённый в отсутствующем исходном файле, компилятор произведёт
поиск, основываясь на значении \variable{CLASSPATH}, и обнаружит
соответствующие файлы классов в архиве.

Получение \Java{}\hyp{}архивов сторонних разработчиков из справочного
дерева реализуется также просто. Просто поместите пути к этим архивам
в переменную \variable{CLASSPATH}. Как уже было замечено, \Makefile{}
может быть очень полезным инструментом управления этим процессом.
Разумеется, функция \function{get\hyp{}file} может быть использована
для автоматического выбора стабильной или бета версии, локальных или
удалённых \Java{}\hyp{}архивов при помощи соответствующего определения
переменной \variable{JAR\_PATH}.

%%--------------------------------------------------------------------
%% Enterprise JavaBeans
%%--------------------------------------------------------------------
\section{Enterprise JavaBeans}

\Java{}\hyp{}компоненты уровня предприятия (Enterprise
Java\-Beans\trademark{}, EJB)~--- это мощная техника инкапсуляции и
повторного использования бизнес\hyp{}логики, каркасом которой является
механизм удалённых вызовов методов (Remote Method Invocation, RMI).
EJB определяет \Java{}\hyp{}классы, используемые для реализации API
сервера, используемого, в конечном счёте, удалёнными клиентами. Эти
объекты и службы настраиваются при помощи специальных файлов в формате
XML.  После написания \Java{}\hyp{}класса и соответствующего ему
конфигурационного XML\hyp{}файла эти файлы нужно упаковать вместе в
\Java{}\hyp{}архив. Затем вызывается специальный EJB\hyp{}компилятор,
создающий код заглушек и связок, реализующих поддержку RPC.

Следующий код может быть добавлен в код универсального \Makefile{}'а
для предоставления поддержки EJB:

{\footnotesize
\begin{verbatim}
EJB_TMP_JAR = $(EJB_TMP_DIR)/temp.jar
META_INF    = $(EJB_TMP_DIR)/META-INF

# $(call compile-bean, jar-name,
#        bean-files-wildcard, manifest-name-opt)
define compile-bean
  $(eval EJB_TMP_DIR := $(shell mktemp -d \
                          $(TMPDIR)/compile-bean.XXXXXXXX))
  $(MKDIR) $(META_INF)
  $(if $(filter %.xml, $2),cp $(filter %.xml, $2) $(META_INF))
  cd $(OUTPUT_DIR) &&                     \
  $(JAR) -cf0 $(EJB_TMP_JAR)              \
         $(call jar-file-arg,$(META_INF)) \
         $(filter-out %.xml, $2)
  $(JVM) weblogic.ejbc $(EJB_TMP_JAR) $1
  $(call add-manifest,$(if $3,$3,$1),,)
  $(RM) $(EJB_TMP_DIR)
endef

# $(call jar-file-arg, jar-file)
jar-file-arg = -C "$(patsubst %/,%,$(dir $1))" $(notdir $1)
\end{verbatim}
}

Функция \function{compile\hyp{}bean} принимает три параметра: имя
\Java{}\hyp{}архива, который требуется создать, список файлов,
входящих в архив, и необязательный файл манифеста. Сначала при помощи
программы \utility{mktemp} создаётся пустой временный каталог, имя
каталога сохраняется в переменной \variable{EJB\_TMP\_DIR}. Поместив
присваивание этой переменной в функцию \function{eval}, мы получаем
гарантию того, что значение \variable{EJB\_TMP\_DIR} будет указывать
на новый временный каталог при каждом вычислении функции
\function{compile\hyp{}bean}. Поскольку функция
\function{compile\hyp{}bean} используется в командном сценарии,
она будет вычисляться только при выполнении сценария. Затем функция
осуществляет копирование всех XML файлов из списка
\variable{bean\hyp{}files\hyp{}wild\-card} в каталог
\filename{META\hyp{}INF}. Именно в этом каталоге хранятся
конфигурационные файлы EJB. После этого функция создаёт временный
\Java{}\hyp{}архив, используемый в качестве входа для
EJB\hyp{}компилятора. Функция \function{jar\hyp{}file\hyp{}arg}
преобразует имена вида \filename{dir1/dir2/dir3} к виду
\filename{-C dir1/dir2 dir3}, поэтому относительные имена файлов
архива корректны. Этот наиболее подходящий формат для передачи команде
\utility{jar} пути к каталогу \filename{META\hyp{}INF}. Поскольку
XML\hyp{}файлы, содержавшиеся в списке, уже скопированы в каталог
\filename{META\hyp{}INF}, мы отсеиваем их из списка аргументов команды
\utility{jar} при помощи функции \function{filter\hyp{}out}. После
сборки временного архива вызывается EJB\hyp{}компилятор Web\-Lo\-gic,
создающий результирующий архив. Затем к составленному архиву
добавляется файл манифеста. Последним действием является удаление
временного архива.

Способ использования новой функции очевиден:

{\footnotesize
\begin{verbatim}
bean_files = com/company/bean/FooInterface.class      \
             com/company/bean/FooHome.class           \
             src/com/company/bean/ejb-jar.xml         \
             src/com/company/bean/weblogic-ejb-jar.xml

.PHONY: ejb_jar $(EJB_JAR)
ejb_jar: $(EJB_JAR)
$(EJB_JAR):
    $(call compile-bean, $@, $(bean_files), weblogic.mf)
\end{verbatim}
}

Список \variable{bean\_files} немного необычен. Пути к файлам классов,
входящих в этот список, указаны относительно каталога
\filename{classes}, в то время как пути к XML\hyp{}файлам будут
вычисляться относительно каталога, в котором располагается
\Makefile{}.

Это всё замечательно, но что если ваш архив содержит много файлов?
Существует ли способ составить список файлов автоматически?
Разумеется:

{\footnotesize
\begin{verbatim}
src_dirs := $(SOURCE_DIR)/com/company/...

bean_files =                                          \
  $(patsubst $(SOURCE_DIR)/%,%,                       \
    $(addsuffix /*.class,                             \
      $(sort                                          \
        $(dir                                         \
          $(wildcard                                  \
            $(addsuffix /*Home.java,$(src_dirs)))))))

.PHONY: ejb_jar $(EJB_JAR)
ejb_jar: $(EJB_JAR)
$(EJB_JAR):
    $(call compile-bean, $@, $(bean_files), weblogic.mf)
\end{verbatim}
}

Этот код подразумевает, что список каталогов с исходными файлами
хранится в переменной \variable{src\_dirs} (в списке могут находится и
каталоги, не содержащие кода EJB\hyp{}компонентов), и что все файлы,
оканчивающиеся строкой \emph{Home.java}, идентифицируют пакеты,
содержащие код EJB\hyp{}компонентов. Выражение для определения
переменной \variable{bean\_files} сначала добавляет суффикс шаблона к
имени каждого каталога в списке, а затем вызывает функцию
\function{wild\-card} для нахождения всех файлов, имя которых
оканчивается строкой \emph{Home.java}. Имена файлов отбрасываются,
полученный список каталогов сортируется, дублирующиеся элементы
удаляются из списка. К каждому каталогу добавляется суффикс
\command{/*.class}, в результате командный интерпретатор заменит
шаблон списком соответствующих файлов классов. Наконец, от каждого
элемента списка отсекается префикс, содержащий имя каталога с
исходными файлами (поскольку такого подкаталога каталога
\filename{classes} не существует). Причиной использования шаблонов
командного интерпретатора вместо функции \function{wild\-card}
является тот факт, что \GNUmake{} не сможет гарантированно выполнить
поиск файлов, соответствующих шаблону, \emph{после} компиляции и
генерации файлов классов. Если \GNUmake{} вычислит функцию
\function{wild\-card} слишком рано, файлы не будут обнаружены, а кэш
содержимого каталогов помешает найти эти файлы позже. Применение же
функции \function{wild\-card} в каталоге с исходными файлами
совершенно безопасно, поскольку мы подразумеваем, что исходные файлы
не будут добавляться во время работы \GNUmake{}.

Предыдущий код будет работать в том случае, если у нас имеется
небольшое число архивов компонентов. Другой стиль разработки
подразумевает помещение каждого EJB\hyp{}компонента в собственный
\Java{}\hyp{}архив. Большие проекты могут содержать десятки архивов.
Для того, чтобы осуществлять автоматическую обработку этой ситуации,
нам нужно составить явное правило для каждого EJB\hyp{}архива. В нашем
примере исходный код EJB\hyp{}компонентов самодостаточен: каждый
компонент располагается в отдельном каталоге вместе с ассоциированным
XML\hyp{}файлом. Определить каталоги, содержащие EJB\hyp{}компоненты,
можно по наличию файлов, оканчивающихся строкой \emph{Session.java}.

Основной подход заключается в поиске EJB\hyp{}компонентов в каталогах
с исходным кодом, построении явного правила для каждого компонента и
записи этих правил в файл. Затем файл с правилами для
EJB\hyp{}компонентов включается в наш \Makefile{}. Создание файла с
правилами для компонентов вызывается через механизм управления
включаемыми файлами \GNUmake{}.

{\footnotesize
\begin{verbatim}
# session_jars - архивы EJB, адресованные относительным путём.
session_jars =
  $(subst .java,.jar,                       \
    $(wildcard                              \
      $(addsuffix /*Session.java, $(COMPILATION_DIRS))))

# EJBS - список всех EJB-архивов.
EJBS = $(addprefix $(TMP_DIR)/,$(notdir $(session_jars)))

# ejbs - Create all EJB jar files.
.PHONY: ejbs
ejbs: $(EJBS)
$(EJBS):
    $(call compile-bean,$@,$^,)
\end{verbatim}
}

С помощью вызова функции \function{wild\-card} со списком всех
каталогов с исходным кодом в качестве аргумента мы находим все файлы,
имя которых оканчивается на \emph{Session.java}. В нашем примере имя
архива образуется из имени найденного исходного файла с добавлением
расширения \filename{.jar}. Архивы будут помещаться во временный
каталог. Переменная \variable{EJBS} содержит список архивов,
адресованных относительным путём от корня дерева бинарных файлов.
Эти архивы являются целью, которую мы хотим обновить. Командным
сценарием является вызов функции \function{compile\hyp{}bean},
реализованной нами ранее. Фокус заключается в том, что список файлов
указан в качестве реквизита каждого архива. Давайте посмотрим, как
они будут создаваться.

{\footnotesize
\begin{verbatim}
-include $(OUTPUT_DIR)/ejb.d

# $(call ejb-rule, ejb-name)
ejb-rule = $(TMP_DIR)/$(notdir $1):            \
             $(addprefix $(OUTPUT_DIR)/,       \
               $(subst .java,.class,           \
                 $(wildcard $(dir $1)*.java))) \
             $(wildcard $(dir $1)*.xml)

# ejb.d - файл зависимостей EJB
$(OUTPUT_DIR)/ejb.d: Makefile
    @echo Вычисляю зависимости ejb...
    @for f in $(session_jars);       \
    do                               \
      echo "\$$(call ejb-rule,$$f)"; \
    done > $@
\end{verbatim}
}

Зависимости для каждого EJB\hyp{}архива записываются в файл
\filename{ejb.d}, включаемый в \Makefile{}. Когда \GNUmake{} первый
раз производит поиск этого файла, файл ещё не существует. Поэтому
\GNUmake{} вызывает правило для обновления включаемого файла. Это
правило записывает по одной строке, подобной следующей, для каждого
EJB\hyp{}архива:

{\footnotesize
\begin{verbatim}
$(call ejb-rule,src/com/company/foo/FooSession.jar)
\end{verbatim}
}

Результатом вычисления функции \function{ejb\hyp{}rule} является
имя целевого архива и списка реквизитов, как показано ниже:

{\footnotesize
\begin{verbatim}
classes/lib/FooSession.jar:                  \
    classes/com/company/foo/FooHome.jar      \
    classes/com/company/foo/FooInterface.jar \
    classes/com/company/foo/FooSession.jar   \
    src/com/company/foo/ejb-jar.xml          \
    src/com/company/foo/ejb-weblogic-jar.xml
\end{verbatim}
}

Таким образом, \GNUmake{} предоставляет возможность управлять довольно
большим количеством архивов без необходимости ручной поддержки набора
явных правил.

