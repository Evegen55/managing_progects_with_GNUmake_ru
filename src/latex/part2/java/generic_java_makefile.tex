%%--------------------------------------------------------------------
%% A generic java makefile
%%--------------------------------------------------------------------
\section{Универсальный \Makefile{} для \Java{}}

Следующий пример демонстрирует универсальный \Makefile{} для сборки
\Java{}\hyp{}проектов. Я объясню каждую из его частей далее в этой
главе.

{\footnotesize
\begin{verbatim}

# Общий makefile для Java-проекта.
VERSION_NUMBER := 1.0

# Определения базовых каталогов
SOURCE_DIR     := src
OUTPUT_DIR     := classes

# Инструменты Unix
AWK            := awk
FIND           := /bin/find
MKDIR          := mkdir -p
RM             := rm -rf
SHELL          := /bin/bash

# Пути для поддержки работы программ
JAVA_HOME      := /opt/j2sdk1.4.2_03
AXIS_HOME      := /opt/axis-1_1
TOMCAT_HOME    := /opt/jakarta-tomcat-5.0.18
XERCES_HOME    := /opt/xerces-1_4_4
JUNIT_HOME     := /opt/junit3.8.1

# Инструменты Java
JAVA           := $(JAVA_HOME)/bin/java
JAVAC          := $(JAVA_HOME)/bin/javac

JFLAGS         := -sourcepath $(SOURCE_DIR)  \
                  -d $(OUTPUT_DIR)           \
                  -source 1.4

JVMFLAGS       := -ea                        \
                  -esa                       \
                  -Xfuture

JVM            := $(JAVA) $(JVMFLAGS)

JAR            := $(JAVA_HOME)/bin/jar
JARFLAGS       := cf

JAVADOC        := $(JAVA_HOME)/bin/javadoc
JDFLAGS        := -sourcepath $(SOURCE_DIR) \
                  -d $(OUTPUT_DIR)          \
                  -link http://java.sun.com/products/jdk/1.4/docs/api

# Jar архивы
COMMONS_LOGGING_JAR := $(AXIS_HOME)/lib/commons-logging.jar

LOG4J_JAR           := $(AXIS_HOME)/lib/log4j-1.2.8.jar
XERCES_JAR          := $(XERCES_HOME)/xerces.jar
JUNIT_JAR           := $(JUNIT_HOME)/junit.jar

# Определяем путь к классам Java
class_path := OUTPUT_DIR          \
              XERCES_JAR          \
              COMMONS_LOGGING_JAR \
              LOG4J_JAR           \
              JUNIT_JAR

# Пробел
space := $(empty) $(empty)

# $(call build-classpath, variable-list)
define build-classpath
  $(strip                                          \
    $(patsubst :%,%,                               \
      $(subst : ,:,                                \
        $(strip                                    \
          $(foreach j,$1,$(call get-file,$j):)))))
endef

# $(call get-file, variable-name)
define get-file
  $(strip                                         \
    $($1)                                         \
      $(if $(call file-exists-eval,$1),,          \
        $(warning Файл, указанный в переменной    \
                  '$1' ($($1)), не найден)))
endef

# $(call file-exists-eval, variable-name)
define file-exists-eval
  $(strip                                      \
    $(if $($1),,$(warning '$1' has no value))  \
    $(wildcard $($1)))
endef

# $(call brief-help, makefile)
define brief-help
  $(AWK) '$$1 ~ /^[^.][-A-Za-z0-9]*:/                   \
          { print substr($$1, 1, length($$1)-1) }' $1 | \
  sort |                                                \
  pr -T -w 80 -4
endef

# $(call file-exists, wildcard-pattern)
file-exists = $(wildcard $1)

# $(call check-file, file-list)
define check-file
  $(foreach f, $1,                       \
    $(if $(call file-exists, $($f)),,    \
      $(warning $f ($($f)) is missing)))
endef

# $(call make-temp-dir, root-opt)
define make-temp-dir
  mktemp -t $(if $1,$1,make).XXXXXXXXXX
endef

# MANIFEST_TEMPLATE - шаблон файла манифеста, предназначенный
#                     для обработки макропроцессором m4
MANIFEST_TEMPLATE := src/manifest/manifest.mf
TMP_JAR_DIR       := $(call make-temp-dir)
TMP_MANIFEST      := $(TMP_JAR_DIR)/manifest.mf

# $(call add-manifest, jar, jar-name, manifest-file-opt)
define add-manifest
  $(RM) $(dir $(TMP_MANIFEST))
  $(MKDIR) $(dir $(TMP_MANIFEST))
  m4 --define=NAME="$(notdir $2)"            \
     --define=IMPL_VERSION=$(VERSION_NUMBER) \
     --define=SPEC_VERSION=$(VERSION_NUMBER) \
     $(if $3,$3,$(MANIFEST_TEMPLATE))        \
     > $(TMP_MANIFEST)
  $(JAR) -ufm $1 $(TMP_MANIFEST)
  $(RM) $(dir $(TMP_MANIFEST))
endef

# Определяем переменную CLASSPATH
export CLASSPATH := $(call build-classpath, $(class_path))

# make-directories - убеждаемся, что выходной каталог существует
make-directories := $(shell $(MKDIR) $(OUTPUT_DIR))

# help - цель по умолчанию
.PHONY: help
help:
    @$(call brief-help, $(CURDIR)/Makefile)

# all - осуществляет полную сборки системы
.PHONY: all
all: compile jars javadoc

# all_javas - временный файл для хранения списка исходных файлов
all_javas := $(OUTPUT_DIR)/all.javas

# compile - компилирует исходный код
.PHONY: compile
compile: $(all_javas)
    $(JAVAC) $(JFLAGS) @$<

# all_javas - составляет список исходных файлов
.INTERMEDIATE: $(all_javas)
$(all_javas):
    $(FIND) $(SOURCE_DIR) -name '*.java' > $@

# jar_list - список всех jar-архивов
jar_list := server_jar ui_jar

# jars - создаёт все jar-архивы
.PHONY: jars
jars: $(jar_list)

# server_jar - создаёт архив $(server_jar)
server_jar_name     := $(OUTPUT_DIR)/lib/a.jar
server_jar_manifest := src/com/company/manifest/foo.mf
server_jar_packages := com/company/m com/company/n

# ui_jar - создаёт архив $(ui_jar)
ui_jar_name     := $(OUTPUT_DIR)/lib/b.jar
ui_jar_manifest := src/com/company/manifest/bar.mf
ui_jar_packages := com/company/o com/company/p

# Создаёт явные правила для каждого архива
# $(foreach j, $(jar_list), $(eval $(call make-jar,$j)))
$(eval $(call make-jar,server_jar))
$(eval $(call make-jar,ui_jar))

# javadoc - создаёт документацию Java doc
.PHONY: javadoc
javadoc: $(all_javas)
    $(JAVADOC) $(JDFLAGS) @$<

.PHONY: clean
clean:
    $(RM) $(OUTPUT_DIR)

.PHONY: classpath
classpath:
    @echo CLASSPATH='$(CLASSPATH)'

.PHONY: check-config
check-config:
    @echo Проверяем конфигурацию...
    $(call check-file, $(class_path) JAVA_HOME)

.PHONY: print
print:
    $(foreach v, $(V), \
      $(warning $v = $($v)))
\end{verbatim}
}
