%%%-------------------------------------------------------------------
%%% Managing large projects
%%%-------------------------------------------------------------------
\chapter{Управление большими проектами}
\label{chap:managing_large_proj}

Какой проект можно назвать большим? Для наших целей мы назовём большим
проект, требующий команды разработчиков, поддерживающий несколько
архитектур, предполагающий несколько релизов и нуждающийся в
поддержке. Конечно, проект не должен иметь все эти признаки, чтобы
называться большим. Миллион строк кода на \Cplusplus{} в
предварительном релизе, предназначенных для одной платформы~--- это
тоже большой проект. Однако программное обеспечение редко остаётся в
стадии предварительного релиза навечно. Если оно успешно, в конечном
итоге кто-то попросит перенести его на другую платформу. Поэтому все
крупные системы программного обеспечения на определённом этапе
становятся похожими.

Большие проекты обычно упрощаются при помощи декомпозиции на отдельные
компоненты, которые обычно собираются в самостоятельные программы или
библиотеки (или и то, и другое). Эти компоненты часто хранятся в
отдельных каталогах файловой системы и управляются при помощи
собственных \Makefile{}'ов. Один из способов сборки всей системы
компонентов подразумевает наличие главного \Makefile{}'а,
вызывающего \Makefile{}'ы компонентов в нужном порядке. Этот подход
называется \newword{рекурсивный \GNUmake{}} (\newword{recursive
  make}), потому что главный \Makefile{} вызывает \GNUmake{}
рекурсивно для обработки \Makefile{}'а каждого компонента. Рекурсивный
\GNUmake{}~--- это общая техника для компонентных
сборок. Альтернатива, предложенная Питером Миллером (Peter Miller) в
1988 году, лишена многих недостатков, присущих рекурсивному
\GNUmake{}, и основана на использовании единственного \Makefile{}'а,
включающего информацию из каталогов компонентов\footnote{
Miller, P.A., \emph{Recursive Make Considered Harmful}, AUUGN Journal
of AUUG Inc., 19(1), pp. 14-25(1998). Эта статья также доступна по
адресу \filename{\url{http://aegis.sourceforge.net/auug97.pdf}} (прим.
автора).}.

Как только проект проходит этап сборки компонентов, обычно на его пути
встают более серьёзные организационные проблемы управления
сборками, включающие управление разработкой нескольких версий проекта,
поддержку нескольких платформ, предоставление эффективного доступа к
исходному коду и исполняемым файлам и осуществление автоматических
сборок. Мы обсудим эти проблемы во второй части этой главы.

%%--------------------------------------------------------------------
%% Recursive make
%%--------------------------------------------------------------------
\section{Рекурсивный \GNUmake{}}
\label{sec:recursive_make}

\index{Рекурсивный \GNUmake{}} Мотивация использования рекурсивного
\GNUmake{} довольно проста: \GNUmake{} отлично работает в рамках
одного каталога (или небольшого набора каталогов), однако его
использование заметно усложняется с ростом числа каталогов. Таким
образом, мы можем использовать \GNUmake{} для сборки большого проекта,
написав для каждого каталога свой простой самодостаточный \Makefile{},
а затем выполнив полученные \Makefile{}'ы по очереди. Мы могли бы
использовать для этой цели сценарий, однако наиболее эффективным
подходом является использование \GNUmake{}, поскольку между
компонентами обычно существуют зависимости более высокого уровня.

Предположим, что мы ведём разработку приложения для воспроизведения
mp3 файлов. Логически его можно разделить на несколько компонентов:
пользовательский интерфейс, кодеки и система управления базой
данных. Эти компоненты могут быть представлены трёмя библиотеками:
\filename{libui.a}, \filename{libcodec.a} и \filename{libdb.a}. Само
приложение является <<клеем>>, связывающим эти три части
воедино. Наиболее простое отображение этих компонентов в структуру
каталогов представлено на рисунке~\ref{fig:file_layout_mp3}.

\begin{figure}
{\footnotesize
\begin{verbatim}
.
|
|--makefile
|
|--include
|  |--db
|  |--codec
|  `--ui
|
|--lib
|  |--db
|  |--codec
|  `--ui
|
|--app
|  `--player
|
`--doc
\end{verbatim}
}
\caption{Структура каталогов проекта mp3 плеера}
\label{fig:file_layout_mp3}
\end{figure}

Более традиционная структура каталогов подразумевает помещение функции
main и <<клея>> в корневой каталог, а не в подкаталог
\filename{app/player}. Я предпочитаю помещать код приложения в
собсвенный каталог, поскольку это делает структуру корневого каталога
более ясной и позволяет легко добавлять в систему новые
модули. Например, если мы решим добавить отдельное приложение для
управления музыкальными каталогами, мы можем аккуратно поместить его в
\filename{app/catalog}.

Если каждый каталог из \filename{lib/db}, \filename{lib/codec},
\filename{lib/ui} и \filename{app/player} содержит собственный
\Makefile{}, то работой головного \Makefile{}'а станет их
последовательный вызов:

{\footnotesize
\begin{verbatim}
lib_codec := lib/codec
lib_db := lib/db
lib_ui := lib/ui
libraries := $(lib_ui) $(lib_db) $(lib_codec)
player := app/player

.PHONY: all $(player) $(libraries)
all: $(player)

$(player) $(libraries):
    $(MAKE) --directory=$@

$(player): $(libraries)
$(lib_ui): $(lib_db) $(lib_codec)
\end{verbatim}
}

Главный \Makefile{} вызывает \GNUmake{} в каждом подкаталоге в рамках
правила, перечисляющего все подкаталоги как цели и выполняющего вызов
\GNUmake{}:

{\footnotesize
\begin{verbatim}
$(player) $(libraries):
    $(MAKE) --directory=$@
\end{verbatim}
}

Переменная \variable{MAKE} должна использоваться всегда при вызове
\GNUmake{} из \Makefile{}'а. \GNUmake{} распознаёт эту переменную и
подставляет на её место реальный путь к исполняемому файлу \GNUmake{},
чтобы все рекурсивные вызовы \GNUmake{} использовали один исполняемый
файл. К тому же, строки, содержащие переменную \variable{MAKE},
обрабатываются особым образом, если используются опции \command{-{}-touch}
(\command{-t}), \command{-{}-just\hyp{}print} (\command{-n}) или
\command{-{}-question} (\command{-q}). Мы обсудим эти детали в
разделе <<\nameref{sec:command_line_options}>> далее в этой главе.

Целевые каталоги помечены как \variable{.PHONY}, поэтому правила
выполняются даже в том случае, когда пересборка целей не
\index{Опции!directory@\command{-{}-directory (-C)}}
требуется. Опция \command{-{}-di\-rec\-to\-ry} (\command{-C})
используется для того, чтобы заставить \GNUmake{} поменять текущий
каталог перед чтением \Makefile{}'а.

Это правило, также довольно тонкое, помогает избежать нескольких
проблем, возникающих при использовании более <<очевидного>> командного
сценария:

{\footnotesize
\begin{verbatim}
all:
    for d in $(player) $(libraries); \
    do                               \ 
        $(MAKE) --directory=$$d;     \
    done
\end{verbatim}
}

Этот командный сценарий не сможет правильно передать ошибки
родительскому \GNUmake{}. Он также не позволит \GNUmake{} выполнять
сборки в разных подкаталогах параллельно. Мы обсудим эту возможность
\GNUmake{} в главе~\ref{chap:improving_the_performance}.

Когда \GNUmake{} планирует выполнение графа зависимостей, реквизиты
цели выглядят для него независимыми. В добавок к этому, две
цели, не связанные зависимостью от третьей цели, также
независимы. Например, библиотеки не  имеют непосредственной
зависимости от цели \target{app/player} или друг от друга. Это
позволяет \GNUmake{} выполнять \Makefile{} для \filename{app/player}
перед сборкой библиотек. Естественно, это вызовет ошибку сборки, так
как компоновка приложения требует наличия библиотек. Для решения этой
проблемы мы добавили дополнительную информацию о зависимостях:

{\footnotesize
\begin{verbatim}
$(player): $(libraries)
$(lib_ui): $(lib_db) $(lib_codec)
\end{verbatim}
}

Этот отрывок сообщает \GNUmake{}, что \Makefile{}'ы в подкаталогах
библиотек должны быть выполнены раньше \Makefile{}'а в каталоге
\filename{player}. Точно так же код библиотеки \filename{lib/ui}
требует, чтобы библиотеки \filename{lib/db} и \filename{lib/codec}
были уже собраны. Это позволяет быть уверенным, что любой код,
требующий генерации (например, исходные файлы на yacc/lex), будет
сгенерирован до того, как начнётся компиляция кода \filename{ui}.

Существует также одна тонкость в отношении порядка сборки
реквизитов. Как и в случае остальных зависимостей, порядок сборки
определяется на основании анализа графа зависимостей, однако когда
реквизиты цели перечисляются в одной строке, GNU \GNUmake{} иногда
собирает их слева направо. Рассмотрим пример:

{\footnotesize
\begin{verbatim}
all: a b c
all: d e f
\end{verbatim}
}

Если нет других зависимостей, требующих рассмотрения, шесть реквизитов
могут быть собраны в любом порядке (например, <<d b a c e f>>), однако
\GNUmake{} в рамках одной строки использует порядок слева направо,
порождая один из следующих результатов: <<a b c d e f>> \emph{или} <<d
e f a b c>>. Несмотря на то, что этот порядок является случайностью
реализации, порядок выполнения будет выглядеть правильным. Легко
забыть, что правильный порядок сборки является счастливой
случайностью, и не предоставить \GNUmake{} полную информацию о
зависимостях компонентов. Рано или поздно анализ зависимостей породит
другой порядок сборки, став причиной ошибок. Таким образом, если набор
целей должен быть собран в определённом порядке, укажите этот порядок
явно при помощи соответствующих реквизитов.

Когда будет выполнен головной \Makefile{}, мы увидим следующий вывод:

{\footnotesize
\begin{verbatim}
make --directory=lib/db
make[1]: Entering directory `/test/lib/db'
Update db library...
make[1]: Leaving directory `/test/lib/db'
make --directory=lib/codec
make[1]: Entering directory `/test/lib/codec'
Update codec library...
make[1]: Leaving directory `/test/lib/codec'
make --directory=lib/ui
make[1]: Entering directory `/test/lib/ui'
Update ui library...
make[1]: Leaving directory `/test/lib/ui'
make --directory=app/player
make[1]: Entering directory `/test/app/player'
Update player application...
make[1]: Leaving directory `/test/app/player'
\end{verbatim}
}

Когда \GNUmake{} определяет, что происходит рекурсивный вызов
\index{Опции!print-directory@\command{-{}-print-directory (-w)}}
\GNUmake{}, он автоматически включает опцию
\command{-{}-print\hyp{}directory} (\command{-w}), руководствуясь
которой, \GNUmake{} печатает сообщения при входе в каталог или выходе
из него. Эта опция также автоматически включается при использовании
опции \command{-{}-directory} (\command{-C}). В добавок ко всему на
каждой строке в квадратных скобках печатается значение переменной
\variable{MAKELEVEL}. В нашем простом примере \Makefile{} каждого
компонента печатает только сообщение о сборке соответствующего
компонента.

%%--------------------------------------------------------------------
%% Command line options
%%--------------------------------------------------------------------
\subsection{Опции командной строки}
\label{sec:command_line_options}

Рекурсивный \GNUmake{}~--- это простая идея, которая очень быстро
становится сложной. Идеальная реализация рекурсивного \GNUmake{} ведёт
себя так, будто множество \Makefile{}'ов системы является одним целым.
Такой уровень координации практически не достижим, поэтому в
реальности всегда приходится идти на компромисы. Тонкие проблемы
станут яснее, когда мы рассмотрим, как должны обрабатываться опции
командной строки.

Предположим, что мы добавили комментарии в заголовочный файл нашего
mp3 плеера. Вместо перекомпиляции всего исходного кода, зависящего от
модифицированного заголовочного файла, мы можем выполнить команду
\index{Опции!touch@\command{-{}-touch (-t)}}
\command{make -{}-touch}, чтобы обновить время модификации всех
файлов. Выполнив эту команду в каталоге с главным \Makefile{}'ом, мы
хотели бы, чтобы \GNUmake{} обновил временные метки всех файлов,
управляемыми дочерними экземплярами \GNUmake{}. Посмотрим, как это
работает.

Когда используется опция \command{-{}-touch}, обычно нормальный
процесс выполнения правил отменяется. Вместо этого \GNUmake{}
производит обход графа зависимостей, обновляя дату модификацию всех
запрошенных неабстракных целей и их реквизитов при помощи команды
\utility{touch}.  Поскольку все наши каталоги помечены как
\variable{.PHONY}, при нормальном ходе событий они будут
проигнорированы (поскольку обновление даты модификации для них смысла
не имеет). Однако мы не хотим, чтобы эти цели игнорировались, нам
требуется выполнение ассоциированных с ними правил. Чтобы обеспечить
правильное поведение, \GNUmake{} автоматически помечает все строки в
сценариях, содержащие переменную \variable{MAKE}, модификатором
\command{+}, в результате чего \GNUmake{} запускает дочерние процессы
\GNUmake{}, несмотря на опцию \command{-{}-touch}.

Когда \GNUmake{} запускает дочерние процессы \GNUmake{}, он должен
позаботиться о передаче им флага \command{-{}-touch}. Это достигается при
помощи переменной \variable{MAKEFLAGS}. Когда \GNUmake{} стартует,
происходит автоматическое добавление большей части опций к переменной
\variable{MAKEFLAGS}. Исключениями являются опции
\command{-{}-directory} (\command{-C}), \command{-{}-file}
(\command{-f}), \command{-{}-old\hyp{}file} (\command{-o}) и
\command{-{}-new\hyp{}file} (\command{-w}). Переменная
\variable{MAKEFLAGS} экпортируется в окружение и считывается дочерними
процессами \GNUmake{} при старте.

Благодаря этой функцональности дочерние процессы \GNUmake{} по
большей части ведут себя так, как вы ожидаете. Рекурсивное выполнение
\variable{\$(MAKE)} и специальная обработка переменной
\variable{MAKEFLAGS}, применяемая к опции \command{-{}-touch}, также
применяется к опциям \command{-{}-just\hyp{}print} (\command{-n}) и
\command{-{}-question} (\command{-q}).


%%--------------------------------------------------------------------
%% Passing variables
%%--------------------------------------------------------------------
\subsection{Передача переменных}

Как уже было замечено, переменные передаются в дочерние процессы
\GNUmake{} через окружение и контролируются при помощи директив
\index{Директивы!export@\directive{export}}
\index{Директивы!unexport@\directive{unexport}}
\directive{export} и \directive{unexport}. Значения переменных, переданные
через окружение, принимаются как значения по умалчанию, однако любое
присваивание изменит их значение. Для того, чтобы разрешить переменным
окружения переопределять локальные присваивания, используйте опцию
\index{Опции!environment-overrides@\command{-{}-environment\hyp{}overrides (-e)}}
\command{-{}-en\-vi\-ron\-ment\hyp{}overrides} (\command{-e}). Вы можете явно
переопределить переменную окружения (даже при включённой опции
\index{Директивы!override@\directive{override}}
\command{-{}-en\-vi\-ron\-ment\hyp{}overrides}) при помощи директивы
\directive{override}:

{\footnotesize
\begin{verbatim}
override TMPDIR = ~/tmp
\end{verbatim}
}

Переменные, определённые в командной строке, автоматически
экспортируются в окружение, если их имена удовлетворяют синтаксису
командного интерпретатора, то есть содержат только буквы, цифры и
подчёркивания. Присваивания переменных в командной строке сохраняются
в переменной \variable{MAKEFLAGS} наряду с другими опциями.

%%--------------------------------------------------------------------
%% Error handling
%%--------------------------------------------------------------------
\subsection{Обработка ошибок}

Что происходит, когда рекурсивный вызов \GNUmake{} обнаруживает
ошибку? На самом деле ничего особенного. Процесс \GNUmake{},
обнаруживший ошибку, завершается с кодом возврата 2. После этого
происходит выход из родительского процесса \GNUmake{}, и ошибка
передаётся вверх по дереву рекурсивных вызовов. Если первый вызов
\GNUmake{} содержал опцию \command{-{}-keep\hyp{}going}
(\command{-k}), она передаётся в дочерние процессы. В этом случае
дочерний процесс \GNUmake{} продолжает нормальное выполнение,
отбрасывает текущую цель и переходит к следующей, не используя цель,
вызвавшую ошибку, в качестве реквизита.

Например, если во время сборки нашего mp3 плеера обнаружится ошибка
компиляции в компоненте \filename{lib/db}, \GNUmake{} закончит
выполнение, вернув код ошибки 2 родительскому процессу. Если мы
\index{Опции!keep-going@\command{-{}-keep\hyp{}going (-k)}}
использовали опцию \command{-{}-keep\hyp{}going} (\command{-k}),
главный процесс \GNUmake{} начнёт обработку следующей независимой
цели, \filename{lib/codec}. Когда сборка этой цели будет закончена,
\GNUmake{} завершит выполнение с кодом возврата 2, поскольку сборка
остальных целей не может быть осуществлена по причине ошибки в
\filename{lib/db}.

\index{Опции!question@\command{-{}-question (-q)}}
Опция \command{-{}-question} (\command{-q}) приводит к похожему
поведению. При включении этой опции \GNUmake{} возвращает код ошибки 1
в случае, если какая-то цель требует повторной сборки, и 0 в противном
случае. Если применить эту опцию к дереву \Makefile{}'ов, \GNUmake{}
будет рекурсивно выполнять \Makefile{}'ы, пока не определит, требует
ли проект сборки. Как только обнаружится файл, требующий сборки,
\GNUmake{} завершит выполняемый в данный момент процесс \GNUmake{} и
<<размотает>> рекурсию.

%%--------------------------------------------------------------------
%% Building other targets
%%--------------------------------------------------------------------
\subsection{Сборка других целей}

Базовые цели для сборки естественны для большинства систем сборок,
однако нам нужны и другие вспомогательные цели, от которых мы зависим,
такие как \target{clean}, \target{install}, \target{print} и так
далее. Поскольку это абстрактные цели, описанная выше техника работает
не очень хорошо.

Например, ниже представлены несколько неработающих подходов:

{\footnotesize
\begin{verbatim}
clean: $(player) $(libraries)
    $(MAKE) --directory=$@ clean
\end{verbatim}
}
или:
{\footnotesize
\begin{verbatim}
$(player) $(libraries):
    $(MAKE) --directory=$@ clean
\end{verbatim}
}

Первый пример не работает потому, что реквизиты цели \target{clean}
вызовут сборку целей по умолчанию в \Makefile{}'ах
\variable{\$(player)} и \variable{\$(libraries)}, а не сборку цели
\target{clean}. Второй пример неверен потому, что для этих целей уже
определён другой командный сценарий.

Один из рабочих подходов основывается на использование цикла
\texttt{for}:

{\footnotesize
\begin{verbatim}
clean:
    for d in $(player) $(libraries); \
    do                               \
      $(MAKE) --directory=$$f clean; \
    done
\end{verbatim}
}

Цикл \texttt{for} не очень хорошо отвечает всем доводам, приведённым
ранее, однако он (вместе с предыдущим неверным примером) приводит нас
к следующему решению:

{\footnotesize
\begin{verbatim}
$(player) $(libraries):
    $(MAKE) --directory=$@ $(TARGET)
\end{verbatim}
}

Добавив к строке с рекурсивным вызовом \GNUmake{} переменную
\variable{TARGET} и выставляя значение этой переменной через командную
стоку, мы можем собирать в дочерних процессах \GNUmake{} произвольные
цели:

{\footnotesize
\begin{alltt}
\$ \textbf{make TARGET=clean}
\end{alltt}
}

К сожалению, это не приведёт к сборке цели \variable{\$(TARGET)} в
головном \Makefile{}'е. Часто это неважно, поскольку головной
\Makefile{} не делает ничего, однако в случае необходимости мы можем
добавить ещё один вызов \GNUmake{}, защищённый функцией \function{if}:

{\footnotesize
\begin{verbatim}
$(player) $(libraries):
    $(MAKE) --directory=$@ $(TARGET)
    $(if $(TARGET), $(MAKE) $(TARGET))
\end{verbatim}
}

Теперь мы можем собрать цель \target{clean} (или любую другую), просто
присвоив соответствующее значение переменной \variable{TARGET} в
командной строке.

%%--------------------------------------------------------------------
%% Cross-Makefile dependencies
%%--------------------------------------------------------------------
\subsection{Общие зависимости}

Специальная поддержка \GNUmake{} переменных командной строки и
коммуникация через переменные окружения подразумевают, что механизм
рекурсивного \GNUmake{} хорошо отлажен. Так в чём же заключаются
упомянутые ранее сложности?

Разделённые \Makefile{}'ы, соединяемые воедино командами
\variable{\$(MAKE)} описывают только наиболее поверхностные
высокоуровневые связи. К сожалению, часто бывают более тонкие
зависимости, скрытые в некоторых каталогах.

Предположим для примера, что модуль \filename{db} включает анализатор,
основанный на \utility{yacc}, для импорта и экспорта мызукальных
данных. Если модуль \filename{ui}, \filename{ui.c}, включает
сгенерированный \utility{yacc} заголовочный файл, на лицо связи между
этими двумя модулями. Если зависимости смоделированы правильно,
\GNUmake{} должен знать, что модуль \filename{ui} требует пересборки в
случае изменения заголовочного файла грамматики. Это нетрудно
организовать, используя технику автоматической генерации зависимостей,
описанную ранее. Однако что если исполняемый файл \utility{yacc} также
изменился? В этом случае после запуска \Makefile{}'а модуля
\filename{ui} корректный \Makefile{} определит, что сначала должна
быть выполнена команда \utility{yacc} для генерации анализатора и
заголовочного файла, и только после этого должна быть осуществлена
компиляция \filename{ui.c}. При нашей декомпозиции этого не случится,
потому что правила для запуска \utility{yacc} находятся в
\Makefile{}'е \filename{db}, а не \filename{ui}.

В этом случае лучшее, что мы можем сделать~--- это убедиться в том,
что \Makefile{} модуля \filename{db} запускается всегда раньше
\Makefile{}'а модуля \filename{ui}. Эта высокоуровневая зависимость
должна быть указана вручную. Мы были достаточно проницательны, чтобы
указать эту зависимость в первой версии нашего \Makefile{}'а, однако в
целом это может стать серьёзной проблемой при поддержке. Поскольку код
добавляется и модифицируется, головной \Makefile{} в какой-то момент
будет неправильно описывать зависимости между модулями.

В продолжение примера предположим, что грамматика \utility{yacc} в
модуле \filename{db} была изменена, и \Makefile{} модуля \filename{ui}
был выполнен до \Makefile{}'а модуля \filename{db} (вручную в обход
головного \Makefile{}'а). \Makefile{} модуля \filename{ui} не содержит
информации о неудовлетворённой зависимости в \Makefile{}'е модуля
\filename{db} и о необходимости запуска программы \utility{yacc} для
изменения головного файла. Вместо этого \Makefile{} модуля
\filename{ui} компилирует программу с устаревшим заголовочным
файлом. Если при модификации были добавлены новые символы, будет
обнаружена ошибка компиляции. Поэтому рекурсивный подход изначально
более хрупок по сравнению с монолитным \Makefile{}'ом.

Ситуация ухудшается с повышением интенсивности использования
генераторов исходного кода. Предположим, в реализации модуля
\filename{ui} был использован генератор заглушек RPC\footnote{
RPC (remote procedure call, вызов удалённых процедур)~--- класс
технологий, позволяющий программному обеспечению вызывать функции,
находящиеся в другом адресном пространстве (прим. переводчика).
},
заголовочные файлы которых используются в модуле
\filename{db}. Теперь нам придётся бороться с перекрёстными ссылками
между модулями. Для решения этой проблемы нам придётся сначала
посетить модуль \filename{db} и сгенерировать заголовочные файлы
\utility{yacc}, затем посетить модуль \filename{ui} и сгенерировать
заглушки RPC, затем вернуться в \filename{db} и произвести компиляцию,
и, наконец, посетить \filename{ui} и завершить процесс компиляции. Число
проходов, требуемое для создания и компиляции исходного кода проекта
зависит от структуры кода и инструментов, при помощи которых он
создаётся. Такой вид перекрёстных зависимостей встречается в сложных
системах довольно часто.

Стандартные решения в настоящих \Makefile{}'ах как правило являются
уловками. Для того, чтобы убедиться, что обновлены все файлы, каждый
\Makefile{} выполняется по команде головного \Makefile{}'а. Заметьте,
что это именно тот подход, который мы использовали в примере с mp3
плеером. Когда происходит запуск головного \Makefile{}'а, каждый из
четырёх дочерних \Makefile{}'ов запускается по очереди. В более
сложных случаях для проверки того, что весь код сначала сгенерирован,
и только затем скомпилирован, дочерние \Makefile{}'ы запускаются по
несколько раз. Чаще всего такие итерации являются напрасной тратой
времени, однако иногда они действительно необходимы.

%%--------------------------------------------------------------------
%% Avoiding duplicate code
%%--------------------------------------------------------------------
\subsection{Избегаем дублирования кода}

Структура каталогов нашего приложения включает три
библиотеки. \Makefile{}'ы этих библиотек очень похожи. Несмотря на то,
что эти библиотеки служат разным целям, все они собираются похожими
командами. Этот тип декомпозиции типичен для больших проектов и ведёт
к большому количеству похожих \Makefile{}'ов и дублированию сценариев.

Дублирование кода~--- это плохо, даже если оно происходит в
\Makefile{}'е. Оно увеличивает стоимость поддержки программного
обеспечения и ведёт к росту количества ошибок. Оно также затрудняет
понимание алгоритмов и определение небольших их вариаций. Поэтому
желательно избежать дублирования кода \Makefile{}'ов, настолько это
возможно. Легче всего достигнуть этого выносом общих частей в
отдельный включаемый файл.

Например, \Makefile{} модуля \filename{codec} содержит следущее:

{\footnotesize
\begin{verbatim}
lib_codec := libcodec.a
sources := codec.c
objects := $(subst .c,.o,$(sources))
dependencies := $(subst .c,.d,$(sources))

include_dirs := .. ../../include
CPPFLAGS     += $(addprefix -I ,$(include_dirs))
vpath %.h $(include_dirs)

all: $(lib_codec)

$(lib_codec): $(objects)
    $(AR) $(ARFLAGS) $@ $^

.PHONY: clean
clean:
    $(RM) $(lib_codec) $(objects) $(dependencies)

ifneq "$(MAKECMDGOALS)" "clean"
  include $(dependencies)
endif

%.d: %.c
    $(CC) $(CFLAGS) $(CPPFLAGS) $(TARGET_ARCH) -M $< | \
    sed 's,\($*\.o\) *:,\1 $@: ,' > $@.tmp
    mv $@.tmp $@
\end{verbatim}
}

Почти весь этот код дублицируется в \Makefile{}'ах модулей
\filename{db} и \filename{ui}. Единственное, что изменяется~--- это
имя библиотеки и исходные файлы. После того, как весь дублированный
код вынесен в файл \filename{common.mk}, мы можем сократить предыдущий
\Makefile{} следующим образом:

{\footnotesize
\begin{verbatim}
library := libcodec.a
sources := codec.c

include ../../common.mk
\end{verbatim}
}

Посмотрим, что вынесено в единственный общий включаемый файл:

{\footnotesize
\begin{verbatim}
MV := mv -f
RM := rm -f
SED := sed

objects      := $(subst .c,.o,$(sources))
dependencies := $(subst .c,.d,$(sources))
include_dirs := .. ../../include
CPPFLAGS     += $(addprefix -I ,$(include_dirs))

vpath %.h $(include_dirs)

.PHONY: library
library: $(library)

$(library): $(objects)
    $(AR) $(ARFLAGS) $@ $^

.PHONY: clean
clean:
    $(RM) $(objects) $(program) $(library) \
          $(dependencies) $(extra_clean)

ifneq "$(MAKECMDGOALS)" "clean"
  -include $(dependencies)
endif

%.c %.h: %.y
    $(YACC.y) --defines $<
    $(MV) y.tab.c $*.c
    $(MV) y.tab.h $*.h

%.d: %.c
    $(CC) $(CFLAGS) $(CPPFLAGS) $(TARGET_ARCH) -M $< | \
    $(SED) 's,\($*\.o\) *:,\1 $@: ,' > $@.tmp
    $(MV) $@.tmp $@
\end{verbatim}
}

Переменная \variable{include\_dirs}, которая раньше была разной для
каждого \Makefile{}'а, теперь одинакова во всех \Makefile{}'ах. Это
достигнуто благодаря переработке пути, используемого для поиска
заголовочных файлов при компиляции: теперь все библиотеки используют
один и тот же путь.

Файл \filename{common.mk} включает также цель по умолчанию для файлов
библиотек. Исходные \Makefile{}'ы использовали в качестве цели по
умолчанию \target{all}. Это вызвало бы проблемы в \Makefile{}'ах
программ, которым требуется указать различные наборы реквизитов для
своих целей по умолчанию. Поэтому включаемая версия кода использует
цель по умолчанию \target{library}.

Заметим, что поскольку общий файл содержит цели, в \Makefile{}'ы
программ он должен включаться \emph{после} цели по умолчанию. Заметим
также, что команда сценария \target{clean} содержит ссылки на
переменные \variable{program}, \variable{library} и
\variable{extra\_clean}. Для \Makefile{}'ов библиотек переменная
\variable{program} содержит пустую строку, в \Makefile{}'ах программ
пустую строку содержит переменная \variable{library}. Переменная
\variable{extra\_clean} добавлена специально для \Makefile{}'а модуля
\filename{db}. Этот \Makefile{} использует переменную для обозначения
кода, сгенерированного программой \utility{yacc}. Код \Makefile{}'а
представлен ниже:

{\footnotesize
\begin{verbatim}
library := libdb.a
sources := scanner.c playlist.c
extra_clean := $(sources) playlist.h

.SECONDARY: playlist.c playlist.h scanner.c

include ../../common.mk
\end{verbatim}
}

При использовании этой техники дублирование кода может быть сведено к
минимуму. Поскольку б\'{о}льшая часть кода вынесена во включаемый
\Makefile{}, со временем он эволюционирует в общий \Makefile{} всего
проекта. Для настройки используются переменные \GNUmake{} и функции,
определяемые пользователем, позволяющие модифицировать общий
\Makefile{} для каждого конкретного каталога.

%%-------------------------------------------------------------------
%% Nonrecursive make
%%-------------------------------------------------------------------
\section{Нерекурсивный \GNUmake{}}

Проекты, содержащие множество каталогов, могут управляться и без
рекурсивного \GNUmake{}. Разница заключается в том, что исходные
файлы, которыми манипулирует \Makefile{}, находятся более чем в одном
каталоге.  Чтобы отразить этот факт, все ссылки на файлы
должны использовать абсолютные или относительные пути к файлам.

Часто \Makefile{} большого проекта содержит множество целей, по одной
для каждого модуля системы. Например, в нашем проекте mp3 плеера нам
понадобились цели для каждой библиотеки и каждого приложения. Также
полезным может быть включение абстрактных целей для групп компонентов,
таких, например, как группа всех библиотек. Цель по умолчанию, как
правило, производит сборку всех этих целей. Часто цель по умолчанию
также производит составление документации и запуск процедур
автоматического тестирования.

Наиболее простой способ использования нерекурсивного \GNUmake{}~---
включение всех целей, ссылок на объектные файлы и зависимостей в один
\Makefile{}. Это часто не устраивает разработчиков, знакомых с
рекурсивным \GNUmake{}, поскольку в этом случае вся информация о
файлах и каталогах сосредоточена в одном файле, в то время как сами
файлы рассредоточены по файловой системе. Для решения этой проблемы
Миллер в своей статье о нерекурсивном \GNUmake{} предлагает добавлять
в каждый каталог включаемый файл, содержащий список файлов модуля и
правила, специфичные для него. Головной \Makefile{} включает все
дочерние \Makefile{}'ы.

Следующий пример демонстрирует \Makefile{} нашего проекта mp3 плеера,
включающий \Makefile{}'ы модулей из соответствующих каталогов.
{\footnotesize
\begin{verbatim}
# Информация о каждом модуле хранится в следующих четырёх
# переменных. Инициализируем их как простые переменные.
programs    :=
sources     :=
libraries   :=
extra_clean :=

objects      = $(subst .c,.o,$(sources))
dependencies = $(subst .c,.d,$(sources))

include_dirs := lib include
CPPFLAGS     += $(addprefix -I ,$(include_dirs))
vpath %.h $(include_dirs)

MV  := mv -f
RM  := rm -f
SED := sed

all:

include lib/codec/module.mk
include lib/db/module.mk
include lib/ui/module.mk
include app/player/module.mk

.PHONY: all
all: $(programs)

.PHONY: libraries
libraries: $(libraries)

.PHONY: clean
clean:
    $(RM) $(objects) $(programs) $(libraries) \
          $(dependencies) $(extra_clean)

  ifneq "$(MAKECMDGOALS)" "clean"
    include $(dependencies)
  endif

%.c %.h: %.y
    $(YACC.y) --defines $<
    $(MV) y.tab.c $*.c
    $(MV) y.tab.h $*.h

%.d: %.c
    $(CC) $(CFLAGS) $(CPPFLAGS) $(TARGET_ARCH) -M $< | \
    $(SED) 's,\($(notdir $*)\.o\) *:,$(dir $@)\1 $@: ,' > $@.tmp
    $(MV) $@.tmp $@
\end{verbatim}
}

Далее приведён пример \Makefile{}'а модуля \filename{/lib/codec}
(\filename{module.mk}):
{\footnotesize
\begin{verbatim}
local_dir  := lib/codec
local_lib  := $(local_dir)/libcodec.a
local_src  := $(addprefix $(local_dir)/,codec.c) 
local_objs := $(subst .c,.o,$(local_src))

libraries += $(local_lib)
sources   += $(local_src)

$(local_lib): $(local_objs)
    $(AR) $(ARFLAGS) $@ $^
\end{verbatim}
}

Таким образом, информация, специфичная для модуля, хранится во
включаемом файле в каталоге соответствующего модуля. Головной
\Makefile{} содержит только список модулей и директивы
\directive{include}. Давайте рассмотрим файл \filename{module.mk}
более детально.

Каждый файл \filename{module.mk} добавляет к переменной
\variable{libraries} имя текущей библиотеки, а к переменной
\variable{sources}~--- пути к исходным файлам. Переменные с префиксом
\variable{local\_} используются для хранения констант или для
предотвращения повторного вычисления значений. Обратите внимание на
то, что каждый модуль использует одинаковые имена \variable{local\_}
переменных. Именно поэтому вместо рекурсивных переменных используются
простые (объявляемые при помощи оператора \texttt{:=}): так сборки,
затрагивающие несколько \Makefile{}'ов, не подвержены риску
повреждения значений переменных в отдельных \Makefile{}'ах. Как уже
упоминалось, имена библиотек и списки исходных файлов используют
относительные пути. Наконец, включаемый файл содержит правила для
сборки текущей библиотеки. Использование \variable{local\_} переменных
в правилах вполне допустимо, так как цели и реквизиты правил
вычисляются при чтении файла.

Первые четыре строки головного \Makefile{}'а определяют переменные,
дополняемые информацией о каждом отдельном модуле. Эти переменные
должны быть простыми, поскольку каждый модуль будет добавлять к ним
данные из локальных переменных:
 
{\footnotesize
\begin{verbatim}
local_src := $(addprefix $(local_dir)/,codec.c)
...
sources   += $(local_src)
\end{verbatim}
}

Если бы переменная \variable{sources} была рекурсивной, финальное её
значение содержало бы просто последнее значение \variable{local\_src},
повторяющееся снова и снова. Поскольку по умолчанию переменные
являются рекурсивными, применяется явная инициализация пустым
значением.

Следующий раздел содержит вычисление списка объектных файлов,
\variable{objects}, и списка файлов зависимостей при помощи значения
переменной \variable{sources}. Эти переменные являются рекурсивными,
поскольку на данном этапе обработки \Makefile{}'а переменная
\variable{sources} содержит пустое значение. Это значение не будет
использоваться до тех пор, пока не будут прочитаны включаемые
\Makefile{}'ы. В нашем случае наиболее разумно было бы поместить
определение этих переменных после директив включения и объявить эти
переменные как простые, однако расположение переменных,
хранящих списки файлов (\variable{sources}, \variable{libraries},
\variable{objects}), рядом друг с другом упрощает понимание
\Makefile{}'а в целом и является хорошей практикой. К тому же, в
более сложных ситуациях перекрёстные ссылки между переменными
потребовали бы использования рекурсивных переменных.

Далее мы специфицируем обработку заголовочных файлов \Clang{}, указывая
значение переменной \variable{CPPFLAGS}. Это позволяет компилятору
находить заголовочные файлы. Для этой цели используется дополнение
значения (оператор \texttt{+=}), поскольку заранее нельзя сказать, что
значение переменной не определено: опции командной строки, переменные
окружения или конструкции \GNUmake{} могли уже придать ей какое-то
значение. Директива \directive{vpath} позволяет \GNUmake{} находить
заголовочные файлы, располагающиеся в других каталогах. Переменная
\variable{include\_dirs} используется для того, чтобы избежать
повторного вычисления списка включаемых каталогов.

Переменные \variable{MV}, \variable{RM} и \variable{SED} используются
для того, чтобы избежать жёсткой привязки к конкретным программам.
Обратите внимание на регистр имён переменных. Здесь мы следовали
соглашениям, принятым в руководству по \GNUmake{}. Имена переменных,
используемых только внутри \Makefile{}'а, состоят из прописных букв,
имена переменных, значение которых можно задать из командной
строки~--- из заглавных.

В следующей секции \Makefile{}'а всё ещё интереснее. Мы начинаем
раздел явных правил с указания цели по умолчанию, \target{all}. К
сожалению, реквизитом цели \target{all} является переменная
\variable{programs}.  Эта переменная будет вычислена незамедлительно,
однако её значение будет известно только после чтения включаемых
файлов. Таким образом, нам требуется прочитать включаемые файлы перед
тем, как определить цель \target{all}. Однако включаемые файлы
содержат цели, первая из которых станет целью по умолчанию. Чтобы
разрешить эту дилему, мы можем указать цель \target{all} без
реквизитов, прочитать включаемые файлы и добавить реквизиты к цели
\target{all} позднее.

Оставшаяся часть \Makefile{}'а уже знакома вам по предыдущим примерам,
однако всё же стоит обратить внимание на то, как \GNUmake{} применяет
неявные правила. Исходные файлы располагаются в подкаталогах. Когда
\GNUmake{} пытается применить стандартное правило \texttt{\%.o: \%.c},
реквизитом будет файл с относительным путём, например,
\filename{lib/ui/ui.c}. \GNUmake{} автоматически распространит
относительный путь на файл цели и попробует собрать
\filename{lib/ui/ui.o}. Таким образом, \GNUmake{} автомагически
(automagically) делает именно то, что нужно.

Есть один неприятный сбой. Несмотря на то, что \GNUmake{} обрабатывает
пути должным образом, не все инструменты, используемые им, делают тоже
самое. В частности, при использовании \utility{gcc} для генерации
зависимостей, результирующий файл не будет содержать относительного
пути к целевому объектному файлу. Вывод команды \texttt{gcc -M} будет
следующим:

{\footnotesize
\begin{verbatim}
ui.o: lib/ui/ui.c include/ui/ui.h lib/db/playlist.h
\end{verbatim}
}

в то время как мы ожидаем увидеть другое:


{\footnotesize
\begin{verbatim}
lib/ui/ui.o: lib/ui/ui.c include/ui/ui.h lib/db/playlist.h
\end{verbatim}
}

Это нарушает обработку файлов реквизитов. Для решения этой проблемы мы
можем настроить команду \utility{sed} так, чтобы она добавляла
информацию об относительных путях:

{\footnotesize
\begin{verbatim}
$(SED) 's,\($(notdir $*)\.o\) *:,$(dir $@)\1 $@: ,'
\end{verbatim}
}

Тонкая настройка \Makefile{}'а для обхода причуд различных
инструментов является естественной частью работы с \GNUmake{}.
Код переносимых \Makefile{}'ов часто бывают очень сложным из-за
капризов различных наборов инструментов, на которые приходится
полагаться.

Теперь у нас есть добротный нерекурсивный \Makefile{}, однако при
поддержке могут возникнуть проблемы. Дело в том, что включаемые файлы
\filename{module.mk} во многом схожи. Изменения в одном из них скорее
всего приведут к необходимости менять другие файлы. Для небольшого
проекта наподобие нашего mp3 плеера это неприятно. Для большого
проекта, содержащего несколько сотен включаемых файлов, это может быть
фатально. При разумном выборе имён переменных и регуляризации
содержимого включаемых файлов эта болезнь поддаётся лечению. Ниже
приводится включаемый файл \filename{lib/codec} после рефакторинга:

{\footnotesize
\begin{verbatim}
local_src := $(wildcard $(subdirectory)/*.c)
$(eval $(call make-library,
         $(subdirectory)/libcodec.a,
		 $(local_src)))
\end{verbatim}
}

Вместо того, чтобы перечислять исходные файлы явно, мы используем
предположение, согласно которому в сборке нуждаются все исходные файлы
в каталоге. Функция \function{make-library} осуществляет набор
операций, общих для всех включаемых файлов. Эта функция определяется в
начале головного \Makefile{}'а нашего проекта:

{\footnotesize
\begin{verbatim}
# $(call make-library, library-name, source-file-list)
define make-library
  libraries += $1
  sources   += $2

  $1: $(call source-to-object,$2)
    $(AR) $(ARFLAGS) $$@ $$^
endef
\end{verbatim}
}

Функция добавляет исходные файлы и имя библиотеки к соответствующим
переменным, затем определяет явные правила для сборки
библиотеки. Обратите внимание на то, что автоматические переменные
используются с двумя знаками доллара, чтобы отложить их вычисление до
выполнения правила. Функция \function{source-to-object} трансформирует
список исходных файлов в список соответствующих объектных файлов:

{\footnotesize
\begin{verbatim}
source-to-object = $(subst .c,.o,$(filter %.c,$1)) \
                   $(subst .y,.o,$(filter %.y,$1)) \
                   $(subst .l,.o,$(filter %.l,$1))
\end{verbatim}
}

В предыдущей версии \Makefile{}'а мы затушевали тот факт, что
настоящими исходными файлами являются \filename{playlist.y} и
\filename{scanner.l}. Вместо этого в списке файлов мы указывали
сгенерированные \filename{.c} файлы. Из-за этого нам приходилось
указывать их явно и включать дополнительную переменную,
\variable{extra\_clean}. Мы решили эту проблему, позволив переменной
\variable{sources} содержать файлы \filename{.y} и \filename{.l} и
возложив на функцию \function{source-to-object} работу по переводу
имён этих файлов в имена соответствующих объектных файлов.

В дополнение к модификации функции \function{source-to-object} нам
нужно добавить ещё одну функцию, вычисляющую имена выходных файлов
\utility{yacc} и \utility{lex}, чтобы позволить цели \target{clean}
должным образом выполнять свою работу. Функция
\function{generated-source} принимает на вход список файлов и
возвращает список промежуточных файлов:

{\footnotesize
\begin{verbatim}
# $(call generated-source, source-file-list)
generated-source = $(subst .y,.c,$(filter %.y,$1)) \
                   $(subst .y,.h,$(filter %.y,$1)) \
                   $(subst .l,.c,$(filter %.l,$1))
\end{verbatim}
}

Другая полезная функция, \function{subdirectory}, помогает избавиться
от локальной переменной \variable{local\_dir}.

{\footnotesize
\begin{verbatim}
subdirectory = $(patsubst %/makefile,%, \
                 $(word                 \
                   $(words $(MAKEFILE_LIST)),$(MAKEFILE_LIST)))
\end{verbatim}
}

Как уже было замечено в разделе <<\nameref{sec:str_func}>>
главы~\ref{chap:functions}, мы можем получить имя текущего
\Makefile{}'а из переменной \variable{MAKEFILE\_LIST}. Использовав
функцию \function{patsubst}, мы можем извлечь относительный путь из
имени последнего прочитанного \Makefile{}'а. Это помогает устранить
одну переменную и уменьшить разницу между включаемыми файлами.

Нашей последней оптимизацией (по крайней мере, в этом примере)
является использование функции \function{wildcard} для получения
списка исходных файлов. Это прекрасно работает в большинстве сред,
поддерживающих чистоту в каталогах с исходными файлами. Однако мне
приходилось работать в проекте, в котором это было не принято. Старый
код хранился в каталогах с исходными файлами <<на всякий случай>>.
Это влекло реальные затраты, выраженные во времени и нервах
программистов, поскольку средства поиска и замены находили символы в
старом коде, и новые программисты (или старые, не знакомые с модулем)
пытались откомпилировать и отладить код, который никогда не
использовался. Если вы используете современную систему контроля версий
(например, CVS), хранение старого кода в каталогах с исходным кодом
совершенно бессмысленно (поскольку весь код уже хранится в
репозитории), и использование \function{wildcard} становится
оправданным.

Директивы \directive{include} также могут быть оптимизированы:

{\footnotesize
\begin{verbatim}
modules := lib/codec lib/db lib/ui app/player
...
include $(addsuffix /module.mk,$(modules))
\end{verbatim}
}

Для больших проектов даже этот код может стать проблемой при
поддержке, поскольку список модулей может вырасти до сотен или даже
тысяч. При некоторых обстоятельствах более предпочтительным является
автоматическое определение модулей при помощи команды \utility{find}:

{\footnotesize
\begin{verbatim}
modules := $(subst /module.mk,,
             $(shell find . -name module.mk))
...
include $(addsuffix /module.mk,$(modules))
\end{verbatim}
}

Мы обрезаем имена файлов, обнаруженных командой \utility{find}, делая
переменную \variable{modules} более полезной как список модулей. Если
вам этого не требуется, тогда, конечно, можно опустить вызовы
\function{subst} и \function{addsuffix} и просто сохранить вывод
команды \utility{find} в переменной \variable{modules}.  Следующий
пример демонстрирует результирующий \Makefile{}.

{\footnotesize
\begin{verbatim}
# $(call source-to-object, source-file-list)
source-to-object = $(subst .c,.o,$(filter %.c,$1)) \
                   $(subst .y,.o,$(filter %.y,$1)) \
                   $(subst .l,.o,$(filter %.l,$1))

# $(subdirectory)
subdirectory = $(patsubst %/module.mk,%, \
                 $(word                  \
                   $(words $(MAKEFILE_LIST)),
				           $(MAKEFILE_LIST)))

# $(call make-library, library-name, source-file-list)
define make-library
  libraries += $1
  sources   += $2
  $1: $(call source-to-object,$2)
      $(AR) $(ARFLAGS) $$@ $$^
endef

# $(call generated-source, source-file-list)
generated-source = $(subst .y,.c,$(filter %.y,$1)) \
                   $(subst .y,.h,$(filter %.y,$1)) \
                   $(subst .l,.c,$(filter %.l,$1)) 

# Информация о каждом модуле хранится в следующих четырёх
# переменных. Инициализируем их как простые переменные.
modules      := lib/codec lib/db lib/ui app/player
programs     :=
libraries    :=
sources      :=
objects      = $(call source-to-object,$(sources))
dependencies = $(subst .o,.d,$(objects))

include_dirs := lib include
CPPFLAGS     += $(addprefix -I ,$(include_dirs))
vpath %.h $(include_dirs)

MV  := mv -f
RM  := rm -f
SED := sed

all:

include $(addsuffix /module.mk,$(modules))

.PHONY: all
all: $(programs)

.PHONY: libraries
libraries: $(libraries)

.PHONY: clean
clean:
    $(RM) $(objects) $(programs) $(libraries) $(dependencies) \
          $(call generated-source, $(sources))

  ifneq "$(MAKECMDGOALS)" "clean"
    include $(dependencies)
  endif

%.c %.h: %.y
    $(YACC.y) --defines $<
    $(MV) y.tab.c $*.c
    $(MV) y.tab.h $*.h

%.d: %.c
    $(CC) $(CFLAGS) $(CPPFLAGS) $(TARGET_ARCH) -M $< | \
    $(SED) 's,\($(notdir $*)\.o\) *:,$(dir $@)\1 $@: ,' > $@.tmp
    $(MV) $@.tmp $@
\end{verbatim}
}

Использование одного включаемого файла для каждого модуля является
весьма работоспособным подходом и имеет свои преимущества, однако я не
могу с уверенностью сказать, что он является наилучшим. Мой
собственный опыт работы с проектом на \Java{} показывает, что
использование головного \Makefile{}'а, эффективно включающего файлы
модулей, является разумным решением. Этот проект включал 997 отдельных
модулей, около двух десятков библиотек и полдюжины приложений.
Для обработки несвязанных подмножеств кода использовались различные
\Makefile{}'ы. Эти файлы в совокупности содержали примерно 2500 строк.
Общий включаемый файл, содержащий глобальные переменные, функции,
определяемые пользователем, и шаблонные правила, содержал ещё примерно
2500 строк.

Выберите ли вы один единственный \Makefile{}, или же поместите информацию
о модулях в отдельные включаемые файлы, нерекурсивный \GNUmake{}
является хорошим подходом к сборке крупных проектов. Он также решает
много традиционных проблем, связанных с использованием рекурсивного
\GNUmake{}. Единственный недостаток этого подхода, о котором стоит
предупредить~-- для разработчиков, использовавших рекурсивный
\GNUmake{}, потребуется смена парадигмы.

%%--------------------------------------------------------------------
%% Components of large systems
%%--------------------------------------------------------------------
\section{Компоненты больших систем}

Мы рассмотрим две популярных на сегодняшний день модели разработки:
модель свободного программного обеспечения и коммерческую модель.

В модели свободного программного обеспечения каждый разработчик в
основном может полагаться только на себя. Проект имеет \Makefile{} и
\filename{README} файл, и ожидается, что разработчикам потребуется
лишь немного помощи для начала работы. Приоритетами таких проектов, как
правило, являются качество и привлечение к участию всего сообщества,
однако наибольшую ценность имеет участие умелых и хорошо
мотивированных членов сообщества. Это не критика. С этой точки
зрения программное обеспечение должно быть высокого качества,
соблюдение временных ограничений имеет меньший приоритет.

В коммерческой модели разработки разработчики могут иметь различный
уровень подготовки, и все из них должны быть способны выполнить работу
к назначенному сроку. Разработчик, который не может разобраться, как
выполнить свою работу, расходует деньги понапрасну. Если система не
компилируется или не запускается должным образом, вся команда
разработчиков может простаивать: этот сценарий является наиболее
затратным из всех возможных. Для решения подобных проблем процесс
разработки управляется командой поддержки, координирующей процесс
сборки, конфигурацию инструментов разработки, поддержку и разработку,
а также менеджмент новых релизов. В подобной среде процессом управляют 
критерии эффективности.

Как правило, для коммерческой модели характерны более продуманные
системы сборки. Основной причиной такого перевеса является стремление
сократить стоимость разработки программного обеспечения за счёт
повышения эффективности труда программистов. Это, в свою очередь,
должно вести к увеличению прибыли. Это именно та модель, которая
требует от \GNUmake{} наибольшего функционала. Тем не менее, техники,
которые мы обсудим, применимы в случае необходиости и к модели
свободного программного обеспечения.

Этот раздел содержит много общей информации, чуть-чуть специфики
и совсем не содержит примеров. Причина этого заключается в том, что
очень многое зависит от языка разработки и используемой операционной
среды. В главах~\ref{chap:c_and_cpp} и \ref{chap:java} мы рассмотрим
примеры реализации многих возможностей, рассмотренных в этом разделе.

%---------------------------------------------------------------------
% Requirements
%---------------------------------------------------------------------
\subsection*{Требования}

Разумеется, требования различны для каждого проекта и каждой рабочей
среды. Здесь мы рассмотрим основные требования, считающиеся важными во
многих коммерческих средах разработки.

Наиболее общей потребностью команд разработчиков является отделение
исходного кода от бинарных файлов. То есть объектные файлы,
полученные в результате компиляции, должны располагаться в отдельном
дереве каталогов. Это, в свою очередь, позволяет включать ещё
множество возможностей. Отделение дерева каталогов бинарных файлов
сулит множество преимуществ:

\begin{itemize}
%---------------------------------------------------------------------
\item Когда расположение дерева каталогов бинарных файлов определено,
гораздо легче управлять дисковым пространством.
%---------------------------------------------------------------------
\item Различные версии деревьев бинарных файлов могут управляться
параллельно. Например, единственному дереву исходных файлов могут
соответствовать оптимизированное, отладочное, и профилировочное
деревья бинарных файлов.
%---------------------------------------------------------------------
\item Существует возможность одновременной поддержки различных
платформ. Правильным образом реализованное дерево исходных файлов
может быть использовано для параллельной компиляции исполняемых файлов
для различных платформ.
%---------------------------------------------------------------------
\item Разработчики могут взять небольшую часть исходного кода и
позволить системе сборки самостоятельно <<заполнять>> недостающие
файлы с помощью зависимостей исходных файлов и деревьев каталогов
объектных файлов. Это не обязательно требует отделения исходных файлов
от объектных, однако без отделения больше вероятность того, что
система сборки не сможет правильно определить, где следует искать
бинарные файлы.
%---------------------------------------------------------------------
\item Дерево исходного кода может быть сделано доступно только для
чтения. Это даёт дополнительную уверенность в том, что сборка отражает
реальное состояние репозитория.
%---------------------------------------------------------------------
\item Некоторые цели, подобные \target{clean}, можно реализовать
тривиальным образом (и выполнять с колоссальным выигрышем в
производительности), если всё дерево каталогов может быть рассмотрено
как отдельная единица, не требующая поиска и манипуляций файлами.
%---------------------------------------------------------------------
\end{itemize}

Б\'{о}льшая часть этих пунктов является важными преимуществами системы
сборки и может быть проектным требованием.

Возможность управления историей сборок проекта часто является важным
качеством системы сборки. Основная идея заключается в том, что сборка
исходного кода осуществляется по ночам, обычно при помощи задачи
\utility{cron}.
\index{Справочные деревья каталогов}
Поскольку результирующие деревья каталогов, содержащие
исходный код и бинарные файлы, являются не модифицируемыми с точки
зрения CVS, я буду называть их справочными. Эта идея имеет множество
применений.

Во-первых, справочное дерево каталогов исходного кода может
использоваться программистами и менеджерами, которым нужно просмотреть
исходный код. Это может показаться банальным, однако когда число
файлов и релизов растёт, извлечение всего исходного кода из
репозитория ради просмотра одного файла может быть не очень разумно. К
тому же, хоть инструменты для просмотра CVS репозиториев достаточно
распространены, они обычно не предоставляют средств для простого
поиска по всему исходному коду проекта. Для этих целей больше подходят
таблицы символов или даже команды \utility{find}/\utility{grep} (или
\utility{grep -R}).

Во-вторых, справочные деревья бинарных файлов являются индикатором
того, что соответствующая сборка прошла успешно. Когда разработчики
начинают утром свою работу, они уже знают, является ли система
работоспособной. Если проект использует систему автоматического
тестирования, свежие сборки могут использоваться для запуска
автоматических тестов. Каждый день разработчики могут проверять отчёты
с результатами тестов, чтобы определить жизнеспособность системы, не
тратя время на запуск тестов. Если же разработчик имеет на руках
только модифицированную версию исходного кода, имеет место дополнительное
сокращение затрат проекта, поскольку в этом случает разработчику не
нужно терять время на получение исходной версии и сборку. Наконец,
справочные сборки могут запускаться разработчиками для тестирования и
сравнения функциональности определённых компонентов.

Есть и другие способы использования справочных сборок. Для проекта,
состоящего из множества библиотек, прекомпилированные библиотеки,
полученные в результате ночных сборок, могут использоваться
программистами для компоновки собственных приложений с теми
библиотеками, которые они не модифицируют. Это позволяет сократить
цикл разработки за счёт исключения необходимости компиляции большей
части исходного кода при запуске собственных сборок. Разумеется,
лёгкий доступ к исходному коду проекта, располагающегося на локальном
файловом сервере, чрезвычайно удобен, если разработчикам, не имеющим
полной рабочей копии проекта, нужно просмотреть исходный код.

При таком разнообразии применений справочных деревьев каталогов
поддержание целостности их структуры становится чрезвычайно важным.
Одним из простых и эффективных способов повышения надёжности является
объявление дерева каталогов с исходным кодом доступным только для
чтения. Это гарантирует, что дерево отражает состояние репозитория в
момент сборки. Этот аспект может потребовать особого внимания,
поскольку во многих случаях система сборки может принимать попытки
записи в дерево каталогов, в частности, при генерации исходного кода
или создании временных файлов. Объявления дерева каталогов с исходным
кодом доступным только для чтения также предотвращает случайное его
повреждение обычными пользователями, что случается наиболее
часто.

Ещё одним общим требованием к проектной системе сборки является
возможность лёгкого управления различными конфигурациями компиляции,
компоновки и развёртывания системы. Система сборки обычно должна
оперировать различными версиями проекта (которые могут быть различными
ветками в репозитории).

Множество крупных проектов зависят от программного обеспечения третих
разработчиков, представленном в форме библиотек или инструментов
разработки. Если нет других инструментов для управления конфигурацией
программного обеспечения (а обычно их нет), использование
\Makefile{}'а и системы сборки для этих целей часто является разумным
выбором.

Наконец, когда программное обеспечение доставляется заказчику, оно часто
упаковывается на базе текущей рабочей версии. Это может быть также
сложно, как конструирование \filename{setup.exe} файла для Windows или
также просто, как редактирование HTML файла и связывание его с
\filename{jar} архивом. Иногда операция инсталляции сочетается с
обычным процессом сборки. Я предпочитаю разделять сборку и генерацию
инсталлятора на два независимых шага, поскольку, как правило, они
используют совершенно разные процессы. В любом случае, скорее всего,
обе эти операции будут влиять на систему сборки.

%%--------------------------------------------------------------------
%% Filesystem layout
%%--------------------------------------------------------------------
\section{Структура файловой системы}
\label{sec:filesystem_layout}

Как только вы решите поддерживать несколько деревьев каталогов,
содержащих бинарные файлы, встаёт вопрос о структуре файловой системы.
В средах, требующих использования нескольких деревьев каталогов, часто
содержится \emph{много} таких деревьев. Чтобы найти способ
поддержания порядка в таких средах требуется немного подумать.

Наиболее общим способом структурирования этих данных является
выделение большого жёсткого диска как хранилища деревьев каталогов с
бинарными файлами. Корневой (или близкий к корню) каталог содержит
один подкаталог для каждого дерева. Одним из разумных способов
идетнификации этих деревьев является включение в имя каждого каталога
именования поставщика, платформы, операционной системы и параметров
сборки бинарного дерева:

{\footnotesize
\begin{alltt}
\$ \textbf{ls}
hp-386-windows-optimized
hp-386-windows-debug
sgi-irix-optimized
sgi-irix-debug
sun-solaris8-profiled
sun-solaris8-debug
\end{alltt}
}

Если требуется хранить множество сборок, произведённых в разные
моменты времени, для их идентификации обычно используются временные
метки, включенные в имя каталога. Из-за удобства сортировки часто
используются форматы \texttt{гг-мм-дд} и \texttt{гг-мм-дд-чч-мм}:

{\footnotesize
\begin{alltt}
\$ \textbf{ls}
hp-386-windows-optimized-040123
hp-386-windows-debug-040123
sgi-irix-optimzed-040127
sgi-irix-debug-040127
sun-solaris8-profiled-040127
sun-solaris8-debug-040127
\end{alltt}
}

Конечно, способ упорядочивания имён компонентов целиком зависит от
ваших потребностей. Каталог верхнего уровня этих деревьев хорошо
подходит для хранения \Makefile{}'а и отчётов о результатах тестов.

Предыдущая структура хорошо подходит для хранения множества сборок,
осуществляемых разработчиками параллельно. Если команда разработчиков
выпускает <<релизы>>, возможно, для внутренних потребителей, вам стоит
рассмотреть возможность добавления хранилища релизов,
структурированное как множество продуктов, каждый из которых может
иметь номер ревизии и временную метку, как показано на
рисунке~\ref{fig:release_tree_layout}.

\begin{figure}
{\footnotesize
\begin{verbatim}
release
|
|--product1
|  |--1.0
|  |  |--040101
|  |  `--040112
|  |
|  `--1.4
|     `--040121
|
`--product1
   `--1.4
      `--031212
\end{verbatim}
}
\caption{Пример структуры дерева каталогов релиза}
\label{fig:release_tree_layout}
\end{figure}

Продукты могут быть библиотеками, производимыми командой разработчиков
для нужд других разработчиков. Конечно, это могут быть и продукты в
привычном их понимании.

Каковы бы ни были структура файловой системы и среда разработки,
реализацией управляет множество однотипных критериев. Каждое дерево
должно легко идентифицироваться. Освобождение ресурсов должно быть
быстрым и ясным. Полезно иметь возможность перемещать и архивировать
деревья. В добавок ко всему, структура файловой системы должна быть
близка структуре процесса разработки организации. Это позволит
перемещаться по хранилищу непрограммистам, таким, как менеджеры,
инженеры по качеству и составители технической документации. 

%%-------------------------------------------------------------------
%% Automating builds and testing
%%-------------------------------------------------------------------
\section{Автоматические сборки и тестирование}

Как правило важно иметь возможность максимально возможной
автоматизации процесса сборки. Это позволит производить сборку
справочных деревьев каталогов по ночам, сохраняя дневное время
разработчиков. Это также позволяет разработчикам запускать сборки на
собственных машинах без предварительной подготовки.

Для программного обеспечения, находящегося в разработке, часто
возникает множество заявок на сборку различных версий различных
продуктов. Для человека, выполняющего эти заявки, возможность
запланировать несколько сборок и <<пойти прогуляться>> часто является
критичной для поддержки и выполнения заявок.

Автоматизированное тестирование создаёт дополнительные трудности.
Для управления процессом тестирования большинства консольных
приложений могут быть использованы простые сценарии. Для тестирования
консольных приложений, требующих взаимодействия с пользователем, можно
\index{dejaGnu}
использовать утилиту GNU \utility{dejaGnu}. Разумеется, каркасы,
\index{JUnit}
подобные JUnit (\filename{\url{http://www.junit.org}}), также
предоставляют поддержку модульного тестирования приложений, не
требующего графической среды.

Тестирование приложений с графическим пользовательским интерфейсом
готовит дополнительные проблемы.  Для систем, использующих X11, я с
успехом применял тестирование по расписанию с использованием
\index{Xvbf}
виртуального оконного буфера (virtual frame buffer), Xvfb. Для Windows
я не смог найти удовлетворительного решения для автоматизированного
тестирования. Все подходы основаны на сохранении тестовой учётной
записи зарегистрированной системе, а экрана~--- не заблокированным.

