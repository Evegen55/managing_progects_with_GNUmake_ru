%%--------------------------------------------------------------------
%% Cygwin
%%--------------------------------------------------------------------
\section{Cygwin}

Несмотря на то, что есть порт \GNUmake{} под Win32, это лишь малая
часть проблемы переноса \Makefile{}'ов на Windows, поскольку командным
интерпретатором, используемым этим портом по умолчанию, является
\filename{cmd.exe} (или \filename{command.exe}). Это, наряду с
отсутствием большинства инструментов \UNIX{}, делает реализацию
кросс-платформенной переносимости очень сложной задачей. К счастью,
проект Cygwin (\filename{\url{http://www.cygwin.com}}) реализовал
для Windows библиотеку совместимости с Linux, с использованием которой
были перенесены многие программы. Я уверен, что Windows разработчики,
желающие достичь совместимости с Linux или получить доступ к
инструментам GNU, не найти смогут лучшего инструмента.

Я использовал Cygwin на протяжении десяти лет для различных проектов,
начиная с CAD-приложения, построенного на смеси \Cplusplus{} и Lisp, и
заканчивая приложением для управления рабочим процессом, написанным на
чистом \Java{}. Набор инструментов Cygwin включает компиляторы и
интерпретаторы многих языков программирования. Однако Cygwin можно
выгодно использовать даже в том случае, если приложение реализовано
без использования набора компиляторов и интерпретаторов Cygwin. Набор
инструментов Cygwin можно использовать как вспомогательное средство
для координации процессов разработки и сборки. Другими словами, совсем
не обязательно писать <<Cygwin>> приложение или использовать средства
разработки Cygwin, чтобы извлечь выгоду из Cygwin-окружения.

Тем не менее, Linux~--- это не Windows (слава богам!), поэтому при
использовании Cygwin инструментов применительно к <<родным>>
приложениям Windows возникает ряд проблем. Практически все эти
проблемы решаются на уровне символов окончаний строки в файлах и форм
путей к файлам, передающихся между Cygwin и Windows.

%---------------------------------------------------------------------
% Line termination
%---------------------------------------------------------------------
\subsection*{Окончания строк}

Файловая система Windows использует для индикации окончания строки
последовательность из двух символов: символа возврата каретки и
символа окончания строки (CRLF). \POSIX{} системы используют для этой
цели один символ~--- символ окончания строки (LF). Иногда это различие
может стать причиной удивления, если программа вдруг сообщит о
синтаксической ошибке или перейдёт к неверной позиции в файле.
Библиотека Cygwin делает всё возможное для избежания этих
неприятностей. Во время установки Cygwin (или при использовании
команды \utility{mount}) вы можете выбрать, следует ли Cygwin
выполнять преобразование файлов, содержащих последовательность CRLF в
качестве индикатора окончания строки. Если выбран формат файлов DOS,
Cygwin будет заменять последовательной CRLF на символ LF при открытии
файла и производить обратное преобразование при записи текста, таким
образом, \UNIX{}-программы могут правильно работать с текстовыми
файлами DOS.  Если вы планируете использовать родные инструменты
наподобие Visual C++ или Sun Java SDK, выбирайте формат файлов DOS.
Если же вы планируете использовать компиляторы Cygwin, используйте
формат \UNIX{} (вы сможете изменить своё решение в любое время).

В добавок ко всему, Cygwin поставляется с набором инструментов для
перевода форматов файлов. Программы \utility{dos2unix} и
\utility{unix2dos} помогут преобразовать файлы в нужный формат в
случае необходимости.

%---------------------------------------------------------------------
% Filesystem
%---------------------------------------------------------------------
\subsection*{Файловая система}

Cygwin предоставляет \POSIX{}-взгляд на файловую систему Windows.
Корневой каталог файловой системы \POSIX{}, \filename{/}, отображается
в каталог, в который установлен Cygwin. Диски Windows доступны из
псевдокаталога \filename{/cygdrive/\texttt{буква}}. Таким образом,
если Cygwin установлен в каталог
\filename{C:\textbackslash{}usr\textbackslash{}cygwin} (я предпочитаю
именно этот каталог), будет производиться отображение каталогов,
представленное в таблице~\ref{tab:cygwin_dir_mapping}.

\begin{table}
\footnotesize
\center
\begin{tabular}{|l|l|l|}
\hline
Путь Windows & Путь Cygwin & Альтернативный путь Cygwin \\
\hline
\hline
\filename{c:\textbackslash{}usr\textbackslash{}cygwin} &%
\filename{/} &%
\filename{/cygdrive/c/usr/cygwin} \\
\hline
\filename{c:\textbackslash{}Program Files} &%
\filename{/cygdrive/c/Program Files} & \\
\hline
\filename{c:\textbackslash{}usr\textbackslash{}cygwin\textbackslash{}bin} &%
\filename{/bin} &%
\filename{/cygdrive/c/usr/cygwin/bin} \\
\hline
\end{tabular}
\caption{Стандартное отображение каталогов Cygwin}
\label{tab:cygwin_dir_mapping}
\end{table}

Поначалу такое преобразование может быть немного непривычным, однако
на работу программ оно никак не влияет. Cygwin также предоставляет
команду \utility{mount}, позволяющую пользователям получать доступ к
файлам и каталогам более удобным способом. Oпция \utility{mount}
\index{Опции!Cygwin!change-cygdrive-prefix@\command{-{}-change-cygdrive-prefix}}
\command{-{}-change\hyp{}cygdrive\hyp{}prefix} позволяет вам изменить
префикс. Мне кажется, что изменение префикса на \filename{/} может
быть полезно, поскольку в этом случае доступ к дискам становится более
естественным:

\begin{alltt}
\footnotesize
\$ \textbf{mount --change-cygdrive-prefix} /
\$ \textbf{ls /c}
AUTOEXEC.BAT            IO.SYS                     WINDOWS
BOOT.INI                MSDOS.SYS                  WUTemp
CD                      NTDETECT.COM               hp
CONFIG.SYS              PERSIST                    ntldr
C\_DILLA                 Program Files              pagefile.sys
Documents and Settings  RECYCLER                   tmp
Home                    System Volume Information  usr
I386                    Temp                       work
\end{alltt}

Как только вы произведёте это действие, предыдущее отображение
каталогов поменяется на отображение, представленное в
таблице~\ref{tab:modified_dir_mapping}.

\begin{table}
\footnotesize
\center
\begin{tabular}{|l|l|l|}
\hline
Путь Windows & Путь Cygwin & Альтернативный путь Cygwin \\
\hline
\hline
\filename{c:\textbackslash{}usr\textbackslash{}cygwin} &%
\filename{/} &%
\filename{/c/usr/cygwin} \\
\hline
\filename{c:\textbackslash{}Program Files} &%
\filename{/c/Program Files} & \\
\hline
\filename{c:\textbackslash{}usr\textbackslash{}cygwin\textbackslash{}bin} &%
\filename{/bin} &%
\filename{/c/usr/cygwin/bin} \\
\hline
\end{tabular}
\caption{Модифицированное отображение каталогов Cygwin}
\label{tab:modified_dir_mapping}
\end{table}

Если вам нужно передать имя файла Windows-программе (например,
компилятору Visual \Cplusplus{}), вы можете просто передать
относительный путь к файлу, использовав \POSIX{} стиль, предполагающий
использование прямых слэшей. Win32 API не различает прямых и обратных
слэшей. К сожалению, некоторые программы, осуществляющие разбор
аргументов командной строки, интерпретируют все прямые слэши как
опции. Одной из таких программ является команда DOS \utility{print},
ещё одним примером является команда \utility{net}.

Если же используется абсолютный путь, синтаксис, основанный на именах
дисков, всегда вызывает проблемы. Несмотря на то, что программы
Windows обычно легко воспринимают прямые слэши в именах файлов, они
совершенно не способны воспринять синтаксис \filename{/c}. Имя диска
всегда должно преобразовываться в формат \filename{c:}. Для
осуществления прямых и обратных преобразований путей \POSIX{} в пути
Windows Cygwin предоставляет команду \utility{cygpath}:

\begin{alltt}
\footnotesize
\$ \textbf{cygpath --windows /c/work/src/lib/foo.c}
c:\textbackslash{}work\textbackslash{}src\textbackslash{}lib\textbackslash{}foo.c
\$ \textbf{cygpath --mixed /c/work/src/lib/foo.c}
c:/work/src/lib/foo.c
\$ \textbf{cygpath --mixed --path "/c/work/src:/c/work/include"}
c:/work/src;c:/work/include
\end{alltt}

\index{Опции!Cygwin!windows@\command{-{}-windows}}
Опция \command{-{}-windows} преобразует заданный путь \POSIX{} в путь
Windows (или, при указании соответствующего аргумента, наоборот). Я
\index{Опции!Cygwin!mixed@\command{-{}-mixed}}
предпочитаю использовать опцию \command{--mixed}, возвращающую путь
Windows, в котором все обратные слэши заменены на прямые (таком
образом, этот путь может использоваться для работы с программами
Windows). Такие пути гораздо удобнее использовать в командном
интерпретаторе Cygwin, воспринимающем обратный слэш как символ
экранирования. Программа \utility{cygpath} имеет множество опций,
предоставляющих переносимый доступ к важным каталогам Windows:

\begin{alltt}
\footnotesize
\$ \textbf{cygpath --desktop}
/c/Documents and Settings/Owner/Desktop
\$ \textbf{cygpath --homeroot}
/c/Documents and Settings
\$ \textbf{cygpath --smprograms}
/c/Documents and Settings/Owner/Start Menu/Programs
\$ \textbf{cygpath --sysdir}
/c/WINDOWS/SYSTEM32
\$ \textbf{cygpath --windir}
/c/WINDOWS
\end{alltt}

Если вы используете \utility{cygpath} в смешанной Windows/\UNIX{}
среде, вы можете захотеть обернуть его вызовы в переносимые функции:

{\footnotesize
\begin{verbatim}
ifdef COMSPEC
  cygpath-mixed         = $(shell cygpath -m "$1")
  cygpath-unix          = $(shell cygpath -u "$1")
  drive-letter-to-slash = /$(subst :,,$1)
else
  cygpath-mixed         = $1
  cygpath-unix          = $1
  drive-letter-to-slash = $1
endif
\end{verbatim}
}

Если вам нужно только преобразовать букву диска в \POSIX{} форму,
функция \function{drive-letter-to-slash} будет работать быстрее, чем
запуск программы \utility{cygpath}.

Наконец, Cygwin не может спрятать все причуды Windows. Имена файлов,
недопустимые в Windows, также недопустимы в Cygwin. Например, такие
имена, как \filename{aux.h}, \filename{com1} и \filename{prn} не могут
использоваться в \POSIX{} путях, даже при наличии расширения.

%---------------------------------------------------------------------
% Program conflicts
%---------------------------------------------------------------------
\subsection*{Конфликты программ}

Несколько программ Windows имеют точно такие же имена, что и
\UNIX{}-программы. Разумеется, программы Windows не принимают тех же
самых аргументов командной строки и не ведут себя совместимым с
\UNIX{}-программами образом. Если вы случайно вызвали Windows версию
программы, обычным результатом является серьёзное недоумение. Наиболее
проблемными программами в этом плане являются \utility{find},
\utility{sort}, \utility{ftp} и \utility{telnet}. Для достижения
максимальной переносимости убедитесь в том, что вы используете
абсолютные пути к этим программам при работе с \UNIX{}, Windows
и Cygwin.

Если вы тесно используете Cygwin и для сборки вам не нужны базовые
инструменты Windows, вы можете спокойно поместить каталог
\filename{/bin} в начало переменной окружения \variable{PATH}. Это
будет гарантией того, что в первую очередь будут использоваться
инструменты Cygwin, а не их Windows аналоги.

Если ваш \Makefile{} использует инструменты \Java{}, учтите, что
Cygwin включает программу GNU \utility{jar}, не совместимую по формату
со стандартными Sun \filename{jar} файлами. Поэтому каталог \Java{}
jdk \filename{bin} следует поместить в вашей переменной
\variable{PATH} раньше каталога Cygwin \filename{/bin}. Это поможет
вам избежать использования программы Cygwin \filename{jar}.
